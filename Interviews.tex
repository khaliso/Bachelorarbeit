\section{Auswertung der Interviews}

Bei der Durchführung der Interviews wurde darauf geachtet, den Interviewten so weit wie möglich die Möglicheit zu geben, über sowohl Zeit als auch Ort der Interviews zu verfügen. Proband 1 (5. bzw 12.04.) entschied sich für ein Kaffee in zentraler Lage (draußen, nur Durchgangsverkehr, Nachbartische nicht besetzt. Neutral, aber dennoch relativ ungestört); das Interview fand im zweiten Anlauf statt.\newline
Das zweite Interview (23.04.) fand in der Wohnung des Interviewten statt; hier konnte schon im Voraus eine persönliche Beziehung auf Vertrauensbasis aufgebaut werden.
Interview nummer drei (02. 05.) fand im EJSA - Jugendcafé in Regensburg im Aufenthaltsraum statt; hier kam es öfter zu Störungen, und der Interviewte war in direkter Umgebung eines Freundeskreises. Dieser Ort wurde auf ausdrücklichen Wunsch des Interviewten gewählt.
Interview Nummer 4 (03.05.) fand in der Wohnung des Probanden statt. Gelegentlich besuchten dessen Mitbewohner das Interview, allerdings ohne zu Unterbrechen. 

 Das fünfte Interview fand in einem Nebenraum des EJSA - Jugendcafé statt.

IT1 :	46:58
IT2: 	67:05
IT3:	27:47
IT4:	35:05
IT5:	73:19
IT6: 	61:31

Average:			51:57.5
total				5:11:45
max/mindeviation	45:32

In Oduntan's Arbeit wurden Informationslücken bei der Reaktion auf die 
\begin{enumerate}
	\item[Ablehnung des Asylbescheids]
		
	\item[Gesetzlichen Grundlagen]
		solicitor
		appeals and court
	\item [Wohnen]
		content
		eviction
		allocation
	\item [Bildung]
		language
	\item [soziales Umfeld]
		extra-curricular
	\item [Informationsquellen]
		Friends
		Internet
		Interpreters
		Caseworkers	
\end{enumerate} 

gefunden.\newline

Da sich Oduntan's Arbeit auf abgelehnte Asylgesuche beschränkte, konnte schon früh davon asugegangen werden, dass sich bestimmte Bereiche zwischen Ihrer Arbeit und dieser Testgröße unterscheiden würden.

Bei den Interviews in dieser Arbeit wurden Informationslücken in folgenden Bereichen entdeckt:

\begin{enumerate}

	\item[Wohnung]
	\item[Bildung]
	\item[Sprachlich]
	\item[Medizinisch]
	\item[Juristisch]
	\item[Sozial]			%sozieale interaktionen und Rassismus
	
\end{enumerate}

Diese wurden mittels eines von Dervin informierten soziokognitiven ansatzes analysiert.

Dervin's Sense - making:

1. Was führte zu dieser Frage?
2. Was hat Sie mit deinem Leben zu tun?
3. Gesellschaft und Machtverhältnisse?
4. Wurde die Frage beantwortet?
5. Wie?
6. Welche Hindernisse gab es?
7. War die Antwort hilfreich?
8. War die Antwort hinderlich?
9. Auf welche Weise?

info gap according to dervin:
	questions, confusions
	muddles, riddles
	angst				
	
Situationen werden in der sense-making methodology in micro-moment timeline steps dargestellt.(cite dervin, p241) Bei jedem dieser Schritte werden die sense-making elemente untersucht:
\begin{enumerate}
	\item	Welche Fragen taten sich auf?
	\item	Welche Gedanken?
	\item	Welche Gefühle?
	\item 	Welche Emotionen?
\end{enumerate}
Jeder dieser Schritte wird dann durch die Situation - Gap - Metapher gejagt: 

situation, lücke, brücke, resultat.

Deshalb 'nur' 6 interviews. -> Gedanken und Emotionen werden selten mit jemandem besprochen, den man nicht kennt. (IT4, min27: Wie hat sich das angefühlt? -> keine Antwort.)

Da Oduntan ausschließlich mit Probanden mit bereits abgelehntem Asylbescheid gearbeitet hatte, wurde eine gewisse Diskrepanz in den Ergebnissen erwartet.\newline


\subsection{Bildung}

Bildung:
    IT1.1, min21:01: Anderes Schul/Bildungssystem in Syrien (franz.)
    Bayrisch (IT1)
    Wie bekomme ich eine AUsbildung?
    Wie gehe ich mit Rassismus am Arbeitsplatz um? (IT3, min21. vgl Rassismus)
    Mache ich nach der Ausbildungen den Meister dazu? (IT4, min12)
    Was mache ich, wenn ich außerhalb der Regelzeit einen Ausbildungs/Arbeitsplatz brauche? (IT5, min39)
    Wie lerne ich den Umgang mit einem Computer ohne Zugang zu einem? (IT5, min40ff)(min40:58 eigener              Lösungsansatz) -> Demotivation(leider ich kann nicht, min48) -> nicht bestanden

\subsubsection{Vorbildung}
\subsubsection{Bildung in Deutschland}
\subsubsection{Sprache}
Sprachlich:
    Deutsch (IT1.1: min8. min7:47: kann jetzt gegenargumentieren)
    Bayrisch (IT1.1 min15:43, vgl sozial)(IT4, min6+min16)(IT6, min28)
            Bayrischer Dialekt (IT5, min65f)(IT1, min16f)
    Wie kommuniziere ich am besten mit dem Arzt, ohne dessen Sprache zu sprechen? (IT2, min28, vgl medizinisch)(IT4, min12)
    Wie lerne ich die Landessprache? (IT3, min9:33)
    Wie kommuniziere ich ohne Kenntnis der Landessprache? (IT4, min
    Wie kommuniziere ich mit der Polizei ohne Kenntnis der Landessprache? (IT6, min14)

Bildung:
    IT1.1, min21:01: Anderes Schul/Bildungssystem in Syrien (franz.)
    Bayrisch (IT1)
    Wie bekomme ich eine AUsbildung?
    Wie gehe ich mit Rassismus am Arbeitsplatz um? (IT3, min21. vgl Rassismus)
    Mache ich nach der Ausbildungen den Meister dazu? (IT4, min12)
    Was mache ich, wenn ich außerhalb der Regelzeit einen Ausbildungs/Arbeitsplatz brauche? (IT5, min39)
    Wie lerne ich den Umgang mit einem Computer ohne Zugang zu einem? (IT5, min40ff)(min40:58 eigener              Lösungsansatz) -> Demotivation(leider ich kann nicht, min48) -> nicht bestanden
    
    
Sozial:
    Trinkkultur (IT1.1: min8:15, 11:14)
        Trinkkultur in Bayern / Alkoholkonsum (wurde nur in IT1 abgefragt.)
    Anderes Verhältnis zu Mitmenschen zwischen D und Syrien (min13:12, alle Freunde)
    Wie kommuniziere ich mit meiner Familie, ohne Sie in Gefahr zu    bringen? (IT2, min15)
    WIe kontaktiere ich meine Familie (IT1.1, min10:16)
    Bayrisch (IT1.1 min15:43, vgl sprachlich)
    Was gibt es in diesem Land für soziale Regeln? (IT3, min10:20 -> peinlich?)
    Wie lerne ich neue Leute kennen? (IT4, min17)
        In Deutschland ist es sozial akzeptiert, nachzufragen, und zu hinterfragen? (IT5, min54/55)
        
Rassismus:
    IT1.1: Wollen die mich dabei haben? (IT1, min17:51), wollen keinen Kontakt (IT1, min19)
    IT2: Wie verhalte ich mich, wenn Ich rassismus erlebe? (IT2, min57)( IT3, min21)
    Wie gehe ich mit Rassismus am Arbeitsplatz um? (IT3, min21. vgl Bildung)
    Basiert Rassismus teilweise auf MEdienberichten? (IT3, min22:17)(IT6 -> das mit dem unmotivierten Zahnarzhelfer)
    Was mache ich, wenn mir der Zutritt zur Disco verwehrt wird, weil die Aufenthaltsgenehmigung nicht als gültiges Ausweisdokument anerkannt wurde? (IT6, min21)
    
Wohnung:
    Wie bekomme ich eine Wohnung ohne permanente Aufenthaltserlaubnis? (IT4, min25)
    Wie finde ich eine Wohnung mit permanenter Aufenthaltserlaubnis? (IT5, min69)
    
Medizinisch:
    Wie kommuniziere ich am besten mit dem Arzt, ohne dessen Sprache zu sprechen? (IT2, min28, vgl sprachlich)(IT3, min20)
    Der Konsum von mehr Schmerzmitteln als von den Betreuer vorgesehen hat auch 
    negative Konsequenzen? (IT2, min29ff., min45)
    Muss ich meinen Arbeitgeber auf meine Schwerbehinderung hinweisen? (IT5, min36f)
    
    
Juristisch:
    Wie bekomme ich eine permanente Aufenthaltsgenehmigung? (IT4, min25/26; min29/30)
    Was gibt es in diesem neuen Land für Regeln? (IT3, min 9:33)
    Wie verhalte Ich mich bei einer Schlägerei?(IT3, min18f)
    Warum wurde mein Asylgesuch abgelehnt? (IT4, min28)
    Wie lang muss ich auf die Antwort des BAMF warten? (IT6, min4)
    
Sprachlich:
    Deutsch (IT1.1: min8. min7:47: kann jetzt gegenargumentieren)
    Bayrisch (IT1.1 min15:43, vgl sozial)(IT4, min6+min16)(IT6, min28)
            Bayrischer Dialekt (IT5, min65f)(IT1, min16f)
    Wie kommuniziere ich am besten mit dem Arzt, ohne dessen Sprache zu sprechen? (IT2, min28, vgl medizinisch)(IT4, min12)
    Wie lerne ich die Landessprache? (IT3, min9:33)
    Wie kommuniziere ich ohne Kenntnis der Landessprache? (IT4, min
    Wie kommuniziere ich mit der Polizei ohne Kenntnis der Landessprache? (IT6, min14)
    
Sonstige:
    Wie kann ic4h Termine pünktlich wahrnehmen (IT2, min62f.)
    Wie bekomme ich neue (Fußball-)Kleidung? (IT3, min10:29)
    (Infolücke: Als Lackierer keine weißen Klamotten anziehen. Wie bekommen? DIe haben's Ihm gesagt! IT4        min15/16)
    Was ist Bayern eigentlich? Wo bin ich hier? (IT6, min17)
    Wie kann ich günstiger Sport treiben? Fitness ist mir zu teuer. (IT6, min36f)
        Ramadan (IT4, min31f)
    
Zukunft nachfragen:
    Alkoholkonsum in Bayern wurde von IT1 genannt, wurde nicht weiter verfolgt
    IT1.2: min 00:27. Woher wusste er, dass er Revision einlegen kann?


subsection{Interview 1}

Demographisch

3 1/2 jahre in Deutschland.
23, männlich
asylstatus: anerkannt
Syrien

Fluchtgrund:
politische unterdrückung + polizeiliche willkür. (IT1.2 min2ff, ) ANgst vor Folter (in1.2 min11:39)
B: Weil, wenn ich dort bleibe, dann muss ich die Waffe tragen. (...) #00:08:53-0#
Mord auf offener STraße (it1.2 min14:50)

Fluchtroute: Boot

Kontakte


Medizinisch


Wohnung


Bildung
Bildungsgrad: Akademischer Abschluss in Syrien

Rassismus


Bayrisch


Inflormationslücken


Notizen

Setting: Wo? Umgebung?
Zeitpunkt: zwischen Tür und Angel?

4 Aspekte einer Nachricht: Fokus auf Inhalts - und Selbstoffenbarungsebene, weniger Beziehung

Treffpunkt und Zeitpunkt von den Interviewees selbst bestimmen lassen






Grund für Deutschland: Raus aus Syrien, wurde in Passaua uf dem Weg in die Niederlande von der Polizei angehalten.
Sozial: breites Netzwerk an Freunden, auch Deutsche (allerdings eher ältere, bei jüngeren Trinkkultur Problemfaktor).

Hat 'immer noch eine Lücke': Kultur und Tradition in Syrien komplett anders als in Deutschland.
Job: Nachbar ist Besitzer einer Bar, kam so an Türsteher - Job.
Ausbildung zum Pharmazeutisch - technischen Assistenten, da Pharmazie/Chemie - Studium nicht anerkannt.
Mentale Stressfaktoren:
Ein Kontaktverlust vor etwa 6 Monaten zum Zeitpunkt des Interviews zu den Eltern (Kriegszeug eben)

Dem Probanden war nach eigener Angabe klar, dass im Falle eines negativen Asylbescheids Revision eingelegt werden kann.

'[..] wenn ich dort bleibe, dann muss ich die Waffe tragen.'

Reise nach Europa: 'einfach' über die Grenze zur Türkei bei Aleppo, dann 2, 3 tage in der Türkei gelebt, dann einen Schleuser bezahlt und getroffen. Dann Übersetzen nach griechenland. Boot mit 6m Länge, 46 personen, mit Elektromotor. (Anderer interviewee: 6x4m, 60Menschen -> Klassenunterschiede?)
Übersetzen 2 1/2 Stunden gedauert
polizei bringt Ihn und sene beiden Brüder von einer Grenze zur nächsten. 


Probleme: 
    Deutschland hat Trinkkultur um Alkohol -> Schwieriger für nicht - Konsumenten, sich zu integrieren. Syrien hat das nicht, feiern ohne Alkohol im Normalfall.
    Auch der Konsum von Schweinefleisch wurde am Frande angemerkt, jedoch nicht mit der selben Gewichtung wie die zum Alkohol.
    
    Ein weiteres Problem in diesem Kontext ist die Sprachbarriere der relativ frisch angekommenen, die Probleme dabei aufweisen, sich in solchen Situationen zu rechtfertigen
    Sprachbarriere zweistufig: Jemand, der Deutsch kann, mag dennoch noch lange nicht mit der bayrischen Sprache zurecht kommen.
    
    Anderes sozialsystem Deutschland VS syrien: 
        D: Unterscheidung zwischen Kollegen, Bekannte, Freunden.
        S: Alles Freunde.
        
    Bildung: Anderes Schulsystem in Syrien: französisches Schulsystem, welches nach der Einschätzung des Interviewten den Lehrkörper in die Verantwortung nimmt, dem Schüler 75\% des Stoffs im Rahmen des Lehrplans beizubringen. In Bayern kam die Sprachbarriere hinzu: Die Lehrkraft sprach nur mit bayrischem Dialekt, dies führte zu Informationslücken. Der Aufforderung, sich bei
    Unverständnis des bearbeiteten Stoffs immer zu melden, nahm der Interviewte aus Angst vor weiterer sozialer Ausgrenzung nur in seltenen Fällen wahr.\newline
    In dieser Ausbildungsklasse befanden sich 18 deutsche und 4 ausländer; 20 Schüler waren weiblich und 2 männlich.
%Fluchtgrund: Festnahme und Erzwingen eines falschen GEständnisses (nur unterschreiben, nicht lesen) durch Polizeigewalt, 'Wenn wir euch auf dieser Straße wieder sehen sperren wir euch länger weg'
%               Anklagepunkte: Sexuelle Übergriffe, Störung der Anwohner, Sachbeschädigung, ..
% Anderer Grund: Installieren eines (fachlich unkundigen) regimetreuen Fakultätsrats, welcher potentielle Regimegegner ausmachen soll. Junge Menschen sollen in die Armee, nicht an die Uni -> Professoren teilen Ansicht.
%Demo an der Uni gegen das Regime wurde u.a. von Panzern zerschlagen
%Demonstranten verschwinden, werden gefoltert und teilweise erst in Leichensäcken an die Familien zurück übergeben.
%Polizei stellt Schläger in Ihren Dienst und ist Blind betreffend der Verbrechen, die diese begehen (Messern eines Oppositionellen vor einer Moschee)



B: Aber so üblich oder so, so groß wie hier in Deutschland, das habe ich nicht erlebt. Aber mir ist(...)ALso mir interessiert überhaupt nicht ob jemand trinkt oder nicht oder sowas. #00:06:38-2#

B: Mich interessiert einfach, dass jemand zu mir kommt und sagt: Warum? #00:06:42-2#

I: hm? #00:06:45-0#

B: Also warum(...)wir trinken schon, warum trinkst du nicht? Ich sag's ganz ehrlich du hast ja, also.. (unv.) Die Antwort kommt sofort: Du bist hier in Bayern. Du musst trinken! #00:06:58-0#

B: Du musst Schwein essen! #00:06:57-9#

B: Hää? Was hat das mit Integration zu tun? #00:07:06-9#

I: //(lacht) Passiert dir das öfter? #00:07:07-0#

B: Ja also für mich ist viele Geschichten entgegen also (unv.) am Anfang der (...)ja..(...) Am Anfang, so 2016, 2017.. #00:07:17-9#

B: ..War das für mich schon so hart, weil ich einfach nicht verstanden hab, weil mein Deutsch ist ganz schlecht, und deswegen ich habe nicht ganz (...) nicht alles verstanden #00:07:31-8#

I: hm (bejahend) #00:07:31-8#

B: Und (...) Und ich kann nicht antworten. #00:07:33-9#

I: Ja.. #00:07:33-9#

B: Also das ist die, die, die.. also, jetzt hab ich schon meine Gründe und warum antworte ich überhaupt nicht. Jetzt kann ich einfach so meine Taten dann begründen. Warum mache ich das? Warum mache ich das? #00:07:47-9#



\subsection{Interview 2}

Demographisch
19, männlich
Afghanistan
Schiite
Muttersprache Dari, kann auch Persisch. (Sind wie Bayrisch zu Deutsch)

Fluchtgrund


Fluchtroute


Kontakte


Medizinisch


Wohnung


Bildung
Bildung: keine Bildung bis zur Ankunft in Deutschland, lernte auch die Schrift seiner Heimat in Deutschland

Rassismus


Bayrisch


Inflormationslücken


Notizen

Setting: Wo? Umgebung?

Zeitpunkt: zwischen Tür und Angel?

4 Aspekte einer Nachricht: Fokus auf Inhalts - und Selbstoffenbarungsebene, weniger Beziehung

Treffpunkt und ZEitpunkt von den Interviewees selbst bestimmen lassen






Grund zur Flucht: Aber mein Mama hat gesagt: 'Du musst gehen. Deshalb du kannst kein andere Wahl jetz, wenn du beim zweiten Mal (...) beim zweiten Mal verhaftet wirst, dann (...) kriegst du echt (...) zwei, vielleicht zwei oder drei Jahre im
Gefängnisstrafen, also Schlagstrafen und Strafen Geld #00:06:36-3#


Flucht über Pakistan in den Iran, dann Iran in die Türkei. Dort mit einem Schlepper nach Griechenland übergesetzt. Griechenland nach Ungarn zu Fuß; auf der Route wurden die GEflücheten von Hilfsklräften mit Lebensmitteln versorgt. Ungarn nach Deutschland mit dem Zug

Wohnen: Alle bisher bezogenen Wohnungen wurden von den örtlichen Hilfsorganisationen gestellt. Der damals zuständige Vormund und die Betreuer der Wohngruppe halfen beim Aussuchen der Wohnung.\newline
Die erste Wohnung wurde aus medizinischen Gründen verlassen; alle Zimmer waren auf zwei Bewohner ausgelegt. Der Proband leidet an schwerer Migräne (Wurde auch mal für nen gehirntumor gehalten, min48) und hätte daraus resultierend ein Einzelzimmer benötigt, welches in diesem Wohnsystem jedoch nicht vorgesehen war. Deshalb wurde er an das Wohnprogramm einer anderen Sozialorganisation verwiesen (min29).

Medizinisch:
In der ersten bezogenen Wohnung gab es gemäß der Aussage des INterviewees wenige Betreuer, jedoch wurde er noch von seinem Vormund individuell betreut. Die Bewohner der Gruppe wurden zur Selbstständigkeit angehalten, i.e. selbstständige Besuche beim Arzt(min28)
Problem: Kein Übersetzer beim Arztbesuch; Kommunikation 'Mit Händen und Füßen' (min28). Führte zu Missverständnissen.
Rekonstruktion eines Arztbesuches:
Der Interviewte machte sich auf den Weg zum Arzt (Ob geschickt oder aus eigenem Antrieb unklar), ohne über suffiziente Deutschenntnisse oder einen Dolmetscher zu verfügen. Betreuer brachten Ihm allerdings einen Satz bei, um sein Leiden zu beschreiben. Bsp.: "Mir ist schlecht"(min29). Die Nachfrage des Arztes nach "wo?" wurde bereits nichtmehr verstanden (min29); Resultierte in einer potentiell Missverständlichen 'Konversation' mit Händen und Füßen.\newline
Weiteres: Der Interviewee leidet (vermutlich genetisch bedingt, min34) an schwerer Migräne und konsumierte schon seit Kinder/Jugendtagen Schmerzmittel (min29). Im Tagesbetrieb der ersten Wohngruppe bekam er jeweils eine Tablette Schmerzmittel vor dem Schlafen gehen.\newline
Als dieser Betreuer jedoch über einen längeren Zeitraum frei hatte, versorgte er den Betreuten mit 20 Tabletten des Schmerzmittels, um Ihn bis zu seiner Rückkehr zu versorgen (Ob ein Ersatzbetreuer für die Wohngruppe zuständig war oder das Aushändigen mit Anweisungen des Betreuers verbunden war hab ich vergessen zu fragen.)\newline
Der zu diesem Zeitpunkt 16-jährige nahm bei einem Migräneanfall an einem Abend drei Tabletten besagten Schmerzmittels, dann 'B: ähm mir war schwindlig, dann
ich (...) mir war schwindlig, dann ich bin einfach auf dem Bett gelegen und geschlafen bis nächstes Tag.  #00:29:28-7#\newline
Tags darauf hatte der interviewte einen Arzttermin, welcher nach einem kurzen Gespräch feststellte, dass binnen sieben Tagen ein Großteil der 20 Tabletten ohne Regulation konsummiert wurde. Daraufhin kontaktierte dieser den Vormund des Betreuten, besagte Betreuung der Wohngruppe, Heimleitung, das Jugendamt, einen Dolmetscher und andere zuständige Stellen.\newline
Der Interviewte erklärte den Anwesenden, dass die Kombination des schnarchenden Mitbewohners mit seiner schweren Migräne der Auslöser für sein Verhalten gewesen seien; Daraufhin wurde der Leiter der Einrichtung aufgefordert, dem Interviewten ein Einzelzimmer zu geben. Da dies jedoch nicht möglich war, wurde er an eine andere soziale Wohneinrichtung weitergeleitet. \newline
In der neuen Einrichtung wurde auch ein geregelter Konsum der verordneten Medikamente (min32) sowie geordnete Kommunikation mit dem behandelnden Arzt (min41) sichergestellt.

(Migräne in Verbindung mit Ohnmachtsanfällen, min32/33)

Nach eigener Aussage war Ihm zu diesem Zeitpunkt der Zusammenhang zwischen Gesundheitlichen Schäden und Tablettenkonsum nicht bekannt. (mehr dazu ab min42) Die Leber des Probanden wurden im Iran durch Tablettenkonsum geschädigt (Flecken auf der Leber, Schmerzen).
Stand jetzt: Nimmt im Monat ein bis zweimal Tabletten (min44)
'damals ich hatte kein Betreuerin. Ich musste einfach meine Schmerzen wegmachen.'
B: Ich habe nix, nicht gedacht: 'Wenn ich Tabletten nehme, ich mache meine Kopfschmerzen weg, sondern ich bekommen davon Leberschmerzen'. Ich habe nicht gewusst. #00:43:43-9#
    -> Informationslücke: Konsum von Medikamenten. Bildungsbedingt?
    Kontext: 
    B: Im Iran habe ich oft Tablette genommen, jeden Tag dreimal. (...) Musste ich, ich war beim Arzt, der hast du gesagt: 'Du musst Tabletten nehmen'. Also ich hatte Kopfschmerzen #00:44:26-6#

Proband hat als Bauarbeiter gearbeitet, um besagte Schmerzmittel zu finanzieren. Diese Arbeit resultierte in andauernden Rückenproblemen, welche in Deutschland zu einem Abbruch einer Ausbildung als Verkäufer führte.

Informationslücke: gegen Rassismus argumentieren können? (I.e. die klauen unsere Jobs/die kriegen Geld vom Staat in Arsch geschoben) (Gefühl der Hilflosigkeit verhindern-> min59 : Ich finde, vielleicht, der hat recht?)

Informationslücke: Verpasste Termine (min63): Ich habe nicht gewusst! Niemand hat mir gesagt! (Informationslücke auf Seite der Betreuer -> anderes Ordnunggsystem?)
%        ist es eine Informationslücke, wenn für einen bestimmte informationen auf dem Weg verloren gehen?
        Effekt: Arzttermin findet zwei Wochen später statt (Dervin Zeug)

Aufgrund schwerer traumatischer Erlebnisse auf der Fluchtroute: 'Wenn ich nach Afghanistan abgeschoben wieder abgeschoben werde, ich werde nie wieder nach Deutschland kommen. Ich will die wieder diese Grenze sehen. Das war furchtbar. Ich will dort hungrig bleiben, nicht sterben, aber nicht wieder (...) solche Grenze sehen. 
Boot: min 12

Ankunft in Regensburg am 19.09.2015 (so genau angeben?), der Sprachkurs Deutsch begann Tags darauf.

Ausländer im Iran stehen häufig vor dem Risiko, verhaftet und abgeschoben zu werden. Sie verfügen über keine gültigen Papiere und sind dementsprechend  zur Arbeit im informellen Sektor gezwungen.

Der Bruder des interviewten wurde, wie der Interviewte selbst, von der iransichen Regierung angehalten, in Syrien zu kämpfen. Dies hätte gemäß deren Aussage zu einer Bleibeerlaubnis (incl Pass) für Ihn und Seine Familie sowie der Bereitstellung monetärer Mittel geführt.
Der Kampf in Syrien gegen den IS wird gemäß der Aussage des Interviewten von den Schiiten als würdevoll gesehen, weshalb viele dazu geneigt sind, sich diesem anzuschließen. Andere wurden gezwungen.(min15)

Der Vater des Interviewten wurde gemäß eigener Aussage von den Taliban 'geschlachtet', als dieser fünf Jahre alt war.(min17): Und bei meiner Familie, meine Bruder Beispiel sollte hingehen. Aber der hat nicht (...) meine Mama hat nicht akzeptiert. Meine Mama hat gesagt: 'Ich hab einmal eure Vater gesehen, wie Taliban hat geschlachtet. Ich will nicht meine Kinder so sehen. (...) Und meine Mama hat gesagt, ich hatte solche Scheiße erfahrung, ich will nicht wieder diese Erfahrung 

Kontakt zur Familie: Kontakt vor einem Jahr, dann Brief v. Iran an Familie (min 15) -> abhören. (min15), min18. 
Letzter Kontakt vor zwei Monaten zur Mutter, Foto: Lag im Krankenhaus. (min34) 
Informationsstand zur Mutter im Iran vor vier Monaten (über Telefon der Nachbarin, min19/20); der Interviewte zeigte sich sichtlich bedrückt von einer
Flut, welche zum Zeitpunkt des Interviews im Iran und Afghanistan viele Menschenleben forderte.

Kontakt zum Bruder: wurde von der Frau des Bruders über FB kontaktiert (min38)

Kontakte in Deutschland: EJSA - Jugendcafé, Campus Asyl, Betreuer der WG, Ehrenamtliche Nachhilfelehrer. Kein Kontakt mehr zum ehem. Vormund

Bildung: Keine Vorbildung in Afghanistan oder dem Iran; der Interviewte lernte sowohl die lateinische als auch arabische Schrift in Deutschland.

Bayrischer Dialekt wurde nicht als Problem und dem Lernprozess hinderlich angesehen. (Wurde als freundlich und respektvoll wahrgenommen. min24/25)

Informationskanäle: Internet, Telefon

Asylstatus: Aufenthaltserlaubnis (bis Dez.)

Interviewee war zum Zeitpunkt der Ankunft in Deutschland nicht volljährig; ein Vormund kümmerte sich deshalb um alle Anfänglichen Aufgaben(min27).

Wie viel über die persönlichen Umstände des Interviewten erzählen? -> Vertraut dennoch in den deutschen Staat
Auf der Flucht: Türkische Polizei hat den ertrinkenden eines gekenterten Bootes auf der Fluchtroute meim Ertrinken zugeschaut, (min 12) dennoch Vertrauen in deutsches Rechtssystem

Potentielle Nachfragepunkte Interview 5: min25, Aber (...) Mein Chef war nicht gut. (...).
                -> potentielles follow-up.
                
\subsection{Interview 3}

Demographisch
Fluchtgrund
B: ähm, hm. (...) Ah, als ich nach Deutschland, bevor dass ich nach Deutschland komme, habe mir gedacht: 'Ja, wenn ich in Deutschland bin, oder in Europa' #00:25:04-6#

I: hm (bejahend) #00:25:04-6#

B: ähm, 'kann ich dann normal leben vielleicht?' Also, ich hab mir so gedacht am Anfang. Dass ich dort normal Leben kann.  #00:25:12-1#

Fluchtroute
Kontakte
Medizinisch
Wohnung
Bildung
Rassismus
Bayrisch
Inflormationslücken

Notizen
Erwähnt mehrmals, wie wichtig Regeln in D sind (zB min14:48)
schließt aus mangel an kommunikationsfähigkeit auf potentiell Ärger mit dem Gesetz?(min16:52)

Situation: Im Jugendcafé, am X.X.2019. Proband umgeben von Freunden und Bekanntschaften, die immer weider an den Tisch kommen. Wurde von Ihm explizit gewünscht, das da zu machen.

Demographisch:
männlich, 21
Geboren im Iran, Afghanische Familie (sieht sich selbst als Afghane, und wird im iran als afghane diskriminiert.)
4 Jahre in Deutschland, seit 2015
Bildung:(min4)	hat im Iran lesen und schreiben gelernt, wurde von Bekannten unterrichtet.
Fluchtroute: (Landroute?)
Fluchtgrund: Unterdrückung?
Asylstatus: Asylverfahren läuft noch (§25a), Abschiebeverbot wgn im Iran geboren, laufendes Verfahren wegen schwerer Körperverletzung)

Kontakte:   Soziale Einrichtung, EJSA - Jugendcafé dient als Ankerpunkt und soziale Stütze?
			Cousin (gemeinsam mit Ihm geflohen, min7)
			Fußballmannschaft (min7)
			Freundin, Ihre Familie
			Familie zu Hause (Internet: Facebook? min24)
			
Berichtet davon, Anfangs Probleme mit den Regeln, der Sprache, (Was ist peinlich?) gehabt zu haben. Hatte zwei Monate vor dem Start seines Deutschkurses komplett 'frei', er sieht dies als positive Erfahrung, welche eine Adaption an die neue Kultur ermöglichte. (min11: B: Genau. Dann haben wir mit durch (unv.) unso weiter haben wir immer ein bisschen mehr verstanden, Regeln und die Sprache undso. Das wird immer schwieriger dann,  #00:10:12-8#

I: hm (bejahend) #00:10:12-8#

B: wenn man nicht versteht und macht irgendwas, dann denkt man nicht ob das (...) schlimm ist undso ne? #00:10:16-6#)

Hebt die Lehrerin Seines Deutschkurses explizit als positiv hervor, hald Ihm, sich wohl zu fühlen und die Kultur zu verstehen.
Vertraut dem Deutschen Gesetzwerk und damit der Polizei; Iranische Polizei: Willkürlich, macht, was Sie will.(min15)

ab min 18: Schlägerei bei Fußballspiel gegen Regenstauf; Ist momentan wegen schwerer Körperverletzung angeklagt.

min22:17: äußeret Wunsch, dass Ausländer differenziert und individuell betrachtet werden (nicht nur als ausländer)

Notiz zu Angst vor der ABschiebung:
Abgeschobene, die dann starben. (min27f)
			
Context -> Situation -> Information need
Informationslücken:
Sprachbarriere nach der Ankunft (min20)
Rassismus (min21) (in Regensburg weniger, in Dresden mehr)

Potentielle Nachfragepunkte Interview 3:

\subsection{Interview 4}
(Nicht sicher, ob der sich so recht wohlgefühlt hat. Interview wurde im Wohnzimmer seiner WG durchgeführt; Platz/Zeit wurden von Ihm ausgewählt. Kurz vor Ramadan, deswegen jetzt noch schnell)

Setting

Demographisch
Fluchtgrund
Fluchtroute
Kontakte
Medizinisch
Wohnung
Bildung
Rassismus
Bayrisch
Inflormationslücken
Notizen

Demographisch:
männlich, 20
Afghanistan, Sunnite
Seit Anfang 2015 in Deutschland
Fluchtroute: Landweg (LKW in der Türkei bestiegen, in Regensburg wieder raus (min2))
Asylverfahren: Duldung (Seit Anfang 2018), laufendes Verfahren nach §25a). Interview beim BAMF:                        min26/27. 2x abgelehnt,
            


Kontakte:
Anfangs keine in Deutschland (min5)
Afghanistan: Onkel
Jugendcafé
Betreuer der WG

Wohnungssuche:
    Nicht möglich, ohne unbefristete Aufenthaltserlaubnis, bei laufendem Asylverfahren eine 'normale' Wohnung zu mieten(min25). Deshalb Sozialwohnung einer Örtlichen Vereinigung (St. VInzent)

Medizinisch:
wurde von den Behörden übernommen

Bildung:
2 Jahre Schulbildung + 1 Jahr Koranschule
Macht vielleicht Meister nach seiner Ausbilung in Deutschland? (min12)

Infolücken:
Sprachbarriere, anfangs Verständigund mit Händen und Füßen(min4). Behandlung der verletzten Beine        nach der Flucht ohne Kommunikationsmöglichkeit mit den behandelnden Ärzten. (Kann sich nicht mehr     an viel erinnern, min12)
sprachbarriere: bayrisch (machte deutsch lernen schwieriger). lösung: arbeit (min 6,'oh! so               funktioniert!' min16)
    (B: Und ich hab auch ein Tag zum Beispiel 'Habadere' gehört, das war mein erste Wort das ich gelernt hab. #00:05:25-3#)
    Hat bei einer vorherigen WG mit deutschen Kindern viel Deutsch gelernt
Wie Praktikum bekommen? (min14) -> In Zusammenarbeit mit den Betreuern und Lehrkräften. Vertrag wird     an die Bedürfnisse des GEflüchteten angeglichen (um die Schulzeit herum)
(Infolücke: Als Lackierer keine weißen Klamotten anziehen. Wie bekommen? Die haben's Ihm gesagt!         min15/16)
Warum reden die nicht mit mir? (min16/17) -> schon vorher in der Schule geklärt: B: Genau, davor haben     wir in die Schule auch gelernt, dass wenn bei deutsche arbeitet oder irgendwas macht, dann die        brauchen länger, dass die dich kennt #00:16:33-0#
Wie Leute kennen lernen? -> Praktikum gemacht, Sprache gelernt etc. min17
Wie bekomme ich eine permanente AUfenthaltserlaubnis? (hier: 25a), min 25/26); min29: Schule,             Ausbildung. Woher die Info? Viele Leute -> diverse Kontakte. NIcht im INternet (min30)
Ablehnung des Asylgesuchs (2x), weiß noch nicht, warum. (min28). War Ihm irgendwann zu viel, er hat's     dann anders probier. (Aber irgendwann habe ich alles gelassen. #00:27:56-3#)


Keine Erfahrungen mit Rassismus in Deutschland (min11)
B: Am Anfang war schon Probleme, aber ich hab das gemerkt, dass die (...) manche Leute so vorsichtig mit mir war. #00:15:55-2#
[..]hald, so, vorsichtig, nicht geredet und so einfach

Notizen:
Trip nach Hamburg mit dem Fahrrad (min19). Erst trainiert, dann losgefahren.
(min21: nicht gewillt, über Autoritäten in Afghanistan zu reden)
Hat gelegentlich als 'Dolmetscher' zwischen Betreuern und NEuankömmlingen fungiert (min23)
Unterschiede zu Heimat: Religiöse Feste (min31f)
    Ramadan von Mitarbeitern akzeptiert. min32

Nachfragepunkte:

\subsection{Interview 5}
Situation/Seiiting: 09.05.19, im Jugendcafé. Nebenraum des Jugendcafé. Interviewee zeigte sich offen und motiviert, wollte seine Geschichte erzählen. Zeit und Ort wurden vom Interviewee ausgewählt: jetzt, hier - vielleicht hinten im Nebenraum?
min54:      Interkulturelle Differenz Deutschland VS naher Osten (nicht verwerflich, nachzufragen)
min55:      hinterfragen der Prozesse -> deutsch, oder an junges Alter in Heimat gebunden?

Demographisch
Männlich, 19
Seit Mitte 2015 in Deutschland
Im Iran geboren; Eltern Afghanen -> Keine iranischen Ausweisdokumente. (min5)
Hat unbefristete Aufenthaltsgenehmigung (min70)


Fluchtgrund:
min8    Bürger zweiter Klasse:B: Wir dürfen nicht in Bäckerei arbeiten. Wir dürfen nicht im Krankenhaus arbeiten. Wir dürfen nicht, zum Besipiel, im ähm (...) Amt arbeiten.[nicht studieren, nicht lernen] #00:07:19-4#
B: Aber hier Gott sei Dank, wir dürfen schon in die Schule gehen. Wir dürfen schon Ausbildung machen. Wir dürfen schon (...) Welche, welche, keine Ahnung, welche Beruf (...) Welche, zum Beispiel welche (...) Wünsche haben wir? Hier, ich glaube (...) am besten. Das, das kann ich nicht einfach mehr, ähm, mehr Sätze dazu sagen. Ich weiß nicht, weil, ähm (...) Wenn du zum Beispiel im Iran bist #00:08:33-2#

Fluchtroute:
min13   Mai 2014 geflüchtet, 11 Monate unterwegs.
min14   Flucht via Schiff, Türkei nach Griechenland. 2 Wochen (10 Tage ohne Essen, mit Spenderniere           Meerwasser getrunken -> Infolücke..min15).
min19   700 Leute Schiff
min34   zu Fuß bis Österreich, dann mit Zug nach Deutschland.

Kontakte
Zwei unbekannte Onkel + zwei Tanten in Afghanistan, die gesehen wegen Gesetzeslage (B: (...) Muss man nochmal schwarz an die Grenze, ja. #00:05:58-7#). 
min 9   Cousin im Iran, Elektriker(privat finanziert). Arbeitet als Einzelhändler, darf nicht                 praktizieren( min 11)
Kollegen in der Arbeit
Betreuer + Mitbewohner in der WG
Jugendcafé
Sportclubs


Wohnung
min69   2 oder 3? wohnt jetzt in einer privaten Wohnung

Medizinisch
min1:   Hat 3 Jahre Dialyse hinter sich (3x wöchentlich, im Iran, start mit 7 oder 8(min3))
min2    Verkleinerte Niere
min3    2012 Nierentransplantation; geht alle 6 Wochen in D zum Arzt+ nimmt MEdikamente
min38   Macht sehr viel SPort in D

Bildung
B: Wir dürfen nicht studieren. Wir dürfen nicht lernen. Als ich krank war, ich ähm (...) als ich klein war, oder jünger war #00:07:26-5# B: Ich war nie in der Schule. #00:07:30-2#
min 8   erste besuchte Schule in Deutschland, mit 15 oder 16
min 8   6 monate Koranschule -> kennt arabische + persische Buchstaben
min 36  2 Jahre Integrationskurs, dann Quali, dann Ausbildung als Elektriker (Vorbildung im Iran, 2,5         Jahre, min55)
min40   Quali in D (nicht bestanden, vgl, Informationslücke)
min49   Beginnt im September Elektrikerausbildung, nachdem er ~6 Monate die Fachbegriffe gelernt hat.         Falls noch nicht bereit, nächsten September
min60   Weiterbildung im Jugendcafé -> Englisch; Schwimmkurs; Kletterkurs

Rassismus

Informationslücken
min36f   Firma sagen, ob Schwerbehindertenkarte vorhanden?
min39   'Was soll ich machen jetzt? -> Ausbildungsplatz war weg; alle Klassen waren voll
min40ff.     Umgang/Maschinenschreiben mit Computern ohne Zugang zu Laptops lernen? (min40:58 eigener              Lösungsansatz) -> Demotivation(leider ich kann nicht, min48) -> nicht bestanden

Bayrisch:
min65f      Kein Problem (nur mit bayerwald-dialekt, min67. wurde aber positiv aufgefasst)

Nachfragepunkte


Notizen
B: Wenn jemand hat (...) ähm, Gesundheit, und sagt: 'Hey, Gott! Warum gibst du mir nicht Geld?' Ich sage: 'Hey, Gott! Gott sei Dank, dass du mir eine, ähm, neue Niere mir gegeben. Und ich gehe nicht zum Dialyse. Ich brauche kein Geld. Ich brauche (...) eigentlich meine Körper einfach gesund wird.  #00:04:05-2#
B: //Schwarz aber sehr gefährlich. Die dürfen schießen an die Grenze #00:06:02-9#
will nie wieder regeln brechen: B: Ja, ähm, jetzt, in Deutschland, wenn ich würde eine, ich wollte eigentlich zum Beispiel (...) eine Straße zum Beispiel ist 100 Meter, ähm, 100 Meter weit weg. Es gibt eine Ampel. #00:15:21-8#
min35f. Optimismus: B: Ich sage nicht, ähm, hey, keine Ahnung: Scheiße Schule, oder Scheiße, keine            Ahnung, die Leute, oder (...) ähm Scheiße Arbeit oder scheiße den oder Scheiße den. Ich sag           garnix! Ich, ich chille! #00:34:20-3#
min51       Irakischer Arbeitskollege kann die Begriffe selber ned, die er Ihm (sofern Sie gepartnert             werden) erklären soll (Irak arabisch, Iran persisch -> kommunikation auf Deutsch. min53)

min55        anderes Verhältnis zu 'Weiß ich nicht, erklär mir das bitte' in Afghanistan
            -> hier einfach nachfragen, in Afghanistan/Iran nicht hinterfragen!

\subsection{Interview 6}

Setting: In der WG des Probanden auf seiner Couch; er kam gerade von der Arbeit heim. Nach einem Gebet begann das Interview. Interviewee brauchte etwas Zeit, um 'aufzutauen', zeigte dann aber Vertrauen.

demographisch:
21, männlich
Afghanistan, im Iran aufgewachsen (mit 8 in den Iran) -> kein Pass
seit 2015 in D
Duldung (25a läuft

fluchtgrund

fluchtroute
min11   Landroute, Türkei anch Österreich im LKW (zu wenig Essen an bord)

kontakte
Jugendcafé
Betreuer der WG
Arbeitskollegen
Freunde in der Stadt (D und Ausländer)
Familie im Iran, über's Telefon

medizinisch
min35   geregelter ablauf f. Boxen -> immer di+do um 17 Uhr.

wohnung
min29   wurde im Bettenlager(erste Wohnung, 20-30 bett-saal) von nem besoffenen RUssen verprügelt
        resultierte in EInzelzimmer, dann X 3,5-4 monate.
min32   dann nach st vinzent, mit dem Kumpel, mit dem er her kam. 1.5 Jahre
min33   dann in's betreute Wohnen, mit besagtem Kumpel (betreutes Wohnen -> keine permanente                  Genehmigung). Kumpel bekam ne permanente und hat jetz ne 'normale' Wohnung

bildung
min5    Ausbildung seit 5 monaten (Karosseriebauer)
min9    Geht man nicht in die Schule. Lernt man nur  in der Werkstatt. #00:08:23-4#
min9    keine Schulbildung in Persien, weder lesen noch schreiben
min10   3 Monate Deutschkurs (erster Tag als sehr hart empfunden)


rassismus
min6    vielleicht Rassismus? Zahntechnikerausbildung fällt weg, wil die keine Ausländer wollen.
        B: Ja. Dort habe ich schon zwei Wochen Praktikum gemacht, und (...) ja, da habe ich schon eine Stelle gefunden, und (...) ja. Und (...) dann (...) ich habe so  zu spät meine Bewerbung abgegeben und die haben eine andere genommen, und (...) der Junge hat schon zwei Monate, drei Monat gearbeitet, und einfach weg(...) gehaut. #00:06:30-7#
min45   alte Frau hat ihn auf offener Straße beschimpft und bespuckt. Alter Mann: Geh dahin wo du             herkommst! Nehmt unser Geld! Geht was arbeiten (war auf dem Weg zum EInkaufen)
        min47: ist nicht darauf eingegangen.
min48   deutscher Freund wird vor 25-30 jährigen angesprochen: 'Hey, was machst du mit dem Ausländer?'         ist nicht darauf eingegangen.
B: ich habe geasgt, wenn du gegen (...) gegenseitig was schimpfst, (...) den Typ, dann bist du auch, so eine (...) so eine Idiot. Dann (...) Wie unterscheidest du zwischen dir und Ihm? #00:49:58-0#


bayrisch
min25   hatte anfänglich Probleme mit bayrisch (wird va in der werkstatt gesprochen) -> mitarbeiter           haben's ihm beigebracht (min26 'die wollen einfach, dass ich bayrisch lerne) 



informationslücken
min4    wie lang muss ich auf's BAMF warten? -> ging nicht mehr zum BAMF, jetzt         §25a
min14   wie mit der Polizei kommunizieren? keiner spricht die Sprache des               anderen, kein Dolmetscher           anwesend. Google Translate! (Freund         konnte persisch lesen und schreiben)
min17   wollten nach Frankfurt zu nem Kumpel. Was ist Bayern? Regensburg?!
min21   mit temporärer aufenthaltserlaubnis zugriff zur Disco verwehrt -> Bürger         2. Klasse? fühlt sich         kacke an. Was machen in der SItuation?
min28   am Anfang: was war das auf bayrisch? -> schrauben falsch gebracht etc
min36   welchen Sport machen? Fitnessstudio is kacke, und alles is verdammt             teuer!(min37 ~100€ mtl.)
        -> selbst im Jugendcafé organisiert.

Notizen
min4    18 Monate keine Antwort vom BAMF auf unbefristete Aufenthaltegenehmigung
min6    wollte eigentlich nicht karosseriebauer werden, hatte schon vier Jahre          im Iran als Mechaniker         gearbeitet. wollte hier was neues machen.
min9    neue ausbildung direkt am ersten Tag des Probearbeitens bekommen
min14   erster kontakt: sehr hilfsbereiter muslimischer taxifahrer
min16   erster kontakt mit polizisten: waren nett, gaben Essen
min20   nur mit Bestechung ausm Knast wieder raus
min23   zist gut zu §25a informiert -> muss ausbildung nicht schaffen.
min39   ausprobieren des Jugendcafé: lernt deutsch, nette leute. er kommt wieder. auch mitgekocht,            aber jetzt überschneidung mit Boxen
min41   Ramadan -> geht nicht ins Jugendcafe (muss npoch für'n Tag darauf               kochen)
min49   Iran: Polizei kacke, Leute fast alle super.
min53   äußert bedenken, dass potentielle rassisten mit 'Heute hat eine                 Ausländer Scheiße gebaut.'            #00:52:31-1# konfrontiert wird. ->         keine individuelle Behandlung des Falles, zuerst der                            'Ausländerfilter'. Objektive AUseinandersetzung mit einem Verbrechen nur         ohne die                     'Ausländer-INfo'
min55   aufenthaltserlaubnis: teil des öffentlichen Lebens; geht zur Arbeit,            trifft Leute.
        Was, wenn die Abschiebung bevor steht? darf nicht in die disco, nicht ausreisen -> Psychicher stress. 'baust du irgendwann scheiße' min56
        min57: hat schon von abgeschobenem nacha fghanistan gehört
        %Interview 3 hatte doch 2, die nach der Abschiebung starben?
        
        B: Dann hörst du solche Geschichten, und dann hast du viel Angst. (...) #00:56:57-0#
        
        B: Die erzählen (...) irgendwelche Geschichte und sagen: 'Ja, du musst dort gehen, und du musst kämpfen, und ja' Möchten natürlich nicht gehen! 'Oh? ähm, du gehst in Gefängnis! (...) vier bis fünf Monaten. Dann, schieben wir dich nach Afghanistan danach.' (...) #00:59:55-5#


\subsection{notes}
Deutschland weniger als erklärtes Zielland, mehr als 'ich will hier weg, irgendwie nach Europa kommen' oder das Besuchen von Bekannten in Deutschland


