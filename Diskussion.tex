\section{Diskussion der Ergebnisse}

Einige der Informationsbedürfnisse und -lücken, die hier festgestellt wurden, decken sich mit denen aus der Literatur. Andere wurden hingegen bislang noch nicht aufgegriffen. Im Folgenden werden die Implikationen dieser Erkenntnisse auf die Literatur und deutsche Integrationssysteme diskutiert.

\subsection{Informationsbedürfnisse aus der Literatur - was ist anders bei deutschen Geflüchteten?}

\citet{oduntan2017investigating} stellten bei Geflüchteten im Vereinigten Königreich Informationslücken in mehreren Gebieten fest:\newline
Rund um abgelehnte Asylsituationen traten verschiedenste Probleme auf. Den Teilnehmern der damaligen Studie war nicht bewusst, weshalb ihr Asylversuch schlussendlich abgelehnt wurde.\newline 
%%mehr
In der vorliegenden Studie kam es zwar bei den Probanden ebenfalls bereits zur Ablehnung des Asylgesuchs, welcher auch in diesem Fall mit Unverständnis und Verwirrung aufgenommen wurde:
\begin{quote}
    ``Ich weiß nicht, warum. Ich habe auch nicht genau verstanden.''
\end{quote}
\centerline{\textit{IT4 min28}}
Allerdings läuft bei allen Teilnehmern, die noch keine unbefristete Aufenthaltserlaubnis erhalten hatten,  noch immer ein Verfahren.\newline
Wie bereits im zweiten Kapitel angesprochen, wurden von Oduntan und Ruthven vier Kategorien identifiziert, in denen Informationslücken in Ihrer Studie auftraten.
\begin{enumerate}
    \item Juristische Informationslücken
    \item Informationslücken bezüglich der Wohnsituation
    \item Informationslücken im Bereich der Bildung
    \item Informationslücken im sozialen Bereich
\end{enumerate}
Die Befragten in der vorliegenden Studie ließen auf keine Informationslücken mit dem rechtlichen Beistand schließen. Dies kann jedoch darauf zurückgeführt werden, dass alle Teilnehmer zum Zeitpunkt der Ankunft in Deutschland noch minderjährig waren. Der staatlich bereitgestellte rechtliche Vormund kümmerte sich um die kritischen ersten Schritte.\newline
Die Teilnehmer dieser Studie äußerten sich deutlich positiver über Ihre Wohnsituation als es in Oduntan's Arbeit der Fall war. Dies ist möglicherweise darauf zurückzuführen, dass die Interviewten entweder in betreuten Wohngruppen oder privat leben. Es kam im Bereich der Wohnsituation zu Problemen, ein Proband wurde während seines kurzen Aufenthaltes in einer Sammelunterkunft von einem Mitbewohner geschlagen (vgl. Wohnen). Jedoch waren in dieser Studie keine Informationslücken und Probleme in der Magnitude festzustellen, wie sie von Oduntan et al. ermittelt wurden.\newline
Für eine klare Aussage bezüglich der Wohnsituation von Geflüchteten und Asylsuchenden in Deutschland sind mehr Daten aus Sammelunterkünften notwendig.\newline 
Im Bereich der Bildung waren sowohl positive als auch negative Erkenntnisse festzustellen. Die Befragten gaben alle an, in Deutschland einen Deutschkurs besucht zu haben. Diese wurden von den zuständigen Behörden verpflichtend angeboten und wies die Beteiligten auf dieses Angebot hin.\newline
Diese Kurse wurden von den Interviewten als lehrreiche und positive Erfahrung empfunden.\newline
Anders als in Oduntan et al.'s Untersuchung traten hier die Probleme vermehrt im späteren Verlauf des Integrationsprozesses auf. Den Interviewten wurde die Möglichkeit gegeben, sich rudimentäre Deutschkenntnisse sowie eine Grundkenntnis der deutschen Kultur und Sozialstruktur anzueignen.\newline
In der Weiterverfolgung dieses Prozesses traten jedoch Lücken auf.\newline
Ein für die Literatur potentiell neuer Aspekt wurde hier entdeckt: Auch wenn die Interviewten in der Lage waren, rudimentär auf Deutsch zu kommunizieren, stellte der lokal geläufige bayrische Dialekt eine große Hürde für alle Befragten dar. Dieser Dialekt weist regional starke Differenzen auf und verfügt über keine konsistente Grammatik.\newline
Die tägliche Zusammenarbeit mit Muttersprachlern, ohne Wertung von diesen zu erfahren, half vielen der Teilnehmer, nach einiger Zeit ohne Probleme mit ihren Mitarbeitern zu kommunizieren.\newline
Ein anderes Problem im Bereich der Bildung war die Zugänglichkeit von teurem Bildungsmaterial. Ein Proband erzählte, dass er den Qualifizierenden Mittelschulabschluss nicht abschloss, weil ihm kein Zugang zu einem Computer zur Verfügung stand. Einige der Prüfungen setzten digitale Vorkenntnisse für die Prüfung voraus.\newline
Im sozialen Bereich wiesen die Interviewten dieser Studie sporatisch Überschneidungen mit den Problemen auf, die in Oduntan et al.'s Untersuchung festgestellt wurden. Ein Interviewter berichtete von großen Problemen, mit seinen altersgleichen Mitschülern in Kontakt zu treten. Schlussendlich gab er seine Bemühungen auf; er fand sozialen Anschluss außerhalb der Schule.\newline
Die Befragten teilten allerdings das Gefühl der Einkerkerung aus Oduntan et al.'s Population nicht, sie zeigten wenig Vorbehalte, mit anderen in Kontakt zu treten. Dies mag allerdings auch durch \textit{sampling bias} erklärbar sein.\newline
Ein Informationsbedürfnis aus Silvio's Arbeit mit südsudanesischen Jugendlichen in den Vereinigten Staaten ließ sich ebenfalls erkennen: Ein Großteil der Befragten wies keine oder mangelnde Bewältigungsstrategien im Umgang mit Rassismus im Alltag auf. Auch diese Gruppe benötigt Informationen, wie am besten mit erlebtem Rassismus umgegangen werden kann.\newline
%Möglicher Umgang: a) Galgenhumor? b) Ein klares Bild vom eingenommenen Platz in der Zielgesellschaft. Wenn auch die Geflüchteten Zweifel an Ihrem Wert für diese Gesellschaft haben, zieht nen Rassismus-Kick in die Selbstwert-zone nochmal deutlich mehr
\citet{hakim2006information} stößt die Überlegung an, ob Sitzungen zum Thema Rassismus für die Betroffenen ins Leben gerufen werden sollten.

\subsection{Was bedeutete das für Systeme, die Menschen mit diesen Bedürfnissen unterstützen wollen?}
Welche Schritte können nun eingeleitet werden?\newline

\subsubsection{Intralinguistische Sprachbarriere}
Ein bisher noch wenig behandeltes Problem in deutschen Integrationssystemen ist die Vielseitigkeit und Fülle an deutschen Dialekten, vor allem im ländlichen Raum. Diese regionalen Dialekte unterscheiden sich oft stark vom Hochdeutschen; die Kenntnis der hochdeutschen Sprache gewährleistet nicht immer die Kommunikationsfähigkeit mit den Einheimischen.\newline
Auf regionale Dialekte (etwa den bayrischen, welcher über keine offizielle Grammatik verfügt) können sich Geflüchtete und Asylsuchende am besten durch konstante Exposition vorbereiten. Dies kann beispielsweise durch das bereitstellen von sprachlich gefärbten Medien erreicht werden.\newline \newline
\subsubsection{Integrationsförderung}
Ein weiterer Weg, die Integration von Migranten zu fördern und gleichzeitig die sozial Schwachen im eigenen Land zu unterstützen, ist die Reform der sozialen Absicherung  und Richtlinien für den Wohnungsbau. Dieser Ansatz wurde unter anderem in Singapur \citep{vasoo2001singapore} \citep{yuen2005squatters} \citep{phang2001housing} und Wien \citep{reinprecht2007social} in verschiedenen Variationen erfolgreich getestet.\newline

\subsubsection{Möglichkeit zur Weiterbildung / digitale Bildung}
Ein weiterer Ansatz zur Verbesserung des aktuellen deutschen Integrations - und Sozialsystems liegt darin, Bedürftigen die Möglichkeit zu geben, sich digital zu bilden. Bei Bedarf können etwa staatlich finanzierte Laptops von ausgewählten Autoritätspersonen, beispielsweise Lehrern, für einen befristeten Zeitrahmen an Bedürftige verliehen werden.\newline

\subsubsection{Rassismus}

Wie können Geflüchtete und Asylsuchende auf Rassismus vorbereitet werden?\newline
Ich möchte hier näher auf einige Phrasen eingehen, an die sich die Interviewten erinnern konnten:
\begin{enumerate}
    \item Die Flüchtlinge kriegen Geld vom deutschen Staat, sind faul und arbeiten nicht.
    \item Die Flüchtlinge nehmen uns unsere Arbeitsplätze weg
    \item Die Flüchtlinge zahlen keine Steuern und kriegen trotzdem Geld vom Staat
\end{enumerate}
Diese Aussagen sind häufig mit dem Vorwurf verbunden, dass durch die Aufnahme von Geflüchteten kein Mehrwert für die deutsche Gesellschaft entsteht.\newline
Wie von \citet{hakim2006information} bereits vorgeschlagen, können Sitzungen für Betroffene ins Leben gerufen werden. So können relevante Informationen wie Bewältigungsstrategien an Leidtragende vermittelt werden. Auch übliche Phrasen wie die oben genannten können in diesen Gruppen analysiert und auf Wahrheitsgehalt überprüft werden.\newline
Im Hinblick auf prognostizierte Entwicklungen könne dieses Konzept auch im Sozialkundeunterricht an staatlichen Schulen versucht werden. Da von einer drastisch steigenden Zahl von Migranten ausgegangen werden kann, sollte auch das Bildungssystem auf diese Herausforderung vorbereitet werden \citep{migration2018report}.

\subsection{Limitationen der Studie}

Diese Studie hat mit der sehr geringen Testgröße (N = 6) sowie dem qualitativen Ansatz zur Datenerhebung keinen Anspruch auf Generalsisierbarkeit.\newline
Ein Problem dieser Studie war es, dass für seine so vielschichtige Gruppe wie "Geflüchtete und Asylsuchende in Deutschland" der Datensatz sehr homogen ausgeprägt war.\newline 
Alle Probanden waren junge Männer aus einem geographisch nah aneinander liegenden Gebiet im Nahen Osten.\newline
Die Probanden waren zum Zeitpunkt der Ankunft minderjährig und hatten damit alle Anspruch auf Bildung sowie einen Vormund, der sie unterstützte.\newline
Da alle Befragten im Jahr 2015 im deutschen Asylsystem registriert wurden, kann keine Aussage über den aktuellen Stand der ersten Instanzen im deutschen Integrationssystem getroffen werden.
In dieser Studie sind beispielsweise keine Daten über die Wohnsituation in den Erstaufnahmeeinrichtungen (den sogenannten Ankerzentren) und deren potentielle Auswirkungen enthalten.\newline
Ein Versäumnis seitens des Interviewers bestand darin, die Barriere des sozialen Alkoholkonsum in Deutschland, welche in IT1 erwähnt wurde, nicht in den Leitfaden mit aufzunehmen.
Auch die Probleme, die durch das Fehlen gültiger Ausweisdokumente bei der Ankunft in Deutschland entstanden, wurden nicht abgefragt.