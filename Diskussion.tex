\section{Diskussion der Ergebnisse}

However, clearly these are not the kind of needs commonly reported in the literature. Herein lies the contribution of your work. You need to explain what makes them different and what this means for a) the literature and b) systems trying to support individuals with such needs.


\subsection{Informationsbedürfnisse aus der Literatur - was is anders bei deutschen Geflüchteten?}

Die Interviewten wurden unter anderem gefragt, weshalb Sie nach Deutschland gezogen seien. Diese Frage rief sehr unterschiedliche Reaktionen hervor:\newline
Einige der Probanden kannten bereits Geflüchtete, welche in Deutschland Asyl erlangt hatten. Sie versprachen sich Hilfe von diesen, und machten sich auf den Weg, um diese zu treffen.\newline
Die Mehrheit der Befragten gab jedoch an, sich aus Angst vor Verfolgung oder dem drohenden Krieg auf den Weg gemacht zu haben. Diese Gruppe hatte Europa als Ganzes als Ziel vor Augen, und war nicht auf eines der Mitgliedsländer fokussiert.\newline


Informationsquellen decken sich mit der vorhandenen Literatur. \cite{oduntan2017investigating} \cite{mykyttschak2018}

-> fisher social cultural skill - based. was is bei meinen anders?

silvio:\cite{hakim2006information}
This is one of the areas about which the youths showed a great deal of concern. They
needed information on how to deal with any racist attitudes and incidents whenever
they encountered them. They also needed information about where to report such
incidents if they ever happened to them and how to be cautious when dealing with a
highly diverse range of people.

conclusion von silvio:
 If possible, even offer sessions on racism and how to deal with it, together with sessions on or brochures about continuing education and how to upgrade their credentials. Librarians and information professionals can also lobby the government on their behalf to make sure that their information needs are understood.


\subsection{Was bedeutete das für Systeme, die Menschen mit diesen Bedürfnissen unterstützen wollen?}

->  einfacheren Zugriff auf vorausgesetzte Lernmaterialien, zB Laptop

Auf regionale Dialekte (etwa den bayrischen, welcher über keine offizielle Grammatik verfügt) können sich Geflüchtete und Asylsuchende am besten durch konstante Exposition vorbereiten. Dies kann etwa durch das bereitstellen von sprachlich gefärbten Medien erreicht werden.?
Ein anderer Weg wurde von den Städten Wien und Singapur vorgezeigt, indem Sozialwohnungen im großen Stil angelegt wurden, welche auf Kulturelle diversität achteten. (Studenten und Flüchtlinge zamschmeißen)
Da alle Befragten im Jahr 2015 im deutschen Asylsystem registriert wurden, kann keine Aussage über die Wirksamkeit der sogenannten Ankerzentren getroffen werden, welche im Jahr 2019 als Erstaufnahmeeinichtungen verwendet werden.
Eine Reform des sozialen Wohnungsbau, wie in SIngapur\cite{vasoo2001singapore}\cite{yuen2005squatters}\cite{phang2001housing} oder Wien\cite{reinprecht2007social}.
        -> keine Ankerzentren, sondern was inklusives (Modell Singapur/Wien?)



\subsection{Limitationen der Studie}

(limitations of the data set?) -> hier qualitative Daten. Da auch nötig?

neues: auch regionale Dialekte der Sprache haben Auswirkungen auf das Integrationsverhalten.

vergessen zu fragen: keine papiere; wie wurde das gelöst?

nur zum Zeitpunkt der Ankunft minderjährige:
    - alle hatten Anspruch auf Bildung
    - gesetzlicher Vormund
    -alle von 2015. da hat sich eineiges geändert. Ankerzentren?
Bericht nur aus dem Blickwinkel der Geflüchteten, kein Kontakt mit der 'Gegenseite' einiger Informationsfragen

    Nicht genug Emphasis auf den Informationsquellen und - kanälen