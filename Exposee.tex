\statement{Was?}

Oduntan und Ruthven stellten im Informationsprozess von Gefl\"uchteten im Asylauswahlverfahrensprozess des Vereinigten K\"onigreichs L\"ucken fest. \cite{oduntan2017investigating}
Mittels semi-strukturierter Interviews nach Dervin's Sense-making Methodology\cite{dervin2003sense} erfolgte eine Zusammenstellung verschiedener Situationen, in denen diese im Rahmen der Studie auftraten - mit dem Hinweis auf die Existenz noch unergr\"undeter Situationen in der Gefl\"uchtetenintegration.\newline
Informationsl\"ucken treten gem\"a\ss{} Derwin aufgrund Abweichungen in Zeit (Ich heute vs. Ich gestern) und Raum (Erfahrung aus dem Blickwinkel verschiedener Kulturen, Werte etc.) auf.\cite{dervin2003sense}\newline
Das Konzept dieser Arbeit ist es, festzustellen, ob auch bei Gefl\"uchteten in Deutschland Informationsl\"ucken w\"ahrend des Integrationsprozesses festzustellen sind.\newline
Die Untersuchung findet im Jahr 2019 statt, in der Folgezeit einer Fl\"uchtlingskrise, welche Europa ab 2015 direkt betraf. \cite{unhcr2015seven}\newline
Hierzu soll das Informationsverhalten von Gefl\"uchteten untersucht werden, um sowohl \"offentliche als auch private Informationsbed\"urfnisse in der deutschen Fl\"uchtlingsintegration festzustellen. Anschlie\ss{}end k\"onnen die zust\"andigen Beh\"orden \"uber eventuelle Unzul\"anglichkeiten informiert werden.

\statement{Warum?}


Ende 2017 waren weltweit 68,5 Millionen Menschen auf der Flucht. Darunter sind Menschen zusammengefasst, welche als interne Vertriebene, Gefl\"uchtete oder Asylsuchende verzeichnet sind. Dies ist die h\"ochste Zahl, die jemals von UNHCR verzeichnet wurde.\cite{uno2018flucht}
Die neuesten Zahlen vom Juni 2018 zeigen keine Ver\"anderung. \cite{uno2018flucht}
Das IPCC geht davon aus, dass sich diese Entwicklung in der n\"aheren Zukunft nicht ver\"andern wird. \cite{sr15ipcc}\newline
Die Zahl der Asylantr\"age in Deutschland ging allerdings von 2016 bis 2018 von 745.545 \"uber 222.683 auf 185.853 zur\"uck. \cite{statistica2019asyl} In der T\"urkei befinden sich (Stand: 2018) 3.5 Millionen Gefl\"uchtete.\newline
\newline
Momentan gibt es Grund zur Annahmne, dass Asylsuchende oft nicht ausreichend auf den Integrationsprozess vorbereitet sind. \cite{oduntan2017information} \cite{gillespie2016mapping}\newline
Das Informationsverhalten von Randgruppen unterscheidet sich oft signifikant von dem der Be- v\"olkerungsmehrheit. \cite{chatman1996impoverished}
Durch einen Mangel an relevanten Informationen wird ein Ausschluss aus der Gesellschaft riskiert, in die die Gefl\"uchteten integriert werden sollen. \cite{andrade2016information}\newline
Als relevant werden in dieser Arbeit alle Informationen gewichtet, die das Leben in der neuen Umgebung beeinflussen - von einem Termin beim \"ortlichen Arzt \"uber Kenntnis der eigenen Rechtslage zur Grundkenntnis der \"ortlichen Kultur und Gesellschaft. \cite{schreieck2017supporting}\newline
Andrade et al. stellten fest, dass ICT's (Information and communications technology) f\"unf Aspekte der Gefl\"uchteten im Hinblick auf soziale Inklusion positiv beeinflussen:
\begin{enumerate}
    \item   Teilnahme an der Gesellschaft
    \item   Effizientere Kommunikation
    \item   Das Verst\"andnis der neuen Gesellschaft
    \item   Pflege sozialer Kontakte
    \item   Ausdr\"ucken einer eigenen kulturellen Identit\"at
\end{enumerate}
Diesbez\"uglich gibt es weitere Ans\"atze:\newline
Schreieck et al. etwa schufen ein Design f\"ur mobile Applikationen, das an verschiedene Kulturkreise angepasst werden kann. \cite{schreieck2017supporting}\newline

%Schreieck: To derive design principles, we applied a Design Science Research approach based on a requirements analysis via a survey among refugees. Subsequently, the mobile application’s design is iteratively improved based on the results of three user studies with 127 participants of Arabic, African, and Western background.

Jones et al. arbeiteten an einem anderen Ansatz: Sie kreierten ein Zuweisungssystem, in dem Gefl\"uchtete und entweder Staaten oder lokale Areale aneinander verwiesen wurden. Hierbei m\"ussen beide Parteien einer Zuweisung zustimmen. \cite{jones2017matching}\newline
Um Projekte wie diese zu verbessern, muss auch erschlossen werden, welche Informationsbed\"urfnisse die jeweiligen Zielgruppen erwarten.\newline
Im Rahmen der sogenannten Fl\"uchtlingskrise wurden in den Aufnahmezentren vom {Bundesamt f\"ur Migration und Fl\"uchtlinge} (BAMF) zahlreiche Neuanstellungen vorgenommen. Bis zum Mai 2017 wurden gem\"a\ss{} des BAMF 454 dieser Neuanstellungen nicht ausreichend trainiert. Dennoch wurde auch dieses Personal damit beauftragt, Asylrechtsentscheidungen zu treffen.\newline
Die Zahl der Untrainierten wurde bis zum Februar 2018 auf 36 reduziert, 769 der 2139 Angestellten hatten allerdings die Ausbildung noch nicht abgeschlossen.\newline
Damit gibt diese Situation Grund zur Annahme, dass Asylsuchende nicht ausreichend auf einige der kommenden Herausforderungen vorbereitet wurden. \cite{asylum2018bamf}
Mittels semi-strukturierter Interviews werden die Erlebnisse der Teilnehmer unter deutscher Jurisdiktion ergr\"undet. Dabei sollen mittels Dervin's sense-making methodology Informationsl\"ucken aufgedeckt werden.\cite{dervin2003sense}\newline
Mit dieser kann festgestellt werden, wie ein Publikum verschiedene Situationen im Lauf des Lebens logisch begr\"undet hat. Um dies zu erm\"oglichen, sind die Probanden eingeladen, Erfahrungen und Schlussfolgerungen mit Hilfe des Forschers so detailliert wie m\"oglich wiederzugeben.\newline
Gem\"a\ss{} Dervin wird dennoch immer davon ausgegangen, dass eine erstellte Kategorisierung falsch, irrelevant oder f\"ur andere Sitautionen unangemessen sein kann. Um diese zu \"uberpr\"ufen, wird zum Dialog zwischen wissenschaftlichen Institutionen und dem allt\"aglichen Leben aufgerufen.\textit{Diese Arbeit kann Aufschluss auf die Anwendbarkeit der in Oduntan et al.'s Forschung erarbeiteten Kategorien f\"ur den deutschsprachigen Raum geben.}\newline
Ein Interview, bei dem Sense-Making angewandt wird, kann auf dem Ansatz der micro-moment timeline basieren. Bei dieser werden die Interviewten gebeten, eine pers\"onliche Situation mit Bezug zum Forschungsfokus genau zu beschreiben (In diesem Fall: Eine Situation abrufen, in der eine Informationsl\"ucke Auswirkungen auf das Leben eines Gefl\"uchteten hatte).
Die von Oduntan et al. festgestellten Kategorien, in denen Informationsl\"ucken geh\"auft auftraten, waren:
\begin{enumerate}
    \item Ablauf des Asylverfahrens\newline
    Auch ein anhaltender Einfluss auf die mentale Gesundheit sowie getroffene Entscheidungen nach der Ablehnung sind hier relevant.
    \item Juristische Komponente \newline
    Beispiel: War dem Gefl\"uchteten klar, dass im Fall einer negativen Asylentscheidung Berufung eingelegt werden kann?
    \item Wohnen \newline
    Wie viele Wohnungen hat der Proband nach seiner Flucht bewohnt, wie lange im Durchschnitt, aus welchen Gr\"unden erfolgten Umz\"uge? Momentane Zufriedenheit mit der Wohnsituation? Wie wurde die momentane Wohnung entdeckt?
    \item Bildung \newline
    beinhaltet Erfahrungen mit dem deutschen Bildungs - und Schulungssystem
    \item Sozial\newline
    Wie ist das soziale Umfeld aufgebaut? Wie steht der Proband in Beziehung zu verschiedenen Demographischen Gruppen?
    \item Medizinisch\newline
    ist ein potentiell empfindlicher Bereich, welcher mit Vorsicht bearbeitet werden sollte. Nur situational abfragbar.
    \item Informationsquellen\newline
    Welche Informationskan\"ale werden von den Probanden verwendet, um sich zu orientieren und offene Fragen zu beantworten?
\end{enumerate}
Besagte Situation wird in Time-Line steps beschrieben, d.h. was passierte als erstes, zweites, etc. Innerhalb eines Schrittes wird behandelt, welche Fragen sich zu diesem Zeitpunkt bildeten, welche Gedanken und Gef\"uhle den Probanden zu dem Zeitpunkt bewegten.\newline
Hierbei ist die Sense-making Metapher zu ber\"ucksichtigen, mit Hinblick auf Situation, Informationsl\"ucke, Br\"ucke und Resultat.\cite{dervin2003sense}\newline
Dies f\"uhrt zu weiteren Fragen:\newline
\begin{enumerate}
    \item Was f\"uhrte zu dieser Frage?
    \item Was hat Sie mit deinem Leben zu tun?
    \item Gesellschaft und Machtverh\"altnisse?
    \item Wurde die Frage beantwortet?
    \item Wie?
    \item Welche Hindernisse gab es?
    \item War die Antwort hilfreich?
    \item War die Antwort hinderlich?
    \item Auf welche Weise?
\end{enumerate}

%context → situation → information need (situational information behaviour) -> infogaps

\newline
Bei der Auswertung der Daten werden die Interviews zun\"achst transkribiert. In Oduntan et al.'s Arbeit wurden die Daten anschlie\ss{}end thematisch analysiert und die Erfahrungen auf Gemeinsamkeiten und Unterschiede \"uberpr\"uft. Mittels dieser Vorgehensweise konnten Verbindungen zwischen individuellen und situationellen Faktoren festgestellt werden, womit Informationsl\"ucken in den Erfahrungen ausgemacht werden konnten. Die verschiedenen Kategorien und Themenbereiche wurden gekennzeichnet.\cite{oduntan2017investigating} Diese Arbeit wird sich an Oduntan et al.'s orientieren.



\statement{Warum so?}

Ein Gro\ss{}teil der Asylsuchenden in Deutschland stammt aus dem Nahen Osten, etwa die H\"afte der Asylsuchenden im Jahr 2017 wurde in Syrien, dem Irak oder Afghanistan geboren\cite{asylum2017seekers}. Daraus ergeben sich potentielle Probleme, welche voraussichtlich am leichtesten mit Interviews angegriffen werden k\"onnen.
Zun\"achst ist die Sprachbarriere zu ber\"ucksichtigen: In einem Interview besteht f\"ur den Interviewenden die M\"oglichkeit, eine Frage nach eigener Einsch\"atzung genauer zu erkl\"aren. Bei zu gro\ss{}en Sprachproblemen kann mit einem Dolmetscher oder Sprachassistenzsystem gearbeitet werden.\newline
Die grundlegende Aufgabe dieser Arbeit ist das Aufdecken von Informationsl\"ucken, ein erkl\"arungs - und redeintensives Unterfangen. Es wird davon ausgegangen, dass die Bed\"urfnisse der Individuen stark variieren. Infolgedessen muss eine potentiell aufgedeckte Informationsl\"ucke durch wiederholtes Nachfragen abgesichert werden, um Missverst\"andnisse zu vermeiden.\newline
Der semi-strukturierte Ansatz wurde ausgew\"ahlt, um den Gespr\"achsfuss so wenig wie m\"oglich zu unterbrechen und gegebenenfalls das Gespr\"ach zu fokussieren.\newline
Mit einem Aufbau der Studie auf diese Weise ist die M\"oglichkeit gegeben, einen direkten Vergleich zu Oduntan et al.'s Arbeit und eventuellen weiteren Folgearbeiten aufzubauen.\newline
Ein anderer Ansatz, etwa ein Fragebogen, ist in dieser Hinsicht nicht Zielf\"uhrend: Dieser ist darauf ausgelegt, f\"ur eine bestimmte Zielpopulation standardisiert werden zu k\"onnen. Dabei werden mehr Informationen vorausgesetzt als derzeit zur Verf\"ugung stehen.\newline
Die qualitativen Daten, welche mit den Interviews generiert werden, k\"onnen allerdings \"ahnliche Arbeiten in Zukunft unterst\"utzen.\newline
