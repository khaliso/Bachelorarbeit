

Integration ist ein vielschichtiges Phänomen, bei dem soziologische und psychologische Konsequenzen von Migration behandelt werden. Dabei werden die sich wandelnden Beziehungen zwischen einer Person und Ihrer Zielgesellschaft untersucht.\newline
Insbesondere die Integration von Geflüchteten und Asylsuchenden kann zahlreiche Komplikationen hervorrufen, da die Betroffenen oft nicht mehr die Option haben, sicher in Ihr Herkunftsland zurückzukehren.\newline
Allerdings werden aufgrund komplexer und regional spezifisch unterschiedlicher Integrationssysteme die Geflüchteten oft an den Rand der Gesellschaft gedrängt. (\cite{oduntan2017investigating})\newline
Der effiziente Aufbau und die zielgerichtete Umsetzung eines Integrationssystems ist deshalb essenziell für eine effektive und erfolgreiche Integration.\newline
Ende 2017 waren weltweit 68,5 Millionen Menschen auf der Flucht. Unter 'Auf der Flucht'  sind Menschen zusammengefasst, welche als interne Vertriebene, Gefl\"uchtete oder Asylsuchende verzeichnet sind. Dies ist die h\"ochste Zahl, die jemals von UNHCR verzeichnet wurde.\cite{uno2018flucht}
Die neuesten Zahlen vom Juni 2018 zeigen keine Ver\"anderung.\newline
Das IPCC geht davon aus, dass sich diese Entwicklung in der n\"aheren Zukunft nicht verbessern wird. \cite{sr15ipcc}\newline
Die Zahl der Asylantr\"age in Deutschland ging allerdings von 2016 bis 2018 von 745.545 \"uber 222.683 auf 185.853 zur\"uck. \cite{statistica2019asyl} In der T\"urkei befanden sich im Jahr 2018 3.5 Millionen Gefl\"uchtete.\newline
\newline
Momentan gibt es Grund zur Annahme, dass Asylsuchende oft nicht ausreichend auf den Integrationsprozess vorbereitet sind. \cite{oduntan2017information} \cite{gillespie2016mapping}\newline
Im Rahmen der sogenannten Fl\"uchtlingskrise wurden in den Aufnahmezentren vom {Bundesamt f\"ur Migration und Fl\"uchtlinge} (BAMF) zahlreiche Neuanstellungen vorgenommen. Bis zum Mai 2017 wurden gem\"a\ss{} des BAMF 454 dieser Neuanstellungen nicht ausreichend trainiert. Dennoch wurde auch dieses Personal damit beauftragt, Asylrechtsentscheidungen zu treffen.\newline
Die Zahl der Untrainierten wurde bis zum Februar 2018 auf 36 reduziert, 769 der 2139 Angestellten hatten allerdings die Ausbildung noch nicht abgeschlossen.\newline
Diese Situation gibt Grund zur Annahme, dass Asylsuchende nicht immer ausreichend auf einige der kommenden Herausforderungen vorbereitet wurden. \cite{asylum2018bamf}\newline



% unten lina's
Zusammenhang mit transnationaler Migration wurde kritisiert, dass bis
dato dem Individuum und seinen Praktiken im Gegensatz zu den
Verhaltensweisen von Gruppen wenig Aufmerksamkeit geschenkt wurde
(Levitt, 2001, S. 211)

Faktoren wie die Befriedigung von Informationsbedürfnissen zur
Steigerung des eudämonischen Wohlbefindens und des Erkennens von
Sinnhaftigkeit sind gemäß Shoham \& Strauss (2008) grundlegender
Bestandteil des Lebens und spielen eine essentielle Rolle beim
erfolgreichen Integrationsprozess.

%_________________________________________
Das Ziel dieser Arbeit war es, Informationslücken im Integrationsprozess in Deutschland zu identifizieren und untersuchen.\newline
Die Untersuchung fand im Jahr 2019 statt, in der Folgezeit der sogenannten Fl\"uchtlingskrise, welche Europa ab 2015 direkt betraf. \cite{unhcr2015seven}\newline


Politische Geflüchtete weisen eine Notwendigkeit von Information in
diversen Bereichen auf. Geeignete Integrationsangebote und -dienste
können jedoch nur angeboten werden sofern die Informationspraktiken,
-präferenzen und -bedürfnisse dieser Bevölkerungsgruppe erfasst
wurden. Dieser Umstand diente als Motivation zur Anfertigung dieser
Masterarbeit, welche die kontextuellen Strategien informationellen,
von geflüchteten individuellen und Jugendlichen und Heranwachsenden innerhalb der Informationssuche untersucht hat. Der Forschungsfokus lag Informationspraktiken;
hierbei im auf  den Besonderen unterschiedlichen über welche Informationsbedürfnisse die Jugendlichen verfügten, welche Quellen und welche Kommunikationskanäle bei der Informationssuche verwendet wurden und welche möglichen Barrieren und Probleme bei der Nutzung
dieser aufgetreten sind. Die Ergebnisse dieser Masterarbeit bieten eine
Grundlage zur Verbesserung von Design-Implikationen, indem die
spezifischen Strategien bei der Informationssuche konstatiert und
dokumentiert wurden.


Das Zusammentragen komplexer,
individualisierter Erfahrungen ermöglicht die Bereitstellung optimierter
Ressourcen. Im weiteren Verlauf der hier vorliegenden Masterarbeit wird zunächst
der aktuelle Forschungsstand und das analytische Rahmenkonzept
vorgestellt sowie die genaue Fragestellung erörtert. Anschließend wird
die Gewährleistung der Einhaltung ethischer Richtlinien dargelegt, die
Akquirierung der Teilnehmer sowie Datenerhebung und
Auswertungsmethodik beschrieben. Danach werden die Ergebnisse der
Studie präsentiert und im Detail diskutiert; in diesem Zusammenhang
wird ebenfalls auf Limitationen der Studie sowie auf Implikationen für
Design und für Implementierung eingegangen. Abschließend folgen ein
Fazit dieser Arbeit und ein Ausblick für mögliche zukünftige
Fragestellungen.

%__________________________________________________________________________________



Mittels semi-strukturierter Interviews werden die Erlebnisse der Teilnehmer unter deutscher Jurisdiktion ergr\"undet. Dabei werden Informationsl\"ucken, die die Probanden in Ihren etwa vier Jahren seit der Ankunft in Deutschland erlebt haben, aufgedeckt.\newline
Diese Arbeit versucht, Oduntan et al.'s Forschung zu Informationslücken im Integrationsprozess für den deutschsprachigen Raum zu ergänzen.\newline

Ein Gro\ss{}teil der Asylsuchenden in Deutschland stammt aus dem Nahen Osten, etwa die H\"afte der Asylsuchenden im Jahr 2017 wurde in Syrien, dem Irak oder Afghanistan geboren\cite{asylum2017seekers}. Daraus ergeben sich potentielle Probleme, von denen ausgegangen wurde, dass Sie voraussichtlich am leichtesten mit Interviews angegriffen werden konnten.
Zun\"achst ist die Sprachbarriere zu ber\"ucksichtigen: In einem Interview besteht f\"ur den Interviewenden die M\"oglichkeit, eine Frage nach eigener Einsch\"atzung genauer zu erkl\"aren.
Die grundlegende Aufgabe dieser Arbeit ist das Aufdecken von Informationsl\"ucken, ein erkl\"arungs - und redeintensives Unterfangen. Es konnte davon ausgegangen werden, dass die Bed\"urfnisse der Individuen variieren. Infolgedessen musste eine potentiell aufgedeckte Informationsl\"ucke durch wiederholtes Nachfragen abgesichert werden, um Missverst\"andnisse zu vermeiden.\newline
Der semi-strukturierter Ansatz wurde ausgew\"ahlt, um den Gespr\"achsfuss so wenig wie m\"oglich zu unterbrechen und gegebenenfalls das Gespr\"ach zu fokussieren.\newline
Mit einem Aufbau der Studie auf diese Weise ist die M\"oglichkeit gegeben, einen Vergleich zu Oduntan et al.'s Arbeit und eventuellen weiteren Folgearbeiten zu ermöglichen.\newline
Ein anderer Ansatz, etwa ein Fragebogen, war in dieser Hinsicht nicht Zielf\"uhrend: Dieser ist darauf ausgelegt, f\"ur eine bestimmte Zielpopulation standardisiert werden zu k\"onnen. Dabei werden mehr Informationen vorausgesetzt als derzeit zur Verf\"ugung stehen.\newline
Die qualitativen Daten, welche mit den Interviews generiert wurden, k\"onnen \"ahnliche Arbeiten in Zukunft unterst\"utzen.\newline

Hannah Arendt beschreibt in Ihrem Essay 'We Refugees'\textit{(1943)} Ihre Erfahrungen aus der perspektive von Immigranten in den Vereinigten Staaten. Sie zeigt sich zugleich verzweifelt und Entschlossen, sich in der neuen Gesellschaft zu beweisen und zu integrieren.\newline



refinition refugee Hannah Arendt VS UN?

Hannah Arendt: 'in the first place, we don't like to be called "refugees". We ourselves call each other "newcomers" or "immigrants." [..] A refugee used to be a person driven to seek refuge because of some act committed or some political opinion held. Well, it is true we have had to seek refuge; but we committed no acts and most of us never dreamt of having any radical opinion. With us the meaning of the term “refugee” has changed. Now “refugees” are those of us who have been so unfortunate as to arrive in a new country without means and have to be helped by Refugee Committees.
[..]Yes, we were “immigrants” or “newcomers” who had left our country because, one fine day, it no longer suited us to stay, or for purely economic reasons. We wanted to rebuild our lives, that was all. In order to rebuild one’s life one has to be strong and an optimist. So we are very optimistic. [..]
We lost our home, which means the familiarity of daily life. We lost our occupation, which means the confidence that we are of some use in this world. We lost our language, which means the naturalness of reactions, the simplicity of gestures, the unaffected expression of feelings. We left our relatives in the Polish ghettos and our best friends have been killed in concentration camps, and that means the rupture of our private lives.
Nevertheless, as soon as we were saved—and most of us had to be saved several times—we started our new lives and tried to follow as closely as possible all the good advice our saviors passed on to us. We were told to forget; and we forgot quicker than anybody ever could imagine.
[..] Apparently nobody wants to know that contemporary history has created a new kind of human beings—the kind that are put in concentration camps by their foes and in internment camps by their friends.

[..]have had such curious notions as to believe that we are not only “prospective citizens” but present “enemy aliens.” In daylight, of course, we become only “technically” enemy aliens—all refugees know this. But when technical reasons prevented you from leaving your home during the dark house, it certainly was not easy to avoid some dark speculations about the relation between technicality and reality.
%Bürger zweiter Klasse?
[..]Once we could buy our food and ride in the subway without being told we were undesirable.[..]We wonder how it can be done; we already are so damnably careful in every moment of our daily lives to avoid anybody guessing who we are, what kind of passport we have, where our birth certificates were filled out—and that Hitler didn’t like us. 
%Rassismus auf der Straße
[..] since passports or birth certificates, and sometimes even income tax receipts, are no longer formal papers but matters of social distinction.
%Nicht in die Disco gelassen

