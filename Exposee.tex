Ende 2017 waren weltweit 68,5 Millionen Menschen auf der Flucht. Unter 'Auf der Flucht'  sind Menschen zusammengefasst, welche als interne Vertriebene, Gefl\"uchtete oder Asylsuchende verzeichnet sind. Dies ist die h\"ochste Zahl, die jemals von UNHCR verzeichnet wurde.\cite{uno2018flucht}
Die neuesten Zahlen vom Juni 2018 zeigen keine Ver\"anderung. \cite{uno2018flucht}
Das IPCC geht davon aus, dass sich diese Entwicklung in der n\"aheren Zukunft nicht ver\"andern wird. \cite{sr15ipcc}\newline
Die Zahl der Asylantr\"age in Deutschland ging allerdings von 2016 bis 2018 von 745.545 \"uber 222.683 auf 185.853 zur\"uck. \cite{statistica2019asyl} In der T\"urkei befanden sich im Jahr 2018 3.5 Millionen Gefl\"uchtete.\newline
\newline
Momentan gibt es Grund zur Annahme, dass Asylsuchende oft nicht ausreichend auf den Integrationsprozess vorbereitet sind. \cite{oduntan2017information} \cite{gillespie2016mapping}\newline

Im Rahmen der sogenannten Fl\"uchtlingskrise wurden in den Aufnahmezentren vom {Bundesamt f\"ur Migration und Fl\"uchtlinge} (BAMF) zahlreiche Neuanstellungen vorgenommen. Bis zum Mai 2017 wurden gem\"a\ss{} des BAMF 454 dieser Neuanstellungen nicht ausreichend trainiert. Dennoch wurde auch dieses Personal damit beauftragt, Asylrechtsentscheidungen zu treffen.\newline
Die Zahl der Untrainierten wurde bis zum Februar 2018 auf 36 reduziert, 769 der 2139 Angestellten hatten allerdings die Ausbildung noch nicht abgeschlossen.\newline
Damit gibt diese Situation Grund zur Annahme, dass Asylsuchende nicht ausreichend auf einige der kommenden Herausforderungen vorbereitet wurden. \cite{asylum2018bamf}\newline

Mittels semi-strukturierter Interviews werden die Erlebnisse der Teilnehmer unter deutscher Jurisdiktion ergr\"undet. Dabei sollens mittels Bates' \textit{socio-cognitive approach}\cite{bates2005introduction}, welcher von Dervin's sense-making methodology\cite{dervin2003sense} inspiriert wurde, Informationsl\"ucken aufgedeckt.\newline
Mit diesem konnte festgestellt werden, wie ein Publikum verschiedene Situationen im Lauf des Lebens logisch begr\"undet hat. Um dies zu erm\"oglichen, wurden die Probanden eingeladen, Erfahrungen und Schlussfolgerungen mit Hilfe des Interviewers so detailliert wie m\"oglich wiederzugeben.\newline
Gem\"a\ss{} Dervin wird dennoch immer davon ausgegangen, dass eine erstellte Kategorisierung falsch, irrelevant oder f\"ur andere Sitautionen unangemessen sein kann.\textit{\textbf{Um diese zu \"uberpr\"ufen, wird zum Dialog zwischen wissenschaftlichen Institutionen und dem allt\"aglichen Leben aufgerufen}} .\textit{Diese Arbeit kann Aufschluss auf die Anwendbarkeit der in Oduntan et al.'s Forschung erarbeiteten Kategorien f\"ur den deutschsprachigen Raum geben.}\newline

Ein Interview, bei dem der Sense-Making approach verwendet wird, kann auf dem Ansatz der micro-moment timeline basieren. Bei dieser werden die Interviewten gebeten, eine pers\"onliche Situation mit Bezug zum Forschungsfokus genau zu beschreiben (In diesem Fall: Eine Situation abrufen, in der eine Informationsl\"ucke Auswirkungen auf das Leben des interviewten hatte).
Die von Oduntan et al. festgestellten Kategorien, in denen Informationsl\"ucken geh\"auft auftraten, waren:
\begin{enumerate}
    \item Ablauf des Asylverfahrens\newline
    Auch ein anhaltender Einfluss auf die mentale Gesundheit sowie getroffene Entscheidungen nach der Ablehnung sind hier relevant.
    \item Juristische Komponente \newline
    Beispiel: War dem Gefl\"uchteten klar, dass im Fall einer negativen Asylentscheidung Berufung eingelegt werden kann?
    \item Wohnen \newline
    Wie viele Wohnungen hat der Proband nach seiner Flucht bewohnt, wie lange im Durchschnitt, aus welchen Gr\"unden erfolgten Umz\"uge? Momentane Zufriedenheit mit der Wohnsituation? Wie wurde die momentane Wohnung entdeckt?
    \item Bildung \newline
    beinhaltet Erfahrungen mit dem deutschen Bildungs - und Schulungssystem
    \item Sozial\newline
    Wie ist das soziale Umfeld aufgebaut? Wie steht der Proband in Beziehung zu verschiedenen Demographischen Gruppen?
    \item Informationsquellen\newline
    Welche Informationskan\"ale werden von den Probanden verwendet, um sich zu orientieren und offene Fragen zu beantworten?
\end{enumerate}
Besagte Situation wird in Time-Line steps beschrieben, d.h. was passierte als erstes, zweites, etc. Innerhalb eines Schrittes wird behandelt, welche Fragen sich zu diesem Zeitpunkt bildeten, welche Gedanken und Gef\"uhle den Probanden zu dem Zeitpunkt bewegten.\newline
Hierbei ist die Sense-making Metapher zu ber\"ucksichtigen, mit Hinblick auf Situation, Informationsl\"ucke, Br\"ucke und Resultat.\cite{dervin2003sense}\newline
Dies f\"uhrt zu weiteren Fragen:\newline
\begin{enumerate}
    \item Was f\"uhrte zu dieser Frage?
    \item Was hat Sie mit deinem Leben zu tun?
    \item Gesellschaft und Machtverh\"altnisse?
    \item Wurde die Frage beantwortet?
    \item Wie?
    \item Welche Hindernisse gab es?
    \item War die Antwort hilfreich?
    \item War die Antwort hinderlich?
    \item Auf welche Weise?
\end{enumerate}

%context → situation → information need (situational information behaviour) -> infogaps

Bei der Auswertung der Daten werden die Interviews zun\"achst transkribiert. In Oduntan et al.'s Arbeit wurden die Daten anschlie\ss{}end thematisch analysiert und die Erfahrungen auf Gemeinsamkeiten und Unterschiede \"uberpr\"uft. Mittels dieser Vorgehensweise konnten Verbindungen zwischen individuellen und situationellen Faktoren festgestellt werden, womit Informationsl\"ucken in den Erfahrungen ausgemacht werden konnten. Die verschiedenen Kategorien und Themenbereiche wurden gekennzeichnet.\cite{oduntan2017investigating} Diese Arbeit orientiert sich an Oduntan et al.'s Arbeit.
\newline
Ein Gro\ss{}teil der Asylsuchenden in Deutschland stammt aus dem Nahen Osten, etwa die H\"afte der Asylsuchenden im Jahr 2017 wurde in Syrien, dem Irak oder Afghanistan geboren\cite{asylum2017seekers}. Daraus ergeben sich potentielle Probleme, welche voraussichtlich am leichtesten mit Interviews angegriffen werden k\"onnen.
Zun\"achst ist die Sprachbarriere zu ber\"ucksichtigen: In einem Interview besteht f\"ur den Interviewenden die M\"oglichkeit, eine Frage nach eigener Einsch\"atzung genauer zu erkl\"aren. Bei zu gro\ss{}en Sprachproblemen kann mit einem Dolmetscher oder Sprachassistenzsystem gearbeitet werden.\newline
Die grundlegende Aufgabe dieser Arbeit ist das Aufdecken von Informationsl\"ucken, ein erkl\"arungs - und redeintensives Unterfangen. Es wird davon ausgegangen, dass die Bed\"urfnisse der Individuen stark variieren. Infolgedessen muss eine potentiell aufgedeckte Informationsl\"ucke durch wiederholtes Nachfragen abgesichert werden, um Missverst\"andnisse zu vermeiden.\newline
Der semi-strukturierte Ansatz wurde ausgew\"ahlt, um den Gespr\"achsfuss so wenig wie m\"oglich zu unterbrechen und gegebenenfalls das Gespr\"ach zu fokussieren.\newline
Mit einem Aufbau der Studie auf diese Weise ist die M\"oglichkeit gegeben, einen direkten Vergleich zu Oduntan et al.'s Arbeit und eventuellen weiteren Folgearbeiten aufzubauen.\newline
Ein anderer Ansatz, etwa ein Fragebogen, ist in dieser Hinsicht nicht Zielf\"uhrend: Dieser ist darauf ausgelegt, f\"ur eine bestimmte Zielpopulation standardisiert werden zu k\"onnen. Dabei werden mehr Informationen vorausgesetzt als derzeit zur Verf\"ugung stehen.\newline
Die qualitativen Daten, welche mit den Interviews generiert werden, k\"onnen allerdings \"ahnliche Arbeiten in Zukunft unterst\"utzen.\newline




refinition refugee Hannah Arendt VS UN?

Hannah Arendt: 'in the first place, we don't like to be called "refugees". We ourselves call each other "newcomers" or "immigrants." [..] A refugee used to be a person driven to seek refuge because of some act committed or some political opinion held. Well, it is true we have had to seek refuge; but we committed no acts and most of us never dreamt of having any radical opinion. With us the meaning of the term “refugee” has changed. Now “refugees” are those of us who have been so unfortunate as to arrive in a new country without means and have to be helped by Refugee Committees.
[..]Yes, we were “immigrants” or “newcomers” who had left our country because, one fine day, it no longer suited us to stay, or for purely economic reasons. We wanted to rebuild our lives, that was all. In order to rebuild one’s life one has to be strong and an optimist. So we are very optimistic. [..]
We lost our home, which means the familiarity of daily life. We lost our occupation, which means the confidence that we are of some use in this world. We lost our language, which means the naturalness of reactions, the simplicity of gestures, the unaffected expression of feelings. We left our relatives in the Polish ghettos and our best friends have been killed in concentration camps, and that means the rupture of our private lives.
Nevertheless, as soon as we were saved—and most of us had to be saved several times—we started our new lives and tried to follow as closely as possible all the good advice our saviors passed on to us. We were told to forget; and we forgot quicker than anybody ever could imagine.
[..] Apparently nobody wants to know that contemporary history has created a new kind of human beings—the kind that are put in concentration camps by their foes and in internment camps by their friends.

[..]have had such curious notions as to believe that we are not only “prospective citizens” but present “enemy aliens.” In daylight, of course, we become only “technically” enemy aliens—all refugees know this. But when technical reasons prevented you from leaving your home during the dark house, it certainly was not easy to avoid some dark speculations about the relation between technicality and reality.
%Bürger zweiter Klasse?
[..]Once we could buy our food and ride in the subway without being told we were undesirable.[..]We wonder how it can be done; we already are so damnably careful in every moment of our daily lives to avoid anybody guessing who we are, what kind of passport we have, where our birth certificates were filled out—and that Hitler didn’t like us. 
%Rassismus auf der Straße
[..] since passports or birth certificates, and sometimes even income tax receipts, are no longer formal papers but matters of social distinction.
%Nicht in die Disco gelassen

The comity of European peoples went to pieces when, and because, it allowed its weakest member to be excluded and persecuted.