\subsection{Wohnen}
Allen Probanden wurde nach ihrer Ankunft im Jahr 2015 von Behörden oder gemeinnützigen Vereinen eine Sozialwohnung gestellt. Informationslücken der Wohnsituation bezogen sich überwiegend auf die Frage, wie eine geeignetere Wohnung gefunden werden kann. Gerade zu Beginn des Integrationsprozesses in den Sammelunterkünften kann es jedoch zu Reibereien kommen. 
%Ankerzentrum in Regensburg erwähnen?

Ein Proband erzählte von seinen Erlebnissen in seiner ersten Unterbringung, einer Sammelunterkunft mit 30-40 Betten. Er kannte zu diesem Zeitpunkt die deutsche Sprache nicht und konnte weder lesen noch schreiben. Dort wurde er von einem Mitbewohner der Unterkunft im Schlaf tätlich angegriffen.

\begin{quote}
    ``Und (...) ja. Ich hab schon einfach Probleme gehabt. Ich war schon, ich war, ich war auf mein Bett und (...) im Schlaf, hat eine gekommen und mich geschlagen. [..] Der hat schon mich auf dem Bett weggeworfen. Ja, wirklich. Der war (...) betrunken. Der war Russe und ähm(...) keine Ahnung, dann (...) ich wusste das nicht. Ich hab aufgestanden, und (...) ja.''
\end{quote}
\centerline{\textit{IT6, min30}}

Der Interviewte rief um Hilfe und die Auseinandersetzung wurde von der Polizei beendet.  Nach der Bestandsaufnahme beantragte er eine Relokalisierung, welche auch genehmigt wurde. Die neue Unterkunft erfüllte die Bedürfnisse des Probanden zu dem Zeitpunkt.\newline
Ein anderer Fall, bei dem die Relokalisation dringend nötig war, ist von IT2 aufgezeigt. Der Proband leidet seit früher Kindheit an schwerer Migräne. In einer Unterkunft litt sein Mitbewohner unter ausgeprägter Rhonchopathie (\textit{Schnarchen}). Um nachts schlafen zu können, nahm der Proband Schmerzmittel.
\begin{quote}
    ``Muss ich immer so stark Tabletten nehmen, bis ich so tief schlafen, nicht zuhören, ne?''
\end{quote}
\centerline{\textit{IT2.2 min32}}
Als die Betreuerin des zu diesem Zeitpunkt minderjährigen Interviewten die Wohngruppe für etwa zwei Wochen verließ, wurde er mit einem Vorrat an Tabletten ausgestattet.\newline
\begin{quote}
    ``Bis ich schlafen konnte. Dann unsere Betreuerin hat immer zu mir gesagt: 'Bevor ich nach Hause gehen (...) ich gebe dir eins, eine Tablette, dann kannst du nehmen, dann schlafen.' (...) Und (...) wir haben, also ich habe meine bekommen, danach der sagt ich mache (...) ich weiß nicht, wie viele Tage Urlaub, dann du kriegst von mir 20 Tabletten.[..]''
\end{quote}
\centerline{\textit{IT2.2 min30}}
Besagte Schmerzmittel waren die selben Medikamente, welche der Teilnehmer bereits im Iran von einem örtlichen Arzt verschrieben bekam. Dort hatte er auf Anraten seines behandelnden Arztes täglich Schmerzmittel konsumiert.
\begin{quote}
    ``Im Iran habe ich oft Tablette genommen, jeden Tag dreimal. (...) Musste ich, ich war beim Arzt, der hast du gesagt: 'Du musst Tabletten nehmen'.''
\end{quote}
\centerline{\textit{IT2.2 min45}}
Der Interviewte nahm bei nächtlichen Migräneanfällen anschließend teilweise mehr als die vorgegebene Dosis von einer Tablette pro Nacht. Bei einem Arzttermin berichtete er dem Arzt davon, was in einer umgehenden Versammlung aller Beteiligten resultierte.
\begin{quote}
    ``Dann (...) dieser Arzt hat meine Vormundin angerufen, meine Betreuerin, der Chefin, (...) und der Chef, und Jugendamt. Alle sind gekommen, und Übersetzer. Wir waren ungefähr neun Leute.''
\end{quote}
\centerline{\textit{IT2.2 min31}}
Hier wurde zunächst die Situation um den Mitbewohner erklärt, welche zum erhöhten Tablettenkonsum führte. Der Arzt wies die Beteiligten an, dem Probanden entweder ein Einzelzimmer zur Verfügung zu stellen, oder eine neue Wohngemeinschaft für ihn zu finden.\newline
Im Anschluss bezog der Asylsuchende eine neue Wohngruppe, in der ihm ein Einzelzimmer zur Verfügung gestellt wurde. Seitdem hat sich der Tablettenkonsum auf zwei bis drei Tabletten monatlich reduziert. \textit{\textbf{mehr dazu im Medizinischen -> Leberflecke}}\newline
Im späteren Verlauf des Lebens in Deutschland beschäftigten sich die Probanden beispielsweise damit, wie eine neue, passende Wohnung gefunden werden kann.\newline 
Es ist nicht allen Geflüchteten oder Asylsuchenden möglich, sich für Wohnungen auf dem öffentlichen Wohnungsmarkt zu bewerben. Asylsuchenden mit befristeter Aufenthaltserlaubnis gaben an, sich nur für Sozialwohnungen bewerben zu dürfen.
\begin{quote}
    `` [..]'Wenn jemand keine Aufenthalt hat, und dann darf man nicht umziehen.'[..] So, zum Beispiel weil man kann entweder ein Jahr oder drei Jahre Aufenthalt hat, dann, dann bekommt man keine Wohnungserlaubnis. ''
\end{quote}
\centerline{\textit{IT4 min25}}
Einige der Interviewten gaben an, dass sie unter anderem eine Ausbildung abschließen wollen, um sich für Wohnungen im allgemeinen Wohnungsmarkt bewerben zu dürfen.\newline