\sectio{Interview 2, Teil 2}

I: Jetz funktioniert's.  #00:00:06-5#

B: Ja, und (...) Was war dein Frage? #00:00:07-9#

I: ähm (...)  wir waren dabei, wie du dann nach Deutschland gekommen bist? #00:00:13-7#

B: Aha (...) dann, ich bin, ich war zum 14 - jähriger war ich im Iran. #00:00:21-5#

I: hm (bejahend) #00:00:23-2#

B: Dann die Iranische Polizei haben mich verhaftet #00:00:25-8#

I: hm (bejahend) #00:00:27-4#

B: Dann (...) mich in's Gefängnis gebracht. Und ich war sieben, acht Tage - ich kann nicht erinnern, ich habe vergessen, schon lange her. Dann, die haben mich nach Afghanistan geschickt, ich war auch verletzt.  #00:00:26-5#

I: Was für eine Verletzung hattest du?  #00:00:28-8#

B: Also, als ich im Gefängnis war, dann die Polizei ist in der Nacht gekommen - ich weiß nicht, wie viele waren. Also das war so im Keller, so dunkel. Wir hatten keine (unv.), sondern nur so eine, so eine kleinen, so eine kleine Tür, wo man schwer reinkommt und rausgehen, sowas. #00:00:50-2#

I: hm (bejahend) #00:00:50-7#

B: Dann die haben mich geschlagen, in's Gesicht, überall. Dann nächste Tag die haben mich nach Afghanistan geschickt. (...) Ich war auf der Grenze, zwei Tage, dann ich hab von der anderen Aghaner ein Telefon geliehen, damm meine Mama angerufen. #00:01:05-7#

I: hm (bejahend) #00:01:05-7#

B: Dann sie hat zu mir gesagt:(...) 'geh nicht weiter, sondern ich nach Kabul oder nach Mazar-e einfach kommen. Also versuch mal wieder nach Iran kommen. Dann wir sind gekommen. (...) Dann wir eigentlich bin von Afghanistan bis Deutschland gekommen. (...) #00:01:25-8#

I: Okay. #00:01:26-9#

B: Weil ich im Iran war, die haben mich nach Afghanistan geschickt, da muss ich von Afghanistan wieder versuchen nach Europa kommen. #00:01:32-3#

I: ähm wie hast du's dann geschafft? War das über ähm über Land oder über's Boot? #00:01:39-0#

B: Also von Afghanistan bis Pakistan wir sind so ein bisschen zu Fuß und (...) ein bisschen mit dem Auto gekommen. #00:01:49-2#

I: hm (bejahend) #00:01:49-2#

B: Und so (...) wir haben versucht von Afghanistan direkt nach Iran kommen

I: hm (bejahend) #00:01:55-8#

B: Auf der Grenze war sehr gefährlich, die Polizei haben einfach erschossen. #00:02:00-6#

I: Hmh? #00:02:00-6#

B: Ja. Erschossen - viele Leute sind verletzt geworden, gestorben. Also damals (unv.) Kinder und Familie, Männer, waren richtig viele Leute. Jetzt ist auch so; jetzt gerade ist auch so. Wenn man auf die Grenze dann man sieht Beispiel wie viele Leute gestorben sind oder wie viele verletzt sind. Manche sind verletzt, die bleiben zwei, sieben Tage so verletzt die erbluten bis sie tot werden. Also niemand kann denen Beispiel zum Krankenhaus sowas bringen, weil Polizei erschießen. Die wollen nicht Beispiel die Afghanen nach Iran fluchten. #00:02:54-2#

I: Die wollen die nicht reinlassen? #00:02:54-2#

B: Sie müssen, also sie (...) sie machen Ihre Aufgabe, also Sie einfach erschießen, so den Leute. #00:03:04-3#

I: Aber du meintest doch mal, dass ähm(...) die Iraner ursprünglich mal Afghanen rekrutieren wollten, um gegen den IS zu kämpfen. #00:03:18-3#

B: Das war ja schon aber die haben(...) #00:03:20-6#

I: //Das ist jetz nichtmehr dann? Also das war dann nicht mehr? #00:03:23-9#

B: Doch, das war schon. Die haben schon zu mir gesagt, bevor, bevor ich in's Gefängnis. Bevor Sie mich in's Gefängnis bringen, Sie haben zu mir gesagt, Sie haben mir ein Blatt gegeben (...) und damals konnte ich nicht so gut Persisch lesen, die haben mich erklären,  gesagt *räuspert sich*: 'Wenn du (...) nach Syrien zum Kämpfen gehst, dann (...) du kriegst Geld und deine Familie haben Sicherheit, du danach, nach den zwei, drei Jahre kriegst du iranische Pass oder Ausweis. Du kannst besser leben, du kannst (...) Beispiel: Eine Wohnung mieten, sowas. Musst nicht schwarz leben.' Dann ich habe nicht akzeptiert. (...) ähm dann die haben mich einfach geschickt nach Afghanistan. #00:04:09-7#

I: hm (bejahend) #00:04:09-7#

B: Also, wie gesagt:ich war auch (...) ein paar Tage im Gefängnis (lacht). Dann (...) #00:04:16-8#

I:// Fuck. #00:04:16-8#

B: Wir wollten vom (lacht) Iran direkt nach (...) von Afghanistan direkt nach Iran kommen, dann nicht geschafft. Dann wir mussten wieder von Afghanistan nach Pakistan fahren #00:04:29-7#

I: hm (bejahend) #00:04:29-7#

B: Mit dem Auto. Wie gesagt, wir sind so ein bisschen zu Fuß gegangen. #00:04:33-9#

I: hm (bejahend) #00:04:33-9#

B: Wir mussten auf den Berge zu Fuß gehen, danach (...)ähm ein Auto auf uns gewartet. Wir waren (...) viele Leute in einem Auto. (unv.). Ich war da hinten vom Auto, ich weiß nicht, was heißt..? #00:04:48-7#

I: Kofferraum? #00:04:47-3#

B: Ja. Wir waren, ich weiß nicht, vier Leute? Fünf Leute?  #00:04:54-5#

I: Im Kofferraum? #00:04:54-5#

B: Mmmh.(lacht) #00:04:59-3#

I: Ooh. (lacht) Wie lange ging das? #00:05:01-0#

B: Ich glaube mehr als acht Stunden. #00:05:04-9#

I: Unbequem. #00:05:06-3#

B: Mmh. Danach meine ganze Körper hat eingeschlafen. #00:05:08-8#

I: Mmmh. #00:05:12-5#

B: Das ist, wenn man so (...) eine Stunde, zwei Stunden sitzt, dann man steht auf, dann die Beine funktioniert nichtmehr. #00:05:22-0#

I: Ja. #00:05:22-0#

B: Bei mir war ganz grob aus dem Kopf. (lacht) #00:05:23-9#

I: (lacht) #00:05:23-9#

B: Ich musste mich wie ein (...) wie eine Schlange auf dem Boden ziehen. #00:05:29-8#

I: Baaaah.. #00:05:29-8#

B: War aber.. (...) Wir hatten Angst, weil da hinten Polizei waren. Wenn die uns wieder verhaften wollten (...) Also (...) #00:05:38-3#

I: Da ist es besser als das Gefängnis. #00:05:40-7#

B: Mmh, dann vielleicht würden noch tiefer sein. #00:05:45-4#

I: hm (bejahend) #00:05:46-1#

B: Dann, wir sind nach Iran (...) geschafft von Afghan (...) von Afghanistan nach Pakistan geschafft, dann von Pakistan nach Iran geschafft. (...) Dann ich habe mich mit meiner Mama in Teheran getroffen #00:05:59-4#

I: hm (bejahend) #00:05:59-4#

B: Sie war nicht zu Hause, weil (...) ich hatte Angst. Ich (...) Depressivität nehmen.(...) Dann (...) meine Mama hat zu mir gesagt, also ich wollte nicht, echt, ich wollte nicht nach Deutschland kommen.  #00:06:13-0#

I: Ja. #00:06:13-0#

B: Meine Mama hat (...) ich meine nicht nach Deutschland, nach Europa. Aber mein Mama hat gesagt: 'Du musst gehen. Deshalb du kannst kein andere Wahl jetz, wenn du beim zweiten Mal (...) beim zweiten Mal verhaftet wirst, dann (...) kriegst du echt (...) zwei, vielleicht zwei oder drei Jahre im Gefängnisstrafen, also Schlagstrafen und #00:06:36-3#

I: hm (bejahend) #00:06:36-3#

B: Strafen Geld. Dann ich hab (...) gedacht, ja die Mama ist, die Mama hat recht dann. Am besten (...) fahre ich weiter. Dann (...) wir sind vom Iran nach (...) in die Türkei gekommen, also auf der Grenze aber richtig (...) gefährlich. Ich hab (...) zwei tote Leute gesehen. War furchtbar. #00:07:00-5#

I: An der Grenze? #00:07:00-5#

B: (...) Mmh. Ich weiß nicht, also der war unkennbar. Also (...) das, also das Gesicht war ganz kaputt. (lacht nervös) Wie ein Obst abgelaufen, sowas? #00:07:13-6#

I: hm (bejahend) #00:07:13-6#

B: Wo.. furchtbar, ich kann nicht ob (...) es war in der Nacht, und (...) wir sind geschafft (...) dann von der Türkei bis (...) Griechenland mit dem Boot gekommen. #00:07:29-6#

I: hm (bejahend) #00:07:30-0#

B: Also, über den Meer. #00:07:31-9#

I: hm (bejahend) #00:07:31-9#

B: (...) Dann von Griechenland bis (...) Hungary? zu Fuß gekommen #00:07:44-6#

I: Von Griechenland bis Ungarn zu Fuß.  #00:07:46-2#

B: Ja, ganze Weg zu Fuß. (...) #00:07:48-4#

I: Das ist ein weiter Weg. Wie lange hat'n des gedauert? #00:07:49-8#

B: (lacht) Weiß ich nicht. Viele Leute waren, das war eine Massenflucht.  #00:07:56-8#

I: Aber (...) sag mal, auf dem (...) Weg von Griechenland nach Ungarn, das sind so viele Leute, wie kriegt ma da was zu Essen und alles? #00:08:09-6#

B: Im Hungary (...) im Hungary wir haben von Deutschland Essen bekommen. #00:08:15-5#

I: Echt? #00:08:15-5#

B: Ja. Wir haben gesehen, viele Leute arbeiten freiwillig, Sie hatten so deutsche (...) wie heißt das (...) deutsche (...) #00:08:25-0#

I: Fahnen? #00:08:25-0#

B: Ja, genau, deutsche Fahne, die hatten da (...) #00:08:28-6#

I: hm (bejahend) #00:08:28-6#

B: die Fahnen, die (...) die haben auch deutsch gesprochen. Wir haben nicht verstanden. Die haben Englisch gesprochen, und (...) aber manche, also Sie miteinander die haben deutsch gesprochen, und mit uns Englisch.  #00:08:41-2#

I: hm (bejahend) #00:08:41-8#

B: Und bis Griechenland, um Griechenland wir haben von Griechenland Hilfe bekommen #00:08:46-1#

I: hm (bejahend) #00:08:46-1#

B: (...) Und auch die Deutsche waren in Griechenland dann. (...) ähm von (...) Österreich, von Hungary bis Österreich, wir sind mit dem Zug gefahren, dann wir waren erst drei (...) (unv.) sieben? Sieben Tage, ich glaube.. #00:09:07-1#

I: hm (bejahend) #00:09:07-1#

B: Ne, ich glaube weniger. Weiß ich nicht, vergessen - zu lange her. Dann wir sind von Griechenland bis Deutschland mit dem Zug wieder gekommen. Aber (...) von Afghanistan (...) bis Griechenland, die Grenzen waren Katastrophe. Als ich auf dem Wasser war, im ähm, im Meer #00:09:35-1#

I: hm (bejahend) #00:09:35-1#

B: Wir waren zwei Boote, und aus Plastik. Die waren nicht Sicherheit, Beispiel: Die war sechs Meter lang und vier Meter breit. Wir waren 60 Leute auf dem (unv.)(lacht). #00:09:44-4#

I: Was?! #00:09:53-1#

B: (Isst Erdbeere) Danke, ne? #00:09:53-1#

I: Ja gerne! Ich probier's auch mal (...) (isst Erdbeere) Hoffentlich schmecken die nicht nur nach Wasser.(lacht) #00:09:58-7#

B: Nur ein bisschen. (...) Aber schmeckt, danke. #00:10:08-0#

I: Gerne. Mal schaun, wir haben auch Erdbeerpflanzen zu Hause #00:10:13-4#

B: hm (bejahend) #00:10:13-4#

I: Dann bring ich dir mal welche mit #00:10:14-8#

B: Danke. (...) Dann (...) Grenze von (...) Afghanistan/Iran war am gefährlichsten. Es war sehr gefährlich. (...) #00:10:30-7#

I: hm (bejahend) #00:10:30-7#

B: Aber (...) wir waren im Meer, dann (...) zwei Boote. Einer war Araber, die war (unv.) ,waren auch viele Familien. #00:10:46-4#

I: hm (bejahend) #00:10:46-8#

B: Dann, ich glaube Ihre Boot waren kaputt. Dann die sind ähm, das Boot hat sich umgedreht. #00:10:55-9#

I: Das ist gekentert? #00:10:57-6#

B: Dann viele Leute ertrunken. Aus dem drei Leute sie haben geschafft wieder auf den Boot kommen. Familien sind gestorben, aber die (...) das war in Türkei. Aber die (...) Türkei haben nicht geholfen. Garnicht. Wir haben die Polizei (unv.)(...), die Polizei haben nur angeschaut. #00:11:17-7#

I: Die haben zugeschaut? #00:11:17-7#

B: Nur zugeschaut. Nur zugeschaut. Garnix gemacht. Wir waren dabei. Und wir hatten auch Angst, also wir haben auch geweint und geschrien (...). Nicht für den, nicht für die Araber, sondern für uns (lacht). Wir waren auch auf dem Boot im Meer. Aber zum Glück wir waren vierundf.. ähm 45 Minuten (...) im Wasser, dann wir haben geschafft. Also es war sehr gefährlich. Von, von Griechenland ist (...) Deutschland war nicht so gefährlich, aber von Afghanistan. Afghanistan, Pakistan, Iran. Die waren am gefährlichsten Grenze. #00:12:03-5#

I: (...) #00:12:03-9#

B: Sehr, sehr gefährlich. #00:12:05-8#

I: hm (bejahend) #00:12:05-8#

B: Ich hab (...) viele tote Leute gesehen, viele Verletzte gesehen. Manchmal ich denke, das war wie ein Albtraum oder wie einen Film. #00:12:17-8#

I: Zum Glück bist du jetzt hier. #00:12:18-9#

B: (lacht) Viele Leute haben geschafft. Wenn ich nach Afghanistan wieder abgeschoben werde, ich werde nie wieder nach Deutschland kommen. Ich will nie wieder diese Grenze sehen. Das war furchtbar. Ich will dort hungrig bleiben, (lacht) nicht sterben aber nicht wieder (...) solche Grenze sehen (lacht). Weißt du, als ich (...) also *räuspert sich* daheim versucht von Afghanistan nach Iran direkt kommen, gerade habe ich schon gesagt, und ich versuche jetzt richtig erklären, was ich genau gesehen habe. #00:12:56-5#

I: Du, musst du nicht. Also, wir, wir können auch wo anders weiter machen. #00:13:00-8#

B: Also (...) #00:13:02-3#

I: Das musst du wirklich ned alles wieder nacherleben. Meinetwegen können wir auch ganz woanders ansetzen wenn du willst. #00:13:09-1#

B: Also die (...) was ich erlaub (...) erlebt habe, das ist (...) wie ein Bild, immer, vor den meinen Augen, weißt du? #00:13:17-0#

I: hm (bejahend) #00:13:17-0#

B: Ich sehe immer (lacht) manchmal, in der Nacht, ich meine die, ähm ich lege mich auf den Bett, dann ich will über gute Sachen denken, dann die kommen einfach die (lacht) die schlechte Sachen, ich will nicht denken, aber (...) Aaaah! #00:13:30-7#

I: Ow, fuck(...) #00:13:35-0#

B: (lacht) Es ist so. #00:13:40-5#

I: ähm, lass was anderes ähm, lass wo anders hinspringen oder? #00:13:42-2#

B: Okay, dann du kannst andere Frage stellen. (lacht) #00:13:44-8#

I: ähm #00:13:46-3#

B: Umso besser(lacht) #00:13:46-3#

I: Erstmal, ähm (...) seit wann bist du in Deutschland, und genau in Regensburg? #00:13:57-8#

B: Ich bin direkt nach Regensburg gekommen. #00:14:01-1#

I: Du bist direkt nach Regensburg gekommen? #00:14:02-8#

B: Ja. #00:14:02-8#

I: Und das war vor (...) drei Jahren gell?  #00:14:05-0#

B: Dreieinhalb Jahren. #00:14:05-8#

I: Dreieinhalb Jahren. #00:14:06-0#

B: 19. 09. 2015 (...) Ich bin in Regensburg angekommen, also in Deutschland. #00:14:13-6#

I: hm (bejahend) #00:14:13-6#

B: Dann (...) #00:14:16-8#

I: ähm #00:14:16-8#

B: zweite Tag, wir haben mit (...)ähm Deutschkurs begonnen. Sprachkurs. #00:14:22-6#

I: Also du bist hier angekommen, und direkt am Tag drauf is der Deutschkurs losgegangen. #00:14:23-8#

B: // Ja, 20.09. wir haben mit dem Sprachkurs begonnen. Also, das damals einei halbe Stunde. #00:14:35-1#

I: Anderthalb Stunden am Tag? #00:14:35-1#

B: hm (bejahend) #00:14:35-1#

I: Hmja, hat sich ausgezahlt (lacht) #00:14:36-1#

B: (lacht) Ja. Nicht so gut geworden, aber ja. Es könnte besser werden. #00:14:43-9#

I: Mit der Zeit wird das  #00:14:46-5#

B: Mh. #00:14:46-5#

I: Funktioniert ja jetz schon ganz gut! #00:14:47-4#

B: So ein bisschen (lacht) #00:14:49-7#

I: Ich meine, wir können uns unterhalten oder? (lacht) #00:14:51-7#

B: Mh. Also ich verstehe mehr als ich spreche. Ich verstehe mehr. Ich versuch's auch ohne Fehler spüren, aber (...) so schwierig. #00:15:03-3#

I: Das kommt mit der Zeit. #00:15:03-3#

B: Hoffe ich (lacht) #00:15:07-8#

I: ähm sag mal mit ähm also momentan hast du keinen Kontakt zu deiner Familie zurück oder? #00:15:16-3#

B: Ne. Als ich ähm nach Deutschland kam, dann ich hatte schon Kontakt bis ein Jahr, dann (...) meine Familie haben zu mir gesagt 'wir haben eine Brief bekommen von iranische Politik, also iranische Staat' #00:15:32-1#

I: hm (bejahend) #00:15:32-1#

B: Die haben gemeint, also damals war Krieg im Syrien #00:15:37-5#

I: hm (bejahend) #00:15:37-8#

B: Meine Mutter hat gemeint, also wir haben am Telefon gesprochen, Sie hat gemeint, (...) im Brief steht, wir wohnen seit mehr als zehn Jahren im Iran schwarz (...)  #00:15:52-2#

I: hm (bejahend) #00:15:52-7#

B: Und entweder wir haben kein Ausweis, wir müssen wieder zurück nach Afghanistan oder (...) Ich habe einen große Bruder, der ist zwei, drei Jahre größer als ich. #00:16:02-0#

I: hm (bejahend) #00:16:02-5#

B: Entweder mein Bruder muss nach Syrien gehen, kämpfen, dann Beispiel meine Familie kann im Iran bleiben und die bekommen Geld. Eigentlich, meine Mama ist sehr religiös, mein Bruder (...) und wir sind Schiiten (...) und meißten Schiiten nach Syrien zum Kämpfen gegangen wegen (...) Ihrer Würde oder (...) sie haben geglaubt, wir kämpfen für unsere Religion, nicht Beispiel für Iran. #00:16:38-6#

I: hm (bejahend) #00:16:38-6#

B: Manche, also, manche sind freiwillig hingegangen, manche haben Geld gekriegt. Manche mussten gehen. (...) Es war so (...) verschiedene Gründen. #00:16:50-5#

I: Ja. #00:16:50-5#

B: Und bei meiner Familie, meine Bruder Beispiel sollte hingehen. Aber der hat nicht (...) meine Mama hat nicht akzeptiert. Meine Mama hat gesagt: 'Ich hab einmal eure Vater gesehen, wie Taliban hat geschlachtet. Ich will nicht meine Kinder so sehen. (...) Und meine Mama hat gesagt, ich hatte solche Scheiße erfahrung, ich will nicht wieder diese Erfahrung (...) #00:17:12-8#

I: Ja #00:17:14-8#

B: Dann mein Bruder (...) dann meine Familie mussten (...) also, die Mama hat gesagt, wenn ich dich rufen an, dann die kontrollieren grad die ganzen Telefonen. Wenn die Leute telefonnieren, dann die kontrollieren, wissen wie viel IS im Iran sind.  #00:17:32-9#

I: hm (bejahend) #00:17:33-4#

B: Im Iran gibts schon viele IS. Die #00:17:35-5#

I: Oh. #00:17:35-5#

B: wollen Anschlagen noch oder so #00:17:38-1#

I: Und deswegen werden die Telefone abgehört? #00:17:40-1#

B: Ja. #00:17:41-7#

I: hm (bejahend) #00:17:42-8#

B: Und meine Mama hat Angst Beispiel, und die (...) kontrollieren auch, kontrollieren auch die Afghaner keine Ausweis haben, die kontrollieren die Telefonnate dann #00:17:51-3#

I: hm (bejahend) #00:17:51-3#

B: Wenn meine (...) viele Afghaner haben so (...) die haben gefunden, Beispiel die Afghaner hatten keinen Ausweis, die haben mit den Kindern gesprochen, dann (...) die Polizei hatten die Name und die Telefonnummer im Internet und die haben, ähm die haben nachgeschaut, die haben gesehen, die Familien wohnen seit schon 20, 30, 15, 17 Jahre schwarz. Entweder sie müssen zum Krieg oder zurück nach Afghanistan #00:18:16-0#

I: hm (bejahend) #00:18:16-3#

B: Viele Afghane sind nach Afghanistan geflüchtet(...) und meine Mama hat Angst die müssen umziehen irgendwo anders. Dann wir hatten eineinhalb Jahre keine Kontakt #00:18:29-5#

I: hm (bejahend) #00:18:29-5#

B: Dann Sie haben sich wieder gemeldet, dann wir haben zwei, dreimal miteinander gesprochen. (...) Dann meine Mama hat wieder aufgehört, also wir haben wieder nicht Kontakt gehabt bis vor 4 Monaten, wir haben wieder gesprochen. (...) Sie hat zu mir gesagt, momentan (...) also vielleicht bald wird besser. Vielleicht wir könnenn mehr unterhalten.  #00:18:54-3#

I: (...)Das wäre gut. #00:18:55-0#

B: Und, ich wünsche mir auch (...) ich hoffe (...) #00:19:00-0#

I: ähm sie ruft dich dann quasi auf deinem Handy an, oder? #00:19:02-6#

B: Ja.  #00:19:03-8#

I: Okay #00:19:03-8#

B: Die haben nicht glaube meine Nummer, ich habe nicht Nummer von denen.  #00:19:07-1#

I: hm (bejahend) #00:19:08-0#

B: Sie haben schon zu mir angerufen, aber nicht von Ihrer Nummer, sondern von mein Nachbarin. #00:19:11-5#

I: hm (bejahend) #00:19:12-0#

B: Ich habe immer Kontakt mit mein Nachbarin, die wohnt im Iran #00:19:15-5#

I: hm (bejahend) #00:19:15-5#

B: Der (...) also ich nenne Sie immer Tante, weil Sie (...) hat von uns viel gemacht (...) weil Sie für uns viel gemacht hat (lacht) #00:19:25-0#

I: Mhm. #00:19:27-3#

B: Ja. (...) #00:19:31-1#

I: Ja (...) Hoffentlich wird das bald besser. #00:19:36-2#

B: Ich hoffe. Trinkst Tee? #00:19:38-3#

I: Ich hab noch, danke! #00:19:38-3#

B: Hmh. #00:19:42-3#

I: ähm mit wem hast du dann hier in ähm in Deutschland Kontakt? Also ich mein, okay, die Leute aus'm Jugendcafé natürlich, und #00:19:56-8#

B: Ja #00:19:56-8#

I: Hallo, ich #00:19:56-8#

B: (lacht) ja.  #00:20:00-7#

I: ähm wer ist denn sonst noch so dabei? #00:20:06-3#

B: Hmh (...) Schon viele Leute kenne ich. #00:20:11-5#

I: hm (bejahend) #00:20:11-5#

B: Ich habe Kontakt, Beispiel: Mein Schwimmlehrerin, mein Schwimmlehrer, die sind beim (...) Mhh (...) Campus Asyl? #00:20:24-8#

I: Ah! Mhm. #00:20:28-8#

B: Vielleicht kennst du auch denen? #00:20:28-8#

I: ähm, mit denen habe ich geschrieben, aber ich muss (...) getehen, dass, ähm dass ich da momentan noch nicht geantwortet hab #00:20:40-3#

B: hm (bejahend) #00:20:40-3#

I: Muss ich noch machen (lacht) (...) ähm Ja. Shit. Muss ich noch machen. #00:20:50-2#

B: Und kenne ich viele andere Leute, die haben mich echt viel geholfen. Beispiel Thomas, macht seit drei Jahren #00:21:00-6#

I: hm (bejahend) #00:21:00-6#

B: Also gibt mir seit drei Jahren Nachhilfelehrer. #00:21:04-3#

I: Der macht vor allem Mathe gell? #00:21:04-3#

B: hm (bejahend), also Mathe, Deutsch, (...) ähm #00:21:11-2#

I: Jetz können wir dann mal Englisch anpacken. #00:21:10-6#

B: Mhhja genau (lacht) #00:21:14-0#

I: (lacht) #00:21:14-0#

B: Und (...) der macht auch einfach Ehr (...) Ehrenamtlich, oder? #00:21:18-2#

I: Der is Ehrenamtlich, genau. ähm(...) #00:21:24-1#

B: Vor drei Jahren, (...) der hat mit mir begonnen  mit dem Lernen. Damals konnte ich nicht Deutsch reden. Also, so, ein bisschen.  #00:21:33-8#

I: hm (bejahend) #00:21:33-8#

B: So (...) habe ich Ausländerdeutsch geredet. Ich gehen, du gehen (lacht) #00:21:39-2#

I: (lacht) Ja gut aber, ähm, ich denk mal da hast du dich dann jetz schon deutlich verbessert oder? #00:21:47-2#

B: Bisschen. (lacht) #00:21:52-5#

I: Ich denk das ist ganz passabel so. (lacht) #00:21:54-1#

B: (lacht) (...) Wie gesagt, es (...) es könnte besser werden. Aber ich versuche es gerade. #00:22:01-2#

I: Das kommt alles mit der Zeit. #00:22:02-8#

B: hm (bejahend) #00:22:03-7#

I: Hast du die Asterixe schon gelesen? #00:22:05-5#

B: So, ja (...) Ich glaube, ich hab ein Buch fertig gemacht, aber (...) wie gesagt, gibt's viele von den neue Worten. #00:22:13-2#

I: ähm also weil wenn du (...) ähm wenn du was anderes lesen willst, dann sag auch Bescheid. Also, wenn dir das ned gefällt (...) #00:22:23-4#

B: Ja, danke #00:22:23-4#

I: Dann such ich irgendwas anderes, und ansonsten, wenn du's fertig gelesen hast, dann bring ich dir neue, da gibts mehr. #00:22:28-7#

B: Ja, okay. Danke. #00:22:28-7#

I: Gerne. (...) ähm Ja. #00:22:33-3#

B: Und, neue Frage? (lacht) #00:22:35-9#

I: ähm okay, ähm (...) Bildung: Du hast in Afghanistan und im Iran warst du nie in der Schule gell? #00:22:45-4#

B: Ne.  #00:22:48-6#

I: Also erst ab hier. #00:22:49-5#

B: Ja, seit dreieinhalb Jahre besuche ich Schule. #00:22:53-0#

I: hm (bejahend) #00:22:53-0#

B: Mache ich Schule. Vorher habe ich nicht besucht. #00:22:57-9#

I: ähm Bezüglich der Schule: ähm da hab ich nen (...) Interview ghabt mit jemandem aus Syrien #00:23:09-2#

B: hm (bejahend) #00:23:10-2#

I: Der meinte, dass seine Lehrer alle bayrisch reden würden und dass er Sie nicht verstehen würde. Is das bei dir auch? #00:23:18-6#

B: Bei mir war so in der Ausbildung. #00:23:22-8#

I: Echt?
 #00:23:24-5#

B: Mein Lehrer hat auf bayrisch gesprochen, ich habe nicht verstanden, dann ich hab gesagt: 'Können Sie bitte, also versuchen Sie bitte auf , auf Deutsch sprechen.' Der sagt: 'Das ist nicht mein Probem, das ist dein Problem. Du musst verstehen.' Das war so. (lacht) #00:23:40-5#

I: Hmh..? (...) Okay. #00:23:43-2#

B: (lacht) Aber der war nett, der hat immer Spaß gemacht mit uns. Aber wegen Sprache der hat gesagt 'Ist nicht mein Problem.' Der konnte nicht Hochdeutsch sprechen. #00:23:51-5#

I: Ach, er konnt's nicht? #00:23:53-0#

B: Er hat gesagt 'Das ist dein Problem. Du bist in Deutschland, du musst die Sprache lernen.' #00:23:58-2#

I: Okay. (lacht) #00:24:02-5#

B: (lacht) Aber der war nett. #00:24:02-5#

I: Ach? Das (...) das ist schön. #00:24:02-3#

B: Der hatte immer zu uns Respekt. #00:24:06-8#

I: hm (bejahend) #00:24:07-4#

B: Außerdem die Sprache war nicht seine Schuld, war unsere Schuld.(lacht) #00:24:08-0#

I: (lacht) #00:24:10-3#

B: Trotzdem ich habe eine vier geschrieben. #00:24:14-8#

I: (lacht) #00:24:16-1#

B: Aber (...) Mein Chef war nicht gut. (...) #00:24:24-5#

I: (...) Warte, mach ma einfach mal die nächste. ähm, also du bsit momentan ein anerkannter ähm Geflüchteter oder? Also dein Asylstatus ist (...) 'angenommen', oder? #00:24:40-9#

B: Ja, so (...) DU kannst hier (unv.) (Zieht Aufenthaltserlaubnis aus dem Geldbeutel) #00:24:48-5#

I: Aufenthaltserlaubnis? #00:24:49-8#

B: Mhm. #00:24:51-8#

I: Ah. (...) Ist ein interessantes Ding. Warte (...) Moment, ich hol mal eben meinen Perso. (...) #00:25:13-0#

B: Ganz unterschiedlich. #00:25:18-8#

I: (...) Ja, aber (...) warte. (...) Ach tatsächlich, der (...) Auch das Wackelbildchen hier drunter ist anders #00:25:32-5#

B: Ja. Ich denke (...)(lacht) #00:25:44-7#

I: Sogar mit deiner (...) mit der zuständigen Behörde mit dazu.  #00:25:52-1#

B: hm (bejahend) #00:25:52-1#

I: Ach warte, das steht bei mir auch drauf. (...) Meh, ja okay, steht bei mir das selbe drauf. #00:25:58-8#

B: (lacht) #00:25:58-8#

I: ähm (...) Aber das heißt, dein Status war von Anfang an anerkannt. Wo hast du den Antrag dann gestellt ghabt? #00:26:12-0#

B: Ich weiß es nicht, meine Vormund hat gemacht. Damals hatte ich Vormundin #00:26:17-2#

I: hm (bejahend) #00:26:18-1#

B: Sie hat alles für mich gemacht und (...) weiß es nicht. #00:26:23-4#

I: Hast du mit der jetz noch Kontakt? #00:26:25-7#

B: ähm ne. #00:26:25-6#

I: Okay. #00:26:25-6#

B: Sie hat, also (...) sich schon beschäftigt. #00:26:34-7#

I: hm (bejahend) #00:26:34-7#

B: Sie macht gerade eine andere Arbeit. #00:26:34-5#

I: Ah. (...) Mmh, und des is jetz hier auch (...) is jetz die zweite Wohnung, in der du hier lebst oder? #00:26:46-5#

B: hm (bejahend) #00:26:46-5#

I: Also, erst war's ein Stock weiter unten, und jetzt hier? #00:26:50-9#

B: // ähm ich war ein Jahr drei Monate in Don Bosco. #00:26:55-7#

I: Achso. #00:26:56-2#

B: Habe dort gewohnt. Dann *räuspert sich* wir hatten keine Betreuer. (...) Eigentlich, ich musste bis 21 dort bleiben. ähm Als ich in Deutschland gekommen bin, dann wir waren selbe ständig. Wir hatten Vormundin aber wenige Betreuer. Wir mussten selbstständig sein und selbst zum Arzt gehen.  #00:27:16-8#

I: hm (bejahend) #00:27:17-3#

B: Und manche Sachen selber erledigen. Wir waren Don Bosco Gruppe 4. #00:27:20-0#

I: ähm hat das funktioniert? #00:27:23-2#

B: ähm nicht so genau. Weiter nur Problem. Dann (...) #00:27:29-4#

I: Was ist denn da schief gelaufen? #00:27:31-2#

B: ähm ich bin öfter zum Arzt gelaufen, und wir hatten keine Übersetzer, keine Betreuer. Und der hat, Beispiel wenn ich Kopfschmerzen hatte, der hat nicht verstanden was ich meine. Der hat einfach meinen Fuß angeschaut oder meinen (...) #00:27:45-2#

I: Du bist zum Arzt gegangen, und du konntest aber damals noch kein Deutsch, und es war kein Dolmetscher dabei? #00:27:50-9#

B: Ne. Und ich hab einfach, wir haben so einen Satz gelernt, wir haben nur gelernt: 'Mir ist schlecht.'  #00:28:01-9#

I: hm (bejahend) #00:28:02-4#

B: Dann, der sagt 'wo?' Ich habe nicht verstanden. Der hat einfach mit seinem angefasst, Beispiel hier, hier, hier, dann ich musste einfach zeigen Beispiel hier oder (...) das war echt schwierig. #00:28:16-1#

I: hm (bejahend) #00:28:16-5#

B: Das war echt so hart, harte Zeit aber (...) Gott sei Dank, es ist schon vorbei (lacht). Und (...) ich hatte Migräne, jetzt habe ich auch, dann musste ich immer Tabletten nehmen #00:28:28-9#

I:  hm (bejahend) #00:28:30-2#

B: Bis ich schlafen konnte. Dann unsere Betreuerin hat immer zu mir gesagt: 'Bevor ich nach Hause gehen (...) ich gebe dir eins, eine Tablette, dann kannst du nehmen, dann schlafen.' (...) Und (...) wir haben, also ich habe meine bekommen, danach der sagt ich mache (...) ich weiß nicht, wie viele Tage Urlaub, dann du kriegst von mir 20 Tabletten. Dann ich hab bekommen, damals war ich 16, 16-jähriger, dann ich hab an einem Nacht, in der Nacht hatte ich so stark Kopfschmerzen, dann ich habe eine genommen. Hat nicht geholfen.  #00:29:11-4#

I: hm (bejahend) #00:29:11-4#

B: Dann ich habe wieder eine genommen. Wieder nicht geholfen. #00:29:11-4#

I: (...) #00:29:11-4#

B: Ich habe drei Tabletten genommen, die waren 600. #00:29:13-9#

I: Oooh. #00:29:15-8#

B: Danach (...) #00:29:15-8#

I: Ibuprofen 600? #00:29:16-1#

B: Weiß ich nicht, das war das aber. War 600, darauf habe ich gesehen dann. #00:29:21-5#

I: Mhm. #00:29:21-5#

B: ähm mir war schwindlig, dann ich (...) mir war schwindlig, dann ich bin einfach auf dem Bett gelegen und geschlafen bis nächstes Tag.  #00:29:28-7#

I: Das sind sehr starke Schmerzmittel. #00:29:32-3#

B: Ja, und nächste Tag, ich hatte einen Termin beim Arzt. #00:29:34-7#

I: hm (bejahend) #00:29:34-7#

B: Ich habe 20 Tabletten bekommen, in (...) sieben Tage ich habe fast alles genommen. (lacht) #00:29:44-3#

I: (...) #00:29:50-9#

B: Dann ich hatte einen Termin bei Hautarzt, dann (...) ähm (unv.), dann die hat zu mir gesagt: 'Wie gehts dir momentan? Wie gehts mit deinen Tabletten? Kriegst du nicht, wie viele nimmst du am Tag Tablette?' Dann ich habe gesagt, meine Betreuerin ist im Urlaub, ich habe fast 20 Tablette in sieben Tage genommen. Dann (...) dieser Arzt hat meine Vormundin angerufen, meine Betreuerin, der Chefin, (...) und der Chef, und Jugendamt. Alle sind gekommen, udn Übersetzer. Wir waren ungefähr neun Leute. #00:30:29-3#

I: Uff. #00:30:29-3#

B: Vom Jugendamt, Chef von unsere Heim, Betreuerin, Sie musste vom Urlaub kommen. #00:30:36-1#

I: Oh, Scheiße. #00:30:36-8#

B: UNd meine Vormundin (lacht), meine Betreuerin. Wir waren alle in einem Raum.  #00:30:40-0#

I: Ahh. (...) #00:30:44-8#

B: Dann hat zu mir gesagt: 'Wo wohnst du?' Ich habe (unv.), also, (...) Ich wohne in Don Bosco, und zwar in Gruppe vier. Wir (...) zwei Leute wohnen in einem Zimmer, und der schnarcht jeden Tag (lacht) jede Nacht #00:30:57-5#

I: (lacht) #00:30:59-0#

B: Muss ich immer so stark Tabletten nehmen, bis ich so tief schlafen, nicht zuhören, ne? #00:31:02-3#

I: Ja. #00:31:02-3#

B: Dann, mich nicht aufwecken. Dann hat zu meinem Chef gesagt: 'Entweder du gibst ihm ein Einzelzimmer, oder (...) er muss von Don Bosco raus.' Dann die haben gesagt: 'Wir können nicht Einzelzimmer geben, weil hier (...) niemand EInzelzimmer hat.' #00:31:23-0#

I: hm (bejahend) #00:31:23-0#

B: Deswegen die haben mich in (unv.) geschickt. Und so war, ich habe Einzelzimmer bekommen, zum Glück dann. Ich habe meine Tabletten regelmäßig bekommen, nicht zwei, drei, sondern immer eine. #00:31:36-9#

I: Ist die Migräne seitdem besser geworden dann? #00:31:40-4#

B: Ja. Früher hatte ich so stark, letztes Jahr mir war (...) immer schwindelig. Ich durfte nicht alleine in die Stadt gehen. #00:31:48-0#

I: Was? ähm (...) Bist du dann auch mal umgekippt oder..? #00:31:55-5#

B: Öfter. #00:31:56-0#

I: Öfter?! #00:31:56-0#

B: Zweimal die Betreuerin und mich vom, von der Straße abgeholt. (lacht) #00:32:03-9#

I: Oi. #00:32:07-1#

B: Dieses Jahr, vor zwei Wochen hatte ich wieder einen Unfall. Wegen Schwindligkeit.  #00:32:09-7#

I: Vor zwei Wochen erst? #00:32:14-5#

B: hm (bejahend). Dann ich war im Krankenhaus, ungefähr vier Stunde, dann die haben gesagt: 'Du hast nix. Geh nach Hause.' Dann bin ich nach Hause gekommen.(lacht) (...) Haben gesagt, das ist einfach (...) ich weiß nicht, der sagt, dass nicht Krankheit einfach (...) irgendwas mit dir (...) nicht stimmt. (lacht) #00:32:35-6#

I: Also, du hast bis jetzt keine Diagnose dazu? #00:32:39-0#

B:// Aber ist nicht schlimm. (...) Was ist das? #00:32:43-3#

I: ähm, die Ärzte wissen nicht genau, was dir fehlt, oder wissen's schon?  #00:33:07-0#

B: Sie haben fast ganze Gruppe untersucht, aber wir (...) du bist gesund ansich. (lacht) #00:32:54-7#

I: ähm (...) #00:32:57-1#

B: Hast du nur Migräne, dann kannst (...) das ist Erbe. Von meine Mama. Die hat auch Migräne. #00:33:02-2#

I: Okay. (...) #00:33:02-2#

B: Und meine Mama hat auch Herzschmerzen, ich hoffe, dass ich nicht die selbe bekomme.  #00:33:12-0#

I: Sind die schlimm? #00:33:15-8#

B: Als ich in Iran war (...) sie hatte viele Probleme. (...) Aber ich weiß nicht, ob sie (...) am Leben ist oder nicht. Letztes mal, ich habe ein Foto bekommen vor zwei Monaten, sie (...) lage im (...) Krankenhaus. (...) Weiß ich nicht, also (...) ich hoffe (...) sie ist am Leben. Ich will Sie wieder sehen. Schauma mal. Das macht mein Schicksal. #00:33:49-2#

I: ähm (...) Hast du (...) seitdem noch irgendwie Kontakt mit deiner Nachbarin? #00:33:54-2#

B: (unv.) Ja so, ich rufe Sie immer an. Also, nicht immer, jedoch zwei, drei mal. #00:34:02-4#

I: Und die weiß auch nicht, was los ist? #00:34:03-6#

B: Ne, sie hat zu mir gesagt, wenn du nach Iran geflüchtet hast, dann können wir zusammen finden. (...) Und vielleicht im Sommer fliege ich nach Iran, vielleicht nicht. Ich weiß nicht. #00:34:17-3#

I: hm (bejahend) #00:34:18-0#

B: Aber ich werde Sie finde. Schauma mal. #00:34:21-4#

I: Du willst dich im Sommer auf die Suche machen, oder wie? #00:34:27-2#

B: Vielleicht. Ich warte auf meinen Reisepass. (...) Wenn ich bekomme, dann ich versuch's denen finden und (...) aber (...) und mein große Bruder wohnt seit zweieinhalb Jahren in Amerika. #00:34:46-3#

I: In Amerika? #00:34:48-1#

B: hm (bejahend). In New York. (...) #00:34:58-0#

I: ähm wird er (...) auch bei der Suche mitmachen, oder (...)? #00:35:03-5#

B: Ja, der macht auch. Der hat gesagt, vielleicht fliegt in drei, oder zwei Monaten nach Iran. #00:35:14-7#

I: Also mit (...) mit Ihm bist du jetzt quasi #00:35:18-7#

B: Kontakt? #00:35:20-3#

I: Im kontakt. #00:35:20-3#

B: hm (bejahend) Also die war (...) der war auch bei meiner Familie vor drei Jahren #00:35:27-3#

I: hm (bejahend) #00:35:27-3#

B: Und (...) mein Bruder musste (...) von meiner Mama, meiner Familie getrennt werden. Dann, Sie haben sich getrennt. Dann, mein Bruder mit seiner Familie von Polizei verhaftet und nach Afghanistan geschickt. Dann, die waren in Afghanistan, dann (...) so der hat zu mir erzählt. Der hat zu mir gesagt: ' Eine Freund hat zu mir gesagt, gehen wir nach Pakistan, wir können uns melden.' Dann (...) also (...) die UNO hat, ähm gibt uns Geld, bis wir in Pakistan bisschen Essen Geld kriegen #00:36:13-9#

I: hm (bejahend) #00:36:13-9#

B: Bis wir am Leben bleiben, dann (...) wir machen Interview in Pakistan bei amerikanische Soldat #00:36:21-9#

I: hm (bejahend) #00:36:22-3#

B: Dann machen (...) vielleicht wenn wir Chance haben, dann wir machen für zwei Monaten, wir bekommen einen Anwalt, dann können wir nach Amerika fliegen. Dann hie haben das gemacht und geschafft. Und mein Bruder von Pakistan nach Amerika geflogt. Geflogen. Ist. Ja, der ist nach Afghanistan geflogen, dann der wohnt, dann wir hatten keinen Kontakt. (...) Vor (...) fünf? Monaten ich habe in Facebook mit jemanden geschrieben. Dann die hat gesagt: 'Bist du Mahdi?' Ich sage: 'Ja'. Dann die sagt: 'Ich bin Frau von deinem Bruder'. Sage ich 'was'? #00:37:01-5#

I: Ui. #00:37:01-5#

B: Sagt er (...) #00:37:05-9#

I: Wie, wie bist, wie bist, hast du Sie erwischt? #00:37:10-4#

B: War furchbar, ich haben nicht gekannt. Weil, ich habe gedacht, mein Bruder ist im Iran, der kann nicht so einfach anrufen, mit mir reden. #00:37:17-0#

I: hm (bejahend) #00:37:17-0#

B: Aber das war, wir haben in Facebook geschrieben. #00:37:19-8#

I: hm (bejahend) #00:37:19-8#

B: Dann die hat gesagt: 'Ich habe deinen Namen in Facebook angegeben, dann ich habe dich gesehen deswegen ich habe dir geschrieben. Dein Bruder ist in der Arbeit, der kommt am Abend. Aber wir können, aber du kannst mich gleich anrufen'. Dann ich habe Ihr angerufen, wir haben gesprochen. (...) Dann Sie hat so einfach über sich erzählt wie wir nach Amerika gekommen sind, war sehr schwierig so, so, so. #00:37:48-2#

I: hm (bejahend) #00:37:48-8#

B: Und (...) mein Bruder hat zu mir gesagt: 'Wenn geht, ich will dich nach Amerika sehen zu mir'. Dann ich habe gesagt: 'Ne, ich bleibe in Deutschland'. Ich will nicht nach Amerika. Vielleicht mache ich Urlaub, aber ich will nicht dort wohnen #00:38:05-3#

I: hm (bejahend) #00:38:05-7#

B: Mh. (...) (unv.) Gefällt mir in Deutschland. Weil, hier man hat (...) weil hier hat man mehr Sicherheit hat, mehr Freiheit. #00:38:21-4#

I: Als in Amerika? #00:38:23-0#

B: Hier. #00:38:23-0#

I: hm (bejahend) #00:38:23-4#

B: In Amerika, jeder Leute hat ein Waffe. #00:38:27-4#

I: Stimmt. #00:38:27-4#

B: Gefährlich. #00:38:31-5#

I: Hier ist es nur auf der Autobahn gefährlich. #00:38:32-4#

B: Ja. (lacht) (...) Nächste? #00:38:39-6#

I: ähm gerne, ja. (...) Nächstes mal bring ich dir'n Tee mit, den trink ich dir ja immer weg #00:38:50-3#

B: Nee, passt! Ich hab viele gekauft, das kostet nicht so viel. Ich glaube. (lacht) #00:38:58-8#

I: (Macht Handgeste) Das ist eine italienische Handgeste.(lacht) #00:39:04-1#

B: Das ist, ähm (...) 60 Cent? Nicht so viel. #00:39:06-3#

I: Danke!  #00:39:09-1#

B: Ich gehen auch immer Tee trinken. Ich kaufe immer viele. Ist garkein Problem. #00:39:14-8#

I: Ich muss mir angewöhnen, mehr Tee zu trinken und weniger Kaffe. #00:39:23-2#

B: Ja #00:39:23-2#

I: Das wäre gesünder. #00:39:24-6#

B: Ja, Kaffee ist nicht so gut. Ich habe heute eine Tasse getrunken (lacht). #00:39:30-5#

I: ähm ich hab ein bisschen mehr. (...) Heute geht sogar, ich glaube es sind nur fünf. Nee warte, sieben. #00:39:45-4#

B: Sieben Tassen? Unglaublich. #00:39:56-2#

I: ähm, aber hat's noch mehr so Situationen gegeben, wie zum Beispiel das mit dem Arzt? Also, dass du (...) vor allem am Anfang (...)ähm nicht die Möglichkeit hattest, mit dem, mit dem du reden musstest, reden zu können? Also, dass hald zum Beispiel kein Dolmetscher dabei war oder irgend sowas in die Richtung? #00:40:22-5#

B: War echt schwierig. #00:40:24-7#

I: Ist das öfter passiert? #00:40:24-7#

B: Ja. Ein halbjahr ist das passiert. Dann, ich bin her gekommen, dann zum Glück meine, immer meine Betreuerin mit mir zum Arzt gegangen sind. Meine Betreuerin war dabei und eine Dolmetscher. #00:40:40-8#

I: ähm #00:40:40-8#

B: Hier war besser als Don Bosco. #00:40:43-0#

I: hm (bejahend) #00:40:43-0#

B: Auch Essen, Platz, (...) und (...) wir, wir hatten so, ich hatte mehr, ähm, Möglichkeiten als in Don Bosco. Don Bosco, die Leute waren sonst nett #00:41:00-9#

I: hm (bejahend) #00:41:00-9#

B: Die waren nett, aber (...) wir hatten wenige Möglichkeiten. #00:41:04-8#

I: Also das Haus ansich war nicht so gut, ähm, nicht so gut wie hier? #00:41:09-1#

B: Also doch, wie gesagt wir waren zwei Leute in einem Zimmer. #00:41:13-8#

I: hm (bejahend) #00:41:14-3#

B: Der war auch krank #00:41:17-3#

I: hm (bejahend) #00:41:17-3#

B: Der hat die ganze Nacht geschnarcht und so. #00:41:20-7#

I: Mh #00:41:22-5#

B: Ja, das (...) war schon richtig laut. #00:41:25-3#

I: Das ist ne (...) schlechte Kombination wenn er sehr laut schnarcht und du Migräne hast. #00:41:29-1#

B: hm (bejahend) Ich musste immer zwei Tablette nehmen bis ich so tief schlafen kann. #00:41:34-4#

I: hm (bejahend) #00:41:34-9#

B: Danach, ich hab viele Tabletten genommen, als ich auch im Iran war. Dann, ich hab immer Schmerzen gefühlt, so hier, diese Seite so, auf dem (...) so diese Seite, linke Seite. Dann, #00:41:49-7#

I: // Auf der linken Seite? #00:41:49-7#

B: Ja. Dann, ich war beim Arzt. Dann die haben gesagt auf deine, auf deine Leber gibt's zwei, dre Flecke #00:42:00-7#

I: hm (bejahend) #00:42:01-2#

B: Die sind wegen Tabletten. Du hast viele genommen. Die sind direkt auf die Leber gegangen. #00:42:05-9#

I: Mm (...) hm (bejahend) #00:42:05-9#

B: Und hat gesagt, musst du jetzt (...) ähm, mehr Wasser trinken, oder Tee trinken, und nicht Tabletten nehmen. Wenn du nimmst, es wird größer, und noch starke schmerzen. Hat gesagt, wenn so wird, müssen wir operieren. #00:42:22-4#

I: hm (bejahend) #00:42:22-4#

B: Dann ich habe Angst gehabt, scheiße! Dann ich hab weniger Tabletten genommen. ähm mehr Tee und Wasser getrunken, und ich war jetztes Jahr wieder beim Arzt, der hat gesagt, is besser geworden. #00:42:36-9#

I: Ist besser geworden? #00:42:37-8#

B: Ja, ein bisschen. Er sagt, vielleicht dauert zwei, drei Jahren, bis ganz weg sind, aber musst du bis dahin (...) wenn du Tabletten nimmst, nicht nehmen, aber ich will jeden Monat eine. Nicht mehr. Also ich habe in diesem Monat zwei Tabletten genommen #00:42:54-7#

I: ähm Schmerzmittel wieder oder? #00:42:56-9#

B: Ja, Schmerzmittel. Manchmal ich habe sehr starke Kopfschmerzen, ich kann nicht mit Wasser oder Tee wegmachen, sondern Tablette. (...) #00:43:06-3#

I: Aber du hast jetzt Tabletten, die dir direkt helfen. #00:43:14-4#

B: Das ist (...) mein Tablette diese früher habe ich bekommen. #00:43:14-9#

I: hm (bejahend) #00:43:14-9#

B: Genauso das. Aber, wie gesagt, ich nehmne jetzt jeden Monat zweimal, vielleich einmal, vielleicht nicht. #00:43:23-2#

I: Das ist sehr viel weniger als 20 in einer Woche. #00:43:28-6#

B: hm (bejahend). Aber wie gesagt, damals hatte ich keine Betreuerin, ich musste einfach meine Schmerzen wegmachen #00:43:32-3#

I: hm (bejahend) #00:43:33-2#

B: Ich habe nix, nicht gedacht: 'Wenn icht Tabletten nehme, ich mache meine Kopfschmerzen weg, sondern ich bekommen davon Leberschmerzen'. Ich habe nicht gewwusst. #00:43:43-9#

I: hm (bejahend) #00:43:43-9#

B: Es würde noch schlimmer sein, also tiefer. Aber zum Glück, nicht so passiert. #00:43:53-2#

I: Wie lang hat das gedauert, bis (...)ähm bis das festgestellt wurde, also bis du quasi so viele Medikamente genommen hast damals? #00:44:03-5#

B: Letztes Jahr. #00:44:05-5#

I: Letztes Jahr? #00:44:05-5#

B: hm (bejahend) #00:44:05-5#

I: Also (...) ähm  #00:44:11-5#

B: Zweieinhalb Jahre. #00:44:13-6#

I: (...) #00:44:14-7#

B: Im Iran habe ich oft Tablette genommen, jeden Tag dreimal. (...) Musste ich, ich war beim Arzt, der hast du gesagt: 'Du musst Tabletten nehmen'. Also ich hatte Kopfschmerzen #00:44:26-6#

I: hm (bejahend) #00:44:27-0#

B: Und damals habe (...) habe ich auch ab - und zu geweint, wegen Kopfschmerzen war so stark. Und (...) das war so eine lustige Geschichte: Ich hatte so stark Kopfschmerzen, ich war, ich glaube (...) 13-jähriger, sowas. #00:44:43-1#

I: hm (bejahend) #00:44:43-5#

B: Dann, wir waren beim Arzt *räuspert sich* (...) Also mein Mama hat gesagt, zu mir gesagt: 'Müssen wir zum Arzt gehen'. Dann wir sind zum Arzt gegangen, dann *räuspert sich* der hat ähm der hat, der Arzt hat zu mir gesagt: 'was ist los bei dir?' Ich sag: 'Ich hab so starke Kopfschmerzen, wenn so stark wird, ich muss weinen. Dann (...) ich muss weinen und (...) irgend, manchmal ich schlagen meinen Kopf an die Wand. Dann der sagt: 'vielleicht hast du im Kopf etwas'. Dann, die haben so, Foto gemacht, ich weiß nicht, was heißt auf Deutsch? #00:45:18-0#

I: ähm (...) nen ähm (...) Röntgen, eine Röntgenaufnahme. #00:45:26-1#

B: Genau. Ja, sowas. Dann, hat zu meine Mama gesagt: ' Nächste Woche kommen Sie bitte hier'. #00:45:38-4#

I: Machen wir's gleich auf Englisch, das ist X-Ray #00:45:36-0#

B: X-Ray ist? (lacht) #00:45:37-2#

I: Entschuldigung. (lacht) #00:45:39-7#

B: Dann, der Arzt hat zu meine Mama gesagt, nächste, nächste Woche kommen zu mir, dann er sagen, welche Problem dein SOhn hat. Bis dahin, wir bekommen ein Ergebnis. #00:45:51-4#

I: hm (bejahend) #00:45:51-4#

B: Dann, ich bin wieder mit meine Mama zum Arzt gegangen, also nächste Woche. #00:45:57-3#

I: hm (bejahend) #00:45:57-8#

B: Dann (...) der Arzt hat zu meine Mama gesagt: ' Deine Sohn hat (...) Tumor. Im Kopf'. Dann hat gesagt: 'Er kann sechs Monaten leben, dann (...) wird gestorben'. (...) Dann, hat meine Mama gesagt: 'Lass Ihn, was er machen will. Der ist nur sechs Monate am Leben'. Dann, meine Mama hat echt richtig viel geweint.
 #00:46:27-0#

I: Verständlich (...) #00:46:27-0#

B: Hm, dann (...) war, ich hab schon alles geholt, dann (...) meine Mama hat, und meine Bruder ist zu mir gekommen, meine Schwester haben zu mir gesagt: 'Was willst du machen? Du weißt alles, was, was wird bei dir?' #00:46:43-2#

I: hm (bejahend) #00:46:43-2#

B: Du weißt, du kannst nur sechs Monaten leben, dann nicht mehr. Sagt, mach, was du willst. Was willst du essen? Wo willst du gehen. Sagt, wir sind immer bei dir. Ich hab zu denen gesagt: 'Ich habe keine Angst vorm Sterben.' Habe gesagt: 'Ich bin schon vorbereitet. Ich habe garkeine Angst. Ist mir echt scheißegal, was Ihr, was Ihr (...)' #00:47:06-2#

I: Du warst damals 13? #00:47:08-6#

B: hm (bejahend)(lacht) #00:47:11-1#

I: Mhm (...) #00:47:14-1#

B: Dann, drei Monaten so gelaufen. Ich war auch dünner geworden, ich hatte schon Angst, aber ich wollte nicht zu meiner Mama, zu meinem Bruder sagen: 'Ich habe Angst.' Weinen undso #00:47:24-5#

I: Ja. #00:47:26-0#

B: Dann, die werden auch trauriger. Deswegen ich habe gesagt: 'Ich werde sowieso tot'. Damals habe ich auch nicht so gut verstanden, was bedeutet sterben. (lacht) #00:47:32-9#

I: hm (bejahend) #00:47:34-1#

B: Ich habe nicht so gut verstanden. Ich habe gedacht: 'Man schläft, dann wird fertig.' Dann (...) nach den drei Monaten (...) da, eine Iraner, also der wohnt in Deutschland, in München. #00:47:52-9#

I: hm (bejahend) #00:47:52-9#

B: Der ist Arzt #00:47:52-9#

I: hm (bejahend) #00:47:52-9#

B: Der ist Arzt, der wohnt in München. Ich glaube in Deutschland geboren ist, aber der ist Iraner. #00:47:58-9#

I: hm (bejahend) #00:47:59-2#

B: Dann, der hat, der hat eine Wohnung im Iran. Eine Villa. Wo wir wohnen. #00:48:07-8#

I: Da habt Ihr gewohnt, in seiner Villa? #00:48:10-7#

B: Ja. #00:48:10-7#

I: hm (bejahend) #00:48:10-7#

B: Dann, eine, ich glaube, war Freund von mein Bruder, hat gesagt: 'Ich habe eine, ähm, Chef. Der wohnt in Deutschland, der ist Arzt. Vielleicht, der versteht besser als diese Arzt zu dem deine Mama gegangen ist.' Dann wir haben mit Ihm ausgemacht, gesprochen, dann der hat gesagt: 'Ich bring dich nach Teheran, und ich schau genau an, was bei dir ist. Vielleicht ist es nicht Tumor.' Er sagt: 'Wenn du Tumor hast, dann du musst bis jetzt ganz dünn werden, du könntest nicht deinen Kopf bewegen. #00:48:44-3#

I: hm (bejahend) #00:48:44-3#

B: Bis jetzt müsstest du im Bett nur bleiben. Er sagt: '(unv.) Ich glaube nicht, dass Tumor.' #00:48:48-5#

I: hm (bejahend) #00:48:49-1#

B: Er hat so gesagt. Dann, er hat wieder mit diese Dinge fotografiert und (...) ein paar Test genacht (...) Und hat zu mir gesagt: 'Ich sage nächste Woche euch Bescheid.' Dann sagt: 'Wenn Tumor ist, dann müssen, also ich kann Dich nach Deutschland bringen.' Der sagt: 'Ich habe dort eine (...) Krankenhaus, so.' Der war echt nett. (...) Dann nach einer Woche, hat zu meinen Mama angerufen, hat gesagt: 'ähm, ich habe eine gute Nachricht für Sie.' Dann meine Mama hat gesagt: 'Kommen Sie bitte zu uns, wir haben für Sie gekocht.' Dann, der ist zu uns gekommen, der hat so mit Süßigkeit gebracht. Und, bei uns ist so, wenn jemand eine Kranke besucht, dann bringe dem hald Süßigkeit. Oder Obst. #00:49:47-1#

I: hm (bejahend) #00:49:47-7#

B: Es ist einfach, ähm (...) unsere (...) Kultur #00:49:53-6#

I: hm (bejahend) #00:49:54-1#

B: Dann, (...) der ist zu uns gekommen, hat geagt: 'Deine Sohn hat keine Tumor, sondern nur eine starke Migräne. Muss Tabletten nehmen, und auch öfter spazieren gehen. Und Wasser trinken. Hat ähm für, für mich so eine (...) Tabletten geschrieben, hat gesagt: 'Du musst dene nehmen.' #00:50:22-1#

I: ähm, verschrieben. #00:50:26-7#

B: Verschrieben. Hat geagt, die, die Tablette sind richtig teuer, aber (...) #00:50:28-6#

I: hm (bejahend) #00:50:28-6#

B: Hat gesagt: 'Ich kann euch ein bisschen helfen, aber nicht so viel.' Nur zum ersten Mal hat geholfen, also ein bisschen bezahlt. Danach, ich musste auch viel arbeiten. Ich hab echt richtig viel gearbeitet, bis ich selber meine Tabletten bezahle. Ich wollte nicht meine Mama und meinen Bruder bezahlt. Ich wollte selber bezahlen. #00:50:50-7#

I: hm (bejahend) #00:50:50-7#

B: Ich habe schon gesehen, meine Mama arbeitet fleißig, mein Bruder. Ich wollte nicht die ganze schwere Sachen einfach von meine Mama, Bruder lassen. Ich hab (...) Ich hab mich festgestellt: 'Ich muss arbeiten. Ich muss (...) mein Tabletten selber bezahlen. Nicht mein Mama und mein Bruder. Ich habe so versucht. Dann ich habe gearbeitet, und immer für meine Tablette habe ich selber bezahlt. Ich war auch oft im Krankenhaus, zwei Wochen, eine Woche. (...). Und (...) und danach (...) ich den Rücken nicht mehr (unv.) gehabt, in der Arbeit schwere Sachen tragen. #00:51:40-3#

I: Du hast da auf Baustellen gearbeitet, gell? #00:51:41-4#

B: Ja. Meine, also, ich glaube es heißt Tandome oder so? Es ist (...) eine Säule sowas? #00:51:53-9#

I: Wirbelsäule? #00:51:53-1#

B: Ja, so. Das hat, also halb ist zerrissen. #00:51:56-6#

I: Hmh? #00:51:58-2#

B: Also, kaputt gegangen. Der Arzt hat geagt: 'Wenn du (...) ein, zwei Monate arbeitest, dann wird die ganze Säule zerrissen. Der sagt: 'Halbes ist schon kaputt. Das ist (unv.), also, hat gesagt: 'Halbes ist schon kaputt.' Der hat so ein Foto uns gezeigt #00:52:17-1#

I: ähm, die Dinger zwischen den Wirbeln inder Wirbelsäule? #00:52:20-1#

B: Also, der gibts die zwei Dinge, die festhalten die Wirbelsäule. Diese Dinge, die #00:52:26-0#

I: Ah, ja! Die Muskeln da. #00:52:30-4#

B: Diese Muskel war, einer war, also diese Seite war richtig verletzt. #00:52:34-2#

I: hm (bejahend) #00:52:35-1#

B: Dann ich musste sechs Monaten zu Hause Pause machen, bis ich vierzehnjäriger geworden bin. #00:52:40-8#

I: Was musstest du auf der Baustelle bitte machen? #00:52:43-6#

B: Ich hab (...) viel gearbeitet. Zement getragen. Stein. Schwere Sachen #00:52:56-4#

I: (...) #00:52:56-4#

B: Also wie gesagt, ich musste meine Tabletten selber bezahlen, ich wollte nicht, mein Mama und mein Bruder bezahlen. #00:52:59-8#

I: hm (bejahend) #00:53:00-4#

B: Sie waren echt unter Druck. #00:53:02-6#

I: hm (bejahend) #00:53:02-9#

B: Mein Bruder hat richtig fleißig gearbeitet. Für unsere, also, Lebensmittel, damit wir selber kaufen können. Dass wir nicht zu den anderen gehen, Geld leihen oder so. Und (...) ja, dann (...) ich weiß nicht, ich bin letzten, ich bin hier gekommen. Und dieses Jahr habe ich wieder bekommen. Als ich in der Ausbildung war. (lacht) #00:53:29-0#

I: Das ist wieder passiert? #00:53:29-0#

B: Ja. (lacht) #00:53:32-3#

I: Mhh. #00:53:33-7#

B: Deswegen ich habe meine Ausbildung aufgehört. #00:53:36-1#

I: Mhh. #00:53:36-1#

B: Aber, jetzt ist wieder besser geworden. #00:53:38-9#

I: Aber, zwischendurch: Was war das eigentlich für eine Ausbildung? #00:53:42-3#

B: Verkäufer. #00:53:44-7#

I: Verkäufer. #00:53:44-7#

B: Ja, ist besser geworden, ähm, ich versuche ab nächstes Jahr wieder eine Ausbildung mache. Ich kann nicht für immer Schule machen. #00:53:52-1#

I: ähm, mit diesen Ruückenmuskeln: Du fängst jetzt morgen den Schwimmkurs an, ne? #00:54:00-8#

B: Ja. #00:54:01-6#

I: ähm für genau diese beiden Rückenmuskeln, also für diese Rückenmuskeln #00:54:07-7#

B: hm (bejahend) #00:54:07-7#

I: ähm (...) Rückenschwimmen. Des soll extrem gut dafür sein. #00:54:13-6#

B: Ich war beim Physio, also, Physiotherapie. #00:54:15-4#

I: hm (bejahend) #00:54:16-2#

B: DIe haben zu mir gesagt: 'Du hast viele Muskulatur im Rücken.' #00:54:19-6#

I: hm (bejahend) #00:54:20-1#

B: Der sagt: 'Du musst viel laufen gehen, bis die locker werden.' #00:54:25-5#

I: Bis sie locker werden? #00:54:25-5#

B: (lacht) #00:54:27-4#

I: ähm, das heißt, du sollt weniger Muskeln am Rücken haben oder wie? (lacht) #00:54:31-1#

B: (lacht) Ich war (...) zwei Monaten beim (...) ich war zwei Monaten beim Physiotherapie, also hast (...) #00:54:39-8#

I: Ja? #00:54:39-8#

B Jeden Woche zweimal. #00:54:41-1#

I: Ja. #00:54:41-9#

B: Die mein Rücken nach der Ausbildung ich musste hingehen. Jetzt die hat immer zu mir gesagt: Du musst mehr laufen gehen, bis deine Muskulatur locker werden. Und ich habe auch so gehört. Ich hab auch viel Laufen gegangen. (...) Und ich glaube ich muss nächsten Monat wieder wegen meinem Rücken hingehen und zeigen. (...) *räuspert sich* Und mein Arzt hat gemeint, (...) ich hab am ganzen Körper sehr starke Muskulatur. #00:55:15-1#

I: Mh ja. #00:55:16-2#

B: Am ganze Körper. #00:55:16-2#

I: Stimmt, hast du. also (...)(lacht) #00:55:16-2#

B:Ja, der hat zu mir gesagt: 'warum so ist?' Ich sage: 'Weil ich als zwölfjähriger in die Arbeit gegangen.' (lacht) #00:55:26-7#

I: (...) #00:55:26-7#

B:  Ich habe als zwölfjähriger wie ein Fitness gegangen. (lacht) #00:55:33-3#

I: Oww. #00:55:35-5#

B: Aber, es is schon vorbei. Schon lange. (...) Zum Glück. #00:55:40-6#

I: Zum Glück. #00:55:40-6#

B: Ja. Jetzt (...) mache ich dieses Jahr Schule. Vielleicht nächstes Jahr ? Danach eine Ausbildung. #00:55:48-8#

I: Nächstes Jahr machen wir'n Quali. #00:55:53-1#

B: Hoffe ich. (...) Ich versuche es. #00:55:57-5#

I: Kriegen wir hin! #00:56:02-1#

B: Ich hoffe. Danke. #00:56:05-8#

I: Ja, bitte! ähm machen wir mal wieder hier ein bisschen weiter. ähm (...) hast du (...) hier, seit du in Deutschland angekommen bist, ähm (...) Rassismus erlebt, in irgeneiner Art und Weise? #00:56:22-3#

B: hm (bejahend), zwei, dreimal. #00:56:22-2#

I: Inwiefern? Also, einmal hast ja gemeint, das mit dem Fahrkartenkontrolleur zum Beispiel.  #00:56:29-4#

B: Ja. #00:56:31-2#

I: hm (bejahend) #00:56:32-2#

B: Und einmal ich war so im Geschäft, Norma.  #00:56:36-7#

I: hm (bejahend) #00:56:36-7#

B: Ich war an der Kasse. Und (...) wir waren viele LEute, und der Mann war nicht so ualt und auch nicht so jung, also so zwischen 50, 55 #00:56:48-6#

I: hm (bejahend) #00:56:48-6#

B: Der hat einfach, angefangen mit dem Sprechen so laut, hat gesagt: 'Die (...) Flüchtlingen sind nach Deutschland gekommen, die (...) können nicht arbei(...) in der Arbeit. Die kriegen von der Stadt Geld, und Sie kriegen unsere Geld, wir geben gerade der Stadt Geld an die (...) die sind faul, und dann aber (...) aber damals war ich in der Ausbildung, das war (...) schon lange her #00:57:24-5#

I: hm (bejahend) #00:57:25-0#

B: Ich hab zu Ihm gesagt, dass das nicht stimmt. Der hat gesagt: 'Wer bist du?' Ich sage, ja, dass das nicht stimmt. Ich mache gerade Ausbildung auf dem meine, also, auf dem Vertrag steht 850€ aber ich kriege 150€. Ich habe zu Ihm gesagt, er sagt: 'Ja, aber du hast drei Jahre das Geld gekriegt. Du gehst zur Schule, du sollst arbeiten.' Sage ich: 'Ja, sie haben recht, aber (...) aber wir lernen gerade die Sprache, ähm sag wir werden später in die Arbeit gehen, Sie sind fast in Rente, Sie bekommen später Ihre Geld wieder zurück.' Und zwei, drei Leute waren, die waren auch Ausländer, zu mir gesagt: 'Einfach halt die Klappe, geh weg von Ihr, warum redest du mit dem, das bringt nix.' Dann, ich hab (...) Ich hab gesehen, die zwei haben recht. Warum ich muss mit dem reden? Das bringt nix. Dann ich bin einfach (...) von Geschäft rausgekommen, nach Hause, ich war (...) so ein Tag, zwei Tage, war ich so ein bisschen traurig. Danach ist, habe ich vergessen alles. Ja. #00:58:34-6#

I: Ja. #00:58:35-6#

B: Ich finde, vielleicht der hat recht? Weiß ich nicht. #00:58:36-4#

I: Nee! #00:58:38-5#

B: (lacht) #00:58:39-8#

I: Das ist tatsächlich nen (...) Trugschluss, also, ähm, er sieht, er kriegt wenig Geld, und er sieht auch: Du kriegst Geld vom Staat. #00:58:52-9#

B: hm (bejahend) #00:58:53-8#

I: Das ist so das, dass er mitbekommt. ähm(...) Das ist allerdings nen relativ kleiner Pool, also (...) Dem Staat ansich steht so viel mehr Geld zur Verfügung. #00:59:08-0#

B: hm (bejahend) #00:59:08-9#

I: ähm, das ist quasi (...) Die zwei Parteien, die eh kein Geld haben, gegeneinander auszuspielen, ähm und währenddessen können andere, die über deutlich mehr Geld verfügen, tatsächlich machen, was Sie wollen. ähm(...) So (...) Das würde jetz ein bisschen lang dauern, aber (...) grundsätzlich (...) Der Mann sollte mehr Geld verdienen. Und dir steht das Geld, das du bekommst, und vor allem deine Ausbildung und deine Schulbildung sthet dir voll und ganz zu. Das ist dein gutes Recht. UNd, also ganz ehrlich: ähm(...) Deutschland braucht dich und andere Leute, die ähm die zuziehen, weil: Du siehst den alten Mann? Der ist momentan 55, in 15, 20 Jahren, der kriegt zwar seine Rente noch dazu, aber es ist nicht sonderlich viel Geld, und irgendwann kann er nichtmehr für sich sorgen. Dann ist er zum Beispiel in nem Pflegehein, und in nem Pflegeheim (...) will keiner arbeiten, weil's momentan genau wie bei Ihm so aussieht: ähm Man kriegt kein Geld und ma hat zum Beispiel beschissene Arbeitszeiten. ähm(...) Und ähm, also vielleicht is es dir nen bisschen aufgefallen: Deutschland wird (...) älter. #01:01:04-3#

B: hm (bejahend) #01:01:04-9#

I: Also es werden weniger junge Menschen geboren, als alte sterben. #01:01:12-1#

B: hm (bejahend) #01:01:12-9#

I: Das heißt, im Laufe der Zeit wird Deutschland immer älter werden (...) weswegen das Land darauf angewiesen ist tatsächlich, dass Leute rein kommen, um zu helfen. Und wenn du jetz zum Beispiel hier ne Ausbildung machst, das ist nen absoluter Glücksfall. #01:01:34-0#

B: hm (bejahend) #01:01:34-0#

I: Also ähm also(...) Lass dir gesagt sein: Der Mann absolut nicht recht. #01:01:41-7#

B: Okay (lacht) #01:01:41-7#

I: (lacht) Ganz ehrlich, wir können froh sein, dass du hier bist. #01:01:46-4#

B: Okay. #01:01:48-2#

I: (lacht): Jetz lassen des Interview mal sein und machen noch nen bisschen was für dich in der Schule wenn du noch Lust hast? #01:01:59-2#

B: (lacht)(schüttelt Kopf) #01:02:01-6#

I: (lacht) Ich pausier das jetz auf jeden Fall mal. (...)(...)(...)(...)(...) Ähm, nochmal: Du, also, du (...) wenn du verpasst manchmal Termine, die für dich ausgemacht worden sind #01:02:24-4#

B: (hm (bejahend) #01:02:24-4#

I: ähm weil dir nicht gesagt worden ist, dass die Termine überhaupt statt finden? #01:02:27-9#

B: Ja, die, manchmal die werden vergessen. Beispiel: heute. Sie (...) sie hat zu mir gesagt: 'Hast du heute, warst du heute beim Termin?' Ich sage: 'Welchem Termin?' 'Beim Stadt!' Sage ich so: 'Wann?' Sage: 'Heute um neun Uhr.' Ich sage: 'Nee. Ich habe nicht gewusst. Niemand hat zu mir gesagt.' 'Oooh. (lacht) Sorry, wir haben vergessen!' #01:02:46-4#

I: (lacht) #01:02:47-1#

B: Oder wenn ich Beispiel irgendwo draußen bin, dann die rufen mich: 'Mahdi, wo bist du?' 'Huh?' 'Du hast gerade einen Termin!' 'Ja okay, ich komme gleich!' #01:02:55-7#

I: (lacht) #01:02:55-7#

B: Die werden vergessen, oder die sagen zu spät. #01:02:58-5#

I: ähm aber sind das dann die Betreuer, die's dir zu spät sagen, oder? #01:03:06-5#

B: Also, man, hier gibt's (...) 21 Leute. #01:03:11-3#

I: hm (bejahend) Also hier im Haus. #01:03:11-8#

B: Ja. Meine ich den erste, ähm, zweite Stock und dritte Stock #01:03:16-1#

I: hm (bejahend) #01:03:16-1#

B: Also ich meine nur für uns. Dann (...) die bringen manchmal durcheinander. #01:03:22-4#

I: hm (bejahend) #01:03:23-8#

B: Beispiel Leute müssen, verliert die Sachen, (...) Termin ausmachen, und Ärztin anrufen, Chef anrufen, meißte machen hier Ausbildung. Deswegen die rufen die Chef an: 'Heute der ist krank, oder der nicht kommen heute. Der kommt. Der ist gekommen. Der war in der Arbeit oder so, so, so. WIrd viele Fragen gestellt. #01:03:44-5#

I: hm (bejahend) #01:03:44-5#

B: Oder, Beispiel: Wenn die von der Ausbildung reinkommen zu Hause, der sagt: 'Mein Chef hatte keine gute Laune oder ich habe meine Kollegen gestritten, oder meine Kollegin, Beispiel, hat mich geschimpft. So, gibts viele Gründe. #01:04:01-7#

I: hm (bejahend) #01:04:01-7#

B: Deswegen vielleicht Sie haben mehr Stress. Aber (...) Die haben öfter schon vergessen, mir Termine sagen. (lacht)  #01:04:10-7#

I: Hmm (lacht) #01:04:10-7#

B: Das Sie haben Beispiel zwei Tage später gesagt. Haben. #01:04:15-1#

I: Ups. #01:04:16-2#

B: Nach die zwei Tagen sie sagt: 'Hast du gestern, vorgestern einen Termin.' Ich habe nicht gewusst. 'Ja okay, wir machen einen neuen Termin aus, aber es dauert wieder zwei Wochen bis du diese Termin bekommst. #01:04:26-7#

I: Ahhhh #01:04:29-1#

B: Deswegen ich bin nicht sicher, weil das öfter passiert. #01:04:35-2#

I: Okay (...) Ja gut, aber das passiert auch mal. #01:04:39-3#

B: hm (bejahend) #01:04:39-3#

I: Danke!