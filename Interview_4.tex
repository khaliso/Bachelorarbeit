\section{Interview 4}

 #00:00:02-2#

I: ..wie des funktioniert. Ah, ja, jetz läufts. Erstmal hi. #00:00:08-0#

B: Hi. #00:00:10-4#

I: (lacht) ähm also ich würd jetz einfach mal so meine Fragen hier runter gehen #00:00:14-4#

B: Ja. Wenn du magst. #00:00:17-4#

I: ähm (...) Wie alt bist du?  #00:00:20-9#

B: ähm ich werde jetzt 21. Ich bin 20 Jahre alt. #00:00:21-4#

I: Du bist 20? #00:00:22-3#

B: Ja. (lacht) #00:00:23-4#

I: Du siehst älter aus. #00:00:23-3#

B: Ich sehe sehr sehr alt aus. (lacht) #00:00:24-8#

I: (lacht) #00:00:27-0#

B: Hatte so viele schlechte Sache da. #00:00:29-9#

I: Echt? #00:00:29-9#

B: Ja, genau. #00:00:31-8#

I: Fuck. ähm, okay - du bist ganz offensichtlich männlich. ähm (...) Aus welchem, wo kommst du ursprünglich her? #00:00:43-6#

B: Ich komme aus Afghanistan. #00:00:45-1#

I: Du kommst aus Afghanistan. #00:00:46-0#

B: Genau. #00:00:47-2#

I: Okay. Und wann bist du nach Deutschland gekommen? #00:00:52-2#

B: Ich bin Anfang 2015 nach Deutschland gekommen. #00:00:54-7#

I: Anfang 2015? #00:00:57-6#

B: Genau. #00:00:59-0#

I: ähm, (...) bis ähm wie bist du nach Deutschland gekommen? #00:01:05-8#

B: Also, ich bin über die Türkei (...) Türkei nach Deutschland bin ich in eine LKW gekommen. #00:01:12-1#

I: Okay. #00:01:13-9#

B: Genau. #00:01:13-9#

I: ähm also (...) quasi über die Landroute über Ungarn dann? #00:01:19-5#

B: ähm das weiß ich nicht ähm, irgendwann war ich in (...) Regensburg #00:01:23-5#

I: Echt? #00:01:22-9#

B: Ja.  #00:01:25-2#

I: Also du bist quasi in der Türkei in den LKW gestiegen #00:01:27-9#

B: Ja. #00:01:28-8#

I: Und bist in Regensburg dann wieder ausgestiegen. #00:01:29-8#

B: Wieder ausgestiegen, ja. #00:01:32-0#

I: Okay #00:01:32-8#

B: Ja. #00:01:34-2#

I: ähm was isn dann passiert? #00:01:39-3#

B: Dann ähm (...) ich war ganz verletzt, meine Füße und sowas, ich konnte garnicht #00:01:41-9#

I: //was? #00:01:41-9#

B: Ja. So, keine Ahnung, mein Fuß war so dick, weil ganze Blut und in LKW war wenig Platz #00:01:47-6#

I: hm (bejahend) #00:01:48-1#

B: Und (...) ich bin zu Fuß hald Stadt gekommen, zum, wo die (...) Pakistanische Geschäft ist. Bin dann gestanden, und ich hab so Leute gesagt wo hald Polizei oder so ist #00:02:02-4#

I: hm (bejahend) #00:02:02-4#

B: Und dann Polizei ist zu mir gekommen und die haben mich geholt #00:02:04-6#

I: hm (bejahend) #00:02:04-6#

B: Und danach die haben zu mich zum Heim geschickt #00:02:08-8#

I: Bist du dann noch ins Krankenhaus gekommen oder so? #00:02:10-7#

B: Genau, dann die haben mich genommen zum Krankenhaus, die haben Blut genommen und alles geschaut, dann musste ich länger warten, bis halt meine Füße besser wird undso. #00:02:21-0#

I: // Was ist da passiert? (...) #00:02:20-5#

B: Habe ungefähr drei Monate gewartet.  #00:02:49-5#

I: Was ist mit deinen Füßen passiert, wenn ich fragen darf? #00:02:24-4#

B: Ja, ich bin so gesessen in LKW, und die ganze Blut war hald hier, so, das war so dick. Dann musste ich ganze Nacht, meine Betreuer und alle, Massagen hald und dass die Blut oben und unter geht. #00:02:37-9#

I: Wie lang bist du so gesessen? #00:02:40-8#

B: Da war so zwei Monate, ein Monat konnte ich nicht gehen, musste ich mit Dings gehen dann. #00:02:44-6#

I: Au. #00:02:44-6#

B: Ja.  #00:02:49-6#

I: Okay (...) ähm konnt (...) konntest du schon Deutsch als du angekommen bist? Also beziehungsweise, wie hat das funktioniert? #00:02:59-7#

B: // Ja ich konnte gar (...) ne ich konnte garkein Deutsch. Nicht Englisch, nix. Ich hab nur irgendwie mit Hände geredet irgendwie #00:03:06-9#

I: // Okay. (...) hm (bejahend) #00:03:06-7#

B: Ja. #00:03:09-0#

I: Es hat irgendwie funktioniert. #00:03:10-2#

B: Irgendwie hat's funktioniert, es war schwer aber (...) konnte nix machen (lacht) #00:03:14-9#

I: Ja. #00:03:15-8#

B: Ja. #00:03:17-2#

I: ähm bist du in Afghanistan zur Schule gegangen? #00:03:22-6#

B: Ich bin zwei Jahre gegangen, ja. Und ein Jahr so Koranschule hald. #00:03:25-9#

I: ähm an ner Koranschule. #00:03:27-9#

B: Ja, ein Jahr. #00:03:28-8#

I: Okay #00:03:28-8#

B: Und zwei Jahre Normale Schule #00:03:30-0#

I: Okay #00:03:30-0#

B: Ja. #00:03:30-0#

I: ähm (...) das (...) wie unterschiedlich sind die Schulen dort zu jetzt ähm, zum Beispiel der Schule in der du jetzt bist? Is des #00:03:46-6#

B: Was meinst du? #00:03:46-6#

I: Ist des zum Beispiel ein anderes System dass du was lernst, oder (...) ähm unterscheidet sich das irgendwie? #00:03:54-7#

B: Achso unterscheidet das was du in normale Schule hald, so zum BEispiel Sozialkunde machst und Mathe und sowas.  #00:03:59-5#

I: hm (bejahend) #00:03:59-5#

B: Und in Koranschule kann nur, liest du das weil dass a, auf arabisch ist. #00:04:04-1#

I: hm (bejahend) #00:04:04-9#

B: Und dann liest du hald nur auf arab, ähm, arabisch. #00:04:05-9#

I: M~hm. #00:04:07-6#

B: Genau, deswegen. Das ist der große Unterschied. Sonst is (...) nicht unter. #00:04:11-9#

I: Okay. #00:04:11-9#

B: Ja. #00:04:11-9#

I: ähm, hattest du, ähm, am Anfang dann irgendwelche Probleme, dass du dich hier zurecht findest? #00:04:21-9#

B: Oh, in Deutschland? #00:04:23-3#

I: Ja. #00:04:23-3#

B: Am Anfang schon Probleme. Keine SPrache, keine Leute, nix. #00:04:28-9#

I: hm (bejahend) #00:04:28-9#

B: Dann, ich war verletzt. #00:04:29-6#

I: Ja. #00:04:30-6#

B: Ja. #00:04:31-5#

I: ähm, hattest, also du hast dann hier vermutlich nen Deutschkurs besucht oder? #00:04:40-4#

B: Ja. Ich bin so, ich hab Sprachkurs besucht vier Monate am Anfang, und danach so, die haben mich zur Mittelschule geschickt. #00:04:47-9#

I: Okay. #00:04:49-2#

B: Genau. #00:04:49-2#

I: ähm (...) hat's dir Probleme bereitet, dass, also in ähm (...) Regensburg wird ja relativ oft auch, ähm oft auch bayrisch geredet. Also ich glaub (...) bei mir hört man's auch nen bisschen raus (...) ähm, hat dir das Probleme bereitet? So dass Bayrisch ja doch so nen bisschen nen anderes Sprachmodell ist wie jetz Hochdeutsch? #00:05:12-1#

B: Schon, schon, manchmal haben wir in dem, im Deutschkurs oder so wenn wir mit dem Lehrer geredet haben, haben wir schon ein bisschen Deutsch gehört #00:05:21-5#

I: hm (bejahend) #00:05:21-5#

B: Undn ich hab auch ein Tag zum Beispiel 'Habadere' gehört, das war mein erste Wort das ich gelernt hab. #00:05:25-3#

I: Habadere? (lacht) #00:05:27-8#

B: Habadere, Genau. Aber ja, sowas. Aber ein bisschen hatten wir, ein bisschen Probleme mit (...) schnell reden und (...) ähm mit bayrisch hald. #00:05:38-4#

I: hm (bejahend) (...) aber jetz funktioniert's oder? #00:05:38-4#

B: Jetz funktioniert's schon, ja, weil ich auch in der Arbeit bin, die reden auch bayrisch. #00:05:41-0#

I: hm (bejahend) #00:05:41-0#

B: Und in die Schule undso es ist kein Problem. #00:05:44-6#

I: ähm was genau für ne Arbeit machstn du jetz? #00:05:46-7#

B: Also ich mache Ausbildung als Fahrzeuglackierer. #00:05:49-5#

I: hm (bejahend) #00:05:49-5#

B: Ja. #00:05:51-3#

I: Cool. ähm, wo hier? #00:05:53-5#

B: ähm, bei (...) hinter bei Handelskammer, bei (unv.)-Straße in der hald. #00:06:00-9#

I: (...) okay, ich weiß ned wo das is. #00:06:00-7#

B: Ja, das is (...) das is der Bus Nummer 10 Richtung Zuckerfabrik und dann die letzte Bushalte Bushaltestelle hald steige ich aus. #00:06:09-4#

I: hm (bejahend) (...) Ah, da hinten. #00:06:10-8#

B: Da hinten, ja. #00:06:11-9#

I: Ja. (...) ähm, hast du (...) jetzt momentan noch Kontakt nach Afghanistan? #00:06:24-5#

B: Ich hab noch mit meine Onkel Kontakt. Sonst habe ich keinen Kontakt #00:06:29-4#

I: Okay #00:06:30-6#

B: Und (...) Freunde hald, sonst #00:06:32-1#

I: hm (bejahend) #00:06:32-1#

B: Nicht viele. #00:06:32-5#

I: ähm und mit wem hast du hier in Regensburg jetz so zu tun? #00:06:36-2#

B: Mit meinem Kumpel, Khsafi, Imdad, und mit denen dass ich mit dene Kicker spiele. Die haben, mit die haben Kontakt. ICh spielen Cricket und (...) ja. #00:06:48-7#

I: Ja, cool #00:06:51-2#

B: Ja. (lacht) #00:06:52-9#

I: ähm (...) du, wie weit ist denn dein Asylverfahren schon? #00:07:01-7#

B: Asylverfahren (...) eigentlich nichts, ich hab Duldung. #00:07:06-6#

I: Hm? Du (...) du hast ne Duldung, okay. #00:07:08-6#

B: Ja, ich haben eine Duldung. #00:07:11-2#

I: Gut, ähm (...) und ändert sich da dann noch irgendwas oder (...)? #00:07:16-5#

B: Ja, ich habe eine Antrag geschrieben von (...) 25a) #00:07:21-9#

I: hm (bejahend) #00:07:22-7#

B: und das läuft noch. Ich weiß nicht mehr wie das geht. #00:07:26-7#

I: Gut. #00:07:26-7#

B: Ja. #00:07:28-4#

I: (...) Ja, mal schaun was da dann raus kommt. #00:07:29-7#

B: Genau, also was da raus kommt, ja. #00:07:31-3#

I: ähm (...) Ihr habt jetz hier in der WG auch immer noch einen Betreuer oder? #00:07:35-7#

B: Ja. #00:07:38-3#

I: Okay. Kommt Ihr mit dem zurecht?  #00:07:48-4#

B: Schon, schon, super. Sie ist nett  #00:07:41-9#

I: //passt. #00:07:41-9#

B: was wir brauchen, SIe ist immer für uns da. #00:07:45-5#

I: (...) Das ist cool #00:07:46-8#

B: Genau. Alles gut. #00:07:49-0#

I: ähm (...) welche Wohnungen hastn du dann seit du hier in Regensburg angekommen bist (...) welche Wohnungen hastn du dann ghabt? Ihr warst erst in ner (...) #00:08:01-1#

B: Ja, erste war ich in (...) ähm (...) O M 1? #00:08:07-8#

I: hm (bejahend) #00:08:08-9#

B: Ja, glaube ich (...) Also erste war ich da. Dann zweite war ich bei deutsche Kinder. #00:08:14-4#

I: hm (bejahend) #00:08:14-9#

B: ach, ähm sieben war deutsche Kinder, sechs und ein war ich dabei. In (unv.) eins, und danach bin ich hierher gezogen. #00:08:23-9#

I: ähm, hat (...) da hab dich vorhin so'n bisschen gehört, dass es da n bisschen Konflikte gegeben hat teilweise. Was is denn da passiert? #00:08:33-6#

B: Ja (...) Eigentlich, bei mir war gut, aber bei Ihm weiß ich nicht #00:08:37-2#

I: Achso #00:08:37-2#

B: Ja, genau, wir waren verschiedene Gruppe. #00:08:38-9#

I: hm (bejahend) #00:08:39-5#

B: Ich wareich war meine in (unv.)mit Deutsche, war alles gut. #00:08:41-0#

I: hm (bejahend) #00:08:41-0#

B: Bin gelaufen. Ich hab mit denen viel geredet, viel gemacht (...)  #00:08:47-9#

I: hm (bejahend) #00:08:47-9#

B: Damit ich hald Deutsch lerne. #00:08:48-3#

I: hm (bejahend) #00:08:48-9#

B: Und jeder Person musste hald pro Woche einmal kochen und einkaufen undso. Das war super. Gut gemacht. #00:08:57-9#

I: Also, ähm ihr hattet da so'n richtiges WG-Leben? So, jeder is mal zum Einkaufen gegangen, und (...) wie viele Leute waren des dann in der WG? #00:09:08-1#

B: Waren wir zu siebt. Sieben Personen waren wir da. #00:09:14-3#

I: Das heißt, jeder war einen Tag für Einkaufen und Kochen zuständig. #00:09:15-5#

B: Genau, ich war zum Beispiel Montag, andere war Dienstag, andere Mittwoch und so weiter.(...) #00:09:22-3#

I: Das (...) hört sich tatsächlich nach Spaß an. #00:09:23-0#

B: Schon, schon. Es war gut. (lacht) #00:09:24-2#

I: (lacht) #00:09:26-5#

B: Hatte Spaß gemacht. #00:09:27-4#

I: Cool. (...) ähm (...) Hast du dann im Jugendcafé auch mal gekocht ode so? #00:09:38-5#

B: Im Jugendcafé hab ich noch nichtgekocht, weil ich nicht kochen kann. (lacht) #00:09:39-3#

I: Nicht? #00:09:41-4#

B: Nein. #00:09:42-8#

I: Hmm (...) Okay, ich schau mal hier weiter. ähm (...) Gut. (...) Hast du (...) hier in (...) also, seit du hier angekommen bist, irgendeine Art von Rassismus erlebt? #00:10:01-3#

B: Ich habe nicht erlebt. #00:10:03-2#

I: Echt? #00:10:03-2#

B: Ja. #00:10:05-6#

I: Cool. (...) Freut mich für meine Stadt (lacht) (...) Also überhaupt nicht, nie? #00:10:13-7#

B: Nein. DIe, viele Leute haben so (...) Demo gemacht undso. Und mich interessiert einfach das nicht. Ich bin wie ich bin und andere Leute mir (...) nur Spaß machen. Demo und (...) keine Ahnung.  #00:10:24-8#

I: Gesunde Einstellung. #00:10:26-1#

B: Ja. Und ich habe auch keine Zeit, dass ich mit denen in Demo gehe undso, so #00:10:30-8#

I: hm (bejahend) #00:10:30-8#

B: Ja, die machen, was die will. (...)  #00:10:34-5#

I: Das heißt, ähm (...) dir gefllt s ganz gut so wie's jetzt grad läuft oder? #00:10:39-0#

B: Ja, in Regensburg gefällts super. Und Leute nett, alles perfekt. Arbeit hab ich, Schule hab ich. Was will ich noch? (...)Ja. #00:10:49-9#

I: ähm (...) was machst du, wenn du mit der Ausbildung fertig bist? #00:10:54-6#

B: ähm, ich bin bei Firma. Wenn ich übernommen bin, dann arbeite ich da zwei, drei Jahre. Und danach (...) weiß ich nicht.  #00:11:03-6#

I: hm (bejahend) #00:11:03-6#

B: Wenn ich gut bin, vielleicht mache ich Meister. #00:11:05-8#

I: hm (bejahend) #00:11:05-8#

B: Oder (...) dann arbeite ich hald da. (lacht) (...) Aber weiß ich nicht (...) sicher. (...) #00:11:14-6#

I: ähm (...) hattest du ähm (...) also ähm, ich jetz mal wieder so n bisschen zurück an Anfang, also nachdem du hier angekommen bist. ähm da (...) bist du zuerst von der Polizei aufgenommen worden und ins Krankenhaus gebracht, dass sich die um deine Beine kümmern. #00:11:42-6#

B: Ja. #00:11:42-6#

I: Und das hat auch alles reibungslos funktioniert? Also (...) #00:11:47-4#

B: Keine Ahnung. Ich wusste damals nicht irgendwas ist funktioniert. #00:11:49-2#

I: hm (bejahend) (...) Okay. #00:11:52-1#

B: Ja ich bin Krankenhaus die haben mich gebracht undso (...) und dann (unv.) hab ich hald vergessen. Schon lange. #00:11:58-3#

I: Ja. ähm und im Nachhinein hast du auch nix mitbekommen, dass (...) dass irgendwie irgendwas aus der Reihe gelaufen wäre? #00:12:10-2#

B: Nein. Dann war ich so Große Heim, da bei Dings. #00:12:14-4#

I: hm (bejahend) #00:12:14-4#

B: Und, die haben mich Don Bosco geschickt. #00:12:18-2#

I: hm (bejahend) #00:12:18-2#

B: Dann war ich eine Woche da. #00:12:18-4#

I: hm (bejahend) #00:12:18-4#

B: Danach war ich in (unv.)heim. War ich in der (unv.) #00:12:21-2#

I: okay. #00:12:23-2#

B: Sonst weiß ich nicht. (lacht) #00:12:23-9#

I: Hmm. (...) Warte (...) Also, ähm worum's grundsätzlich in meiner Arbeit geht ist so (...) ähm, diese Momente ähm, wenn du (...) was machen musst, und dann hast du dir später gedacht, so: 'Ah! Oh, so hätte das funktionieren sollen!' Hast du das auch zum BEispiel in der Arbeit nie gehabt? Wie funktioniert denn des da, so die Zusammenarbeit mit deinen Mitarbeiter. Also, mit den anderen bei dir in der Firma. Wie funktioniert denn des? Also so, von dem Moment an wo du da reingekommen bist. #00:13:13-6#

B: An die, am Anfang war dich (...) Praktikant. #00:13:17-5#

I: hm (bejahend) #00:13:17-5#

B: Ich hab Praktikum gemacht. Und, ja, ich wusste auch nicht, wie du gesagt, (...) wie man zum Beispiel (...) eine Flüchtlinge eine Ausbildungsstelle bekommt oder so. #00:13:27-6#

I: hm (bejahend) #00:13:27-6#

B: Dann aheb ich Praktikum gemacht, und (...) #00:13:29-7#

I: ähm wie hast du denn die (...) Ausbilungsstelle, die Stelle dann bekommen, des Praktikum? #00:13:34-3#

B: Ja, ich war im B1 #00:13:37-6#

I: hm (bejahend) #00:13:38-1#

B: ähm (...) (unv.) hald, in die Schule. #00:13:41-7#

I: hm (bejahend) #00:13:42-0#

B: Und ich hab mit meinem Lehrer zusammen gesucht, wo hald Lackiererei, er sucht so (...)  #00:13:46-8#

I: Ah. #00:13:48-2#

B: Wo Lackiererei ist oder so #00:13:48-9#

I: hm (bejahend) #00:13:48-9#

B: Im Computer, dann haben wir Lackiererei gesucht und wir hald gefunden, telefonniert, und wir kommen so hin #00:13:55-6#

I: hm (bejahend) #00:13:55-6#

B: Mein Chef geredet und dann sind wir hingegangen wegen Praktikum, ich und mein Betreuerin #00:14:01-2#

I: hm (bejahend) #00:14:01-6#

B: Und dann haben wir gefragt, ob ich Praktikum bekomme, aber dass ich am, pro Woche zweimal kommen kann, weil drei Tage habe ich (unv.) und Schule, und zwei Tage kann ich zur Arbeit kommen. #00:14:14-3#

I: hm (bejahend) #00:14:14-8#

B: Mein Chef hat gesagt: 'Ja.' Und dann haben wir zwei Monate Vertrag geschrieben, pro Woche zweimal, und dann habe ich ein Praktikum gemacht und danach haben die gesagt: 'Ja, du kannst schon ab September hald hier Ausbildung machen.' #00:14:26-1#

I: Cool. #00:14:27-1#

B: Ja. #00:14:28-7#

I: (...)ähm wie hat's dann ausgesehen, nachdem du im BEtrieb drin warst? ähm, des (...) du bist quasi reingekommen und des war (...) ma hat dir dann direkt gezeigt, was du machen kannst, und (...) oder wie hat des da dann ausgesehen? #00:14:47-7#

B: Ja, ich bin reingekommen. #00:14:49-8#

I: hm (bejahend) #00:14:49-8#

B: Und ich hatte hald so (...) normale Kleidung an, weiß. #00:14:53-4#

I: hm (bejahend) #00:14:53-4#

B: Wie Doktor hald. #00:14:55-1#

I: hm (bejahend) #00:14:55-6#

B: Und dann die haben zu mir gesagt: 'zuerst tauschst du deine Kleidung.' #00:14:59-6#

I: hm (bejahend) #00:14:59-6#

B: 'Wir haben in Werkstatt anderes Kleidung. Ziehst du die da an.' Und dann habe ich auch gecheckt: 'Oh, so funktioniert!' #00:15:05-5#

I: hm (bejahend) #00:15:05-9#

B: Und dann die haben gesagt hald, da ist Kleber, da ist das, weil ich nicht wusste, wie das heißt und wie das funktioniert. #00:15:13-0#

I: hm (bejahend) #00:15:13-5#

B: Und dann hab (...) hab verstanden, was die wollen in der Arbeit. #00:15:17-2#

I: Also die haben die dann auch quasi (...) direkt die, ähm, deutschen Worte für die Sachen, die du brauchst, dann beigebracht. #00:15:24-9#

B: Genau, die haben beigebracht. #00:15:24-9#

I: Und die, die warn dann quasi alle auch direkt freundlich oder? #00:15:30-5#

B: Die waren am Anfang schon freundlich, und immer noch jetz. #00:15:33-9#

I: Cool #00:15:35-1#

B: Ja. #00:15:36-1#

I: ähm (...) Es freut mich irgendwie, dass das alles so reibungslos funktioniert. (...) hat's ähm (...) aber es (...) hat dann nie irgendwelche Probleme gegeben oder? #00:15:49-5#

B: Am Anfang war schon Probleme, aber ich hab das gemerkt, dass die (...) manche Leute so vorsichtig mit mir war. #00:15:55-2#

I: Inwiefern? #00:15:58-4#

B: Naja, so (...) nicht geredet undso. Weil die nicht wusste, wie ich bin und (...) #00:16:02-8#

I: hm (bejahend) #00:16:04-0#

B: Wie du reingekommen bist undso. #00:16:06-1#

I: Also die warn am Anfang einfach noch skeptisch oder (...)? #00:16:10-5#

B: Ja, so, hald, so, vorsichtig, nicht geredet und so einfach. #00:16:13-8#

I: hm (bejahend) #00:16:14-8#

B: Angst gehabt? Ich weiß nicht, so (...) #00:16:16-2#

I: ähm (...) bei, (...) ähm, bei uns Deutschen, des dauert des manchmal bisschen bis wir auftauen. #00:16:23-3#

B: Genau, davor haben wir in die Schule auch gelernt, dass wenn bei deutsche arbeitet oder irgendwas macht, dann die brauchen länger, dass die dich kennt #00:16:33-0#

I: hm (bejahend) #00:16:33-0#

B: und dir vertraut undso. #00:16:34-0#

I: hm (bejahend) #00:16:34-4#

B: Und das war genau so, dass ich dann Praktikum gemacht habe, und danach habe ich langsam Leute kennen gelernt und geredet undso. #00:16:42-2#

I: hm (bejahend) #00:16:42-2#

B: Jetzt ist, habe ich kein Problem, und dann geht's so. #00:16:44-6#

I: (...) Cool. ähm (...) Hast du, ähm (...) Hast du außerhalb von Regensburg, hast du dir Deutschland auch schon nen bisschen angeschaut? #00:17:00-3#

B: Schon, da war ich eine mal in (...) Hamburg und in München. So, zwei Städte hab ich angeschaut. #00:17:08-5#

I: Was hältst du davon. #00:17:11-9#

B: Ja, Hamburg war super. Viele Leute, schöne Stadt. Und München (...) gefällt mir nicht, weil, das ist irgendwie so (...) komisch. (lacht) #00:17:19-0#

I: Seh ich tatsächlich ganz ähnlich. #00:17:21-0#

B: Ja. #00:17:23-5#

I: Ich weiß (...) ich weiß ned, ähm (...) München muss (...) bis, ähm, vor 20 Jahren, 20 - 30 Jahren was, muss das ne wunderschöne Stadt gewesen sein. (...) ähm, aber ich weiß ned, ich find's mittlerweile irgendwie unsympathisch. Aber is auch nur so, meine persönliche Einschätzung. Keine Ahnung. #00:17:45-6#

B: Ja. Mir ist auch nicht gefallen. #00:17:48-2#

I: hm (bejahend) Aber Hamburg will ich mir umbedingt mal anschaun. #00:17:55-0#

B: Ja, ich war Hamburg, ich war in 2016. #00:17:55-0#

I: hm (bejahend) #00:17:55-0#

B: Und dann war ich Regensburg nach Hamburg mit dem Fahrrad unterwegs. Mit andere Leute.  #00:18:05-6#

I: hm (bejahend) #00:18:02-5#

B: Und so, ja genau. Waren dann da, haben Urlaub gemacht, dann zurück mit Auto gefahren hald. #00:18:07-9#

I: Ihr seid von Regensburg nach Hamburg mim Fahrrad gefahren? #00:18:11-8#

B: Mim Fahrrad gefahren, ja. #00:18:12-9#

I: Waas? Sehr cool. (lacht)  #00:18:16-9#

B: Ja.  #00:18:18-1#

I: Wie seid Ihr da drauf gekommen? #00:18:18-1#

B: Ja, ich war hald (...) ähm mein Betreuerin kennt (...) ein Person und der war in (...)ähm eine Gruppe hald und der war Betreuer und der hat mit andere Jungs auch geredet, dass die auch mitfahren sollen. Und die haben mich gefragt, ob ich auch mitkomme undso. Dann haben hier in Regensburg zuerst trainiert und (...) vorbereitet, danach waren wir hald unterwegs. #00:18:41-5#

I: Wie viele Leute wart Ihr dann? #00:18:44-8#

B: Waren wir so acht Personen waren. (...) #00:18:49-3#

I: ähm, wie lang seid Ihr dann gefahren? #00:18:52-7#

B: Sind wir (...) ungefähr sechs Tage gefahren. #00:18:53-6#

I: Von Regensburg nach Hamburg in sechs Tagen? #00:18:58-1#

B: In sechs Tagen, ja. In der Nacht haben wir geschlafen, und am Tag dann eben Fahrrad gefahren. #00:19:01-5#

I: (...) Ich muss gestehen ich weiß nicht wie viele Kilometer das sind? #00:19:08-0#

B: Das war ungefähr 780 so, Kilometer. #00:19:14-5#

I: 780 Kilometer in sechs Tagen? #00:19:14-5#

B: So ungefähr glaube ich, ja. #00:19:18-7#

I: Das heißt das sind dann (...) pro Tag deutlich über 100 Kilometer. #00:19:22-3#

B: Über 100 Kilometer. 140, 130. So sind wir gefahren. Manchmal auch bisschen, ja, kürzer #00:19:30-5#

I: hm (bejahend) #00:19:30-5#

B: Wenn regnet undso 80 so. Aber ungefähr sind wir 700. #00:19:34-6#

I: Sportlich. #00:19:34-6#

B: Am Anfang schon. (lacht) #00:19:37-5#

I: (lacht) (...) ähm, warte aber des war 2016, da war dann des mit deinen Beinen auch noch garned so lange her oder? #00:19:47-0#

B: Ja meine Beine waren nur wegen Blut, also nicht verletzt oder so. Da ist alles gesund geworden. #00:19:51-7#

I: Achso. #00:19:52-4#

B: Ja. #00:19:52-4#

I: Und dann (...) absolut keine Probleme mehr seitdem. #00:19:55-0#

B: Keine Probleme seitdem. #00:19:58-7#

I: Passt. (...) ähm (...) also die, ahja eins hab ich noch: also in Afghanistan ist ja (...) die Lage relativ angespannt, also auch zur, ähm Polizei jetz denke ich mal. Also ich hab's jetzt von Syrien und im Iran mitbekommen, dass da auf jeden Fall, ähm (...) die Polizei und die Regierung macht quasi was Sie will. Also so hab ich das jetzt mitbekommen. ähm (...) Ich weiß ned, welche Erfahrungen du da gemacht hast? Aber (...) Hat des dann irgendwelche Auswirkungen drauf, wie du zum Beispiel die Polizei, die draußen auf der Straße rumfährt, ähm (...) ob du denen dann vertraust oder, hat des da irgendwelche Auswirkungen drauf? #00:20:57-7#

B: ähm (...) Das habe ich garnicht gemerkt, wie das da sein soll. Keine Ahnung. #00:20:59-8#

I: Echt #00:21:01-0#

B: Ja. #00:21:03-1#

I: Okay. #00:21:03-1#

B: Ich weiß nicht. (...) #00:21:08-2#

I: (...) Ja, gut. Das heißt bei dir ist tatsächlich die Integration ziemlich reibungslos gelaufen oder? #00:21:21-4#

B: Keine Ahnung, da war so jung, ich weiß es nicht. Polizei undso. Hat kein Kontakt, nix. #00:21:25-1#

I: N-ne, ich mein jetz hier in Deutschland. #00:21:27-2#

B: In Afghanistan auch nicht! Keine Ahnung. (...) #00:21:32-9#

I: (...)ähm (...) Aber, du bist (...) ähm Bist du, seit du hier nach Deutschland gekommen bist, irgendwann auf, ähm (...), irgend nen Problem gestopen, was weiß ich, ähm, dass du zum BEispiel zum Arzt musstest und es war kein Dolmetscher da zum BEispiel? Irgendwas in die Richtung. (...) #00:22:18-8#

B: Ja zum Arzt ich bin schon oft gegangen. Und das war kein Dolmetscher #00:22:21-6#

I: hm (bejahend) #00:22:21-6#

B: Ich habe selber geredet. (...) #00:22:26-8#

I: (...) hm (bejahend) Okay. #00:22:29-1#

B: Ja. Von Heim war ich mit im Heim auch, ich bin manchmal mit Jungs und mit die andere Gruppe gegangen #00:22:34-5#

I: hm (bejahend) #00:22:34-8#

B: Betreuer hat dann mich mitgebracht, wegen hald Übersetzen undso. Ich konnte ein bisschen Deutsch und die anderen sind neu gekommen #00:22:41-6#

I: hm (bejahend) #00:22:41-6#

B: Und die konnten auch nicht. (Begrüßung eines Mitbewohners) Ich hab hald denen auch geholfen indso, Sprache undso. #00:22:52-5#

I: Also dann hast du quasi (...) ähm den Job des Dolmetschers übernommen. #00:22:56-2#

B: Ja. Nur zwei, drei Worte hald. (lacht) #00:23:00-0#

I: Ja (lacht) #00:23:00-0#

B: Nicht viel (lacht) #00:23:00-2#

I: Aber immerhin so, dass es irgendwie funktioniert hat. #00:23:03-4#

B: Ja, irgendwie hat's funktioniert(lacht) #00:23:05-6#

I: Ja, cool! (Begrüßung eines Mitbewohners) (...) ähm wie habt ihr denn die Wohnung hier gefunden? #00:23:27-3#

B: Ah, die ist von St. Vinzent. #00:23:29-7#

I: Die gehört zu St. Vinzent #00:23:28-9#

B: Die alle gehören St. Vinzent, ja. #00:23:30-1#

I: Okay. #00:23:30-7#

B: Die haben uns hald her geschickt. #00:23:33-1#

I: ähm (...) Also ihr habt quasi mit den Leuten bei St. Vinzent geredet, ähm dass Ihr ne Wohnung sucht und ob die irgendwas wissen? #00:23:41-6#

B: Ja. #00:23:42-9#

I: Und (...) die meinten dann: 'Jo, wir haben hier die Wohnung, schaut euch die doch mal an'? #00:23:48-9#

B: Also es war so damals, wir, ich wollte auch umziehen, und (...) Die wollten auch umziehen. Und St. Vinzent hat gesagt: 'Wenn jemand keine Aufenthalt hat, und dann darf man nicht umziehen.' #00:24:01-5#

I: hm (bejahend) #00:24:01-5#

B: Und die haben hald uns genommen und in diese Wohnung geschickt. (...) #00:24:07-0#

I: ähm also man darf ohne Aufenthaltserlaubnis nicht umziehen? #00:24:11-4#

B: Nur, nich umziehen und darf man nicht hald Wohnung mieten oder so. #00:24:15-5#

I: hm (bejahend) (...) ähm und die Aufenthaltserlaubnis habt Ihr momentan (...) doch die habt Ihr schon, oder? #00:24:24-0#

B: Nein, wir haben hald so, drei Jahre oder ein Jahr hatten wir nicht. #00:24:30-3#

I: hm (bejahend) #00:24:30-3#

B: So, zum Beispiel weil man kann entweder ein Jahr oder drei Jahre Aufenthalt hat, dann, dann bekommt man keine Wohnungserlaubnis #00:24:36-2#

I: Mmh #00:24:36-2#

B: Und wir haben hald sechs Monate und ich habe selbst , weiß nicht, drei Monate, und ich darf überhaupt nicht umziehen. #00:24:44-5#

I: Und des wird dann quasi immer nach Ablauf wieder verlängert, oder wie läuft das dann? #00:24:50-9#

B: Genau. Wieder anch drei Monate dann gehe ich hin, und muss ich verlängern lassen. #00:24:57-6#

I: hm (bejahend) ähm und die permanente Aufenthaltserlaubnis könntest du kriegen, wenn das rechtliche erledigt ist oder? Also das ist das, wo du meintest, dass das schon initiiert worden is oder? #00:25:13-0#

B: ähm ich hab zweimal Ablehnung und jetz habe ich hald diese Antrag 25a) und das läuft noch.  #00:25:19-2#

I: hm (bejahend) #00:25:19-8#

B: Weiß ich nicht mehr. #00:25:21-0#

I: ähm wie ist denn das mit der Ablehnung damals gelaufen? #00:25:25-3#

B: ähm ich war in Interview im BAMF #00:25:26-6#

I: hm (bejahend) #00:25:27-0#

B: Und die haben hald alles gesagt, nein, abgelehnt. Dann habe ich nochmal geklagt #00:25:31-6#

I: hm (bejahend) #00:25:31-6#

B: Und ich hab nochmal Einladung bekommen, Interview, ich bin zu Gericht gegangen. #00:25:35-4#

I: hm (bejahend) #00:25:36-2#

B: Und die haben nochmal hald abgelehnt. Und danach habe ich gesagt 'jetz mache ich nichtmehr.' #00:25:42-6#

I: Also, und dann hattest du keine Lust mehr oder (...) ? #00:25:49-8#

B: Ja. Keine Lust, und ich konnte nicht warten. #00:25:52-2#

I: hm (bejahend) #00:25:53-0#

B: Und sonst (...) genau. #00:25:56-2#

I: (...) Ähm, magst du mir des noch nen bisschen genauer erzählen, wie des um die Ablehnung (...) ähm wie des rundrum ausgeschaut hat? #00:26:05-2#

B: ähm das weiß ich nicht. #00:26:06-2#

I: Okay. Wann war das denn? #00:26:06-9#

B: Das war in 2017 glaube ich. #00:26:12-7#

I: 2017? #00:26:12-7#

B: Also, nicht sicher, aber sowas war, ja. #00:26:15-7#

I: hm (bejahend) #00:26:15-7#

B: Und Anfang 2018 habe ich Duldung hald, ungefähr ein Jahr und ein paar Monate. #00:26:22-5#

I: Okay. #00:26:23-9#

B: Ja. #00:26:24-8#

I: ähm wie viele Ablehnungen hast du da dann bekommen? #00:26:35-4#

B: Zwei. #00:26:35-4#

I: Zwei Ablehnungen. #00:26:35-4#

B: Ja. #00:26:35-5#

I: ähm (...) Wie hat sich das angefühlt? (...) #00:26:41-3#

B: (...)  #00:26:44-4#

I: (...) Okay. #00:26:44-4#

B: Ja. (...) #00:26:53-2#

I: (...) Ähm was (...) was denkst du, warum (...) ähm entschuldige, dass ich da jetzt drauf rein bohre, aber das passt leider zu meinem Thema. ähm, oder soll ich das lassen? #00:27:15-5#

B: Wie du willst. #00:27:15-5#

I: Okay (...) also, wenn du keine Lust mehr hast dann sag's einfach. #00:27:18-9#

B: Okay. #00:27:20-4#

I: ähm (...) aber, rund um die Ablenung, was denkst du denn, warum's dazu gekommen ist? Und (...) ob du da dran was ändern hättest können. #00:27:37-1#

B: Ich weiß nicht, warum. Ich habe auch nicht genau verstanden. #00:27:40-7#

I: hm (bejahend) #00:27:41-1#

B: Und (...) ich weiß es nicht. Ich habe alles ganz vergessen. Ist schon lange her, seit zwei Jahre undso. Ich weiß nicht warum. Aber irgendwann habe ich alles gelassen. #00:27:56-3#

I: hm (bejahend) #00:27:56-7#

B: Ich konnte da nichtmehr warten, auch nochmal Klagen undso, dann habe ich Dulung. Lieber Duldung, besser. Muss wieder meine Ausbildung machen, muss ich viel lernen. Sprache lernen. Neue Leute kennen lernen. #00:28:08-4#

I: hm (bejahend) #00:28:09-0#

B: Das war mir wichtig, nicht andere. #00:28:10-9#

I: Ja. #00:28:12-0#

B: Ja. #00:28:12-8#

I: ähm und wenn, wenn du jetz dann die (...) ähm Ausbildung erledigt hast, dann könntest du auch ne permanente Aufenthaltsgenehmigung bekommen oder?
 #00:28:23-4#

B: Ich glaube schon, ja. Wenn ich dann sage, dass ich Hauptschulabschluss hab, und  #00:28:29-2#

I: hm (bejahend) #00:28:29-2#

B: Ausbildung geschafft hab und alles #00:28:30-3#

I: hm (bejahend) #00:28:30-8#

B: Dann haeb ich hald Arbeit #00:28:32-8#

I: hm (bejahend) #00:28:33-7#

B: Das Vertrag, dann vielleicht (...) Weiß ich auch nicht sicher. #00:28:38-6#

I: ähm hast du das dann von deinen Betreuern so mitbekommen, dass das so läuft, oder über wen? Oder war das über's BAMF oder (...) ? #00:28:50-8#

B: Wegen Ausbildung meinst du? #00:28:51-1#

I: ähm ja. Also, dass zu dum Beispiel dann(...) erstens, dass es die Möglichkeit zu ner Ausbildung gibt, und wenn du die erledigst und dann ne Schulbildung hinter dir hast, dass du dann gute Karten für ne permanente Aufenthaltserlaubnis hast. #00:29:05-6#

B: Ja das habe ich von viele Leute gehört #00:29:08-1#

I: hm (bejahend) #00:29:08-1#

B: Haben Lehrer, Betreuer, Chef (...) und selber #00:29:15-0#

I: hm (bejahend) #00:29:15-0#

B: ähm und ich habe Leute auch gesehen, dass die hald Ausbildung gemacht haben und danach kamen, die haben auch ein Jahr oder so, hald so Erlaubnis und Karte bekommen. #00:29:24-4#

I: Cool. ähm (...) Hast du auch im, ähm, im Internet irgendwas zum deutschen Asylverfahren gelesen oder so? Oder irgendwo anders? #00:29:36-8#

B: Nein. #00:29:38-0#

I: Nö? Okay. (...) ähm hat's was gegeben, so in der ähm Kultur, also ich denk mal dass Afghanistan und Deutschland (...) sind ja relativ unterschiedliche Kulturen im großen und ganzen würd ich sagen oder? #00:30:06-4#

B: Ja. #00:30:06-4#

I: ähm (...) was sind Unterschiede, die dir aufgefallen sind? #00:30:17-5#

B: (...) Ja, große Unterschied ist zum Beispiel: Weihnachtsfeier, wir bei uns, ist keine Weihnachtsfeier. #00:30:26-4#

I: hm (bejahend) #00:30:26-4#

B: Und bei uns ist zum Beispiel nach Ramadan so kleine Fest undso #00:30:30-8#

I: hm (bejahend) #00:30:31-3#

B: und #00:30:33-4#

I: //ähm wann (...) Entschuldigung, zwischendurch: Wann geht Ramadan jetz los? #00:30:36-3#

B: Ab nächste Woche Montag. #00:30:38-4#

I: Ab nächsten Montag läuft das? #00:30:39-9#

B: Ja. #00:30:39-9#

I: Danke dass du davor jetzt noch zeit gehabt hast. #00:30:41-5#

B: (lacht) #00:30:42-4#

I: (lacht) #00:30:43-0#

B: Und noch nicht so offen unter Ramadan, dann ein, ein Monat (...) und danach haben wir hald drei Tage, zwei Tage, so Fest hald. #00:30:48-2#

I: hm (bejahend) #00:30:48-7#

B: Wir keins (unv.) hatten wir gehabt und hier, mussten wir arbeiten leider. (lacht) #00:30:54-6#

I: hm (bejahend) #00:30:57-0#

B: Und nach diese Fest haben dann zwei Monate, nach zwei Monate, dann haben wir große Fest gehabt. Des die unterschiedliche (unv.). #00:31:04-7#

I: ähm, also du bist ja jetzt schon a paar Jahre hier, hat des bisher (...) immer funktioniert in der Arbeit, dann in der Zeit des Ramadan? Weil, ich mein des muss ja extrem stressig für dich sein. #00:31:18-9#

B: Eigentlich schon, schon funktioniert, ja. #00:31:20-6#

I: Echt? #00:31:20-6#

B: Ja, schon.  #00:31:22-3#

I: Weil ich mein du darfst, ähm, während'm Ramadan unter Tags ja weder essen noch trinken oder? #00:31:28-6#

B: Schon, schon. #00:31:30-6#

I: Schon? #00:31:30-6#

B: Ja, ich kann nicht trinken, nein. #00:31:30-7#

I: Genau, meine ich. ähm (...), und des is aber immer auch so von deinen Kollegen auch akzeptiert worden und (...) ähm das is hald einfach so. #00:31:44-7#

B: Ja, die sagt nichts. Mein Chef sagt: 'Deine Religion und deine is anderes, und meine Arbeit ist was anderes. Egal was du machst, aber ich brauche Arbeit für dich.' #00:31:52-4#

I: hm (bejahend) #00:31:52-4#

B: Genau. Von dir. #00:31:55-5#

I: Passt. #00:31:55-5#

B: Ich arbeite und mache Ramadan, kein Stress. Echt, alles Gut. Genau. #00:32:01-2#

I: ähm, sind dir noch andere Unterschiede aufgefallen? #00:32:05-9#

B: Das ist auch große Unterschiedlich dass ich mache Ramadan und die andere nicht (lacht) #00:32:09-1#

I: Jaa (lacht) (...) Kannst sie ja vielleicht überzeugen, dass Sie auch mitmachen. #00:32:15-6#

B: nein, das kann ich nicht. (lacht) #00:32:17-8#

I: (lacht) #00:32:17-8#

B: Jeder macht was er will. #00:32:21-2#

I: Ja. #00:32:21-2#

B: Ich kann nicht. #00:32:22-1#

I: Ja. ähm, aber abgesehen davon, dass jetz zum Beispiel hier wird ja Weihnachten oder Ostern oder was weiß ich gefeiert, ähm und (...) in Afghanistan dann der Ramadan zum Beispiel, ähm (...) Gibt's andere Unterschiede, die dich (...) sagen wir mal, verwundert haben, als du hier hergekommen bist? (...) #00:32:47-2#

B: Keine Ahnung. Vielleicht gibt es schon am Anfang, aber ich weiß es nichtmehr. Ich bin schon seit fünf Jahre ungefähr da #00:32:53-3#

I: hm (bejahend) #00:32:53-3#

B: Bin jetzt daran gewohnt #00:32:56-1#

I: Ja. #00:32:56-1#

B: Und viele weiß ich nicht, ich weiß nicht. #00:32:58-3#

I: Also es ist jetz (...) nichts quasi hängen geblieben, dass du (...) das dich so sehr verwundert hat, ähm dass es (...) hier komplett anders läuft als in Afghanistan, dass das bis jetzt hängen geblieben wäre oder so. #00:33:18-1#

B: Ne, eigentlich nicht, nein. #00:33:18-5#

I: Okay. (...) warte (...) ähm (...) Also, und die (...) du warst ja insgesamt drei Jahre in der Schule in Afghanistan, ne? #00:33:39-4#

B: Ja. #00:33:40-5#

I: ähm (...) was ist denn da so abgedeckt worden? ähm, waren des dann so (...) Grundsachen, so lesen und schreiben, und rechnen und dann eben die Koranschule? Oder was hast du da damals gemacht? #00:33:59-4#

B: ich weiß es nicht genau. Da war ich glaube wir haben so Mathe und Sozialkunde undso gemacht. #00:34:03-6#

I: Hmh? #00:34:05-1#

B: ähm, in die Schule. Aber das ist schon lange her. Ich weiß nicht. #00:34:09-5#

I: Okay. #00:34:10-8#

B: Ja. Das ist echt lange her. #00:34:11-8#

I: hm (bejahend) #00:34:11-8#

B: (...)Ich mach mal eben ne Pause. /Kurzes Gespräch abeits #00:34:40-3#

I: Also in welcher Bevölkerungsgruppe ist du denn eigentlich in Afghanistan? #00:34:42-0#

B: ähm Ich bin Sunnite hald. #00:34:47-3#

I: Okay, Danke! ähm, dann (...) ürd ich jetz sagen mal, lass ma's für heut, ähm wenn mir noch was einfallen würde, kann ich mich dann wieder bei dir melden? #00:35:01-2#

B: Hm, ja, wenn ich Zeit habe. ABer ich glaube nicht. (lacht) #00:35:03-5#

I: Okay(lacht) Ja gut, jetzt kommt Ramadan. #00:35:05-6#

B: Genau, Ramadan, muss ich viel lernen, alles viel Stress, ja. #00:35:14-2#

I: Also, Danke. #00:35:15-0#

B: Bitteschön.