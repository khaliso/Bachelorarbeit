
\subsection{Bildung}

Bildung:
    IT1.1, min21:01: Anderes Schul/Bildungssystem in Syrien (franz.)
    Bayrisch (IT1)
    Wie bekomme ich eine AUsbildung?
    Wie gehe ich mit Rassismus am Arbeitsplatz um? (IT3, min21. vgl Rassismus)
    Mache ich nach der Ausbildungen den Meister dazu? (IT4, min12)
    Was mache ich, wenn ich außerhalb der Regelzeit einen Ausbildungs/Arbeitsplatz brauche? (IT5, min39)
    Wie lerne ich den Umgang mit einem Computer ohne Zugang zu einem? (IT5, min40ff)(min40:58 eigener              Lösungsansatz) -> Demotivation(leider ich kann nicht, min48) -> nicht bestanden
    
Unter den Interviewten befand sich 

Fünf der Probanden hatten vor Ihrer Flucht nach Europa nur eine rudimentäre Grundbildung erhalten; zwei arbeiteten bereits in den Bereichen, in denen Sie nun tätig sind.\newline

IT1 Hochschulreife, Chemiestudium                                    jetzt Ausbildung P-T-A
IT2 keine, als Bauarbeiter gearbeitet                           jetzt schüler
IT3 brachte sich selbst Lesen und Schreiben bei                 jetzt in der Gastronomie, ausb.
IT4 2 jahre Schule + 1 Jahr Koranschule                         jetzt lackierer, ausb.
IT5 6 Monate Koranschule, hat als Elektriker gearbeitet         jetzt elektriker, Praktikum
IT6 1 Jahr Koranschule, hat als Mechaniker gearbeitet           jetzt Karosseriebauer, wollte was andres                                                                              machen

Der akademiker hatte
Akademischer Abschluss der Universität Aleppo in Deutschland nicht anerkannt; macht jetzt eine Ausbildung zum Pharmazeutisch-Technischen Assistenten. Aufgrund mehrerer Komplikationen, etwa einer Bayrisch sprechenden Lehrkraft (vgl. Sprache) hat IT1 diese Ausbildung abgebrochen und beginnt die selbe Ausbildung erneut.
Allen Teilnehmern wurde die Möglichkeit gegeben, sich in Deutschland zu bilden und wurden von Betreuern und Lehrern in diesem Aspekt normalerweise untertsützt.
IT1 berichtete von Problemen, sich an das deutsche Bildungssystem anzugleichen; Er hatte in Syrien die Schule besucht und später einen akademischen Abschluss erworben. Das Bildungssystem in Syrien hatte sich am französischen Bildungssystem orientiert
    
Participants had no information on what education is
accessible. This particularly concerned the minors in the
population. A young participant said
“I never knew I go to school, I stayed at home for 7
months until a day my mum took me to a meeting
with the council that was when we were told I go to
school”

Although, adult refused asylum seekers are not entitled to
formal education, minors are entitled to formal education
until the end of high school.

The minor participants had no information on how to fulfil
the education curriculum requirements. For instance the
education curriculum requires work experience as part of
high school however immigration laws prevents the asylum
seeker from any form of employment.
“All my friends are on work experience but I can’t
because I am asylum seeker so I stay at home”
This was a major dilemma for the young participants and
they expressed how much they wanted to fully complete the
high school education curriculum.        

\subsubsection{Vorbildung}
\subsubsection{Bildung in Deutschland}

    Mit Sprache verbinden -> viele lernen erst in der Arbeit richtig Deutsch
    
    IT4:
    I: Also die haben die dann auch quasi (...) direkt die, ähm, deutschen Worte für die Sachen, die du brauchst, dann beigebracht. #00:15:24-9#

B: Genau, die haben beigebracht. #00:15:24-9#

\subsubsection{Sprache}
Sprachlich:
    Deutsch (IT1.1: min8. min7:47: kann jetzt gegenargumentieren)
    Bayrisch (IT1.1 min15:43, vgl sozial)(IT4, min6+min16)(IT6, min28)
            Bayrischer Dialekt (IT5, min65f)(IT1, min16f)
    Wie kommuniziere ich am besten mit dem Arzt, ohne dessen Sprache zu sprechen? (IT2, min28, vgl medizinisch)(IT4, min12)
    Wie lerne ich die Landessprache? (IT3, min9:33)
    Wie kommuniziere ich ohne Kenntnis der Landessprache? (IT4, min
    Wie kommuniziere ich mit der Polizei ohne Kenntnis der Landessprache? (IT6, min14)
    
Das Erlernen der deutschen Sprache wurde von allen sechs Teilnehmern als Informationsbedürfnis genannt. Alle sechs gaben an, von den zuständigen Behörden an einen deutschen Sprachkurs verwiesen worden zu sein. IT2 gab sogar an, direkt am Tag nach seiner Ankunft seinen ersten Deutschunterricht gehabt zu haben:
\begin{quote}
    ``[..]zweite Tag, wir haben mit (...)ähm Deutschkurs begonnen.''
\end{quote}
Aus der Zeit nach der Ankunft erinnerten sich die Interviewten an einige Situationen, welche Sie vor große Probleme stellten oder in unangenehme Situationen brachten:\newline
IT1 erinnert sich an ein Gefühl der Machtlosigkeit, anfangs Seine Taten nicht mit Worten begründen zu können.
\begin{quote}
    ``[..]jetzt hab ich schon meine Gründe und warum antworte ich überhaupt nicht. Jetzt kann ich einfach so meine Taten dann begründen. Warum mache ich das? Warum mache ich das?''
\end{quote}
\textbf{kann jetz gegenargumentieren. Erwähnen?}
Der Interviewte IT2 gab etwa an, mit rudimentärsten Deutschkenntnissen ausgestattet zum Arzt gegangen zu sein. 
\begin{quote}
    ``ich bin öfter zum Arzt gelaufen, und wir hatten keine Übersetzer, keine Betreuer. Und der hat, Beispiel wenn ich Kopfschmerzen hatte, der hat nicht verstanden was ich meine. Der hat einfach meinen Fuß angeschaut oder meinen[..]''
\end{quote}
Ein Betreuer ließ Ihn eine kurze Beschreibung seiner Symptomatik auf Deutsch auswenig lernen, welche er beim Arztbesuch wiedergab. 
\begin{quote}
    ``Ne. Und ich hab einfach, wir haben so einen Satz gelernt, wir haben nur gelernt: 'Mir ist schlecht.' ''
\end{quote}
Da dies zu Missverständnissen führte, wurde schließlich mit 'Händen und Füßen' (pantomimisch?) kommuniziert.\newline
\begin{quote}
    ``Dann, der sagt 'wo?' Ich habe nicht verstanden. Der hat einfach mit seinem angefasst, Beispiel hier, hier, hier, dann ich musste einfach zeigen Beispiel hier oder (...) das war echt schwierig.''
\end{quote}
IT4 wurde direkt nach seiner Ankunft in Deutschland in ärztliche Behandlung überführt; es waren keine Dolmetscher anwesend und er zeigte sich der Situation ausgeliefert. Er hat keine weiteren Erinnerungen an die Situation.
\begin{quote}
    ``Ich wusste damals nicht irgendwas ist funktioniert.''
\end{quote}
Er gab weiter an, Neuankömmlingen in ehemaligen Wohngruppen später selbst als Dolmetscher zu Arztbesuchen begleitet zu haben.
\begin{quote}
    ``Betreuer hat dann mich mitgebracht, wegen hald Übersetzen undso. Ich konnte ein bisschen Deutsch und die anderen sind neu gekommen[..]''
\end{quote}
Von IT6 wurde ein Kommunikationsansatz erwähnt, welcher erst seit wenigen Jahren möglich ist:\newline
Er selbst konnte zum Zeitpunkt der Ankunft in Deutschland weder lesen noch schreiben. Sein Begleiter und Er befanden sich auf dem Weg nach Frankfurt, um einen Bekannten, welcher sich bereits länger in Deutschland befand, um Rat zu fragen. Im Zug wurde eine Polizeikontrolle durchgeführt; Sie wurden vor das Problem gestellt, mit deutschen Polizisten auf der Polizeiwache ohne Dolmetscher zu kommunizieren. Diese Informationslücke wurde schließlich mit dem Google Übersetzer überbrückt:
\begin{quote}
    ``[..] War ein Kumpel von mir, der konnte schon persische Schrift und der hat schon auf Deutsch, die Polizisten hat schon auf Deutsch geschrieben und auf Persisch übersetzt und, und uns gezeigt.[..]''
\end{quote}
Die Situation wurde für alle Beteiligten zufriedenstellend aufgelöst; die Geflüchteten wurden registriert und erhielten Getränke und Nahrung.
Ein anderer Faktor stellte viele der Probanden vor unerwartete Probleme:\newline
Auch nachdem Sie sich rudimentäre Kenntnisse der deutschen Sprache angeeignet hatten, stießen Sie auf teils große Probleme im Verständnis des bayrischen Dialektes. Diese Komplikationen rangierten von holpriger Kommunikation mit den Kollegen über die Fehlinterpretation von Anweisungen bis hin zu einem Gefühl der absichtlichen Ausgrenzung durch die einheimische Bevölkerung. \newline
\begin{quote}
    ``An Anfang war ganz schlecht. Ich hab schon so alle, keine Ahnung, Schrauben verkehrt gebracht und die haben schon was anderes gemeint und ich hab schon (...) was anderes gebracht.''
\end{quote}
IT6, 27:02
\begin{quote}
    ``Aber die reden nur bayrisch. Und wenn ich mit denen rede, dann.. Pff ja also wenn ich denen nicht verstehe und Sie mich nicht verstehen, da gibts keinen Kontakt dabei''
\end{quote}
IT1.1, 16:10
\begin{quote}
    ``Ja, am Anfang war schon. Mit Absicht. Ich weiß es nicht, vielleicht es war mein Gefühl ist falsch. Aber ich hab das Gefühl, dass einfach in Ihren Augen war: ' Es ist zu viel für einen Flüchtling, dass er dieses Beruf lernt.''
\end{quote}
IT1.1, 17:51
IT4 erzählt auch, dass 'Habaderce', eine bayrische Begrüßung, sein erstes gelerntes Wort in der neuen Kultur gewesen sei.
IT5 
\begin{quote}
    ``Die haben gesagt: '(unv.)'. Ich habe gesagt: 'Wie bitte?' '(unv.)' ''
\end{quote}
IT5, min 66:40

IT1 berichtete, dass einer der Gründe für den Abbruch Seiner Ausbildung eine Bayrisch sprechende Lehrkraft war. Es fiel Ihm schwer, dem Unterricht zu folgen, kam jedoch der Aufforderung des Lehrkörpers sich bei Unklarheiten zu melden, nicht nach. \textit{Dies tat er aus einem Gefühl der sozialen Verantwortung und der Angst von Ausgrenzung durch seine Mitschüler?}
In der Klasse befanden sich 18 deutsche und 4 ausländische Schüler.

\begin{quote}
    ``B: Ich hab mit Ihr schon gesprochen 'könnten Sie vielleicht das auf Hochdeutsch dann umstellen vielleicht oder so?' - 'Ja, wenn du das Gefühl hast, dass du einfach das nicht verstehst, mach einfach die Hände auf, also, Hände hoch, und dann fragst du mich nach.'[..]Okay, wie lange dauert das? Soll ich jede Minute dann die (...) die (...) mein Hand hoch gehen und dann 'Entschuldigung, ich hab noch eine Frage!' ''
\end{quote}



Notizen:
B: Und dann (...) ja. Die haben schon gefragt und wir haben gesagt: 'Keine Pass und keine Beizeug.' #00:17:32-8#
Bei der Kontrolle im Zug -> Wie wurde das kommuniziert?
    IT3 9:33    gab an, dass der Deutschkurs echt wichtig für Regeln und SPrache war. Hatte 2 Monate verzögerung



