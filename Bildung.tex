\subsection{Bildung}

%IT1 Hochschulreife, Chemiestudium                               jetzt Ausbildung P-T-A
%IT2 keine, als Bauarbeiter gearbeitet                           jetzt schüler
%IT3 brachte sich selbst Lesen und Schreiben bei                 jetzt in der Gastronomie, ausb.
%IT4 2 jahre Schule + 1 Jahr Koranschule                         jetzt lackierer, ausb.
%IT5 6 Monate Koranschule, hat als Elektriker gearbeitet         jetzt elektriker, Praktikum
%IT6 1 Jahr Koranschule (aber illiterate), hat als Mechaniker gearbeitet    jetzt Karosseriebauer, wollte was andres machen

\subsubsection{Vorbildung}
Fünf der Probanden hatten vor Ihrer Flucht nach Europa nur eine rudimentäre Grundbildung erhalten; zwei arbeiteten bereits in den Bereichen, in denen Sie nun tätig sind.\newline
Ein Teilnehmer berichtete von Problemen, sich an das deutsche Bildungssystem anzugleichen; Er hatte in Syrien die Schule und später die Chemiefakultät der Universität Aleppo besucht. Das Bildungssystem in Syrien hatte sich am französischen Bildungssystem orientiert, was anfänglich zu Verwirrung führte.
Er beschloss, die selbe Ausbildung in Deutschland erneut zu beginnen, unter anderem da er nun besser mit dem deutschen System vertraut ist.

\begin{quote}
    ``Und ich hab mich nicht gewöhnt auf den (...) (unv.) System von der Schule. Also die deutsche System ist ganz anders als bei uns''
\end{quote}
\centerline{\textit{IT1.1, min22}}
\subsubsection{Bildung in Deutschland}

Allen Teilnehmern wurde die Möglichkeit gegeben, sich in Deutschland zu bilden. Sie wurden von Betreuern und Lehrern in diesem Aspekt in den abgefragen Punkten unterstützt.
Auch hier traten einige Probleme auf:
Ein Teilnehmer berichtet von einem Gefühl der Machtlosigkeit, als er am Arbeitsplatz mit Rassismus konfrontiert war:
\begin{quote}
    ``(...) Schon, aber danach die meinen: 'Ja, das ist ja Spaß, ne?' Ist voll gemein erstmal, dann wenn die merken, dass es (...) Scheiße war, dann ja 'es war Spaß' undso''
\end{quote}
\centerline{\textit{IT3, min21}}
%Was mache ich bei Rassismus am Arbeitsplatz?
Die Situation ereignete sich im Tagesgeschäft der Ausbildungsstelle, im Gespräch mit den Kollegen am Arbeitsplatz. Der Teilnehmer zeigte sich während des Interviews über die Situation verletzt, thematisierte Sie allerdings gegenüber den Kollegen nicht weiter. Er scheint zum momentanen Zeitpunkt noch keine Antwort auf seine Frage gefunden zu haben, wie Er mit Rassismus am Arbeitsplatz umgehen soll. In diesem Fall waren nur gleichgestellte Kollegen beteiligt.\newline
%Mache ich nach der Ausbildungen den Meister dazu? (IT4, min12)
Auf die Frage nach Plänen für die Zukunft meinte ein Proband, er überlege, sich nach seiner Ausbildung auf die Meisterprüfung vorzubereiten, sollte er eine Festanstellung finden und die Ausbildung zu seiner Zufriedenheit verlaufen.
%Was mache ich, wenn ich außerhalb der Regelzeit einen Ausbildungs/Arbeitsplatz brauche? (IT5, min39)
Ein weiterer Proband erzählte, sich Ende 2018 in einer bedrängenden Lage befunden zu haben:
Er hatte zwei Wochen Praktikum in einem kleinen Unternehmen absolviert und erbat bei seinem Vorgesetzten um Erlaubnis, seine Ausbildung bei diesem Betrieb anzumelden. Dies wurde genehmigt.\newline
\begin{quote}
    ``Also, zwei Wochen war ich in der Arbeit, und ich hab gesagt: 'Ja, machen wir irgendwann den ähm, (...) den (...) ähm Vertrag unterschreiben?' - 'Ist wurscht, ist kein Problem!' (...) Dann ich hab Schule angemeldet, ich war ähm, zwei Tage in die Schule''
    %\caption{IT5, min37}
\end{quote}
\centerline{\textit{IT5, min37}}
Nach zwei Unterrichtstagen in der zuständigen Berufsschule wies er besagten Vorgesetzten darauf hin, dass er aufgrund einer Nierentransplantation 50\% schwerbehindert sei. Daraufhin kündigte der Vorgesetzte das Ausbildungsverhältnis auf.
\begin{quote}
    ``Und, bin ich dort angemeldet und ich war zwei Tage in der Schule, dann bevor ich unterschreiben, dann ich hab gesagt so: 'Hey, ich hab Schwerbehinderungausweis.' Und er hat gesagt: 'Echt?' Und ich hab geagt: 'Ja.' Und er hat gesagt: 'Also leider kann ich nicht ähm, dich nehmen.' ''
   % \caption{IT5, min37}
\end{quote}
\centerline{\textit{IT5, min37}}
Begründte wurde dies mit den potentiellen Risiken für seinen Kleinebtrieb. Der Proband nahm dies teils als Kränkung (-> er ist sportlich, spielt in 3 vereinen!) und teils mit Verwirrung auf.\newline
Da er nun seinen Ausbildungsplatz verloren hatte und die Frist zum Beginn einer Ausbildung bereits verstrichen war, war er mit der Frage konfrontiert:
\begin{quote}
    `` 'Was soll ich machen jetzt?' ''
    %\caption{IT5, min39}
\end{quote}
\centerline{\textit{IT5, min39}}
Zunächst wurde Ihm von den Betreuern seiner Wohngruppe geraten, den Deutschkurs bis zum Beginn des nächsten Ausbildungszyklus erneut zu besuchen. Da er jedoch keinen Sinn darin sah, einen Kurs für ein bereits erlangtes Sprachniveau zu besuchen, lehnte er ab. Auch den Vorschlag, den Zeitraum mit Praktika zu überbrücken, lehnte er mit Hinweis auf monetäre Probleme zurück.\newline
In Zusammenarbeit mit seinen Betreuern und Lehrkräften konnte er schließlich eine Stelle in einem Elektronik - Fachbetrieb finden, bei welchem er zu Beginn des nächsten Ausbildungszyklus seine Lehre beginnen möchte.\newline
%    Wie lerne ich den Umgang mit einem Computer ohne Zugang zu einem? (IT5, min40ff)(min40:58 eigener              Lösungsansatz) -> Demotivation(leider ich kann nicht, min48) -> nicht bestanden
Eine anderer Fall ereignete sich während der Schulzeit des Interviewten, in der Vorbereitungsphase zum Qualifizierten Hauptschulabschluss. Bei der Prüfung wurde unter anderem von den Geprüften gefordert, in 45 Minuten eine Bewerbung am Computer zu schreiben.
Der Interviewte brachte gegenüber den Betreuern seiner Wohngruppe einen Vorschlag:
\begin{quote}
    `` Ich hab gesagt: 'Sie müssen nicht für mich selbst kaufen.' [..] Das ist, Sie kaufen eine von Gruppe (...) wenn ich brauch heute, dann benutze ich. Wenn nicht, dann gebe ich zum Beispiel die andere Person weiter. [..] Hat sie Gesagt: 'Nein.' Nicht Nein, 'Ja, mache ich! Ich schreibe beim Campus Asyl, ich schreibe beim (...) BFZ, ich schreibe beim BSZ.' Das ist die, die, die Leute, die haben zum Beispiel paar Laptop oder Computer, vielleicht kriegen wir geschenkt oder vielleicht nicht. ''
    %\caption{IT5, min42}
\end{quote}
\centerline{\textit{IT5, min42}}
Dieser Vorschlag trug nach geraumer Zeit keine Früchte, was in der Demotivation des Probanden resultierte.
\begin{quote}
    `` Und ich hab gesagt: 'Nein.' Zum Beispiel ich brauche Laptop, ich hab am Anfang des Jahr hab ich gesagt, also, unsere Lehrerin hat gesagt, ähm: 'Ihr braucht Laptop!' ''
\end{quote}
\centerline{\textit{IT5, min45}}
Als die schriftlichen Prüfungen schließlich bevor standen, gibt der Proband an, unter anderem deshalb nicht für die Prüfungen bereit gewesen zu sein, was in einem Nichtbestehen des Qualifizierten Hauptschulabschlusses resultierte.
\begin{quote}
    ``  Und, die haben gesagt: 'Hey!' Meine Lehrer hatten mit mir gestreiten, 'Was ist los mit dir? Also jetzt, Prüfung, musst du anfangen!' Ich habe gesagt: 'Leider, ich kann nicht.' [..] Ich, mit dem, mit dem (...) Kopf bin ich nicht bereit. ''
\end{quote}
\centerline{\textit{IT5, min48}}
%Wie lerne ich die richtigen Fachbegriffe für das Werkzeug? (IT5+6 (falsche Schrauben))
Während der Ausbildung waren die Probanden mit den (deutschen) Fachbegriffen für verschiedene Werkzeuge konfrontiert. Die beiden Probanden, welche schon vor Ihrer Flucht in den respektiven Ausbildungszweigen tätig waren, war etwa zwar das persische, nicht aber das englische oder deutsche Fachvokabular für verschiedene Werkzeuge geläufig. Es gaben mehrere Befragte an, von Seiten Ihrer Kollegen tatkräftig unterstützt worden zu sein. Dies trägt auch zur Verbesserung der Deutschkenntnisse der Geflüchteten bei.
 
\begin{quote}
    ``Und du kannst zum Beispiel, was bedeutet Spitzzange, was bedeutet Zange, und solche Sachen. Weißt du schon. [..]  Aber bei mir, wenn ich weiß nicht. [..] Bei mir werden noch schwieriger. ''
\end{quote}
\centerline{\textit{IT5, min50}}
\begin{quote}
    `` An Anfang war ganz schlecht. Ich hab schon so alle, keine Ahnung, Schrauben verkehrt gebracht und die haben schon was anderes gemeint und ich hab schon (...) was anderes gebracht.(lacht) ''
\end{quote}
\centerline{\textit{IT6, min28}}
\begin{quote}
    ``Also die haben die dann auch quasi (...) direkt die, ähm, deutschen Worte für die Sachen, die du brauchst, dann beigebracht. ''
\end{quote}
\centerline{\textit{IT4, min16}}

Ein afghanischer Proband erzählte, dass er mittlerweile lediglich Probleme bei der Kommunikation mit einem Irakischen Mitarbeiter habe, welcher selbst noch Probleme mit vielen der Fachbegriffe habe.

\subsubsection{Sprache}

Ein Teilnehmer berichtete von mehreren Komplikationen, etwa einer Bayrisch sprechenden Lehrkraft (vgl. Sprache), aufgrund welcher Er seine Ausbildung abgebrochen und neu begonnen habe.


Das Erlernen der deutschen Sprache wurde von allen sechs Teilnehmern als Informationsbedürfnis genannt. Alle sechs gaben an, von den zuständigen Behörden an einen deutschen Sprachkurs verwiesen worden zu sein. IT2 gab sogar an, direkt am Tag nach seiner Ankunft seinen ersten Deutschunterricht gehabt zu haben:
\begin{quote}
    ``[..]zweite Tag, wir haben mit (...)ähm Deutschkurs begonnen.''
\end{quote}
\centerline{\textit{IT2.2, min15}}
Aus der Zeit nach der Ankunft erinnerten sich die Interviewten an einige Situationen, welche Sie vor große Probleme stellten oder in unangenehme Situationen brachten:\newline
IT1 erinnert sich an ein Gefühl der Machtlosigkeit, anfangs Seine Taten nicht mit Worten begründen zu können.
\begin{quote}
    ``[..]jetzt hab ich schon meine Gründe und warum antworte ich überhaupt nicht. Jetzt kann ich einfach so meine Taten dann begründen. Warum mache ich das? Warum mache ich das?''
\end{quote}
\centerline{\textit{IT1.1, min8}}
\textbf{kann jetz gegenargumentieren. Erwähnen?}
Der Interviewte IT2 gab etwa an, mit rudimentärsten Deutschkenntnissen ausgestattet zum Arzt gegangen zu sein. 
\begin{quote}
    ``ich bin öfter zum Arzt gelaufen, und wir hatten keine Übersetzer, keine Betreuer. Und der hat, Beispiel wenn ich Kopfschmerzen hatte, der hat nicht verstanden was ich meine. Der hat einfach meinen Fuß angeschaut oder meinen[..]''
\end{quote}
\centerline{\textit{IT2.2, min28}}
Ein Betreuer ließ Ihn eine kurze Beschreibung seiner Symptomatik auf Deutsch auswenig lernen, welche er beim Arztbesuch wiedergab. 
\begin{quote}
    ``Ne. Und ich hab einfach, wir haben so einen Satz gelernt, wir haben nur gelernt: 'Mir ist schlecht.' ''
\end{quote}
\centerline{\textit{IT2.2, min29}}
Da dies zu Missverständnissen führte, wurde schließlich mit 'Händen und Füßen' (pantomimisch?) kommuniziert.\newline
\begin{quote}
    ``Dann, der sagt 'wo?' Ich habe nicht verstanden. Der hat einfach mit seinem angefasst, Beispiel hier, hier, hier, dann ich musste einfach zeigen Beispiel hier oder (...) das war echt schwierig.''
\end{quote}
\centerline{\textit{IT2.2, min29}}

IT4 wurde direkt nach seiner Ankunft in Deutschland in ärztliche Behandlung überführt; es waren keine Dolmetscher anwesend und er zeigte sich der Situation ausgeliefert. Er hat keine weiteren Erinnerungen an die Situation.
\begin{quote}
    ``Ich wusste damals nicht irgendwas ist funktioniert.''
\end{quote}
\centerline{\textit{IT4, min12}}
Er gab weiter an, Neuankömmlingen in ehemaligen Wohngruppen später selbst als Dolmetscher zu Arztbesuchen begleitet zu haben.
\begin{quote}
    ``Betreuer hat dann mich mitgebracht, wegen hald Übersetzen undso. Ich konnte ein bisschen Deutsch und die anderen sind neu gekommen[..]''
\end{quote}
\centerline{\textit{IT4, min23}}
Von IT6 wurde ein Kommunikationsansatz erwähnt, welcher erst seit wenigen Jahren möglich ist:\newline
Er selbst konnte zum Zeitpunkt der Ankunft in Deutschland weder lesen noch schreiben. Sein Begleiter und Er befanden sich auf dem Weg nach Frankfurt, um einen Bekannten, welcher sich bereits länger in Deutschland befand, um Rat zu fragen. Im Zug wurde eine Polizeikontrolle durchgeführt; Sie wurden vor das Problem gestellt, mit deutschen Polizisten auf der Polizeiwache ohne Dolmetscher zu kommunizieren. Diese Informationslücke wurde schließlich mit dem Google Übersetzer überbrückt:
\begin{quote}
    ``[..] War ein Kumpel von mir, der konnte schon persische Schrift und der hat schon auf Deutsch, die Polizisten hat schon auf Deutsch geschrieben und auf Persisch übersetzt und, und uns gezeigt.[..]''
\end{quote}
\centerline{\textit{IT6, min15}}
Die Situation wurde für alle Beteiligten zufriedenstellend aufgelöst; die Geflüchteten wurden registriert und erhielten Getränke und Nahrung.
Ein anderer Faktor stellte viele der Probanden vor unerwartete Probleme:\newline
Auch nachdem Sie sich rudimentäre Kenntnisse der deutschen Sprache angeeignet hatten, stießen Sie auf teils große Probleme im Verständnis des bayrischen Dialektes. Diese Komplikationen rangierten von holpriger Kommunikation mit den Kollegen über die Fehlinterpretation von Anweisungen bis hin zu einem Gefühl der absichtlichen Ausgrenzung durch die einheimische Bevölkerung. \newline
\begin{quote}
    ``An Anfang war ganz schlecht. Ich hab schon so alle, keine Ahnung, Schrauben verkehrt gebracht und die haben schon was anderes gemeint und ich hab schon (...) was anderes gebracht.''
\end{quote}
\centerline{\textit{IT6, 27:02}}
\begin{quote}
    ``Aber die reden nur bayrisch. Und wenn ich mit denen rede, dann.. Pff ja also wenn ich denen nicht verstehe und Sie mich nicht verstehen, da gibts keinen Kontakt dabei''
\end{quote}
\centerline{\textit{IT1.1, 16:10}}
\begin{quote}
    ``Ja, am Anfang war schon. Mit Absicht. Ich weiß es nicht, vielleicht es war mein Gefühl ist falsch. Aber ich hab das Gefühl, dass einfach in Ihren Augen war: ' Es ist zu viel für einen Flüchtling, dass er dieses Beruf lernt.''
\end{quote}
\centerline{\textit{IT1.1, 17:51}}
IT4 erzählt auch, dass 'Habaderce', eine bayrische Begrüßung, sein erstes gelerntes Wort in der neuen Kultur gewesen sei.
\begin{quote}
    ``Die haben gesagt: '(unv.)'. Ich habe gesagt: 'Wie bitte?' '(unv.)' ''
\end{quote}
\centerline{\textit{IT5, min67}}
IT1 berichtete, dass einer der Gründe für den Abbruch Seiner Ausbildung eine Bayrisch sprechende Lehrkraft war. Es fiel Ihm schwer, dem Unterricht zu folgen, kam jedoch der Aufforderung des Lehrkörpers sich bei Unklarheiten zu melden, nicht nach. Andere Gründe für den Abbruch lagen darin, dass Er Probleme hatte, sozial Anschluss an die Mitschüler zu finden (Mehr dazu in \textit{Soziales Umfeld}). 
In der Klasse befanden sich 18 deutsche und 4 ausländische Schüler.

\begin{quote}
    ``B: Ich hab mit Ihr schon gesprochen 'könnten Sie vielleicht das auf Hochdeutsch dann umstellen vielleicht oder so?' - 'Ja, wenn du das Gefühl hast, dass du einfach das nicht verstehst, mach einfach die Hände auf, also, Hände hoch, und dann fragst du mich nach.'[..]Okay, wie lange dauert das? Soll ich jede Minute dann die (...) die (...) mein Hand hoch gehen und dann 'Entschuldigung, ich hab noch eine Frage!' ''
\end{quote}
\centerline{\textit{IT1.1, min25}}