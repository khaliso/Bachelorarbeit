
\subsection{Bildung}

Bildung:
    IT1.1, min21:01: Anderes Schul/Bildungssystem in Syrien (franz.)
    Bayrisch (IT1)
    Wie bekomme ich eine AUsbildung?
    Wie gehe ich mit Rassismus am Arbeitsplatz um? (IT3, min21. vgl Rassismus)
    Mache ich nach der Ausbildungen den Meister dazu? (IT4, min12)
    Was mache ich, wenn ich außerhalb der Regelzeit einen Ausbildungs/Arbeitsplatz brauche? (IT5, min39)
    Wie lerne ich den Umgang mit einem Computer ohne Zugang zu einem? (IT5, min40ff)(min40:58 eigener              Lösungsansatz) -> Demotivation(leider ich kann nicht, min48) -> nicht bestanden
    
    
Participants had no information on what education is
accessible. This particularly concerned the minors in the
population. A young participant said
“I never knew I go to school, I stayed at home for 7
months until a day my mum took me to a meeting
with the council that was when we were told I go to
school”

Although, adult refused asylum seekers are not entitled to
formal education, minors are entitled to formal education
until the end of high school.

The minor participants had no information on how to fulfil
the education curriculum requirements. For instance the
education curriculum requires work experience as part of
high school however immigration laws prevents the asylum
seeker from any form of employment.
“All my friends are on work experience but I can’t
because I am asylum seeker so I stay at home”
This was a major dilemma for the young participants and
they expressed how much they wanted to fully complete the
high school education curriculum.        

\subsubsection{Vorbildung}
\subsubsection{Bildung in Deutschland}
\subsubsection{Sprache}
Sprachlich:
    Deutsch (IT1.1: min8. min7:47: kann jetzt gegenargumentieren)
    Bayrisch (IT1.1 min15:43, vgl sozial)(IT4, min6+min16)(IT6, min28)
            Bayrischer Dialekt (IT5, min65f)(IT1, min16f)
    Wie kommuniziere ich am besten mit dem Arzt, ohne dessen Sprache zu sprechen? (IT2, min28, vgl medizinisch)(IT4, min12)
    Wie lerne ich die Landessprache? (IT3, min9:33)
    Wie kommuniziere ich ohne Kenntnis der Landessprache? (IT4, min
    Wie kommuniziere ich mit der Polizei ohne Kenntnis der Landessprache? (IT6, min14)
    
Das Erlernen der deutschen Sprache wurde von allen sechs Teilnehmern als Informationsbedürfnis genannt. Alle sechs gaben an, von den zuständigen Behörden an einen deutschen Sprachkurs verwiesen worden zu sein. Ein Teilnehmer gab sogar an, direkt am Tag nach seiner Ankunft seinen ersten Deutschunterricht gehabt zu haben \textbf{such mla raus, welcher}.\newline
Aus der Zeit nach der Ankunft erinnerten sich die Interviewten an einige Situationen, welche Sie vor große Probleme stellten oder in unangenehme Situationen brachten:\newline
Der Interviewte IT2 gab etwa an, mit rudimentärsten Deutschkenntnissen ausgestattet zum Arzt gegangen zu sein. Ein Betreuer ließ Ihn eine kurze Beschreibung seiner Symptomatik auf Deutsch auswenig lernen, welche er beim Arztbesuch wiedergab. \textbf{finden + mehr} Da dies zu Missverständnissen führte, wurde schließlich mit 'Händen und Füßen' (pantomimisch?) kommuniziert.\newline
IT4 wurde direkt nach seiner Ankunft in Deutschland in ärzliche Behandlung überführt; es waren keine Dolmetscher anwesend und er vertraute darauf, \textbf{dass des scho so passen wird}
Er gab weiter an
Von IT6 wurde ein Kommunikationsansatz erwähnt, welcher erst seit wenigen Jahren möglich ist:\newline
Er selbst konnte zum Zeitpunkt der Ankunft in Deutschland weder lesen noch schreiben, sein Begleiter jedoch schon. Beide wurden von der Polizei befragt

wie haben sie's gelernt?



An das Erlernen der deutschen Sprache waren jedoch multiple Probleme geknüpft:
Die Interviewten ließen jedoch verlauten, dass Sie teils große Probleme mit dem in Regensburg üblichen deutsche Dialekt hatten


Language
English language classes were a major difficulty on arrival,
especially for those from countries where English is not
their first language. Participants had no information on how
to get into language classes. It appeared there were many
English language classes available however getting into an
English class proved difficult. All participant lamented
about how long it took to get into a class. A recently arrived
asylum seeker broke down in tears during the interview as
she discussed her ordeal with language.

Participants had no information on the boundaries of the
roles of service providers during integration. Half the
research population got into language classes by themselves
after waiting for the service providers for a period of time.


