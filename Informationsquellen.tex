\subsection{Informationsquellen}
		Friends
		Internet
		Interpreters
		Caseworkers

Von den Interviewten wurde überwiegen das soziale Umfeld zur Informationsgewinnung verwendet

\cite{mykyttschak2018}



Information sources are also outcomes of information
behaviour studies. The identified information sources
during refused asylum include:
Friends
Friends were the most popular source of information
consistent with previous research (Lingel, 2011), (Silvio,
2006), (Courtright, 2005) (Fisher et al., 2004). A participant
said her friend from the English class told her she could go
to a computer school. Another participant heard about
volunteering from friends. For another it was a friend that
told him to go to service providers for transport support to
fulfil an identity request by the Home Office.
“We did not know these things existed but when you
talk to people we get information about what is for
us”
Friends included inmates at the detention center. A
participant currently in their 11 th year as a refused asylum
seeker said:
“All through this times I look for friends to talk to,
some who are in the same situation or someone who
can speak my language”
Internet
The Internet was a common source of information in this
study population.
‘I went to the library close to the accommodation
and used the computers to get information about
schooling’
Only 3 participants did not use the Internet. This is as a
result of their inability to speak English.
Interpreters
Interpreters emerged as a key source of information for the
non-native English speakers in the study population.
“I trust to go with him go the lawyer because he can
speak my language”
This applied especially for getting a lawyer and local travel.
Caseworkers
A caseworker is assigned by service providers to facilitate
the provision of services to refugees and asylum seekers.
“My caseworker gave me advice, she gave me
everything in a paper and showed me everything and
where I can go to college”
These sources of information are not completely surprising
considering the participant were dealing with asylum
refusal. Friends such as people in similar situations will be
more comfortable sources of information. As one of them
accurately compared
“The condition is really different between asylum
seeker and refugee, if you come as a refugee lots of
people help you there is benefit and support from
service providers but who come as asylum seeker no
one help them they start from zero”
Furthermore, the information sources could be implied to
emphasize the role of social capital and its types in the
integration of refugees explicated in Oduntan (2017).
Bonding social capital of close family and friends in this
case just friends. Bridging social capital of loose friends
and workmates as seen in participants that got information
from inmates at the prison. Linking social capital of people
out further as in the participants that got information from
case workers and interpreters.