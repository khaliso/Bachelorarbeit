\subsection{Informationsquellen}

Mykyttschak identifizierte 2018 personelle und digitale Informationsquellen in minderjährigen Geflüchteten in Deutschland.\newline
Die personellen Quellen waren in
\begin{enumerate}
    \item als vertrauenswürdig eingeschätzte Personen, beispielsweise Betreuer oder Familie
    \item Personen in einer ähnlichen Lebenslage, beispielsweise Freunde, Mitbewohner oder Landsleute
    \item Experten, beispielsweise Lehrer oder Ärzte
\end{enumerate}
aufgeteilt.\newline
Als digitale Informationsquellen (Internet) wurden
\begin{enumerate}
    \item Suchmaschinen (Google, ..)
    \item Google Translate
    \item soziale Medien (Facebook, ..)
    \item Wikipedia
    \item Andere Applikationen
\end{enumerate}
Die persönliche Kommunikation wurde in dieser Studie eindeutig präferiert.
In dieser Arbeit lag kein Fokus auf diesen Aspekten, deshalb wurden Sie nur am Rand befragt. Auch hier konnte festgestellt werden, dass Informationen bevorzugt persönlich ausgetauscht wurden.

\subsubsection{Vertraute}

Der Kontakt zur Familie in dieser Arbeit diente überwiegend der Vergewisserung, dass diese nach wie vor wohlauf seien. Mehrere Befragte gaben an, zeitweise monatelang den Kontakt zu ihren Familien verloren zu haben, was sie teils schwer bedrückte.\newline
\begin{quote}
    ``Der Krieg war so dort, und das können wir einfach (...) die sind einfach verschwunden vor ungefähr sechs Monaten da hatten wir kontaktlos mit denen. Aber danach haben wir hald dann wieder Kontakt mit ihnen''
\end{quote}
\centerline{\textit{IT1.1 min11}}
Ein Interviewter gab an, dass der telefonische Kontakt zu seiner Mutter in Afghanistan sich sehr schwer gestalte, da dieser sie potentiell gefährden könne.
\begin{quote}
     ``[..]Mama hat gesagt, wenn ich dich rufen an, dann die kontrollieren grad die ganzen Telefonen. Wenn die Leute telefonieren, dann die kontrollieren, wissen wie viel IS im Iran sind.'' 
\end{quote}
\centerline{\textit{IT2.2 min18}}
 Aufgrund der schwierigen sicherheitspolitischen Lage in seinem Heimatland werden nach seiner Information viele Telefonate abgehört; Da seine Mutter nicht über die iranische Staatsbürgerschaft verfüge, könne ein Anruf ihren Aufenthaltsort an die zuständigen staatlichen Behören verraten.\newline
 Ein anderer Interviewter in einer ähnlichen Situation teilte jedoch seine Bedenken nicht.
\begin{quote}
    ``In die Iran zu meine Familie? Ja, klar. Wir reden schon telefonisch. ''
\end{quote}
\centerline{\textit{IT6 min59}}
Alle Interviewten, die in Deutschland in einer sozialen Wohngruppe leben oder lebten gaben an, dass Ihre Betreuer für Sie eine wichtige Informationsquelle gewesen seien.
Wohnungen wurden im Normalfall in Kollaboration mit den Bereuern gefunden. Als ein Proband plötzlich seinen Ausbildungsplatz verloren hatte, wandte er sich an seine Betreuer und Lehrer.
\begin{quote}
    ``B: Dann ich hab gesagt: 'Was soll ich machen jetzt?'[..] ähm, dann, ich war so Zuhause und mein, ähm, ich hab (...) mit Frau (unv.), die (...) Ich war bei Frau (unv.) in Wohngruppe. Oben. Beim, ähm, betreutes Wohnen.
\end{quote}
\centerline{\textit{IT5 min39}}

\subsubsection{Freunde}

Komplexere soziale Informationsbedürfnisse, etwa das Finden einer passenden sportlichen Aktivität in der Umgebung, kann auch durch eine Kollaboration gelöst werden. Ein Proband beschäftigte sich damit, dass seine Mitgliedschaft in einem örtlichem Fitnessstudio  ihm keine Freude mehr bereitete und seine finanziellen Mittel überbeanspruchte. Daraufhin entdeckte er zunächst über einen Freund den Boxsport.
\begin{quote}
    ``Ja (lacht). Und, ja, dann Kumpel von mir war Boxer, und der hat gesagt: ´'Ja, komm mit! Gehen wir (...) Boxen. Ich nehme mit. Und es ist besser als Fitnessstudio.' Und dann ich hab gesagt: 'Ja, ok.' Dann war ich schon bei einem Boxverein' ''
\end{quote}
\centerline{\textit{IT6 min36}} 
Die Betreuer im EJSA - Jugendcafé erklärten sich später bereit, einen Raum zum Trainieren zur Verfügung zu stellen.

\subsubsection{Internet}

Das Internet als Informationsquelle wurde in dieser Arbeit nur am Rande angesprochen.  Ein Proband erzälte, dass der Google Übersetzer von einem Freund, der zu diesem Zeitpunkt Lesen und Schreiben konnte, und einem deutschen Polizisten zur Kommunikation verwendet wurde.
\begin{quote}
    ``[..]War ein Kumpel von mir, der konnte schon persische Schrift und der hat schon auf Deutsch, die Polizisten hat schon auf Deutsch geschrieben und auf Persisch übersetzt und, und uns gezeigt.''
\end{quote} 
\centerline{\textit{IT6 min15}}
Der selbe Interviewte erzählte, dass er, nachdem er in Deutschland Lesen und Schreiben gelernt hatte, über das Internet seinen heutigen Ausbildungsplatz gefunden hatte.
\begin{quote}
    ``Ja. Und dann habe ich schon im Internet gesehen, dass die Firma Reisinger suchen eine Lehrling, und ich hab mich (...) Telefonnummer rausgeholt und angerufen und Termin ausgemacht, Praktikum.''
\end{quote}
\centerline{\textit{IT6, min8}}