\subsection{Informationsquellen}

\subsubsection{Lehrer und Betreuer}
Alle interviewten, die in Deutschland in einer sozialen Wohngruppe leben oder lebten gaben an, dass Ihre Betreuer für Sie eine wichtige Informationsquelle gewesen seien.

Betreuer kümmerten sich auch um die Terminplanung

Wohnungen wurden im Normalfall in Kollaboration mit den Bereuern gefunden. Als ein Proband plötzlich seinen Ausbildungsplatz verloren hatte, wandte er sic

Dann ich hab gesagt: 'Was soll ich machen jetzt?'[..]

I: Wo bist du dann hin? #00:38:05-7#

B: ähm, dann, ich war so Zuhause und mein, ähm, ich hab (...) mit Frau (unv.), die (...) Ich war bei Frau (unv.) in Wohngruppe. Oben. Beim, ähm, betreutes Wohnen. h an L um Rat zu fragen.


\subsubsection{Freunde}

Der Freundeskreis war für die Interviewten in vielen Situationen die beanspruchte Informationsquelle. Etwa zum Finden einer passenden sportlichen Aktivität in der Umgebung:
\begin{quote}
    ``Ja (lacht). Und, ja, dann Kumpel von mir war Boxer, und der hat gesagt: ´'Ja, komm mit! Gehen wir (...) Boxen. Ich nehme mit. Und es ist besser als Fitnessstudio.' Und dann ich hab gesagt: 'Ja, ok.' Dann war ich schon bei einem Boxverein' ''
\end{quote}
 #00:35:53-8#


\subsubsubsection{Internet}
Google und Google Translate


Ja. Und dann habe ich schon im Internet gesehen, dass die Firma Reisinger suchen eine Lehrling, und ich hab mich (...) Telefonnummer rausgeholt und angerufen und Termin ausgemacht, Praktikum. IT6, min8

Bei den
IT1 steht noch im Kontakt zu seinen beiden Brüdern, die sich auch hier in Deutschland befinden. Ansonsten informiert Er sich über das Internet oder seine sozialen Kontakte, welche sowohl aus älteren Einheimischen als auch aus Migranten bestehen.
Von einem dieser Einheimischen Kontakte, seinem Nachbarn, hatte Er einen Job als Türsteher beschafft bekommen.

\subsubsection{Familie}
Von den Interviewten wurde überwiegen das soziale Umfeld als Informationsquelle verwendet.

Familie zu Hause (Verbindung über Telefon oder Internet. Telefon risky?) Einige der Interviewten, die angaben, noch im Kontakt zu Ihrer Familie zu Hause zu stehen gaben an, dass es gelegentlich zu monatelangen Kontaktabbrüchen komme. 
\begin{quote}
    ``Der Krieg war so dort, und das können wir einfach (...) die sind einfach verschwunden vor ungefähr sechs Monaten da hatten wir kontaktlos mit denen. Aber danach haben wir hald dann wieder Kontakt mit ihnen'' #00:10:27-4# IT1.1
\end{quote}
Diese stellen nach Ihrer Angabe eine massive psychische Belastung für Sie dar; IT2 gab an, sich auf die Suche nach seiner Mutter in Afghanistan zu machen, sobald er den deutschen Reisepass erlangt habe.

 also, die Mama hat gesagt, wenn ich dich rufen an, dann die kontrollieren grad die ganzen Telefonen. Wenn die Leute telefonnieren, dann die kontrollieren, wissen wie viel IS im Iran sind.   #00:17:32-9#
IT2.2

\cite{mykyttschak2018}