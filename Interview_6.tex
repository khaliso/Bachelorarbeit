\section{Interview 6}

 #00:00:02-8#

I: Ok. (...) Dann läuft des jetz. Erstmal danke, dass du heute Zeit hast! #00:00:09-0#

B: Danke. (lacht) Gerne #00:00:11-1#

I: ähm(...) Dann würd ich jetz erstmal die, ähm(...) allgemeinen Sachen machen. Also: Du bist ganz offensichtlich männlich. #00:00:21-2#

B: hm (bejahend) #00:00:22-4#

I: ähm(...) wie alt bist du denn? #00:00:27-2#

B: Ja ich bin schon 21. #00:00:30-5#

I: Du bist 21? #00:00:30-5#

B: Ja.(...) #00:00:34-1#

I: (lacht)Ich bin 24. #00:00:34-7#

B: Ach du bist 24? #00:00:37-1#

I: Ja, ja. #00:00:37-1#

B: Naja, schaust du nicht (unv.). #00:00:41-7#

I: Hab einein Jungbrunnen. (lacht) #00:00:44-4#

B: (lacht) Ok. Ja. #00:00:49-1#

I: ähm(...) #00:00:50-8#

B: Du hast keine Bart, aber ich habe schon. (lacht) #00:00:52-6#

I: (lacht)Der wächst bei mir ned so wirklich, der will einfach ned. (lacht) #00:00:55-6#

B: Ja ich weiß schon, bei manche ist schon, ja. #00:00:58-4#

I: ähm(...) du bist (...) Aus welchem Land bist du denn? #00:01:03-4#

B: ähm, aus Afghanistan. #00:01:05-8#

I: (...), auch in Afghanistan geboren. #00:01:08-3#

B: Ja. Ich bin schon in Afghanistan geboren, aber (...) aufgewachsen im Iran. #00:01:17-5#

I: De(...) des haben sehr viele genau so erlebt, oder? #00:01:19-5#

B: Genau. Als Kind hatte ich schon (...) schwieriger. Zuhause, Probleme. Ja, ich, und meine Eltern. Ja, das ist wahr. Das hatten meine Eltern gehabt, und (...) als Kind (...) konnte ich nicht (...) dort leben. Deswegen ich bin mit acht Jahren auf den (unv.).(...) #00:01:44-4#

I: (...)ähm(...), also, grundsätzlich noch: Du bist Moslem, oder? #00:01:49-9#

B: Ich bin Moslem, ja. #00:01:49-3#

I: ähm(...) welcher, ähm, Bevölkerungsgruppe gehörst du denn an? Also, bist du Schiite, Sunnite(...)? #00:01:57-2#

B: Nein, ich bin Sunnite. #00:01:59-2#

I: Du bist Sunnite? #00:01:59-2#

B: hm (bejahend) #00:02:01-0#

I: ähm(...) seit wann bist du denn in Deutschland? #00:02:10-0#

B: (...)2015 war(...)ähm(...) weiß es nichtmehr, welcher Monat. Aber 2015 war. #00:02:20-3#

I: 2015. #00:02:23-1#

B: hm (bejahend) 2015. #00:02:24-1#

I: ähm(...)und(...) dann noch des letzte zu dem Allgemeinen jetz: Was hast du denn für nen Asylstatus? Also (...)ähm, hast du ne Aufenthaltsgenehmigung, oder(...)? #00:02:40-8#

B: Nein. Ich habe keine Aufenthaltsgenehmigung. Hier normal, Dultung. #00:02:46-8#

I: Ne Duldung. #00:02:47-2#

B: Eine Duldung. #00:02:47-2#

I: Also (...) das, dass alle paar Monate wieder erneuert werden muss, oder? #00:02:55-7#

B: Früher war schon. Weil ich hab (...) ich habe schon einmal Interview gehabt, und ich das is schon (...) abgelehnt. #00:03:01-3#

I: hm (bejahend) #00:03:01-3#

B: Von der BAMF. Und dann (...) ich hab wieder zweimal geklagt, und (...) 18 Monate hat gedauert. Keine Antwort bekommen.  #00:03:15-5#

I: Was? #00:03:15-5#

B: Ja. 18 Monat. Fast 18, ja, 18 Monat hat gedauert; keine Antwort bekommen von der BAMF. Musste ich, ja, einfach warten, warte. Ich wusste nicht, bis wann? Oder (...) wie lange noch? Dann (...) es gibt schon Paragraph, von (...) hier in Deutschland. Wenn man in Deutschland vier Jahre ist, und (...) ja. Schule besucht. Das und, ja. Gute (...) Kenntnisse hat. Dann darf man schon hier bleiben, diese Paragraph beantragen. #00:03:51-8#

I: Das ist(...) #00:03:51-8#

B: //Dann habe ich schon diese Paragraph beantragt und den, und (...) und BAMF hab ich schon, ja. Dann habe ich (unv.) gesagt, das brauche ich nicht. Hab gesagt. (...) #00:04:05-1#

I: hm (bejahend) #00:04:05-1#

B: Ja. #00:04:07-4#

I: ähm das ist dieser Paragraph 25a) oder? #00:04:12-4#

B: Paragraph 25a), genau. #00:04:13-1#

I: hm (bejahend) #00:04:14-4#

B: Ja. #00:04:16-0#

I: ähm, was sin ddann jetzt deine nächsten Schritte so?  #00:04:24-0#

B: Meine nächsten Schritt (...) Meine nächsten Schritt (...) dass ich meinen Ausbildung schaffe. (lacht) Und (...) ja, danns chaun ma mal (lacht). #00:04:32-2#

I: ähm, wie lang machst'n die jetzad schon? #00:04:36-5#

B: Jetzt fast fünf Monat. Seit fünf Monat#00:04:37-7#

I: Seit fünf Monaten? #00:04:38-8#

B: hm (bejahend) #00:04:39-9#

I: Du machst Jackierer gell? #00:04:41-3#

B: Nein, Karosseriebau.  #00:04:44-3#

I: (unv.) war Lackierer! #00:04:44-2#

B: (unv.) macht Lackierer, genau.  #00:04:45-8#

I: Ah, ja. Okay. #00:04:48-3#

B: Genau. #00:04:48-3#

I: ähm (...) gefällt's dir? #00:04:55-4#

B: Ganz gut. #00:04:55-4#

I: Ok. #00:04:55-4#

B: Ja. #00:04:56-9#

I: ähm wie hast'n die Ausbildung bekommen wenn ich fragen darf? #00:05:01-8#

B: Mh, den Ausbildung (...) ich wollten eigentlich (...) Mh, nicht Karosseriebauer werden. Und, ich wollte was anderes machen weil, ich hab schon im Iran (...) vier bis (...) vier Jahre gearbeitet als Mechaniker. #00:05:17-2#

I: hm (bejahend) #00:05:18-3#

B: Und ich wollte was anderes (...) anders machen. #00:05:19-5#

I: hm (bejahend) #00:05:19-5#

B: Und (...) nicht Hände schmutzig machen. (lacht) #00:05:23-5#

I: (lacht) #00:05:23-5#

B: Ja, dann habe ich schon alles probiert, und nicht geklappt. #00:05:27-8#

I: Mh, was hast'n alles ausprobiert so? #00:05:28-9#

B: Ja, ja. Ich hab Zahntechniker. Das habe ich probiert. Das hat mir gut gefallen, aber (...) es gibt schon weniger (...) Labor in Regensburg, und die brauchen, die meißten brauchen keine Ausländer #00:05:45-6#

I: Hm? #00:05:46-8#

B: Ja.(lacht) Ja, ist so, ja. Die meißten brauchen keine Ausländer. Die akzeptieren keine Ausländer. Die einer (...) in Doktor Gessler Straße, ich weiß es nicht ob du kennst oder nicht. Da oben beim Königswiesen fast in der Nähe #00:06:01-0#

I: (...)ähm, ne ich glaub ned. #00:06:04-8#

B: Ja. Dort habe ich schon zwei Wochen Praktikum gemacht, und (...) ja, da habe ich schon eine Stelle gefunden, und (...) ja. Und (...) dann (...) ich habe so  zu spät meine Bewerbung abgegeben und die haben eine andere genommen, und (...) der Junge hat schon zwei Monate, drei Monat gearbeitet, und einfach weg(...) gehaut. #00:06:30-7#

I: Mh? #00:06:33-0#

B: Ja. #00:06:34-1#

I: Wie? #00:06:35-5#

B: Der hat nicht weiter gemacht. Der hat einfach (...) #00:06:38-7#

I: Der, der die Stelle anstatt dir bekommen hat? #00:06:39-1#

B: Genau, ja genau. Ja, der hat schon zwei, drei Monaten gearbeiten und hat einfach (...) weggehabt. Ja, und (...) dann, wir haben schon wieder versucht anzurufen, und die haben schon gesagt: Nein. Wir brauchen jetzt keine Aus(...) keine Ausländer, weil wir können nicht vertrauen. Die, die sind nicht so fleißig, und (...) #00:07:03-4#

I: Scheiße #00:07:03-4#

B: Ja, deswegen wir haben schon gesagt: 'Ja, es (...) es kommt (...) ein Mensch hald, wie du in der Arbeit bist undso.' #00:07:14-1#

I: hm (bejahend) #00:07:14-1#

B: Immer nicht gleich. Jeder ist nicht gleich undso. Aber trotdem die haben gesagt: 'Nein. Wir werden Ihn jetzt nicht.' Ja. Und dann habe ich schon im Internet gesehen, dass die Firma Reisinger suchen eine Lehrling, und ich hab mich (...) Telefonnummer rausgeholt und angerufen und Termin ausgemacht, Praktikum. #00:07:40-0#

I: hm (bejahend) #00:07:40-0#

B: Drei Tage habe ich schon Praktikum gemacht, und erster Tag habe ich schon (unv.) (...) Stelle bekommen. (lacht) #00:07:47-3#

I: Cool! #00:07:48-2#

B: Genau, ja. Weil ich konnte mich schon mit dem Auto undso klarkommen undso. Und deswegen die haben schon gesagt: 'Ja, wir nehmen dich.' Obwohl es schon so spät war und (...) vier Monaten zu spät war. Die haben gesagt: 'Gerne!' #00:08:04-1#

I: Gut, da hast du schon (...)ähm vier Jahre im, im Iran gearbeitet. Des kannst du dann. #00:08:10-4#

B: Genau. Ja. Ja. Ja, im Iran ist so, kann man schon arbeiten, aber (...) Schule und (...) lernt man nur einfach. Geht man nicht in die Schule. Lernt man nur  in der Werkstatt. #00:08:23-4#

I: hm (bejahend) #00:08:23-4#

B: Besucht man keine Schule. Es ist so. #00:08:26-7#

I: ähm(...), hast du dann im, ähm im Iran oder in Afghanistan grundsätzlich ne Schule besucht? #00:08:33-8#

B: Nein. #00:08:35-0#

I: Ok. Also, erst hier in Deutschland dann. #00:08:38-8#

B: Es (...) ja, habe ich schon angefangen, ja. Null. (lacht) #00:08:41-6#

I: Du hast komplett von Null angefangen? #00:08:41-6#

B: Komplett, ja. #00:08:45-4#

I: Das muss echt hart gewesen sein oder? #00:08:46-0#

B: War schon, ja, sehr hart. Und, ich war schon (...) im Deutschkurs, und (...) wenn, im Lernwerkstatt hab ich schon drei, drei? (...) Ja, drei Monate habe ich schon Deutschkurs besucht. Und, erster Tag bis (...) erster Monat war sehr, sehr, sehr, sehr schwierig für mich. #00:09:09-1#

I: hm (bejahend) #00:09:09-6#

B: Weil, ich konnte garnix. Ich konnte keine Buchstaben, einfach. Und, ich konnte auch nicht schreiben (...) und lesen. #00:09:17-1#

I: hm (bejahend) #00:09:17-1#

B: Das war sehr, sehr, sehr hart, und ich hab mal auch gewein, und (lacht), ja, die anderen hat schon geschrieben und ich hab, ich hab gesagt: 'Hä, wieso, wieso kann ich nicht?(...) Und, wieso kann ich nicht, wenn (...) wieso kann ich nicht schreiben?' Und dann auf einmal hab ich schon Gas gegeben. (...) Schreiben gelernt. Sprache gelernt. (...) Ja. #00:09:42-2#

I: Also jetz (...) also, du sprichst auf jeden Fall gut Deutsch. #00:09:45-8#

B: Ja, jetzt geht schon. #00:09:49-1#

I: ähm, hast du dann auch (...) die persische Schrift hier gelernt? #00:09:53-9#

B: Nein, auch nicht. Wenn man nicht in die Schule geht, dann, wie kann man schon in Persisch lesen? #00:09:59-0#

I: hm (bejahend) ähm, ich hab's nur mitbekommen von a paar andren aus'm Jugendcafé, die eben dann hier in Deutschland Persisch, also die Schrift gelernt haben dazu. #00:10:10-2#

B: Ahso? Ok. Ne, habe ich nicht. #00:10:16-5#

I: hm (bejahend) ähm, über welche Fluchtroute bist du denn nach Deutschland gekommen wenn ich fragen darf? #00:10:26-9#

B: Ja, war sehr schwierig. (lacht) (...) Iran habe ich , bin nach Türkei gekommen und (...) Türkei waren wir schon so (...) keine Ahnung was, sechs, sieben Leute. Ja, sowas (...) ähm, mit eine LKW. Ja, dann (...) Österreich, Österreich, Deutschland. #00:10:55-5#

I: Also du bist in der Türkei in nen LKW gestiegen #00:10:56-0#

B: hm (bejahend) #00:10:56-0#

I: und in Österreich wieder raus. #00:10:57-1#

B: Genau. (...)  #00:11:02-8#

I: Das hat länger gedauert oder? #00:11:04-0#

B: WAr schon länger, keine Ahnung. Für uns war alles Nacht. (lacht) #00:11:09-3#

I: Ohh. #00:11:12-4#

B: War auch alles Nacht, ja. (...) #00:11:15-3#

I: Ow. #00:11:17-7#

B: hm (bejahend) (...) Trinken haben wir gehabt, aber Essen (...)? #00:11:21-6#

I: Es gab nichts zu Essen? #00:11:23-1#

B: hm (verneinend) #00:11:23-1#

I: (...) ähm (...) und in Österreich seid Ihr dann wieder raus gekommen? #00:11:36-2#

B: Österreich, die haben uns raus geschmissen. #00:11:39-0#

I: WIe, einfahc die Türen raus, und (...) alle raus? #00:11:42-3#

B: Mh, genau. Ja, den Fahrer hat uns raus geschmissen. Tür raus, ausgemacht. Und hat gesagt: 'Geht (...) raus.' Und wir waren schon auf den Straße und ich hab schon einen Taxi geruft und bis Hauptbahnhof und den Taxifahrer war schon (...) gut. Und die haben (...) die hat gesagt: 'Wo, wo wollen Sie hin?' Natürlich mit Englisch, und (...) wir konnten schon ein bisschen Englisch #00:12:11-5#

I: Als Englisch konntest, ähm konntet Ihr? Okay. #00:12:11-2#

B: Ja, genau. Ja. Und (...) die haben gesagt: Ja, jetzt gehen wir nach Deutschland. Und dann (...) die anderen, jeder hat schon einfach und (...) weggegangen. UNd ich war schon mit meinem Kumpel und (...) wir warten zu zweit und die hat (...) den Fahrer hat schon uns, wir haben Geld gegeben und der hat schon Ticket gekauft und (...) Bar bezahlt und (...) ja. #00:12:38-6#

I: hm (bejahend) #00:12:38-6#

B: Dann (...) #00:12:41-7#

I: Also, der Taxifahrer? #00:12:41-1#

B: Der Taxifahrer. #00:12:44-4#

I: Des is nett! #00:12:44-4#

B: Ja, der war Moslem und der hat schon gesagt: 'Ja, seid Ihr Moslem?' Hab gesagt, wir haben gesagt: 'Ja, wir sind Moslem' und (...) der hat schon, einfach (...) menschlich, ja. Geholfen und (...) #00:13:01-7#

I: Cool. #00:13:01-7#

B: Ja. #00:13:01-7#

I: (...) ähm, wie hat's denn ausgschaut, sobald Ihr nach Deutschland gekommen seid? #00:13:13-7#

B: Mh, war schon ganz, ganz (...) gut. War, war echt ganz gut weil (...) bei uns ist (...) Polizei, die schlagen uns, ne? Wenn wir keine (...) Fehler machen oder (...) in Türkei oder im Iran undso. #00:13:28-6#

I: hm (bejahend) #00:13:29-1#

B: Aber, damals. Wir waren schon im Zug und die Polizei, die Kontrollen sind gekommen und Sie haben uns gefragt nach dem Pass oder Reisepass oder Ausweis und wir haben gesagt: 'Ja, wir haben garkeins. Nur Ticket.' Und (...) die haben ganz nett, ganz nett gefragt: 'Ja, Ihr müsst zu uns kommen und (...) IHr müsst garnicht Angst haben, und Ihr sind (...) sind in Sicherheit und (...) wir gehen zum Polizeistadtion und nur einfach fragen wir wo (...) woher kommen (...) Sie und (...) #00:14:10-2#

I: War des auf (...) Englisch oder war da nen Dolmetscher dabei? #00:14:13-8#

B: Nein. War ein Kumpel von mir, der konnte schon persische Schrift und der hat schon auf Deutsch, die Polizisten hat schon auf Deutsch geschrieben und auf Persisch übersetzt und, und uns gezeigt. #00:14:25-3#

I: Habt Ihr mim Google Übersetzer oder wie? #00:14:28-0#

B: Genau. (lacht) #00:14:28-0#

I: Nein! (lacht) #00:14:28-2#

B: Wirklich(lacht) #00:14:29-9#

I: Hah, cool! (lacht) #00:14:31-8#

B: War ganz netter Polizist. (...) Ja, der war so chillig. Ja. #00:14:37-6#

I: (...) Des hatte ich mal in ähm(...) im Flughafen #00:14:44-5#

B: hm (bejahend)? #00:14:46-0#

I: Da waren ähm(...) nen paar Leute aus Russland, die(...) des war in, ähm, Reykjavík. ähm, war ich mit nem Freund #00:14:54-2#

B: hm (bejahend) #00:14:54-2#

I: Und ähm (...) da waren nen paar Leute aus Russland, die umbedingt nach Barcelona mussten #00:15:00-1#

B: Okay? #00:14:59-1#

I: Aber absolut kein Englisch gesprochen haben und keine Ahnung hatten, wie Sie da hinkommen sollten. #00:15:05-7#

B: Oh. (lacht) #00:15:06-2#

I: Das heißt, da haben wir das aller selbe gemacht: Auch über den Google Übersetzer, ähm, und irgenwann haben die dann ne provisorische Kreditkarte bekommen #00:15:19-9#

B: hm (bejahend) #00:15:19-9#

I: ähm, haben's eben bei ner Bank ähm (...) beantragt und dann konnten Sie fliegen. Aber find ich irgenwie nett, dass der Google Übersetzer so (...) #00:15:27-0#

B: Genau, ja. Die Polizist hat schon dort geschrieben, und auf, auf Persisch dann übersetzt und (...) und sein Zeit und (...) dann die Kumpel von mir, die konnte schon Persisch lesen und schreiben, und die hat schon wieder die Antwort geschrieben und (...) hingegeben. (lacht) #00:15:46-8#

I: (lacht) #00:15:46-8#

B: Wir haben so gemacht, ja. Aber die waren ganz, ganz, ganz, ganz nett. Und wir haben schon (...) wir haben, ja. Wir waren schon ganz (...) hunger, und hungrig, und die haben uns Essen gegeben und Obst gegeben und (...) #00:16:03-6#

I: (...) Cool. #00:16:03-6#

B: Ja. #00:16:05-8#

I: War das dann direkt hier in Regensburg? #00:16:09-4#

B: WIr wollten durch, ähm(...), Österreich nach ähm(...) Frankfurt gehen #00:16:18-1#

I: hm (bejahend) #00:16:18-1#

B: Ja. Wir wussten da nicht. Bayern? Wo ist Bayern? Wo ist Regensburg? #00:16:23-5#

I: Ja. #00:16:23-5#

B: Ja, wir haben gesagt, ja gehen wir nach Frankfurt und (...) Kumpel von uns war dort. Und der hat gesagt, er muss nachfragen (...) und dann wir haben schon Österreich nach Frankfurt Ticket gekauft. Und dann, Kontrolle. Den Polizei war von Regensburg.  #00:16:46-9#

I: hm (bejahend) #00:16:47-5#

B: Und die haben schon uns kontrolliert, und (Handy des Interviewten leutet) #00:16:50-4#

I: Ups, t'schuldigung. #00:16:51-2#

B: Ja, dann wieder nach Regensburg gebracht. Eigentlich wir waren schon #00:16:50-7#

I: Ich pausier eben. #00:16:56-8#

B: Ja. #00:16:57-6#

I: ähm, Ihr seid auf der Strecke von Österreich nach Frankfurt  #00:17:03-1#

B: genau #00:17:04-3#

I: ähm, in Nürnberg dann kontrolliert worden? #00:17:06-1#

B: Nein, in Regensburg die Polizei hat schon (...) ähm, ja, eingestiegen. und dann, nach dem Zug Kontrolle einfach alle hat kontrolliert, und lange gedauert, und wir waren schon ganz vorne beim Zug #00:17:20-7#

I: hm (bejahend) #00:17:21-1#

B: Und dann (...) ja. Die haben schon gefragt und wir haben gesagt: 'Keine Pass und keine Beizeug.' #00:17:32-8#

I: hm (bejahend) #00:17:32-8#

B: Dann wir waren schon dann im (unv.). Die haben schon (...) gezeigt, wir müssen jetzt aussteigen und wieder nach Regensburg fahren. #00:17:41-1#

I: hm (bejahend) #00:17:41-8#

B: hm (bejahend) #00:17:43-3#

I: Okay. (...) ähm, aber (...) hattest du dann zum Beispiel im (...) du hattest im Iran vermutlich auch keinen Pass oder? #00:17:57-0#

B: Nein. Ich hab schon illegal im Iran gelebt. #00:18:02-9#

I: ähm, weil (...) ähm, die anderen, mit denen ich jetz geredet hab #00:18:06-8#

B: hm (bejahend) #00:18:06-8#

I: die meinten auch, also (...) Sie wären eben ursprünglich aus (...) Afghanistan und hätten im Iran dann gelebt #00:18:15-8#

B: hm (bejahend) #00:18:15-8#

I: ähm, aber (...) dass ma als Afghane im Iran quasi immer Illegaler is, außer, OK, man schließt sich dem Militär an oder so? #00:18:26-6#

B: hm (bejahend) #00:18:27-3#

I: Aber das des so ziemlich die einzige Möglichkeit is, sich (...) nen legalen Status da zu verschaffen #00:18:33-8#

B: Ja. #00:18:34-9#

I: Das is hart. #00:18:36-8#

B: Das is echt hart, wenn, wenn jemand (...) illegal lebt dort, dann mit Polizei. (...) Ja. In (...) kontrolliert, und (...) sag, dass du Afghaner bist oder, keine Ahnung, als Afghaner dort lebst, dann (...) legal dann kriegst du schwierig Probleme. Erstes Mal musst du Gefängnis, dann (...) keine Ahnung. Ein Monat (...) ein Monat, eineinhalb Monat, dann, die schicken, die schieben dich nach Afghanistan. #00:19:06-0#

I: hm (bejahend) #00:19:06-4#

B: Weil, damals. Ich konnte schon die Sprache, weil ich als Kind dort war. #00:19:11-6#

I: hm (bejahend) #00:19:12-3#

B: Ich konnte schon gut den Sprache, und die haben schon garnicht (...) gemerkt, dass ich Afghaner bin undso. Ja. Aber (...) paarmal hatte ich schon Schwierigkeiten, ganz ganz ganz Schwierigkeiten bekommen. Ich war schon im Gefängnis undso. #00:19:30-6#

I: Shiit. #00:19:30-6#

B: Ja, genau. Und dann, wir haben schon (...) ja, den Richter gefunden, und wir haben schon viel Geld gegeben, und dann die haben schon uns raus gelassen. Eigentlich (unv.) #00:19:42-5#

I: Also es lief (...) quasi nur über Bestechung, da wieder raus zu kommen. #00:19:46-0#

B: Ja. Nur dass du Afghaner bist und legal dort lebst und (...)ja. (...) #00:19:56-5#

I: Ich hoff mal hier läuft's besser? (lacht) #00:19:58-1#

B: (lacht) Jetzt läuft gut. (...) Ja. Bisschen Schwierigkeiten mit den (...) Behörden aber sonst passt alles. #00:20:12-6#

I: Mh, was denn? #00:20:11-9#

B: Mit den Behörden, ne? #00:20:14-1#

I: Was lief denn da? #00:20:17-3#

B: Mh, mit Ausweis und Aufenthaltserlaubnis und Asyl und sowas. Soweit. #00:20:21-6#

I: hm (bejahend) #00:20:23-4#

B: Dass man keine Aufenthaltserlaubnis hat. Darf man nicht irgendwohin Reisen undso, deswegen sage ich. Eigentlich, sonst passt.(...) #00:20:21-8#

I: Gut, des is echt ungut irgendwie. #00:20:24-4#

B: Ja (...) es ist schon ungut. Nunja, manchmal du willst (...) Disco gehen, und (...) gehst so. Dann die Security fragt: 'Ja, Ausweis?' Dann (...) gibst du ein Papier und der sagt: 'Ne was, was für ein Ausweis ist? Darfst du nicht mit solche Dokumenten rein!' (...) Dann hast du andere Gefühl, weißt du? Und (...) ja. Du hast keine richtige Ausweis, das (...)  #00:20:58-2#

I: Warte, die Aufenthaltserlaubnis wird dann nicht als gültiges, ähm(...) Ausweisdokument akzeptiert? #00:21:05-5#

B: Die akzeptieren manchmal nicht. Manchmal sagen: 'Ja (...) du darfst nicht. Also (...) Ausweiß nicht da rein. #00:21:17-1#

I: Das hört sich irgendwie nach Blödsinn an? #00:21:17-1#

B: Ja ich war schon paarmal, und die haben mich gefragt: 'Bitte Ausweis.' Ich hab schon meine Ausweis abgegeben! Geben mir die einfach und (...) alle gesehen und die hat gesagt: 'Ja mit solche Dokumente darfst du nicht rein. Zeig deine richtige Ausweis.' Ich hab gesagt: 'Das ist mein richtiger Ausweis!' Und, ja (...) hat gesagt: 'Nein! Musst du ein richtig Ausweis!' (...) Ja. Keine Ahnung. Es ist so. (lacht) #00:21:49-2#

I: Oh, Scheiße. #00:21:48-1#

B: Ja. #00:21:48-9#

I: ähm, aber wenn du jetzt die Ausbildung dann hast, dann hast du ja, ähm, ne Chance ne permanente Aufenthaltsgenehmigung zu kriegen oder? #00:22:01-1#

B: Zum Aufenthaltsgenehmigung die §25 brauchst du nicht umbedingt den Aubildung schaffen, oder (...) ob du Ausbildung hast oder nicht.  #00:22:14-0#

I: hm (bejahend) #00:22:14-0#

B: Du kriegst das, weil du (...) vier Jahre hier bist und (...) #00:22:18-6#

I: Die hast du jetzt dann. #00:22:18-6#

B: Ja, genau. Und Schule besucht hast und alles. Keine (...) Strafentäter bist oder keine Ahnung, keine Schlägerei und (...) keine Scheiße gebaut hast. Ja, dann kriegst du. Ja. Aber natürlich musst du für dich einen Job suchen. Und selbstständig sein; ob Ausbildung oder (...) Arbeiten einfach. #00:22:40-9#

I: hm (bejahend) #00:22:40-9#

B: Ja. Nicht, dass du daheim einfach sitzt und (...) nix machst und (...)  #00:22:46-0#

I: Ok. #00:22:46-8#

B: Das darfst du nicht. #00:22:47-5#

I: ähm, das willst du machen? #00:22:53-6#

B: Ja. Letztes mal, ja, ich bin schon in der Ausbildung, wir haben (...) es ist schon echt schwierig, und mit der Ausbildung. Und schaue ich einfach weiter und (...) ob ich nicht schaffe, das ist auch wurscht. Aber versuche ich! #00:23:06-5#

I: Ja. #00:23:06-5#

B: Ich gebe meine besten undso. #00:23:09-5#

I: ähm, du, und wirklich, wenn  (...) wenn ich dir dabei helfen kann, dann meld dich einfach. #00:23:15-5#

B: Ja. #00:23:16-3#

I: Also, meine Nummer hast du. #00:23:17-5#

B: Ich hab schon dein Nummer. Schreiben wir. #00:23:19-2#

I: Ja. #00:23:20-6#

B: Ja. #00:23:21-7#

I: Also, ähm, wenn ich irgenwie Zeit hab helf ich dir da echt gerne. #00:23:26-5#

B: Ok, Danke. #00:23:28-8#

I: Ja, ähm, gerne. (...)ähm(...) Eins würd mich interessieren, des hab ich in, ähm, meinem ersten Interview gehabt. #00:23:40-9#

B: hm (bejahend) #00:23:41-3#

I: ähm(...) der meinte, dass (...) okay, bei, bei Ihm war's in der Schule, ähm, dass der, ähm, ziemliche Probleme hatte mit (...) Also hier in Regensburg wird da relativ viel Bayrisch noch gesprochen. #00:23:57-5#

B: (lacht) #00:23:59-5#

I: Dass er damit Probleme hatte. #00:24:02-0#

B: Ja (...) Hochdeutsch und Bayrisch ist natürlich hier (...) wenn jemand nicht bayrisch verstehst, dann (...) es ist schwierig. Echt, es ist schwierig.  #00:24:13-7#

I: hm (bejahend) #00:24:15-0#

B: Weil den (...) die manche (...) reden ganz Dialekt auf Bayrisch und (...) du verstehst (...) garnicht was der (...) sagt, oder was der meint er? Und, ja, es ist schon bei mir jetzt dann auch schon (...) auf an(...) ähm Anfang habe ich schon Probleme gehabt in der Werkst att, und weil (...) die Leute sind alle (...) die reden (...) #00:24:44-4#

I: Die reden alle Bayrisch? #00:24:44-4#

B: Bayrisch, und, ja (...) am Anfang habe ich garnicht verstanden. #00:24:48-7#

I: Oh. (lacht) #00:24:48-7#

B: Ja. Und, danach (...) bin mitbekommen, und einfach (...) #00:24:55-0#

I: Mittlerweile funktioniert's? #00:24:56-3#

B: Ja, genau. Funktioniert. #00:24:57-7#

I: Und, und (...) die ham dir dann quasi auch geholfen? ähm #00:25:01-1#

B: Ja, die reden garnicht mit dir Hochdeutsch! #00:25:03-8#

I: Nur Bayrisch? #00:25:03-8#

B: Nur Bayrisch. #00:25:06-0#

I: (lacht) #00:25:06-0#

B: Nur Bayrisch, ja. Obwohl du sagst: 'ja bitte, ich verstehe nicht Bayrisch.' Und die (...) #00:25:14-7#

I: ähm #00:25:15-3#

B: Die reden überhaupt. #00:25:18-1#

I: ähm, können Sie nicht oder wollen Sie nicht? #00:25:17-8#

B: Die wollen nicht. #00:25:18-6#

I: Wollen nicht? #00:25:19-7#

B: Vielleicht, die können nicht. Bayr(...)ähm Hochdeutsch. #00:25:22-3#

I: hm (bejahend) #00:25:22-8#

B: Kann sein, die können nicht Hochdeutsch, oder die haben schon immer Bayrisch gesprochen und(...) #00:25:29-4#

I: Was ist dein Eindruck, was glaubst du? #00:25:32-0#

B: Ich glaube die können schon Bayrisch, ähm, Hochdeutsch, und (...) die wollen einfach, dass ich Bayrisch lerne. Und (...) #00:25:40-7#

I: Also, also eher so (...)ähm (...) Als Spielerei, Freundschaftlich? #00:25:44-8#

B: Ja, genau. Wirklich, Spielerei, Freundschaftlich. #00:25:47-3#

I: Also jetz nicht irgendwie so in die Ausgrenzungsrichtung. #00:25:52-4#

B: Ne, ne. #00:25:53-9#

I: Okay. #00:25:53-7#

B: Ja. (...) #00:25:56-6#

I: Ja, gut. #00:25:58-2#

B: Die wollen, dass ich einfach Bayrisch, weil ich einfach haben (...) einmal gesagt: 'Wieso reden (...) Ihr nicht mit mir Hochdeutsch?' Und die haben geagt: 'Ja, wir wollen, dass du auch Bayrisch #00:26:07-5#

I: hm (bejahend) #00:26:07-5#

B: redest.' Und, ja, Bayrisch mit uns redest. Und dann, ja, wenn du ein Wort öfter (...) mitbekommst, dann, natürlich, dann (...) verstehst du und (...) kennst du. Aber es ist schon auch schwierig, bayrisch zu reden. (lacht) #00:26:21-7#

I: (lacht) #00:26:24-4#

B: Aber ich kann es irgendwann(lacht) #00:26:23-4#

I: //Jetzt muss (...) jetzt musst du Deutsch und Bayrisch lernen. (lacht) #00:26:25-1#

B: Ja, trotzdem Bayrisch. Ja, wenn, natürlich, wenn du in Bayern bist, dann (...) musst du beides können, ne? (...) Ja. #00:26:36-9#

I: (lacht) Vermutlich kannst du jetzt schon besser Bayrisch als mancher Deutscher hier. #00:26:37-2#

B: ja, ne, ne, ne (lacht) #00:26:38-0#

I: ähm, (...) aber (...) das hat quasi so für dich jetz nich wirklich prob(...) nen Problem her(...) dargestellt oder? #00:26:44-7#

B: jetzt ist nicht mehr. Jetzt ist nicht mehr. Manchmal  #00:26:47-3#

I: //Wie war's am Anfang? #00:26:47-3#

B: Manchmal vielleicht (...) ähm, wie bitte? #00:26:50-6#

I: Wie war's am Anfang? #00:26:50-7#

B: An Anfang war ganz schlecht. Ich hab schon so alle, keine Ahnung, Schrauben verkehrt gebracht und die haben schon was anderes gemeint und ich hab schon (...) was anderes gebracht.(lacht) #00:27:02-8#

I: (lacht) #00:27:03-9#

B: Genau. #00:27:05-7#

I: Ja gut, aber (...) das passiert am Anfang. #00:27:07-5#

B: Ja. #00:27:10-4#

I: (...) ähm (...)Aber so im großen und ganzen ist es für dich ne Positive (...) ist es (...) schön hier oder? Also, ne positive Erfahrung. #00:27:26-4#

B: Ja. #00:27:30-4#

I: ähm (...) wie viele Wohnungen hast du denn bisher gehabt? Da war nämlich letztes mal, kann ich mich erinnern (...)ähm(...) dass, da, da gab's doch irgendwann (...) ähm, meintest du, Reibereien in ner Wohngemeinschaft oder? #00:27:52-4#

B: In Wohngemeinschaft, ja. Es ist immer so, weil (...) dort sind (...) verschiedene Nationalitäten, wohnen dort #00:28:07-1#

I: hm (bejahend) #00:28:07-1#

B: Und (...) ja, wenn (...) die eine nett ist, dann der andere ist bestimmt (...) nicht nett ist. Und es ist so, die manche kommen mit dir klar, aber die manche wollen nicht mit dir klar kommen einfach. (...) Ja, es ist so. Ich hab schon dort, ähm, (unv.) gewohnt. Nicht lange, ein Woche, als ich neu war. Hier, in Regensburg. #00:28:40-1#

I: hm (bejahend) #00:28:40-1#

B: Und (...) ja. Ich hab schon einfach Probleme gehabt. Ich war schon, ich war, ich war auf mein Bett und (...) im Schlaf, hat eine gekommen und mich geschlagen. #00:28:57-2#

I: Was?! #00:28:57-2#

B: Ja, wirklich. Auf dem bett weggeworfen. Ja. #00:29:02-0#

I: Hä?! #00:29:02-0#

B: Der hat schon mich auf dem Bett weggeworfen. Ja, wirklich. Der war (...) betrunken. Der war Russe und ähm(...) keine Ahnung, dann (...) ich wusste das nicht. Ich hab aufgestanden, und (...) ja. #00:29:15-8#

I: Der hat (...) der hat auch da gewohnt oder wie? #00:29:17-5#

B: Ja. (...) Auf eine Zimmer, keine Ahnung, 20, 30 Bett? Und waren viele verschiedene Leute, und (...) der hat mich geschlagen und (...) ja. Und dann (...) hab gesagt: 'Hey, was, was, was ist los?' und 'was willst du?' Und der hat schon einfach weiter gemacht. Und dann habe ich schon Hilfe gerufen. #00:29:41-3#

I: //Der is einfach weiter auf dich, hat geschlagen? #00:29:44-1#

B: //ja, genau. ja. Ja. #00:29:47-1#

I: Besoffener Wichser. #00:29:47-1#

B: Ja, dann, dann Polizei gekommen und dann ich hab gesagt: 'Es ist so' und 'Ich will nicht hier wohnen' #00:29:54-5#

I: hm (bejahend) #00:29:54-5#

B: Und dann die haben mich (...) ein Zimmer gegeben. #00:29:59-7#

I: Dann hast du nen Einzelzimmer bekommen. #00:30:01-4#

B: Dann habe ich schon eine Eilzelzimmer bekommen. #00:30:04-8#

I: Ja. War denn sowas(...) #00:30:05-9#

B: //zwei Tage.  #00:30:07-5#

I: Zwei Tage? #00:30:09-5#

B: Ja, zwei Tage. Dann, die haben mich (...) nach (...) ja, St. Vinzent gebracht.  #00:30:17-0#

I: hm (bejahend) #00:30:17-0#

B: Dann war ich schon dort auch, dreieinhalb Monate bis vier Monate, habe ich schon dort gewohnt. #00:30:24-7#

I: Wie war's da? #00:30:25-6#

B: War gut. Ein, eine Apartement (...) eine, ein große Apartement war dort #00:30:34-4#

I: hm (bejahend) #00:30:34-4#

B: und sechs, sieben (...) acht Leute. (...) #00:30:43-4#

I: Und die waren alle okay? #00:30:43-2#

B: der war schon, ja, die meißten war okay. Die meißten war nicht, ja. (lacht) Es ist so, passt, war gut, einfach. #00:30:52-9#

I: hm (bejahend) #00:30:53-3#

B: Auch nicht so schlecht, war gut. Und, ja, ich hab schon ein Zimmer gehabt, mit eine geteilt. Weil, wir waren schon zu zweit in einem Zimmer. #00:31:04-1#

I: hm (bejahend) #00:31:04-1#

B: Dann war okay. Dann (...) Nach dem drei Monat, drei, vier Monat, die haben gesagt: Ja, jetzt, jetzt musst du weiter. (...) Dann die haben mich nach St. Vinzent gebracht in Liebigstraße. Dort war schon auch eine Wohngruppe #00:31:25-6#

I:hm (bejahend)  #00:31:25-6#

B: mit sechs Leute. Dann (...) sechs, sieben Leute. Ja, dann habe ich schon auch ein Zimmer bekommen, mit mein Kumpel. #00:31:35-7#

I: Der, mit dem du (...)ähm #00:31:39-3#

B: nach Detschland. #00:31:40-3#

I: Cool! #00:31:41-4#

B: gekommen. Ja, wir waren schon zusammen dort, und, ja, dann wir haben gesagt: 'wir wollen schon auch zusammen, wenn irgendwo hingehen, dann (...) gehen zusammen.' Dann die haben uns ein Zimmer gegeben. Dann war ich schon eineinhalb Jahre dort #00:31:58-4#

I: hm (bejahend) #00:31:58-4#

B: Und (unv.). (lacht). Ja, dann habe ich gesagt, dass es mir reicht, und ich will nicht ähm(...) in Wohngruppe. Und die haben schon mir betreutes Wohnen gegeben, hier. In drei (unv.) #00:32:12-7#

I: Also nach der, ähm, nach der Wohnung bist du hierher gezogen. #00:32:16-5#

B: Genau. #00:32:18-8#

I: Was macht dein Kumpel jetz? Is der noch drin? #00:32:20-4#

B: Der war schon hier! #00:32:22-0#

I: Der war hier? #00:32:22-0#

B: Der war schon hier. Genau. #00:32:25-1#

I: Und, wo ist der jetz? #00:32:26-2#

B: Der hat schon Aufenthaltserlaubnis bekommen, und der, der durfte schon einfach (...) #00:32:32-4#

I: der konnte jetzt ne normale Wohnung (...) #00:32:35-0#

B: Normale Wohnung mieten, genau. Genau. ja. Der hat schon eine Wohnung gefunden, und (...) weggezogen. #00:32:43-2#

I: Machst du des dann (...)ähm, aus Regensburg raus, oder (...)? #00:32:47-8#

B: Ne. Der wohnt jetzt in Pentling, weil #00:32:48-0#

I: //Schon hier, ok #00:32:48-0#

B: der arbeitet dort. Der arbeitet auch ähm(...) als Karosseriebauer. Als Lackierer. #00:32:54-7#

I: hm (bejahend) #00:32:55-1#

B: Und (...) ja. (...) Dort hat schon, ja, hat schon ein Zimmer dann bekomm(...) gefunden und er hat geagt: 'Ja, es ist (...) gut für Ihn', weil (...) der hat, der arbeitet auch in Pentling. #00:33:13-2#

I: hm (bejahend) #00:33:13-2#

B: Der hat gesagt: 'Wenn ich dort gehe, es ist schon besser. Es ist kürzer.' Von der Arbeit her. #00:33:19-8#

I: Ja. #00:33:19-8#

B: Ja. #00:33:20-9#

I: ähm, wie lang dauerts denn so am Tag, wenn du hier raus kommst zund zur Arbeit? Wie lang bistn da unterwgs? #00:33:30-3#

B: (...) Unterwegs? Meinst du jeden Tag? #00:33:35-1#

I: hm (bejahend) #00:33:37-1#

B: Ja, acht (...) acht Stunde arbeite ich. Und der Rest, komme ich nach Hause und (...) manche Tag (...) zweimal in der Woche gehe ich zum Sport. #00:33:49-9#

I: Ah? #00:33:49-9#

B: Ja. #00:33:49-9#

I: Was machst du? #00:33:52-2#

B: Boxen! (lacht) #00:33:52-2#

I: Okay #00:33:54-4#

B: Ja, genau. Mit anderen, mit Humind und (...) im Jugendcafé. #00:33:57-9#

I: Des geht um (...)ähm 17 Uhr los meintest du oder? #00:34:00-8#

B: genau, ja. Dienstag und Donnerstag. #00:34:07-0#

I: Am Dienstag ist des auch? #00:34:07-0#

B: hm (bejahend) #00:34:07-0#

I: Auch um 17 Uhr? #00:33:59-2#

B: Um 17 Uhr. (...) Ja (lacht) #00:34:02-2#

I: Ach, des Boxen würd mich schon interessieren. #00:34:15-4#

B: Eh, Boxen ist cool, cooles Sport, aber ein bisschen anstrengend. #00:34:20-5#

I: Ja bitte, muss es sein! #00:34:22-7#

B: Ja (lacht) #00:34:24-5#

I: (lacht) #00:34:24-5#

B: ist schon viel anstrengend, aber, das ist gut. (...) #00:34:29-5#

I: ähm, wie bist'n du zum Boxen gekommen? Hast du des im Iran schon gemacht oder hier dann? #00:34:34-6#

B: nein. Im Iran (...) war ich schon im Fitnessstudio, paarmal, so (...) habe ich schon auch ein bisschen Sport gemacht, aber nicht mit Boxen undso. (...) Und, Tae Kwon Do habe ich schon gemacht, ein Jahr #00:34:51-9#

I: Tae Kwon Do? #00:34:51-9#

B: hm (bejahend) Im Iran. Und (...) Fitnessstudio. Ich war schon auch ganz sportlich da. (...) Und dann, hier (...) zwei Jahre. Ein Jahr war ich schon im Fitnessstudio, und dann (...) war langweilig. Echt, war langweilig. #00:35:10-6#

I: hm (bejahend) #00:35:12-7#

B: Und wenn du jeden Tag so ne Stunde zum Fitnessstudio gehst, zum pumpst deine Muskeln. War echt langweilig. #00:35:21-0#

I: War ich noch nie, will ich nie machen. #00:35:21-0#

B: (lacht) #00:35:21-9#

I: (lacht) #00:35:22-6#

B: Ja, du bekommst ganz Muskeln. Und wenn du nicht machst, dann geht alles weg! #00:35:28-7#

I: Vor allem, des sind total nutzlose Muskeln #00:35:31-1#

B: Genau. #00:35:31-1#

I: Also (...) Ehh. #00:35:34-3#

B: Ja (lacht). Und, ja, dann Kumpel von mir war Boxer, und der hat gesagt: ´'Ja, komm mit! Gehen wir (...) Boxen. Ich nehme mit. Und es ist besser als Fitnessstudio.' Und dann ich hab gesagt: 'Ja, ok.' Dann war ich schon bei einem Boxverein #00:35:53-8#

I: hm (bejahend) #00:35:53-8#

B: Und (...) wir waren schon sechs bis (...)acht Monaten dort. Und jedes Monat, wir haben 30€ bezahlt. #00:36:04-5#

I: hm (bejahend) Mh, des is relativ viel Geld. #00:36:07-1#

B: ähm waren echt viel, 30€. Fitnessstudio 30€. #00:36:14-4#

I: Sind 360€ im Jahr. #00:36:16-5#

B: Genau, (...) dann 30€ Boxen, war 60€, dann Internet undso weiter. 100€ gibst du alles weg! #00:36:28-7#

I: hm (bejahend) #00:36:28-7#

B: Dann wir haben schon überlegt, und wo, wo müssen wir einen Platz, einen (...) Platz suchen. Eine Halle zum Boxen. #00:36:42-3#

I: hm (bejahend) #00:36:42-3#

B: Dort wir haben schon beim Jugendcafé gefragt und die haben uns gegeben ja. Wegen (unv.) #00:36:51-2#

I: //Ach des habt quasi Ihr euch selber aufgebaut im Jugendcafé? #00:36:55-1#

B: Genau. Wir haben schon selber gegründet, also(...) #00:36:58-1#

I: Cool! Also, ähm, der Raum war quasi leer, also(...)? #00:37:06-1#

B: Der Raum war nicht ganz leer. Die anderen auch schon damals benutzt, die anderen Tage. Die, ähm, die eine Frau hat schon (...) den kleine Mädels. Die hat schon, ähm(...) Tanzunterricht gegeben. #00:37:23-5#

I: hm (bejahend) #00:37:23-5#

B: Ja, genau. Und (...) so und war schon leer, und wir haben schon gesagt: 'Ja, dann andere Tage. Wenn frei ist, dann nehmen wir.' #00:37:36-5#

I: Cool. #00:37:36-5#

B: Und zweimal in der Woche trainieren wir. Ja, ok. (...) ja. (...) #00:37:44-1#

I: ähm(...), Wie bist du denn eigentlich zum Jugendcafé gekommen? (...) #00:37:54-9#

B: (...)Jo, gute Frage. Ich war schon (...) bei der B - Klasse. (...) Ja, absch(...) normal, Schulabschluss. #00:38:06-3#

I: hm (bejahend) #00:38:06-8#

B: drei, zwei Jahre war schon bei Berufsschule, zwei, habe ich schon Schulabschluss gemacht, und dort eine Sozialpädagogen hat schon uns (...) gesagt: Es gibt schon. So eine Jugendcafé. Wenn Ihr wollt, ähm, dann (...) könnt Ihr (...) kommen. Dann war ich schon einen Tag. Dann viele Leute gesehen, dann habe ich schon gesagt: 'Es ist gut, wenn ich herkomme. Und es ist schon auch gut von der (...) deutsche Sprache und kann man schon auch was lernen und die helfen uns!' und dann (...) war schon dabei. Bin immer gekommen, und jeder frei, jeder Donnerstag habe ich schon gekocht. #00:38:54-4#

I: Ja. Jeden Donnerstag kochst? #00:38:56-7#

B: Jeden Donnerstag, ja. Früher habe ich schon jeden Donnerstag habe ich schon gekocht. #00:39:00-0#

I: Aber jetzt nichtmehr oder? #00:39:02-6#

B: Jetzt nichtmehr. Jetzt habe ich keine Zeit.(lacht) #00:39:03-9#

I: (lacht) #00:39:05-2#

B: Weil Donnerstag trainiere ich schon #00:39:06-8#

I: hm (bejahend) #00:39:06-8#

B: und (...) kochen Zeit ist nur am Donnerstag. #00:39:14-2#

I: hm (bejahend) (...) Aber des heißt, du bist dann jetz quasi erst (...) ähm, knapp zwei Jahre da drin oder? #00:39:21-4#

B: (...) drei (...) Jahre. #00:39:21-3#

I: Drei Jahre? #00:39:24-2#

B: hm (bejahend) #00:39:24-2#

I: Ah! #00:39:24-2#

B: Drei. (...) Vier. 2015. Es ist 2015. #00:39:33-4#

I: hm (bejahend) #00:39:34-8#

B: Vier Jahre. #00:39:40-9#

I: ähm, du gehst doch morgen sicher weider relativ früh ins Jugendcafé oder? #00:39:48-8#

B: Morgen, nein. #00:39:48-8#

I: Morgen nicht? #00:39:49-3#

B: Ne. Weil Ramadan Zeit ist so. Und dann (...) müssen wir selbst was daheim kochen, und (...) #00:40:01-0#

I: hm (bejahend) #00:40:01-0#

B: Ja. #00:40:01-0#

I: ähm(...) also, beim Ramadan isst man auch zu Hause dann? #00:40:09-8#

B: nein, es ist so, musst du nicht umbedigt. Zu Hause. Essen. Wenn ich im Jugendcafe geht, dann kriege ich nur (...) Abendessen, ne? #00:40:17-9#

I: hm (bejahend) #00:40:17-9#

B: Dann muss ich auch mit(...) in die Mitternacht, muss ich auch aufstehen und was Essen. #00:40:22-0#

I: hm (bejahend) #00:40:22-0#

B: Drei Uhr. Dann muss ich auch selbst was kochen. #00:40:26-6#

I: Ja. #00:40:27-6#

B: Dass die übrig bleibt. Und (...) ja. deswegen. #00:40:33-3#

I: Du kannst nicht einfach was mitnehmen von dort? ich mein (...) #00:40:34-4#

B: Geht schon, aber (...) #00:40:36-7#

I: Ja gut, dann musst du's rumschleppen. #00:40:36-4#

B: Ja (...) es ist nich so, schleppen. ist kein Problem. Aber vielleicht kommen viel Leute, und bleibt nicht übrig. Und dann, was kannst du machen? #00:40:47-0#

I: Es muss ja nicht übrig bleiben. Also ich mein (...)ähm, du kannst dir ja direkt am Anfang was wegnehmen dann. #00:40:57-0#

B: Ja es ist so, dann (...) du kannst machen. Aber wenn viele Leute kommen, und nicht zu Essen (...) bekommen, dann fühlst du dich nicht gut. Weil (...) #00:41:09-5#

I: Ja, versteh ich. #00:41:09-5#

B: Weil dann der andere kriegen nichts Essen, dann krieg zwei Portion. Dann nicht so gut, ne? Deswegen (...) #00:41:16-1#

I: hm (bejahend) #00:41:16-1#

B: koche ich. Ja. (...) #00:41:22-2#

I: (...)ähm, aber des heißt, du gehst morgen auch garned hin. #00:41:26-7#

B: Ne. #00:41:28-1#

I: Also, jetzt über die komplette Zeit des Ramadan. #00:41:30-3#

B: Komplette Zeit Ramadan. #00:41:31-4#

I: Okay. #00:41:31-4#

B: Danach (unv.) gehe ich schon. Zweimal in der Woche. Einmal zum Training und einmal zum Training und auch zum Essen. (lacht) Ja. #00:41:45-3#

I: ähm(...) Wie lang geht denn der Ramadan noch? #00:41:51-4#

B: (...) Außer dieser Woche noch zwei Wochen. (...) Bis fünfte (...) jetzt Mai zuende? (...) Bis fünfte nächsten Monat. #00:42:03-6#

I: ähm Juni. #00:42:07-1#

B: Juni, genau. Bis fünfte Juni. #00:42:07-5#

I: Okay. ähm(...) hat's eigentlich für dich ähm nach (...) irgendwelche anderen Auswirkungen gehabt, dass du dich quasi an den Ramadan hältst? Also irgendwie in der Arbeit oder so? #00:42:23-6#

B: Wie meinst du? #00:42:26-8#

I: ähm, naja du (...)ähm isst ja dann zum Beispiel auch nix. #00:42:29-6#

B: Ja. #00:42:30-4#

I: ähm(...) hat des dann irgendwie (...) irgendwelche Auswirkungen? Hat des (...) keine Ahnung, nen Mitarbeiter mal kommentiert oder was weiß ich? #00:42:41-7#

B: (...)ne. normal. #00:42:44-9#

I: Okay. #00:42:44-9#

B: Oder (...) einfach normal, ich gehe (...) der Arbeit. Ich mache meine Arbeit. #00:42:48-8#

I: hm (bejahend) #00:42:48-8#

B: Egal, was ich bekomme. #00:42:51-1#

I: hm (bejahend) #00:42:51-1#

B: Egal, wie (...) schwierig meine Arbeit sein. Mache meine Aufgabe, bekomme ich. Und ja.(...) #00:43:01-6#

I: Ja. #00:43:02-8#

B: Ja. #00:43:05-5#

I: Deine ganzen Mitarbeiter sind dann eigentlich auch echt nett oder? #00:43:08-4#

B: Meine (...) ja. ja, schon. Die sind alle nett. (...)  #00:43:14-7#

I: Glücksgriff? #00:43:17-2#

B: Ja. #00:43:18-5#

I: ähm(...) vielleicht ne Dumme Frage, aber (...) hast du, seit du hier angekommen bist, ähm, offenen Rassismus erlebt? #00:43:32-8#

B: (...) mhm. #00:43:32-8#

I: Schon? #00:43:32-8#

B: ja. (lacht) ja.(...) Parrmal habe ich schon erlebt. Nicht oft, aber paarmal. Zweimal mit eine Frau, die gleiche Frau. #00:43:48-3#

I: Hmh? #00:43:49-9#

B: Und einmal eine (...) ein Mann. Irgendein Mann einmal. #00:44:00-4#

I: //ähm, was (...) #00:44:00-4#

B: Eine alter Mann. #00:44:02-0#

I: Ein alter Mann? #00:44:04-0#

B: hm (bejahend). Ich war unterwegs, und (...) in Maximilianstraße war ich schon, (...) Hotel (...) Bahnhof gehen, und im Arcaden gehen, was kaufen. Und (...) vorne (...) auf Vorne Seite, kam da ein Frau und, zu mir, und einfach sie hat schon auf mich gespucken. #00:44:24-1#

I: Gespuckt?! #00:44:24-1#

B: hm (bejahend). Gespuckt. Und (...) die hat schon (...) keine Ahnung, geschimpft. #00:44:32-4#

I: Wie? Du gehst, ähm, die Maxstraße entlang, in Richtung Arcaden. #00:44:35-9#

B: Ja, genau. #00:44:35-7#

I: Und dann kommt dir ne Frau entgegen, und beschimpft dich und bespuckt dich? #00:44:41-1#

B: Ja danach. Genau, ja. Die hat schon einfach entgegen gekommen und die hat schon (...) gespuckt. Und (...) geschimpft und ich hab gesagt: 'Ja, passt schon.' Es war einmal, nicht einmal. (lacht) Die gleiche Frau, die gleiche war. #00:45:00-5#

I: Es ist, ähm, dann, in der, wo is denn des andere passiert? #00:45:03-4#

B: Die andere war auch schon dort #00:45:05-2#

I: hm (bejahend) #00:45:05-2#

B: Die andere war schon bei der Ampel. (...)(lacht) Genau. bei der Ampel.  #00:45:12-0#

I: Ja? #00:45:12-0#

B: In der Max. Maximilianstraße. Und ich wollte schon auch im andere Seite gehen. Und (...) ich wollte den zehner Bus, den (...) Bus zehner nehmen, und (...) keine Ahnung, war schon irgendwo (...) unterwegs. Und da, bei der Ampel, eine alter Man gekommen und er hat uns geschimpft und er hat gesagt: 'Was nacht Ihr? Müsst  wieder zurück!' Und (...) ja. Ja, er will Geld von uns und keine Ahnung, will Arbeit von uns und irgendwo. #00:45:46-3#

I: Also, ähm, er, er is an der Ampel gestanden und hat dich beschimpft, dass (...) warte: 'Er will Geld, und Ihr sollt was arbeiten!' oder wie? #00:45:58-2#

B: Der hat gesagt: 'Ja, die ähm, die Ausländer, die blöde Ausländer! Was macht Ihr?' und ja, 'wir müssen weggehen!' und ja, einfach weiter geschimpft und (...) #00:46:09-7#

I: Hmm, was macht Ihr? Auf jeden Fall keine leute an der Ampel beschimpfen. (lacht) #00:46:14-2#

B: Ja. (lacht) Seine Arbeiten, jammer, 'unsere Arbeitsplatz geht weg und unsere Geld, unsere Geld und alles!' Und er hat schon weitergeschimpft und ich habe gesagt: 'Ja, passt. (...) Wiedersehen' #00:46:32-2#

I: Uh.(...) #00:46:32-2#

B: Ja. #00:46:32-6#

I: Hat dich das danach noch länger beschäftigt, oder (...) passt? #00:46:38-1#

B: Ne, einfach schon (...) geschimpft und weiter gegangen. #00:46:39-4#

I: Was für ein Idiot! #00:46:42-0#

B: Okay. (lacht) Ja. #00:46:42-6#

I: Uh. #00:46:42-6#

B: Ja, wir hatten einmal, war ich schon auch (...) in Dachauplatz. (...) ähm, mit eine (...) eine Junge. Ein deutscher Junge. War mein Kumpel. (...) Wir haben schon da einfach gestanden, und einige, einfach einige zu uns gekommen und (...) nicht zu mir direkt gekommen. Zu meine Kumpel gekommen. Und der hat gesagt: 'Hey, du bist doch Deutscher! Was machst du mit dem Ausländer?'(...) #00:47:20-1#

I: Was? #00:47:20-1#

B: Ja. Der hat gesagt: 'Du bist doch Deutscher! Was machst du mit der Ausländer? Schämst du dich nicht? (...) Ja. Schämst du dich nicht?' Er hat gesagt: 'Wieso? Wiesoll ich mich schämen?' Und er hat geasgt: 'Ist doch Ausländer! Du musst Ihn schlagen! Ist Ausländer!' #00:47:38-9#

I: //(lacht) #00:47:41-9#

B: 'Ist schon gut! Der versteht überhaupt nicht! Ja, de rmuss zurück gehen!' Ja, dann der hat gesagt: 'Ja, passt schon. Geh (...) geh weg.' #00:47:51-8#

I: ähm, der der (...) #00:47:56-1#

B: Der war eigentlich ganz sauer, (...) aber sagt: 'Ja, was soll ich machen? Nix!' #00:48:02-3#

I: ähm, der, der des gesagt hat, war des auch nen alter Mann oder (...)? #00:48:06-4#

B: Ne, war ein Junge. 25? 30? (...) #00:48:15-7#

I: (...) Und der is zu dem Freund von dir #00:48:19-0#

B: //Zu mein Freund, ja. Zu mein Freund von von mir gekommen, und der war Deeutscher und der hat gesagt: 'Hey, was machst du da?!' #00:48:24-9#

I: (lacht) #00:48:24-9#

B: (lacht)(...) Ja. #00:48:32-8#

I: Was is denn des für seine komische Situation? (...) #00:48:34-4#

B: Keine Ahnung. (lacht) Keine Ahnung. #00:48:37-6#

I: Uh. #00:48:38-1#

B: Ich hab schon echt, im Iran, ähm(...) solche Situationen habe ich schon garnicht erlebt. Ja, aber (...) nun, hast du im Iran mit die Polizei Probleme. Aber mit andere Leute? Wenn du normale Leute hast und ein Problem? Garkein Problem!  #00:48:55-4#

I: hm (bejahend) #00:48:55-8#

B: Ja, wenn du nicht natürlich selbst Scheiße baust, aber sonst (...) alle sind (...) meißtens sehr freundlich. Es gibt schon Arschloch, aber (...) (unv.) aber, er schon auch natürlich hier. Gibt schon auch andere Seite, viele nette Leute. (...)  #00:49:18-5#

I: (...) Des irritiert mich grad voll. (lacht) #00:49:20-0#

B: (lacht) #00:49:20-0#

I: Ja, wenigstens, ähm(...) hier auf die Polizei verlassen. #00:49:28-4#

B: hm (bejahend) Ja. kann man nicht machen. Wenn du (...) (unv.) dann, bist du auch selbst (unv.). (...) Wie Ihr oder wie Sie(...) #00:49:41-0#

I: Nochmal? #00:49:41-0#

B: ich habe geasgt, wenn du gegen (...) gegenseitig was schimpfst, (...) den Typ, dann bist du auch, so eine (...) so eine Idiot. Dann (...) Wie unterscheidest du zwischen dir und Ihm? #00:49:58-0#

I: Ja. Am besten einfach garnicht anfangen. #00:50:00-5#

B: Ja. (...) #00:50:03-3#

I: (lacht) Ich muss gestehen, ich hätt zwar vermutlich selber angefangen zu schimpfen, aber (...) das ist vermutlich die klügere Entscheidung, das nicht zu machen. (...) #00:50:17-5#

B: Ja, manchmal hast du auch keine Bock. Du willst auch (...) den schimpfen zurück, aber manchmal sagst: 'Hey, lass doch! Das (...) der geht weg.' und, sagst du: 'Überhaupt keine Stress. Geh. Bitte.' #00:50:41-5#

I: (...)ähm(...) aber sind dir noch andere Sachen aufgefallen, die(...) ähm, in Deutschland jetz anders sind als im Iran? Also(...) so: Die Leute im Iran sind dann wohl ein bisschen freundlicher?(lacht)(...) #00:51:02-5#

B: (...) in jede Land gibt es schon gute Leute und schlechte Leute. #00:51:05-4#

I: hm (bejahend) #00:51:05-8#

B: Ja. #00:51:07-6#

I: Klar, sind alles Menschen. #00:51:07-7#

B: Alle, alle sind Menschen. Ja. genau. Aber (...) im Iran hast du schon (...) eigentlich keine Sicherheit. #00:51:18-6#

I: hm (bejahend) #00:51:19-1#

B: Und (...) mehr Probleme. Aber hier nicht. Dafür manche Rassisten, die schimpfen dich. #00:51:26-0#

I: hm (bejahend) #00:51:26-6#

B: Die schlagen dich natürlich nicht, aber die schimpfen dich. Und (...) zum Schimpfen (...) bekommst du garnicht was. (...) Musst du nur einfach (...) aus und raus (...). #00:51:48-7#

I: Ja. (...) wenn du's einfach ignorieren kannst, dann (...) dann isses ja gut. #00:51:54-4#

B: Ja. #00:51:54-4#

I: Sollte aber trotzdem ned passieren, also ganz ehrlich. Es ist 2019. #00:51:59-1#

B: Ich kann schon auch verstehen, weil (...) #00:52:01-2#

I: Echt? #00:52:01-2#

B: Ja, natürlich. Kann schon auch den Leute verstehen. Weil (...) die, die manche Ausländer, die sind auch nicht, keine gute Leute. Die manchen sind einfach Idioten. Die manchen sind einfach Arschloch. Und die bauen einfach Scheiße und (...) #00:52:16-1#

I: Ja. #00:52:16-1#

B: Weil, wenn die (...) weniger, die Rassist sind. Die seht im, im Fernseher. Oder, keine Ahnung. Im Radio. Und im Zeitung. Die lest was. #00:52:26-3#

I: hm (bejahend) #00:52:26-3#

B: 'Heute hat eine Ausländer Scheiße gebaut.' #00:52:31-1#

I: hm (bejahend) #00:52:31-6#

B: Dann der sagt: 'Hey (...) Ja. Alle sind schlecht und alle sind gleich.' #00:52:39-9#

I: Ja, ähm, tatsächlich wenn (...) wenn nen Ausländer irgendwie Scheiße baut, dann (...) ähm wird das(...) sehr viel schneller von den Nachrichten aufgegriffen. Das ist (...) auch unpraktisch dann. #00:52:53-9#

B: Das ist natürlich auch unpraktisch. Weißt du, wenn, wenn ein Ausländer (...) keine Ahnung, hald Scheiße bauen, dann die sagen: 'Ja (...) Ausländer. Islamischer, keine Ahnung, so so so.' Dann, wenn eine Deutscher macht das, dann die sagen nicht, dass (...) keine Ahnung, die war Schuld, sondern die sagen: 'Ja, die war krank!' Ja. Die war (...) keine Ahnung, psychisch krank oder keine Ahnung #00:53:26-3#

I: hm (bejahend) #00:53:26-3#

B: Einfach, ähm, (...)die sagen, dass die krank war, die hat schon so gemacht. Aber, ja. #00:53:33-4#

I: Das wird dann grundlegend anders behandelt. #00:53:35-0#

B: Ja, ja. Genau. Wird anders behandelt, genau. (...) Die Behandlung sind nicht gleich (...) ja. #00:53:40-4#

I: ähm (...) gestern, oder heute? Habe ich in der Zeitung gelesen, dass (...) ähm rechtsextreme Straftaten haben jetzt (...) deutlich zugenommen tatsächlich. Also (...) ähm, im, ähm, die Ausländer machen (...) also, da hat sich die Zahl der Verbrechen nicht wirklich verändert, aber die Zahl der, ähm, rechtsextrem motivierten Straftaten (...) die is tatsächlich deutlich nach oben gegangen. (...) Also, mal schaun in welche Richtung das jetz geht. #00:54:16-5#

B: hm (bejahend) #00:54:17-1#

I: Aber (...) Ja. #00:54:20-2#

B: Das ist alles so (...) wenig, wenig, wenig auf Person. Wenn ich Aufenthaltserlaubnis habe #00:54:27-7#

I: hm (bejahend) #00:54:28-1#

B: dann dürfe ich alles machen, ne? Dann habe ich keinen Stress. #00:54:33-4#

I: hm (bejahend) #00:54:33-4#

B: Dann gehe ich jeden Tag zum Arbeit, komme ich zurück, heim. #00:54:35-3#

I: Ja. #00:54:36-1#

B: Dann habe ich was zu tun. #00:54:36-5#

I: Ja. #00:54:38-4#

B: Wenn ich keine Aufenthaltserlaubnis hätte? (...) Und, garnicht in der Hand (...) #00:54:44-0#

I: hm (bejahend) #00:54:44-0#

B: Dann darf ich nicht (unv.). Kann ich was machen? Ich kann nicht in die Stadt gehen. Ich darf nicht in der Disco gehen. Ich darf nicht, ähm(...), nach Österreich gehen. Ich darf nicht (...) arbeiten. #00:54:57-5#

I: hm (bejahend) #00:54:58-5#

B: Ich darf nicht garnicht, mich ähm, beschäftigen. Dann (...) das ist alles Problem, und das ist alles (...) Stress und (...) und dann überlegst du, wenn du auch gute Mensch bist. #00:55:07-8#

I: hm (bejahend) #00:55:07-8#

B: Dann baust du irgendwann Scheiße. Dann du kannst auch nicht (unv.).  #00:55:11-8#

I: hm (bejahend) #00:55:12-3#

B: Du gehst in die Straße, und keiner (unv.) eine dich? Und du hast ganz viele Stress im Kopf? #00:55:21-0#

I: hm (bejahend) #00:55:21-0#

B: Und du machst gleich ein Fehler! Und schlagst Ihn. #00:55:21-5#

I: hm (bejahend) #00:55:22-1#

B: Oder keine Ahnung, du gibst eine Antwort. Dann? So ist das. #00:55:28-0#

I: Dann stheht's in den Nachrichten. #00:55:28-9#

B: Genau, dann steht in der Nachricht: 'Ja, Ausländer hat schon Deutschen (...) ja, geschlagen', oder keine Ahnung. So. #00:55:40-0#

I: Ja, natürlich. wenn du (...) Tatsächlich, wenn ma keine Aufenthaltsgenehmigung hat, ähm(...) dann is man quasi noch in Deutschland und wartet auf die Abschiebung. #00:55:51-9#

B: Genau. #00:55:53-9#

I: Ab(...) und, diese Deadline kommt immer näher und man kann nichts machen! #00:55:57-6#

B: hm (bejahend). Du kannst nicht dich konzentrieren, du sagst jeden Tag: 'Ja, ob ich bleiben darf? Ob Ihr, ob (...) keine Ahnung. Ich bin dabeim. Vielleicht, die kommen. Die Bullen kommen jetzt und (...) die nehmen mich fest und die schieben mich ab! Nach Afghanistan.' Du bist garnicht, du fühlst dann nicht (...) du fühlst dich garnicht gut. #00:56:16-5#

I: Ne. #00:56:16-5#

B: Ja. Ist schon passiert schon bei dem, bei anderen Leute. Und (...) #00:56:22-3#

I: // Kennst du (...) kennst du Leute, denen des passiert is? #00:56:24-4#

B: (...) Ich kenne schon eine, ja genau. Der war schon ähm (...) in Regensburg, und die hat schon nach Re(...) ähm, nach Nürnberg gegangen. #00:56:35-9#

I: hm (bejahend) #00:56:36-5#

B: Und (...) zwölf Uhr in der Nacht (...) drei Uhr in der Nacht! Polizei (...) sind gekommen und die haben schon (...) den festgenommen und (...) abgeschoben nach (...) Afghanistan. #00:56:50-6#

I: Scheiße. #00:56:52-1#

B: Dann hörst du solche Geschichten, und dann hast du viel Angst. (...) #00:56:57-0#

I: Ja. (...) Aber hey, Afghanistan ist sicher! (...) #00:56:59-3#

B: Ja, die sagen einfach sicher. (...) #00:57:04-1#

I: Fuck. #00:57:05-3#

B: Die drei (...) drei Jahren, und dann (...) echt viel Leute gestorben. Ganz, ganz, ganz viele Leute gestorben. #00:57:11-9#

I: Ja. (...) ich versteh's nicht. #00:57:19-1#

B: Ich verstehe auch nicht. Weil (...) keine Ahnung #00:57:23-0#

I: (...) #00:57:25-5#

B: Die anderen machen Geschäft und wir sind im Arsch.(...) #00:57:42-0#

I: (...)ähm(...) des wär dann auch schon direkt das Letzte: ähm (...) mit wem bist du denn momentan noch so direkt im Kontakt? Also natürlich die Leut aus'm Jugendcafé. Hast du auch noch Kontakt in den Iran zu deiner Familie oder so? #00:58:10-3#

B: In die Iran zu meine Familie? Ja, klar. Wir reden schon telefonisch. #00:58:13-4#

I: Ihr könnt auch ganz normal telefonnieren? #00:58:17-1#

B: Ganz normal telefonnieren. #00:58:19-0#

I: Okay, weil (...) ähm, aus nem andren hab ich's mitbekommen, dass (...) der sich Gedanken gemacht hat, dass zum Beispiel des Telefonnat dann verwanzt ist? Weil's im Iran wohl teilweise, ähm(...) auf der Suche nach den IS-Leuten #00:58:41-1#

B: hm (bejahend) #00:58:41-1#

I: ähm(...) Telefonnate abgehört werden, und wenn's ähm, jemanden der illegal dorten lebt, ähm #00:58:50-4#

B: Ja. Es ist auch so. Es ist auch so, weil (...) als (...) eine Afghaner. Das ist (...) als eine Afghaner, du lebst dort und (...) du lebst auch illegal #00:59:05-8#

I: hm (bejahend) #00:59:05-8#

B: Wenn die Polizei dich sehen und festnehmen, dann sagen: 'Ja, wir geben dich Geld. Schicken wir dich nach Syrien. Oder nach Irak.' #00:59:20-0#

I: //und dann #00:59:20-0#

B: 'Du kämpfst. Für uns!' #00:59:22-4#

I: ja. #00:59:22-4#

B: 'Gegen. Gegen Leute. Wo dort sind.' Gegen Amerikaner, ne? Das sind Amerikaner oder keine Ahnung. Oder gegen IS, ähm Taliban oder keine Ahnung. Die sagen irgendwas! #00:59:34-3#

I: Ja. #00:59:34-3#

B: Die erzählen (...) irgendwelche Geschichte und sagen: 'Ja, du musst dort gehen, und du musst kämpfen, und ja' Möchten natürlich nicht gehen! 'Oh? ähm, du gehst in Gefängnis! (...) vier bis fünf Monaten. Dann, schieben wir dich nach Afghanistan danach.' (...) #00:59:55-5#

I: Also, entweder du ziehst in den Krieg, an die Front. Oder du gehst in's Gefängnis. #01:00:01-2#

B: Genau. #01:00:02-7#

I: verdammte Scheiße. (...) #01:00:06-9#

B: Du kannst garnicht machen. Du hast keine Rechte. #01:00:10-6#

I: hm (bejahend) #01:00:10-6#

B: Du hast garkeine Rechte. (...)  #01:00:14-8#

I: Ich glaub ich wär' in der Situation (...) wär ich auch, wär ich auch abgehauen. (...) #01:00:20-4#

B: Ja, das kommt drauf an. (lacht) Wo du lebst. #01:00:25-5#

I: Ja. (...) Aber du kannst mit deinen Leuten Zuhause ganz normal telefonnieren (...) #01:00:37-1#

B: Ich kann schon ganz normal telefonnieren. #01:00:36-3#

I: Das ist gut (...) #01:00:40-1#

B: Ja. #01:00:41-0#

I: ähm, mit wem bist du denn dann hier in Deutschland sonst noch in Kontakt? Also, vermutlich Campus Asyl denk ich mal? #01:00:50-3#

B: Campus Asyl gehe ich nicht. #01:00:51-3#

I: Ne? #01:00:52-8#

B: Ne, war ich garnicht in Campus Asyl. Und (...) ja. Nur Jugendcafé. #01:01:02-1#

I: hm (bejahend) #01:01:02-1#

B: Manche Leute aus Jugendcafé. Und ich kenne schon auch (...) paar Leute. (...) Auch in der Stadt. In Regensburg. Ich kenne schon viele Leute in Regensburg. Deutsche, Ausländer, alle. Kenne ich. #01:01:19-9#

I: Ja. #01:01:20-8#

B: Die gute Leute habe ich schon immer Kontakt. (lacht) Die schlechte Leute nicht. #01:01:22-0#

I: (lacht) Genau so musst du's machen! #01:01:27-4#

B: Ja. (...) Ja. (...) #01:01:33-2#

I: Dann glaub ich, ähm, lass ma's jetzt hier direkt mal. Danke, dass du (...) ähm, dass du mitgemacht hast! #01:01:41-0#

B: Gerne, gerne. #01:01:49-9# 