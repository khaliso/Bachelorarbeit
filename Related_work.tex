\section{Theoretische Rahmenbedinungen}

\subsection{Aktueller Forschungsstand}

\subsubsection{Definitionen}

Zunächst die Definitionen einiger Begrifflichkeiten, welche im Rahmen der Arbeit wiederholt aufgegriffen werden.\newline

\textbf{Refugees - Geflüchtete}

\begin{quote}
    \textit{Refugees are people fleeing conflict or persecution. They are defined and protected in international law, and must not be expelled or returned to situations where their life and freedom are at risk.}\cite{unhcr2017refugees}
%\caption{Definition der UNHCR}
\end{quote}
\centerline{\textit{Definition der UNHCR}}

Gemäß der Definition der Vereinten Nationen ist unter einem Geflüchteten jede Person zu verstehen, welche vor Konflikt oder Verfolgung flieht. Sie steht unter dem Schutz des internationalen Rechts und darf nicht in eine Situation entlassen werden, in der Leben oder Freiheit nicht gesichert sind.\newline


\textbf{Asylum seekers - Asylsuchende}

\begin{quote}
    \textit{An asylum-seeker is someone whose request for sanctuary has yet to be processed.}\cite{unhcr2015asylum}
%\caption{Definition der UNHCR}
\end{quote}
\centerline{\textit{Definition der UNHCR}}


Ein Asylsuchender ist eine Person, deren Antrag auf Asyl in einem Land noch nicht fertig bearbeitet wurde.

\textbf{Migrants - Migranten}

\begin{quote}
    \textit{Migrants choose to move not because of a direct threat of persecution or death, but mainly to improve their lives by finding work, or in some cases for education, family reunion, or other reasons. Unlike refugees who cannot safely return home, migrants face no such impediment to return. If they choose to return home, they will continue to receive the protection of their government.}\cite{unhcr2016migrant}
%\caption{Definition der UNHCR}
\end{quote}
\centerline{\textit{Definition der UNHCR}}

Ein Migrant zieht nicht aufgrund Verfolgung oder Lebensgefahr weiter, sondern um eine Verbesserung der Lebensbedingungen zu erreichen. Anders als Geflüchtete, deren Sicherheit am Herkunftsort nicht gewährleistet ist, genießt ein Migrant weiter den Schutz des Herkunftsstaates.
Der Akt der Migration beschreibt die Handlung, von einem Ort zu einem anderen zu ziehen - innerhalb eines Landes oder über Grenzen hinweg.

\textbf{Immigrants - Immigranten}

Ein Immigrant ist ein Migrant, welcher sich dazu entschlossen hat, permanent in einem Land zu verweilen.

\textbf{Integration}


Oduntan
The UN convention is operated in an asylum system in
which refugees arrive under different elements. The
systems incorporate varied levels of access to provisions
including support, benefits, work entitlements and rights to
remain. Thus refugee integration starts on arrival in the host
country through the transition process.


\textbf{Informationslücke}

Information Gap:
\cite{dervin2003sense}sense-making incorporates a repertoire of potenteial procedures for accomplishing the goals discussed preciously. All these are drawn from a xenral metaphor, shown in Figure 12.2(Dervin's dings einfügen). Here, one sees a human moving across time and space, facing a gap, building a bridge across the gap, and then constructing and evaluating the uses of the bridge. This metaphor rests on a descontinuity assumption - that gappiness is pervasive both in and between moments in time and space and in and between people. Gappiness is assumed to occur because of differences across time (e.g. self today vs self yesterday and scientific fact today vs. scientific fact tomorrow) and across space (e.g. the experiece of a particular condition on different cultures, contexts, communites, material circumstances and the sense of an experience physically and the articulation of it verbally)
continue on Dervin p.238


\subsubsection{Stand der Literatur}
%%%%%%%%%%%%%%%%%%%%%%%%%%%%%%%%%%%%%
%LINA'S
%%%%%%%%%%%%%%%%%%%%%%%%%

Oduntan und Ruthven stellten 2017 im Informationsprozess von Gefl\"uchteten im Asylauswahlverfahrensprozess des Vereinigten K\"onigreichs informationsl\"ucken fest. \cite{oduntan2017investigating}
Mittels semi-strukturierter Interviews nach Dervin's Sense-making Methodology\cite{dervin2003sense} erfolgte eine Zusammenstellung verschiedener Situationen, in denen diese im Rahmen der Studie aufgedeckt wurden - mit dem Hinweis auf die Existenz noch unergr\"undeter Situationen in der Gefl\"uchtetenintegration.\newline
Informationsl\"ucken treten gem\"a\ss{} Derwin aufgrund Abweichungen in Zeit (Ich heute vs. Ich gestern) und Raum (Erfahrung aus dem Blickwinkel verschiedener Kulturen, Werte etc.) auf.\cite{dervin2003sense}\newline
Das Ziel dieser Arbeit ist es, Informationslücken im Integrationsprozess in Deutschland zu identifizieren und untersuchen.\newline
Die Untersuchung fand im Jahr 2019 statt, in der Folgezeit der sogenannten Fl\"uchtlingskrise, welche Europa ab 2015 direkt betraf. \cite{unhcr2015seven}\newline
Hierzu wurde das Informationsverhalten von Gefl\"uchteten untersucht, um Informationsbed\"urfnisse in der deutschen Fl\"uchtlingsintegration festzustellen.\newline
In der Literatur wurden bereits zahlreiche Informationsbedürfnisse und Informationslücken offengelegt\cite{oduntan2017investigating}:
Die Informationsbedürfnisse von Migranten in Queens, NY, wurden von Fisher et al. in Sozialen, Kulturellen und Fähigkeitsbasierten Informationsbedürfnissen kategorisiert. \cite{fisher2004information}
Courtright identifizierte des weiteren medizinische Betreuung, Bildung, Wohnen und Berufstätigkeit bei Lateinamerikanischen Immigranten in Indiana, US. \cite{courtright2005health}
Silvio untersuchte südsudanesische Immigranten in Ontario, US; Ihre Informationsbedürnisse waren auf Bildung, medizinische Betreuung, Berufstätigkeit und politische Informationen bezogen, und suchten nach Mitteln und Wegen, angebracht mit Rassismus umzugehen.\cite{hakim2006information}\newline
Lloyd\cite{lloyd2014building} und Lloyd et al.\cite{lloyd2013connecting}'s Arbeit sind spezifisch auf Geflüchtete und Asylsuchende fokussiert. Lloyd ermittelte Informationsbedürfnisse im Gesundheitssektor, und Lloyd et al. entdeckte Informationsbedürfnisse im Alltag und der Anpassung an soziale Normen; beide Arbeiten beschäftigten sich mit Mündigkeit im Umgang mit Informationen.\newline
All diese Arbeiten sind Personenzentriert; es wird mit relativ kleinen Populationen gearbeitet, bei denen die Informationsbedürfnisse der Individuen untersucht werden. Ausgehend von diesen festgestellten Informationsbedürfnissen kann die Perspektive auf eine größere Population erweitert werden. So können Zusammenhänge zwischen allgemeinem menschlichem Informationsverhalten (\textit{Human Information Behaviour}) und für Geflüchtete und Asylsuchende spezifische Informationsfaktoren herstellt werden.\cite{oduntan2017investigating}\newline
Dies ist insofern wichtig, da etwa Chatman feststellte, dass sich das Informationsverhalten von Randgruppen oft signifikant von dem der Bev\"olkerungsmehrheit unterscheidet. \cite{chatman1996impoverished}
\newline
\newline 
%ICT's
Um den Integrationsprozess zu erleichtern, gab es mehrere Projekte, mit sogenannten ICT's (Information and communications technology) die Geflüchteten zu unterstützen. Andrade et al. stellten fest, dass ICT's  f\"unf Aspekte der Gefl\"uchteten im Hinblick auf soziale Inklusion positiv beeinflussen:
\begin{enumerate}
    \item   Teilnahme an der Gesellschaft
    \item   Effizientere Kommunikation
    \item   Das Verst\"andnis der neuen Gesellschaft
    \item   Pflege sozialer Kontakte
    \item   Ausdr\"ucken einer eigenen kulturellen Identit\"at
\end{enumerate}
Diesbez\"uglich gibt es weitere Ans\"atze:\newline
Schreieck et al. etwa schufen ein Design f\"ur mobile Applikationen, das an verschiedene Kulturkreise angepasst werden kann. \cite{schreieck2017supporting}\newline
Jones et al. arbeiteten an einem anderen Ansatz: Sie kreierten ein Zuweisungssystem, in dem Gefl\"uchtete und entweder Staaten oder lokale Areale aneinander verwiesen wurden. Hierbei m\"ussen beide Parteien einer Zuweisung zustimmen. \cite{jones2017matching}\newline
Um Projekte wie diese zu verbessern, muss auch erschlossen werden, welche Informationsbed\"urfnisse die jeweiligen Zielgruppen erwarten.\newline
%___________________________________________________________________________________________

In einer Studie bezüglich der Problematiken bei der
psychologischen und sozio-kulturellen Anpassung von russischsprachigen Immigranten_innen in Neuseeland zeigte sich, dass
die Motivation zur Migration einen der größten Einflüsse auf den Prozess
der Sozialisation besitzt (Maydell-Stevens, Masgoret & Ward, 2007).

Immigranten_innen stellen eine distinkte Gruppe mit spezifischen
Informationsbedürfnissen dar, mit allgemeinen Bedürfnissen zu Beginn
der Einwanderung hin zu individuellen, spezifischen Bedürfnissen im
Laufe der Integrationszeit (Shoham & Strauss, 2008).