\section{Theoretische Rahmenbedinungen}

\subsection{Aktueller Forschungsstand}


\subsubsection{Stand der Literatur}

Oduntan und Ruthven stellten 2017 im Informationsprozess von Gefl\"uchteten im Asylauswahlverfahrensprozess des Vereinigten K\"onigreichs informationsl\"ucken fest \citep{oduntan2017investigating}.\newline
Mittels semi-strukturierter Interviews erfolgte eine Zusammenstellung verschiedener Situationen, in denen diese im Rahmen der Studie aufgedeckt wurden - mit dem Hinweis auf die Existenz noch unergr\"undeter Situationen in der Gefl\"uchtetenintegration.\newline
In deren Arbeit wurden Informationslücken auf dem Gebiet abgelehnter Asylgesuche untersucht. Diese wurden in 
\begin{enumerate}
    \item Juristische Informationslücken
    \item Informationslücken bezüglich der Wohnsituation
    \item Informationslücken im Bereich der Bildung
    \item Informationslücken im sozialen Bereich
\end{enumerate}
kategorisiert. Außerdem wurden die verwendeten Informationsquellen identifiziert. Die vorliegende Studie orientierte sich an dieser Kategorisierung.\newline
Informationsl\"ucken treten gem\"a\ss{} Derwin aufgrund Abweichungen in Zeit (Ich heute vs. Ich gestern) und Raum (Erfahrung aus dem Blickwinkel verschiedener Kulturen, Werte etc.) auf \citep{dervin2003sense}.\newline
Hier wurde das Informationsverhalten von sechs Gefl\"uchteten und Asylsuchenden untersucht, um Informationsbed\"urfnisse in der deutschen Fl\"uchtlingsintegration festzustellen.\newline
In der Literatur wurden bereits zahlreiche Informationsbedürfnisse und -lücken offengelegt:\newline
Die Informationsbedürfnisse von Migranten in Queens, NY, wurden von Fisher et al. in sozialen, kulturellen und fähigkeitsbasierten Informationsbedürfnissen kategorisiert \citep{fisher2004information}.\newline
Courtright identifizierte des weiteren medizinische Betreuung, Bildung, Wohnen und Berufstätigkeit bei lateinamerikanischen Immigranten in Indiana, US \citep{courtright2005health}.\newline 
Silvio untersuchte südsudanesische Immigranten in Ontario, US; Ihre Informationsbedürnisse waren auf Bildung, medizinische Betreuung, Berufstätigkeit und politische Informationen bezogen und suchten nach Mitteln und Wegen, angebracht mit Rassismus umzugehen \citep{hakim2006information}.\newline
Lloyd\citep{lloyd2014building} und Lloyd et al.\citep{lloyd2013connecting}'s Arbeiten sind spezifisch auf Geflüchtete und Asylsuchende fokussiert. Lloyd ermittelte Informationsbedürfnisse im Gesundheitssektor, und Lloyd et al. entdeckte Informationsbedürfnisse im Alltag und der Anpassung an soziale Normen; beide Arbeiten beschäftigten sich mit Mündigkeit im Umgang mit Informationen.\newline
Mykyttschak beschäftigte sich 2018 mit den Informationsbedürfnissen von minderjährigen Geflüchteten in Deutschland \citep{mykyttschak2018}. 
Gemäß Mykyttschak lassen sich deren Informationsbedürfnisse in Grundbedürfnisse, Befürfnisse bezüglich Freizeit, Arbeit und Bildung und persönliche Interessen untergliedern. Als Informationsquellen dienten im Besonderen personelle Quellen sowie das Internet. Die Geflüchteten wandten sich im personellen Bereich an 
\begin{enumerate}
    \item  von Ihnen als \textbf{vertrauenswürdig} eingeschätzten Personen \textit{(Betreuer, Familie)}
    \item Personen in einer \textbf{ähnlichen Lebenslage} \textit{(Freunde, Mitbewohner, Landsleute)} \item\textbf{Experten} \textit{(Lehrer, Ärzte, Deutsche)} 
\end{enumerate}
und im digitalen an
\begin{enumerate}
    \item Suchmaschinen
    \item Soziale Medien
    \item Wikipedia
\end{enumerate}
All diese Arbeiten sind personenzentriert: es wird mit relativ kleinen Populationen gearbeitet, bei denen die Informationsbedürfnisse der Individuen untersucht werden. Ausgehend von diesen festgestellten Informationsbedürfnissen kann die Perspektive auf eine größere Population erweitert werden. So können Zusammenhänge zwischen allgemeinem menschlichem Informationsverhalten und für Geflüchtete und Asylsuchende spezifische Informationsfaktoren herstellt werden \citep{oduntan2017investigating}.\newline
Dies ist insofern wichtig, da etwa Chatman feststellte, dass sich das Informationsverhalten von Randgruppen oft signifikant von dem der Bev\"olkerungsmehrheit unterscheidet. \citep{chatman1996impoverished}\newline \newline 
%ICT's
Um den Integrationsprozess zu erleichtern, gab es mehrere Projekte, mit sogenannten ICT's \textit{(Information and communications technology)} die Geflüchteten zu unterstützen. Andrade et al. stellten fest, dass ICT's  f\"unf Aspekte der Gefl\"uchteten im Hinblick auf soziale Inklusion positiv beeinflussen:
\begin{enumerate}
    \item   Teilnahme an der Gesellschaft
    \item   Effizientere Kommunikation
    \item   Das Verst\"andnis der neuen Gesellschaft
    \item   Pflege sozialer Kontakte
    \item   Ausdr\"ucken einer eigenen kulturellen Identit\"at
\end{enumerate}
Diesbez\"uglich gibt es weitere Ans\"atze:\newline
Schreieck et al. etwa schufen ein Design f\"ur mobile Applikationen, das an verschiedene Kulturkreise angepasst werden kann. \citep{schreieck2017supporting}\newline
Jones et al. arbeiteten an einem anderen Ansatz: Sie kreierten ein Zuweisungssystem, in dem Gefl\"uchtete und wahlweise Staaten oder lokale Areale aneinander verwiesen wurden. Hierbei m\"ussen beide Parteien einer Zuweisung zustimmen. \citep{jones2017matching}\newline
Um Projekte wie diese zu verbessern, muss auch erschlossen werden, welche Informationsbed\"urfnisse die jeweiligen Zielgruppen erwarten.\newline

Durch einen Mangel an relevanten Informationen wird ein Ausschluss aus der Gesellschaft riskiert, in die die Gefl\"uchteten integriert werden sollen. \citep{andrade2016information}\newline
Als relevant werden in dieser Arbeit alle Informationen gewichtet, die das Leben in der neuen Umgebung beeinflussen - von einem Termin beim \"ortlichen Arzt \"uber Kenntnis der eigenen Rechtslage bis hin zur Grundkenntnis der \"ortlichen Kultur und Gesellschaft. \citep{schreieck2017supporting}\newline