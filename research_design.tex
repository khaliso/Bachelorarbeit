
\section{Aufbau der Studie}
Diese Studie arbeitet wie Oduntan et al.\cite{oduntan2017investigating} mit dem von Dervin's \textit{Sense-making methodology} beeinflussten Sozio-Kognitiven Ansatz nach Bates\cite{bates2005introduction}.

\subsection{Akquirieren der Teilnehmer}
Die Interviewten, welche in Afghanistan oder dem Iran geboren wurden, frequentieren in regelmäßigen Abständen das EJSA - Jugendcafé Regensburg. Als Vorbereitung auf die Interviews wurde Zeit vor und zwischen den Interviews genutzt, um mit den Probanden Vertrauensverhältnis aufzubauen, i.e. Nachhilfestunden und Partizipation an den Aktivitäten des Jugendcafé.\newline
Der Partizipant aus Syrien erklärte sich über persönliche Kontakte bereit, an der Studie teilzunehmen.

\subsection{Demographische Angaben}

Die demographischen Informationen sind in Tabelle 1 zusammengefasst. Sie betreffen das Geschlecht, das Alter, das Geburtsland, das Jahr der Ankunft in Deutschland, den Bildungsgrad vor der Flucht, den aktuellen Bildungsweg, den aktuellen Asylstatus der Geflüchteten sowie die bewältigte Fluchtroute.

\begin{table}[h!]
  \begin{center}
    \caption{Überblick}
    \label{tab:table1}
    \begin{tabular}{l|S|r|l}
      \textbf{Kategorie} & \textbf{Anzahl}\\
%      $\alpha$ & $\beta$ & $\gamma$ & $\delta$ \\ % <--
        \newline
      \textit{Geschlecht}\\
      \hline
      männlich & 6\\
      weiblich & 0\\
      \hline
      \newline
      \textit{Alter}\\
      \hline
      23 Jahre & 1\\
      21 Jahre & 2\\
      20 Jahre & 1\\
      19 Jahre & 2\\
      \hline
      \newline
      \textit{Geburtsland}\\
      \hline
      Afghanistan & 3\\
      Iran & 2\\
      Syrien & 1\\
      \hline
      \newline
      \textit{Ankunftsjahr in Deutschland}\\
      \hline
      2015 & 6\\
      \hline
      \newline
      \textit{Bildungsgrad vor der Flucht}\\
      \hline
      Akademiker & 1\\
      Berufliche Ausbildung & 2\\
      Rudimentäre Bildung & 2\\
      Keine Vorbildung & 1\\
      \hline
      \newline
      \textit{Momentaner Bildungsweg}\\
      \hline
      Ausbildung & 4\\
      Praktikum & 1\\
      Schule & 1\\
      \hline
      \newline
      \textit{Asylstatus}\\
      \hline
      Laufendes Asylverfahren & 4\\
      Aufenthaltserlaubnis  & 1\\
      Duldung & 1\\
      \hline
      \newline
      \textit{Fluchtroute}\\
      \hline
      Land, LKW & 2\\
      Seeweg & 3\\
      Unbekannt & 1 (IT3)\\
      \end{tabular}
  \end{center}
\end{table}
Von den sechs Teilnehmern der Studie waren alle männlich. Sie waren zum Zeitpunkt des jeweiligen Interviews zwischen 19 und 23 Jahre alt (Mittelwert 20,5 Jahre). Drei der Interviewten wurden in Afghanistan geboren, zwei im Iran und einer in Syrien. \newline 
Hier muss erwähnt werden, dass die beiden im Iran geborenen Interviewten in afghanischen Familien geboren wurden und von iranischen Autoritäten als Afghaner betrachtet wurden.\newline
Alle Probanden wurden im Lauf des Jahres 2015 in Deutschland als Asylsuchende oder Geflüchtete registriert. Der als Akademiker beschriebene Geflüchtete hatte vor Seiner Flucht an der Universität in Aleppo studiert, die Hochschulreife wurde jedoch in Deutschland nicht anerkannt.\newline
Unter rudimentärer Bildung werden in dieser Studie bis zu zwei Jahre Schulbesuch, der Besuch einer Koranschule, oder das erlernen der Schrift durch eigenen Antrieb gewertet.\newline
Unter beruflicher Ausbildung werden mehr als zwei Jahre Aktivität in einem Berufsfeld vor der Flucht verstanden.\newline
Zum Zeitpunkt der respektiven Interviews befanden sich vier der Teilnehmer in einer beruflichen Ausbildung nach deutschem Verständnis.\newline Einer befand sich in einem Praktikum als Vorbereitung auf seine Ausbildung, und einer hatte bereits eine Ausbildung angefangen, welche aus medizinischen Gründen abgebrochen wurde. Er befindet sich nun in der Vorbereitung auf seinen Qualifizierten Hauptschulabschluss.\newline
Die gewählte Fluchtroute konnte von fünf der sechs Teilnehmer in Erfahrung gebracht werden. Zwei flohen auf der Ladefläche eines LKW nach Europa. Die anderen drei erreichten auf dem Seeweg Europa.