\section{Aufbau der Studie}
Diese Studie arbeitet wie Oduntan et al. \citep{oduntan2017investigating} mit dem von Dervin's \textit{Sense-making methodology} beeinflussten sozio-kognitiven Ansatz nach Bates \citep{bates2005introduction}.

\subsection{Ethische Richtlinien}

Bei der Durchführung der Interviews wurde darauf geachtet, den Interviewten die Möglicheit zu geben, über sowohl Zeit als auch Ort der Interviews zu verfügen. Die Interviewten wurden vor Beginn darauf hingewiesen, dass Sie das Interview zu jedem Zeitpunkt abbrechen können und Ihre Anonymität gewährleistet ist.\newline

\subsection{Demographische Angaben}

Die demographischen Informationen sind in Tabelle 1 zusammengefasst. Sie betreffen das Geschlecht, das Alter, das Geburtsland, das Jahr der Ankunft in Deutschland, den Bildungsgrad vor der Flucht, den aktuellen Bildungsweg, den aktuellen Asylstatus der Geflüchteten sowie die bewältigte Fluchtroute.

\begin{table}[h!]
  \begin{center}
    \caption{Überblick}
    \label{tab:table1}
      \begin{tabular}{l|c|}
      \textbf{Kategorie} & \textbf{Anzahl}\\
%      $\alpha$ & $\beta$ & $\gamma$ & $\delta$ \\ % <--
        \newline
      \textit{Geschlecht}\\
      \hline
      männlich & 6\\
      weiblich & 0\\
      \hline
      \newline
      \textit{Alter}\\
      \hline
      23 Jahre & 1\\
      21 Jahre & 2\\
      20 Jahre & 1\\
      19 Jahre & 2\\
      \hline
      \newline
      \textit{Geburtsland}\\
      \hline
      Afghanistan & 3\\
      Iran & 2\\
      Syrien & 1\\
      \hline
      \newline
      \textit{Ankunftsjahr in Deutschland}\\
      \hline
      2015 & 6\\
      \hline
      \newline
      \textit{Bildungsgrad vor der Flucht}\\
      \hline
      Akademiker & 1\\
      Berufliche Ausbildung & 2\\
      Rudimentäre Bildung & 2\\
      Keine Vorbildung & 1\\
      \hline
      \newline
      \textit{Momentaner Bildungsweg}\\
      \hline
      Ausbildung & 4\\
      Praktikum & 1\\
      Schule & 1\\
      \hline
      \newline
      \textit{Asylstatus}\\
      \hline
      Laufendes Asylverfahren & 3\\
      Aufenthaltserlaubnis  & 2\\
      Duldung & 1\\
      \hline
      \newline
      \textit{Fluchtroute}\\
      \hline
      Land, LKW & 2\\
      Seeweg & 3\\
      Unbekannt & 1 (IT3)\\
      \end{tabular}
  \end{center}
\end{table}
Von den sechs Teilnehmern der Studie waren alle männlich. Sie waren zum Zeitpunkt des jeweiligen Interviews zwischen 19 und 23 Jahre alt (Mittelwert 20,5 Jahre). Drei der Interviewten wurden in Afghanistan geboren, zwei im Iran und einer in Syrien. \newline 
Hier muss erwähnt werden, dass die beiden im Iran geborenen Interviewten in afghanischen Familien geboren wurden. Die iranischen Autoritäten betrachteten sie als Afghaner.\newline
Alle Probanden wurden im Lauf des Jahres 2015 in Deutschland als Asylsuchende oder Geflüchtete registriert. Der als Akademiker beschriebene Geflüchtete hatte vor seiner Flucht an der Universität in Aleppo studiert, die Hochschulreife wurde jedoch in Deutschland nicht anerkannt.\newline
Unter rudimentärer Bildung werden in dieser Studie bis zu zwei Jahre Schulbesuch, der Besuch einer Koranschule, oder das Erlernen der Schrift durch eigenen Antrieb gewertet.\newline
Unter beruflicher Ausbildung werden mehr als zwei Jahre Aktivität in einem Berufsfeld vor der Flucht verstanden.\newline
Zum Zeitpunkt der respektiven Interviews befanden sich vier der Teilnehmer in einer beruflichen Ausbildung nach deutschem Verständnis.\newline Einer befand sich in einem Praktikum als Vorbereitung auf seine Ausbildung, und einer hatte bereits eine Ausbildung angefangen, welche aus medizinischen Gründen abgebrochen wurde. Er befindet sich nun in der Vorbereitung auf seinen Qualifizierenden Mittelschulabschluss.\newline
Die gewählte Fluchtroute konnte von fünf der sechs Teilnehmer in Erfahrung gebracht werden. Zwei flohen auf der Ladefläche eines LKW nach Europa. Die anderen drei erreichten auf dem Seeweg Europa.

\subsection{Akquisition der Teilnehmer}
Die Interviewten, welche in Afghanistan oder dem Iran geboren wurden, frequentieren in regelmäßigen Abständen das EJSA - Jugendcafé Regensburg. Als Vorbereitung auf die Interviews wurde Zeit vor und zwischen den Interviews genutzt, um mit den Probanden Vertrauensverhältnis aufzubauen, i.e. Nachhilfestunden und Partizipation an den Aktivitäten des Jugendcafés.\newline
Der Partizipant aus Syrien erklärte sich über persönliche Kontakte bereit, an der Studie teilzunehmen.

\subsection{Durchführung der Interviews}

Ähnlich vorausgegangener Studien zur Erforschung und Dokumentierung des Informationsverhaltens von Immigranten \citep{courtright2005health}\citep{oduntan2017investigating}\citep{mykyttschak2018}\citep{lingel2011information}\citep{shoham2008immigrants} wurden die Daten mittels retrospektiver Tiefeninterviews erhoben. Es wurden sechs Interviews durchgeführt, wobei die Interviewten der Interviews IT2 bis IT6 den Interviewer bereits aus dem EJSA - Jugendcafé in Regensburg kannten. Somit konnte eine Vertrauensbasis hergestellt werden.\newline
Die sechs Interviews werden in der Arbeit mit dem Kürzel IT, gefolgt von der Nummer des Transkripts, angesprochen.\newline
IT1 (15.04.) fand in einem Kaffeehaus in zentraler Lage statt. Das Interview wurde an einem Tisch im Außenbereich nahe der Fußgängerzone durchgeführt. Eine neutrale und ungestörte Atmosphäre war gewährleistet. Das Interview fand im zweiten Anlauf statt. Der Interviewte und der Interviewee kannten sich vor dem treffen nicht, der Kontakt wurde über gemeinsame Bekannte hergestellt. Das Interview ist in zwei Teile, IT1.1 und IT1.2 aufgeteilt.\newline
IT2 (23.04.) fand in der Wohnung des Interviewten statt; hier konnte schon im Voraus eine persönliche Beziehung auf Vertrauensbasis über Nachhilfeunterricht und das EJSA - Jugendcafé aufgebaut werden. Das Interview ist in zwei Teile, IT2.1 und IT2.2, aufgeteilt.\newline
Interview IT3 (02. 05.) fand im EJSA - Jugendcafé in Regensburg im allgemeinen Aufenthaltsraum statt. Es kam zu Störungen, da das Interview in der direkten Umgebung des Freundeskreises des Interviewten durchgeführt wurde. Dieser Ort wurde auf ausdrücklichen Wunsch des Interviewten gewählt.\newline
Interview IT4 (03.05.) fand in der Wohnung des Probanden statt. Gelegentlich besuchten dessen Mitbewohner das Interview, allerdings ohne zu unterbrechen. Das Interview wurde so gelegt, dass es keine Überschneidung mit dem am 05.05. beginnenden Fastenmonat Ramadan hatte.\newline
IT5 fand am 09.05. in einem Nebenraum des EJSA - Jugendcafés statt. Der Interviewte kannte den Interviewer bereits und erklärte sich spontan bereit, das Interview durchzuführen.
IT6 fand im zweiten Anlauf am 15.05. statt. Es wurde in der Wohnung des Interviewten durchgeführt, nachdem dieser von seiner Ausbildungsstelle zurückkehrte. IT6 ist gläubiger Moslem und orientiert sich strikt an den Regeln des Ramadan. Das heißt, dass er nach Sonnenuntergang und vor Sonnenaufgang isst und trinkt.\newline
Die Interviews dauerten im Durchschnitt 51 Minuten und 35 Sekunden.\newline

%IT1 :	46:58
%IT2: 	67:05
%IT3:	27:47
%IT4:	35:05
%IT5:	73:19
%IT6: 	61:31

%Average:			51:57.5
%total				5:11:45
%max/mindeviation	45:32

Der Interviewleitfaden, an dem sich orientiert wurde, befindet sich in Anhang 2. In diesem wurden zunächst allgemeine Fragen gestellt, gefolgt von Fragen bezüglich der Forschungsfrage .\newline
Die behandelten Forschungsfragen im Leitfaden bezogen sich auf Informationslücken   seit der Ankunft in Deutschland. Als Orientierung dienten die von Oduntan et al. festgestellten Informationslücken von Geflüchteten im Vereinigten Königreich\citep{oduntan2017investigating}.\newline
Diese Fragen wurden nach dem ersten Interview um den Aspekt der Auswirkungen des regionalen Dialektes auf die Integration ergänzt.\newline
Die Vorgehensweise zur Analyse einer Informationslücke nach Dervin wurde zu Veranschauungszwecken hinzugefügt, wurde jedoch während der Interviews nicht benötigt.\citep{dervin2003sense}

\subsection{Methodik}

Mittels semi-strukturierter Interviews werden die Erlebnisse der Teilnehmer unter deutscher Jurisdiktion ergr\"undet. Dabei werden mittels Bates' \textit{socio-cognitive approach}\citep{bates2005introduction}, welcher von Dervin's sense-making methodology\citep{dervin2003sense} inspiriert wurde, Informationsl\"ucken aufgedeckt.\newline
Mit diesem kann festgestellt werden, wie ein Publikum verschiedene Situationen im Lauf des Lebens logisch begr\"undet hat. Um dies zu erm\"oglichen, wurden die Probanden eingeladen, Erfahrungen und Schlussfolgerungen mit Hilfe des Interviewers so detailliert wie m\"oglich wiederzugeben.\newline
Gem\"a\ss{} Dervin wird dennoch immer davon ausgegangen, dass eine erstellte Kategorisierung falsch, irrelevant oder f\"ur andere Sitautionen unangemessen sein kann.\textit{\textbf{Um diese zu \"uberpr\"ufen, wird zum Dialog zwischen wissenschaftlichen Institutionen und dem allt\"aglichen Leben aufgerufen}}. Diese Arbeit kann Aufschluss auf die Anwendbarkeit der in Oduntan et al.'s Forschung erarbeiteten Kategorien f\"ur den deutschsprachigen Raum geben. Allerdings ist hier zu erwähnen, dass die Testpopulation anders als bei Oduntan et al. nicht nur aus Migranten mit abgelehntem Asylgesuch bestanden.\newline
Ein Interview, bei dem der \textit{Sense-Making Approach} verwendet wird, kann auf dem Ansatz der micro-moment timeline basieren. Bei dieser werden die Interviewten gebeten, eine pers\"onliche Situation mit Bezug zum Forschungsfokus genau zu beschreiben (In diesem Fall: Eine Situation abrufen, in der eine Informationsl\"ucke Auswirkungen auf das Leben des interviewten hatte).
Die von Oduntan et al. festgestellten Kategorien, in denen Informationsl\"ucken geh\"auft auftraten, waren:
\begin{enumerate}
    \item Ablauf des Asylverfahrens\newline
    Auch ein anhaltender Einfluss auf die mentale Gesundheit sowie getroffene Entscheidungen nach der Ablehnung sind hier relevant.
    \item Juristische Komponente \newline
    Beispiel: War dem Gefl\"uchteten klar, dass im Fall einer negativen Asylentscheidung Berufung eingelegt werden kann?
    \item Wohnen \newline
    Wie viele Wohnungen hat der Proband nach seiner Flucht bewohnt, wie lange im Durchschnitt, aus welchen Gr\"unden erfolgten Umz\"uge? Momentane Zufriedenheit mit der Wohnsituation? Wie wurde die momentane Wohnung entdeckt?
    \item Bildung \newline
    Erfahrungen mit dem deutschen Bildungs - und Schulungssystem
    \item Sozial\newline
    Wie ist das soziale Umfeld aufgebaut? Wie steht der Proband in Beziehung zu verschiedenen Demographischen Gruppen?
    \item Informationsquellen\newline
    Welche Informationskan\"ale werden von den Probanden verwendet, um sich zu orientieren und offene Fragen zu beantworten?
\end{enumerate}
Besagte Situation wird in \textit{Time-Line Steps} beschrieben, d.h. was passierte als erstes, zweites, etc. Innerhalb eines Schrittes wird behandelt, welche Fragen sich zu diesem Zeitpunkt bildeten, welche Gedanken und Gef\"uhle den Probanden zu dem Zeitpunkt bewegten.\newline
Hierbei ist die \textit{Sense-Making Metapher} zu ber\"ucksichtigen, mit Hinblick auf Situation, Informationsl\"ucke, Br\"ucke und Resultat \citep{dervin2003sense}.\newline
Dies f\"uhrt zu weiteren Fragen:\newline
\begin{enumerate}
    \item Was f\"uhrte zu dieser Frage?
    \item Was hat Sie mit dem Leben zu tun?
    \item Gesellschaft und Machtverh\"altnisse?
    \item Wurde die Frage beantwortet?
    \item Wie wurde sie beantwortet?
    \item Welche Hindernisse gab es?
    \item War die Antwort hilfreich?
    \item War die Antwort hinderlich?
    \item Auf welche Weise?
\end{enumerate}

\subsection{Auswertungsmethodik}
Die sechs semi-strukturierten Interviews wurden zun\"achst während des Interviews als Audiodatei aufgezeichnet. Diese wurden anschließend transkribiert. Die so generierten Daten wurden thematisch analysiert und codiert. So konnten die einzelnen identifizierten Informationslücken auf Gemeinsamkeiten und Unterschiede \"uberpr\"uft werden. Mittels dieser Vorgehensweise konnten Verbindungen zwischen individuellen und situationellen Faktoren festgestellt werden, womit Informationsl\"ucken in den Erfahrungen ausgemacht werden konnten. Die verschiedenen Kategorien und Themenbereiche wurden gekennzeichnet. Die in der Auswertung wiedergegebenen Zitate wurden immer von den Interviewten, nicht vom Interviewer ausgesprochen. Die Zitate wurden mit dem Kürzel des Interviews versehen, in dem Sie auftraten. Auch die Minute des Zitats wird angegeben. Klarnamen im Transkript, welche auf die Identität eines Interviewten hinweisen könnten, wurden zensiert. Alle Transkripte befinden sich im Anhang dieser Arbeit.