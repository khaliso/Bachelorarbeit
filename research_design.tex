\section{Aufbau der Studie}
Diese Studie arbeitet wie Oduntan et al.\cite{oduntan2017investigating} mit dem von Dervin's \textit{Sense-making methodology} beeinflussten Sozio-Kognitiven Ansatz nach Bates\cite{bates2005introduction}.

\subsection{Akquisition der Teilnehmer}
Die Interviewten, welche in Afghanistan oder dem Iran geboren wurden, frequentieren in regelmäßigen Abständen das EJSA - Jugendcafé Regensburg. Als Vorbereitung auf die Interviews wurde Zeit vor und zwischen den Interviews genutzt, um mit den Probanden Vertrauensverhältnis aufzubauen, i.e. Nachhilfestunden und Partizipation an den Aktivitäten des Jugendcafé.\newline
Der Partizipant aus Syrien erklärte sich über persönliche Kontakte bereit, an der Studie teilzunehmen.

\subsection{Demographische Angaben}

Die demographischen Informationen sind in Tabelle 1 zusammengefasst. Sie betreffen das Geschlecht, das Alter, das Geburtsland, das Jahr der Ankunft in Deutschland, den Bildungsgrad vor der Flucht, den aktuellen Bildungsweg, den aktuellen Asylstatus der Geflüchteten sowie die bewältigte Fluchtroute.

\begin{table}[h!]
  \begin{center}
    \caption{Überblick}
    \label{tab:table1}
      \begin{tabular}{l|c|}
      \textbf{Kategorie} & \textbf{Anzahl}\\
%      $\alpha$ & $\beta$ & $\gamma$ & $\delta$ \\ % <--
        \newline
      \textit{Geschlecht}\\
      \hline
      männlich & 6\\
      weiblich & 0\\
      \hline
      \newline
      \textit{Alter}\\
      \hline
      23 Jahre & 1\\
      21 Jahre & 2\\
      20 Jahre & 1\\
      19 Jahre & 2\\
      \hline
      \newline
      \textit{Geburtsland}\\
      \hline
      Afghanistan & 3\\
      Iran & 2\\
      Syrien & 1\\
      \hline
      \newline
      \textit{Ankunftsjahr in Deutschland}\\
      \hline
      2015 & 6\\
      \hline
      \newline
      \textit{Bildungsgrad vor der Flucht}\\
      \hline
      Akademiker & 1\\
      Berufliche Ausbildung & 2\\
      Rudimentäre Bildung & 2\\
      Keine Vorbildung & 1\\
      \hline
      \newline
      \textit{Momentaner Bildungsweg}\\
      \hline
      Ausbildung & 4\\
      Praktikum & 1\\
      Schule & 1\\
      \hline
      \newline
      \textit{Asylstatus}\\
      \hline
      Laufendes Asylverfahren & 3\\
      Aufenthaltserlaubnis  & 2\\
      Duldung & 1\\
      \hline
      \newline
      \textit{Fluchtroute}\\
      \hline
      Land, LKW & 2\\
      Seeweg & 3\\
      Unbekannt & 1 (IT3)\\
      \end{tabular}
  \end{center}
\end{table}
Von den sechs Teilnehmern der Studie waren alle männlich. Sie waren zum Zeitpunkt des jeweiligen Interviews zwischen 19 und 23 Jahre alt (Mittelwert 20,5 Jahre). Drei der Interviewten wurden in Afghanistan geboren, zwei im Iran und einer in Syrien. \newline 
Hier muss erwähnt werden, dass die beiden im Iran geborenen Interviewten in afghanischen Familien geboren wurden und von iranischen Autoritäten als Afghaner betrachtet wurden.\newline
Alle Probanden wurden im Lauf des Jahres 2015 in Deutschland als Asylsuchende oder Geflüchtete registriert. Der als Akademiker beschriebene Geflüchtete hatte vor Seiner Flucht an der Universität in Aleppo studiert, die Hochschulreife wurde jedoch in Deutschland nicht anerkannt.\newline
Unter rudimentärer Bildung werden in dieser Studie bis zu zwei Jahre Schulbesuch, der Besuch einer Koranschule, oder das erlernen der Schrift durch eigenen Antrieb gewertet.\newline
Unter beruflicher Ausbildung werden mehr als zwei Jahre Aktivität in einem Berufsfeld vor der Flucht verstanden.\newline
Zum Zeitpunkt der respektiven Interviews befanden sich vier der Teilnehmer in einer beruflichen Ausbildung nach deutschem Verständnis.\newline Einer befand sich in einem Praktikum als Vorbereitung auf seine Ausbildung, und einer hatte bereits eine Ausbildung angefangen, welche aus medizinischen Gründen abgebrochen wurde. Er befindet sich nun in der Vorbereitung auf seinen Qualifizierten Hauptschulabschluss.\newline
Die gewählte Fluchtroute konnte von fünf der sechs Teilnehmer in Erfahrung gebracht werden. Zwei flohen auf der Ladefläche eines LKW nach Europa. Die anderen drei erreichten auf dem Seeweg Europa.


\subsection{Ethische Richtlinien}

Bei der Durchführung der Interviews wurde darauf geachtet, den Interviewten die Möglicheit zu geben, über sowohl Zeit als auch Ort der Interviews zu verfügen. Die Interviewten wurden vor Beginn darauf hingewiesen, dass Sie das Interview zu jedem Zeitpunkt abbrechen können und Ihre Anonymität gewährleistet ist.\newline

\subsection{Durchführung der Interviews}


IT1 (12.04.) entschied sich für ein Treffen bei einem Kaffeehaus in zentraler Lage. Das Interview wurde an einem Tisch im Außenbereich nahe der Fußgängerzone durchgeführt, es war dennoch eine neutrale und ungestörte Atmosphäre gewährleistet. Das Interview fand im zweiten Anlauf statt. Der Interviewte und der Interviewee kannten sich vor dem treffen nicht, der Kontakt wurde über gemeinsame Bekannte hergestellt.\newline
IT2 (23.04.) fand in der Wohnung des Interviewten statt; hier konnte schon im Voraus eine persönliche Beziehung auf Vertrauensbasis über Nachhilfeunterricht aufgebaut werden.
Interview IT3 (02. 05.) fand im EJSA - Jugendcafé in Regensburg im allgemeinen Aufenthaltsraum statt. Es kam es öfter zu Störungen, da der Interviewte sich in direkter Umgebung seines Freundeskreises befand. Dieser Ort wurde auf ausdrücklichen Wunsch des Interviewten gewählt.\newline
Interview IT4 (03.05.) fand in der Wohnung des Probanden statt. Gelegentlich besuchten dessen Mitbewohner das Interview, allerdings ohne zu Unterbrechen. Das Interview wurde so gelegt, dass es keine Überschneidung mit dem am 05.05. beginnenden Fastenmonat Ramadan hatte.\newline

IT5 fand in einem Nebenraum des EJSA - Jugendcafé statt.
IT6 fand im zweiten Anlauf statt
Die Interviews dauerten im Durchschnitt 51.57,5 Minuten, bei einer max/min Abweichung von 45:32 Minuten.

%IT1 :	46:58
%IT2: 	67:05
%IT3:	27:47
%IT4:	35:05
%IT5:	73:19
%IT6: 	61:31

%Average:			51:57.5
%total				5:11:45
%max/mindeviation	45:32

Einzelne Settings beschreiben
wann waren die einzelnen Interviews?
Wie lange haben's gedauert?

Leitfaden?
%Im Verlauf der Interviews wurde der Leitfaden um die Frage ergänzt, auf welche Weise sich der bayrische Dialekt auf Ihr Leben ausgewirkt hatte.
Modifizierter Leitfaden?

\subsection{Methodik}


Durch einen Mangel an relevanten Informationen wird ein Ausschluss aus der Gesellschaft riskiert, in die die Gefl\"uchteten integriert werden sollen. \cite{andrade2016information}\newline
Als relevant werden in dieser Arbeit alle Informationen gewichtet, die das Leben in der neuen Umgebung beeinflussen - von einem Termin beim \"ortlichen Arzt \"uber Kenntnis der eigenen Rechtslage bis hin zur Grundkenntnis der \"ortlichen Kultur und Gesellschaft. \cite{schreieck2017supporting}\newline


Ein Interview, bei dem der Sense-Making approach verwendet wird, kann auf dem Ansatz der micro-moment timeline basieren. Bei dieser werden die Interviewten gebeten, eine pers\"onliche Situation mit Bezug zum Forschungsfokus genau zu beschreiben (In diesem Fall: Eine Situation abrufen, in der eine Informationsl\"ucke Auswirkungen auf das Leben des interviewten hatte).
Die von Oduntan et al. festgestellten Kategorien, in denen Informationsl\"ucken geh\"auft auftraten, waren:
\begin{enumerate}
    \item Ablauf des Asylverfahrens\newline
    Auch ein anhaltender Einfluss auf die mentale Gesundheit sowie getroffene Entscheidungen nach der Ablehnung sind hier relevant.
    \item Juristische Komponente \newline
    Beispiel: War dem Gefl\"uchteten klar, dass im Fall einer negativen Asylentscheidung Berufung eingelegt werden kann?
    \item Wohnen \newline
    Wie viele Wohnungen hat der Proband nach seiner Flucht bewohnt, wie lange im Durchschnitt, aus welchen Gr\"unden erfolgten Umz\"uge? Momentane Zufriedenheit mit der Wohnsituation? Wie wurde die momentane Wohnung entdeckt?
    \item Bildung \newline
    beinhaltet Erfahrungen mit dem deutschen Bildungs - und Schulungssystem
    \item Sozial\newline
    Wie ist das soziale Umfeld aufgebaut? Wie steht der Proband in Beziehung zu verschiedenen Demographischen Gruppen?
    \item Informationsquellen\newline
    Welche Informationskan\"ale werden von den Probanden verwendet, um sich zu orientieren und offene Fragen zu beantworten?
\end{enumerate}
Besagte Situation wird in Time-Line steps beschrieben, d.h. was passierte als erstes, zweites, etc. Innerhalb eines Schrittes wird behandelt, welche Fragen sich zu diesem Zeitpunkt bildeten, welche Gedanken und Gef\"uhle den Probanden zu dem Zeitpunkt bewegten.\newline
Hierbei ist die Sense-making Metapher zu ber\"ucksichtigen, mit Hinblick auf Situation, Informationsl\"ucke, Br\"ucke und Resultat.\cite{dervin2003sense}\newline
Dies f\"uhrt zu weiteren Fragen:\newline
\begin{enumerate}
    \item Was f\"uhrte zu dieser Frage?
    \item Was hat Sie mit deinem Leben zu tun?
    \item Gesellschaft und Machtverh\"altnisse?
    \item Wurde die Frage beantwortet?
    \item Wie?
    \item Welche Hindernisse gab es?
    \item War die Antwort hilfreich?
    \item War die Antwort hinderlich?
    \item Auf welche Weise?
\end{enumerate}