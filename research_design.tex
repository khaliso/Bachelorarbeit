Diese Studie arbeitet wie Oduntan et al.\cite{oduntan2017investigating} mit dem von Dervin's \textit{Sense-making methodology} beeinflussten Sozio-Kognitiven Ansatz nach Bates\cite{bates2005introduction}.



\subsection{Demographische Angaben}

\begin{table}[h!]
  \begin{center}
    \caption{Allgemeine Informationen}
    \label{tab:table1}
    \begin{tabular}{l|S|r|l}
      \textbf{Kategorie} & \textbf{Anzahl}\\
%      $\alpha$ & $\beta$ & $\gamma$ & $\delta$ \\ % <--
        \newline
      \textit{Geschlecht}\\
      \hline
      männlich & 6\\
      weiblich & 0\\
      \hline
      \newline
      \textit{Alter}\\
      \hline
      23 Jahre & 1\\
      21 Jahre & 2\\
      20 Jahre & 1\\
      19 Jahre & 2\\
      \hline
      \newline
      \textit{Geburtsland}\\
      \hline
      Afghanistan & 3\\
      Iran & 2\\
      Syrien & 1\\
      \hline
      \newline
      \textit{Ankunftsjahr in Deutschland}\\
      \hline
      2015 & 6\\
      \hline
      \newline
      \textit{Bildungsgrad vor der Flucht}\\
      \hline
      Akademiker & 1\\
      Berufliche Ausbildung & \\
      Rudimentäre Bildung & \\
      Keine Vorbildung &\\
      \hline
      \newline
      \textit{Momentaner Bildungsweg}\\
      \hline
      23 Jahre & 1\\
      21 Jahre & 2\\
      20 Jahre & 2\\
      19 Jahre & 2\\
      \hline
      \newline
      \textit{Asylstatus}\\
      \hline
      23 Jahre & 1\\
      21 Jahre & 2\\
      20 Jahre & 2\\
      19 Jahre & 2\\
      \hline
      \newline
      \textit{Fluchtroute}\\
      \hline
      Land, LKW & 2\\
      Land, Fußweg & 1\\
      Seeweg & 2\\
      Unbekannt & 1 (IT3)
      \end{tabular}
  \end{center}
\end{table}


Demographische Infos
Fluchtgrund
Fluchtroute
Asylstatus

\subsection{Methodik}

%1
Ende 2015 (->3 1/2 jahre in Deutschland.)
23, männlich
asylstatus: anerkannt.
Bildungsgrad: Akademischer Abschluss in Syrien
Fluchtroute: Boot
Syrien
Grund für Deutschland: Raus aus Syrien, wurde in Passaua uf dem Weg in die Niederlande von der Polizei angehalten.
Sozial: breites Netzwerk an Freunden, auch Deutsche (allerdings eher ältere, bei jüngeren Trinkkultur Problemfaktor).

%2
19, männlich
Afghanistan
19.09.2015
Schiite
Bildung: keine Bildung bis zur Ankunft in Deutschland, lernte auch die Schrift seiner Heimat in Deutschland
Muttersprache Dari, kann auch Persisch. (Sind wie Bayrisch zu Deutsch)

%3
männlich, 21
Geboren im Iran, Afghanische Familie (sieht sich selbst als Afghane, und wird im iran als afghane diskriminiert.)
4 Jahre in Deutschland, seit 2015

%4
männlich, 20
Afghanistan, Sunnite
Seit Anfang 2015 in Deutschland
Asylverfahren: Duldung (Seit Anfang 2018), laufendes Verfahren nach §25a). Interview beim BAMF:                        min26/27. 2x abgelehnt,

%5
Männlich, 19
Seit Mitte 2015 in Deutschland
Im Iran geboren; Eltern Afghanen -> Keine iranischen Ausweisdokumente. (min5)
Hat unbefristete Aufenthaltsgenehmigung (min70

%6
21, männlich
Afghanistan, im Iran aufgewachsen (mit 8 in den Iran) -> kein Pass
seit 2015 in D
Duldung (25a läuft