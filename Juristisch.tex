\subsection{Juristisch}
    
Eine Frage, die alle Interviewten zu einem Zeitpunkt im Integrationsprozess bewegte, war, wie eine unbefristete Aufenthaltserlaubnis erlangt werden konnte. Bei mehreren Befragen läuft momentan das der Antrag gemäß \textit{§25a} (Anhang), der Aufenthalsgewährung bei gut integrierten Jugendlichen und Heranwachsenden.\newline
\textit{§25a} kann beantragt werden, wenn:
    \begin{enumerate}
        \item Sich der Geflüchtete seit mindestens vier Jahren legal in Deutschland befindet
        \item Seit vier Jahren eine Schule besucht oder einen Schul - oder Berufsabschluss erworben hat
        \item Der Antrag vor dem 22. Geburtstag gestellt wird
        \item Integration in der deutschen Gesellschaft zu erwarten ist
        \item Keine falschen Informationen vom Geflüchteten angegeben wurden
    \end{enumerate}
\centerline{\textit{Vgl. Anhang}}
Sie haben von der Möglichkeit, über \textit{§25a)} eine unbefristete Aufenthaltserlaubnis zu erlangen,  entweder von Freunden, dem sozialen Umfeld oder von Betreuern erfahren. Es wurde angegeben, das Verfahren nach diesem Paragraphen nach mehrmaliger Ablehnung des Asylgesuchs eingeleitet zu haben. 
%Warum wurde mein Asylgesuch abgelehnt? (IT4, min28)
Weshalb ein Asylgesuch abgelehnt wurde, war den Probanden nicht immer klar.
\begin{quote}
    ``Ich weiß nicht, warum. Ich habe auch nicht genau verstanden.''
\end{quote}
Ein Proband gab an, mit den Regeln, Vorschriften und Gesetzten in den Tagen und Monaten nach der Ankunft Probleme gehabt zu haben. Er gab an, sich in seinem Herkunftsland, dem Iran, nicht auf die unabhängige Gesetztestreue der staatlichen Exekutive verlassen zu können. Ähnliches wurde auch von allen anderen Probanden angegeben. Er äußerte sich jedoch erleichtert über die Unabhängigkeit der Staatsgewalt in Deutschland\textit{, obwohl im Moment ein Verfahren wegen schwerer Körperverletzung gegen Ihn läuft},
\begin{quote}
    ``Also, ich kann auf jeden Fall vertrauen, weil da läuft alles über Regeln, ne? Aber im Iran ist das nicht so''
\end{quote}
Der Deutschkurs wurd ein diesem Kontext insbesondere positiv hervorgehoben, da eine präzise Vermittlung der "\textit{Regeln}" gewährleistet schien.
\begin{quote}
    ``Genau. Da waren wir dann (...) keine Ahnung, sechs Monaten vielleicht? Da sind wir dann in die Deutschkurs gegangen. Da war super. Weil (...) es war so (...) wir, wir konnten nix, keine Sprache, keine Regel undso''
\end{quote}
\textbf{Wie genau geh ich da drauf ein? Thema ist ja nur allgemein Geflüchtete in Deutschland.}
Ein Problem, mit dem die afghanisch - und iranisch - stämmigen Geflüchteten konfrontiert waren, waren gültige Ausweispapiere.\newline
Alle fünf betroffenen Interviewten entstammten Afghanischen Familien. Sie waren dementsprechend der afghanischen Minderheit im Iran zugeordnet, selbst wenn Sie im Iran geboren worden waren. Sie hatten dementsprechend (nach iranischer Sicht?) keinen Anspruch auf die iranische Staatsbürgerschaft, und damit den iranischen Pass.\newline
Mehrere Interviewte gaben an, von den iranischen Behörden das Angebot erhalten zu haben, mit Ihnen zu kooperieren - etwa um Sie zu militärischem Engagement in den Einsatzgebieten der iranischen Armee zu gewinnen. Alle Interviewten lehnten das Angebot ab \textit{(und wurden nach dem Verrichten einer Haftstrafe nach Afghanistan abgeschoben.)}. 
Der afghanische Pass wiederum ist nach Angabe der Interviewten aufgrund der prekären politischen Lage in Afghanistan nur durch einen langwierigen Prozess zu erlangen.\newline
Nachdem in Deutschland das Asylverfahren eingeleitet wurde, erhielten die Interviewten ein Ausweisdokument, welches Sie als in Deutschland registrierte Geflüchtete identifiziert. Ein Geflüchteter war vor die Problematik gestellt, dass besagtes Ausweisdokument vom Türsteher einer Diskothek nicht akzeptiert wurde.
\begin{quote}
    ``Ja (...) es ist schon ungut. Nunja, manchmal du willst (...) Disco gehen, und (...) gehst so. Dann die Security fragt: 'Ja, Ausweis?' Dann (...) gibst du ein Papier und der sagt: 'Ne was, was für ein Ausweis ist? Darfst du nicht mit solche Dokumenten rein!' (...) Dann hast du andere Gefühl, weißt du? Und (...) ja. Du hast keine richtige Ausweis, das (...)''
\end{quote}
\caption{IT6, min21}
Dies konfrontierte Ihn damit, dass der Türsteher seine befristete Aufenthaltserlaubnis als Asylsuchender nicht als legitimes Ausweisdokument akzeptierte. Sein Freundeskreis und Er zogen schließlich weiter.


Betreffende Personen zeigen sich nun wieder zuversichtlicher. Mehrere berichteten jedoch im Kontext der Ablehnung des Asylbescheids oder einer bevorstehenden Abschiebung von negativen Emotionen, einem Gefühl der Verzweiflung oder Hoffnungslosigkeit. Insbesondere mit einer potentiellen bevorstehenden Abschiebung werden Berichte von ähnlichen Fällen verbunden, welche unter anderem mit dem Tod der Betroffenen endeten.

\begin{quote}
    ``Dann darf ich nicht (unv.). Kann ich was machen? Ich kann nicht in die Stadt gehen. Ich darf nicht in der Disco gehen. Ich darf nicht, ähm(...), nach Österreich gehen. Ich darf nicht (...) arbeiten.[..]Ich darf nicht garnicht, mich ähm, beschäftigen. Dann (...) das ist alles Problem, und das ist alles (...) Stress und (...) und dann überlegst du, wenn du auch gute Mensch bist.''
\end{quote}
\begin{quote}
    ``Du kannst nicht dich konzentrieren, du sagst jeden Tag: 'Ja, ob ich bleiben darf? Ob Ihr, ob (...) keine Ahnung. Ich bin dabeim. Vielleicht, die kommen. Die Bullen kommen jetzt und (...) die nehmen mich fest und die schieben mich ab! Nach Afghanistan.' Du bist garnicht, du fühlst dann nicht (...) du fühlst dich garnicht gut.''
\end{quote}
\begin{quote}
    `` Ich habe auch mal mitgekriegt, dass die Leute, die, die aus (...) Europa raus, aus Deutschland nach Afghanistan zurück (...) abgeschoben wurden [..] Die wurden (...) also ich hab auf jeden Fall ge (...) also mitgekriegt, dass zwei Leute da beim Kämpfen, oder (...) solche Sachen, gestorben wurden.''
\end{quote}
