
\section{Interview 1, Teil 1}

I: (...)Ah, jetz (...) ähm /also es qird jetz quasi aufgenommen und dann später setz ich mich an' Computer, hör's ma nomml an und tipp's runter #00:00:13-2#

B: okay #00:00:11-9#

I: ähm Beziehungsweise wir ham auch ne Software ähm da muss ich mich aber erst noch ähm genau informieren wie des dann aussieht #00:00:23-1#

B: okay #00:00:23-1#

I: aber grundsätzlich die(..)meine Arbeit bezieht sich da drauf ähm im bei im Integrationsprozess ähm dann ähm Lücken zu finden also wann hast du zum Beispiel irgendwann mal so nen Erlebnis ghabt 'Oh! So hätte des funktioniert!' ähm so ja ist jetz vielleicht ne schlechte Erklärung. ähm ich weiß ned seit wann bist du denn in Deutschland? #00:00:51-2#

B: Also ich bin seit drei einhalb Jahr in Deutschland #00:00:53-2#

I: Seit drei einhalb Jahren. #00:00:54-2#

B: Seit drei einhalb Jahren. #00:00:55-3#

I: hm (bejahend). Du sprichst phänomenal gut Deutsch. #00:00:58-9#

B: Es ist nicht Wunder. (lacht) Es ist kein Wunder. Es gibt schon viele, die mehr als ich(...)Schöner als ich sprechen können auch. Es gibt Leute die einfach bayrisch gelernt haben. Also ich bin nicht der einzige deswegen (...) es ist kein Wunder also wenn man in einem Land lebt dann muss ich die Sprache beherrschen oder? #00:01:16-5#

I: hm (bejahend) Trotzdem du(...)du sprichst gut. ähm darf ich dich fragen wie alt du bist? #00:01:21-5#

B: also ich bin 23 #00:01:24-6#

I: du bist 23?! #00:01:22-9#

B: ja(...)sehe ich älter aus oder was? #00:01:26-9#

I: du siehst älter aus. ich bin 24. #00:01:26-8#

B: (lacht) echt jetzt? #00:01:27-8#

I: ja(lacht) #00:01:29-6#

B: ne ich bin immer noch(...)noch jung #00:01:31-5#

I: mhm #00:01:33-5#

B: ja. Vielleicht wegen dem Bart oder sowas dann sehe ich älter aus(lacht) #00:01:36-6#

I: Hab ich keinen anständigen Bartwuchs leider.(lacht) #00:01:37-7#

B: (lacht)ja. #00:01:41-7#

I: ähm warte(...)was war für dich so die Entscheidung ähm warum hast du dich entschieden dass Du nach Deutschland kommen willst? #00:01:55-8#

B: Also zuerst die Entscheidung war nicht dass ich nach Deutschland komme #00:01:57-8#

I: sondern einfach nur Europa? #00:01:56-3#

B: Es ist eifach dass ich raus von Syrien komme.hm  #00:02:01-1#

I: hm (bejahend) #00:02:00-6#

B: Egal wo ich den Weg mich nehmt.  #00:02:04-2#

I: hm (bejahend) #00:02:04-2#

B: Und als ich dann weg von Syrien war habe ich mich entschieden dass ich in die Niederlande dann (...) #00:02:09-0#

I: hm (bejahend) #00:02:09-0#

B: Also in die Richtung der Niederlande #00:02:11-8#

I: hm (bejahend) #00:02:11-8#

B: Aber daher hat die Polizei mich angehalten und ich darf nicht weiter fahren. #00:02:16-6#

I: ähm wo bist du da angehalten worden? #00:02:20-2#

B: In Passau. #00:02:20-2#

I: In Passau? #00:02:22-8#

B: In Passau. #00:02:22-8#

I: Okay. #00:02:22-8#

B: Da ist die Polizei und die ahben mich (...) also die haben mich (unv.) um die dann den Fingerabdruck zu machen #00:02:30-7#

I: hm (bejahend) #00:02:30-7#

B: Und ich hab gesagt, ich wollte dann weiter in die Niederlande fahren. #00:02:33-9#

I: Ja? #00:02:33-9#

B: Nein. Das machst du vielleich später. #00:02:38-3#

I: Okay.. #00:02:38-3#

B: Das war ja ein Dolmetscher dabei und ich hab mit dem geredet und ich wollte ja weiter (...) ne, leider funktioniert jetzt nicht.(...) Okay(...) #00:02:51-4#

I: hm (bejahend) #00:02:51-4#

B: Okay. Aber naja also ich bereue mich überhaupt nicht (lacht). Hier habe ich auch nicht #00:02:55-5#

I: //Herzlich willkommen.(lacht) #00:02:56-2#

B: Danke(lacht) #00:02:57-9# Ja, sozusagen habe ich hier bis jetzt mich schon gut integriert #00:03:07-0#

I: ähm #00:03:07-0#

B: Die Sprache gelernt.. #00:03:06-9#

I: Ja #00:03:06-9#

B: Und viele Freunde auch habe ich schon(...)ja(...)Aber man fühlt sich immer so dass man immer noch ein Lücke hat. Weißt du, es geht darum dass ich vielleicht in Syrien aufgewachsen bin, und dann bin ich nach Deutschland gekommen #00:03:23-2#

I: hm (bejahend) #00:03:23-2#

B: Weil es spielt schon große Rolle. Wenn man in einem Land aufgewachsen ist und dann kommt auf ein anderen Land und der hat sich gewöhnt wie er dort(...)was er macht und was für Ihn angeboten wurde, und hier dann etwas anderes. #00:03:42-9#

I: ähm #00:03:44-4#

B: Die Kultur spielt eine Rolle, die traditionelle spielt auch eine Rolle #00:03:49-5#

I: Wie unterschiedlich sind die beiden Kulturen? #00:03:53-0#

B: 180 Grad. #00:03:53-0#

I: Echt?(lacht) #00:03:53-0#

B: Ja. 180 Grad. #00:03:54-6#

I: ähm inwiefern? Also ich.. #00:04:00-1#

B: //INwiefern.. Also.. #00:04:00-1#

I: Ich war noch nie in Syrien #00:04:02-3#

B: Ja (...) in Syrien, also  ich sag's immer den(...)den dieses Sprichwort, für mich ist das immer so: Nicht alle Finger gleich. #00:04:13-6#

I: Ja.. #00:04:13-6#

B: Es sind nicht immer alle Finger gleich. Aber wo ich gelebt habe und wo ist mein Gebiet ist und wo ich aufgewachsen bin(...)da war für mich ja, wenn ich das vergleichen, da ist die (unv.). Aber jetzt, also vielleicht es ist am Anfang. Aber jetzt habe ich mich gewöhnt. #00:04:31-8#

I: Ja. #00:04:31-8#

B: Und es gibt immer noch schwere Punkte. #00:04:38-4#

I: Zum Beispiel? #00:04:38-4#

B: Ja, also strenger Muslim bin ich nicht, aber zum Beispiel ich trinke Alkohol. #00:04:42-5#

I: Hab ich letztes mal mitbekommen (lacht) #00:04:45-7#

B: Was hast du? #00:04:44-6#

I: ähm bei unserem ersten Termin #00:04:47-8#

B: Okay #00:04:46-3#

I: ähm wir hattens ja schonmal ausgemacht ghabt ähm das war ganz lustig dann du hattest ja dann gschrieben 'Oh scheiße, du warst gestern bis fünf was trinken #00:05:00-2#

B: Ne also ich war nicht beim trinken sondern das war eine Party. #00:05:06-8#

I: hm (bejahend) #00:05:06-8#

B: Eine Geburtstagsfeier #00:05:08-9#

I: hm (bejahend) #00:05:08-9#

B: Aber wir machen das(...) also wir feiern aber ohne zu Trinken. Es ist(...) es geht einfach. #00:05:15-2#

I: ähm ein Freund von mir, also ich bin draußen im Jugendcafé, vielleicht kennst du das? #00:05:21-4#

B: Wo ist das? #00:05:22-3#

I: ähm das ist.. #00:05:25-7#

B: Trefft ihr euch jeden Donnerstag.. #00:05:29-3#

I: Jeden Donnerstag, genau #00:05:29-3#

B: Okay, ja (lacht) #00:05:29-7#

I: also da hinten draußen, hinter'm Bahnhof. #00:05:32-5#

B: Ja, eine Freundin von mir die kommt immer. #00:05:36-4#

I: Ah? #00:05:35-1#

B: Ja. #00:05:35-1#

I: ähm also nen Freund von mir ist eben dort und der hat da auch seinen Geburtstag gefeiert #00:05:41-6#

B: hm (bejahend) #00:05:41-6#

I: ähm(...)und da is kein Tropfen Alkohol getrunken worden #00:05:49-2#

B: Braucht man nicht(lacht) #00:05:49-2#

I: Genau! #00:05:50-0#

B: Das braucht man nicht. #00:05:51-3#

I: Das war für mich tatsächlich komplett neu #00:05:54-5#

B: Ja. #00:05:55-1#

I: Eine Geburtstagsfeier, auf der nur getanzt wird. #00:05:57-6#

B: ja. #00:05:57-6#

I: UNd es war schön! Es hat Spaß gemacht! #00:05:57-6#

B: Ja. #00:06:00-2#

I: ähm(...)ja(...)Andere Kultur, aber es is.. #00:06:04-1#

B: Ja es ist andere Kultur. Also es ist nicht 'Schock' für mich also ich sag's nicht dass in Syrien wird nie jemals getrunken oder sowas #00:06:13-9#

I: hm (bejahend) #00:06:13-9#

B: Doch, schon! #00:06:14-3#

I: Ja #00:06:14-3#

B: Und es gibt schon, und es gib schon Laden für den Alkohol, wo man reinkommt und dann kauft für sich einfach was er will. #00:06:23-7#

I: hm (bejahend) #00:06:23-9#

B: Aber so üblich oder so, so groß wie hier in Deutschland, das habe ich nicht erlebt. Aber mir ist(...)Also mir interessiert überhaupt nicht ob jemand trinkt oder nicht oder sowas. #00:06:38-2#

I: Ja. #00:06:38-2#

B: Mich interessiert einfach, dass jemand zu mir kommt und sagt: Warum? #00:06:42-2#

I: hm? #00:06:45-0#

B: Also warum(...)wir trinken schon, warum trinkst du nicht? Ich sag's ganz ehrlich du hast ja, also.. (unv.) Die Antwort kommt sofort: Du bist hier in Bayern. Du musst trinken! #00:06:58-0#

I: Nein? #00:06:58-0#

B: Du musst Schwein essen! #00:06:57-9#

I: Ahh.. #00:06:57-9#

B: Wieso willst du dich integrieren? Hää? #00:07:02-4#

I: Was?(lacht) #00:07:02-4#

B: Hää? Was hat das mit Integration zu tun? #00:07:06-9#

I: //(lacht) Passiert dir das öfter? #00:07:07-0#

B: Ja also für mich ist viele Geschichten entgegen also (unv.) am Anfang der (...)ja..(...) Am Anfang, so 2016, 2017.. #00:07:17-9#

I: Ja? #00:07:17-9#

B: ..War das für mich schon so hart, weil ich einfach nicht verstanden hab, weil mein deutsch ist ganz schlecht, und deswegen ich habe nicht ganz (...) nicht alles verstanden #00:07:31-8#

I: hm (bejahend) #00:07:31-8#

B: Und (...) Und ich kann nicht antworten. #00:07:33-9#

I: Ja.. #00:07:33-9#

B: Also das ist die, die, die.. also, jetzt hab ich schon meine Gründe und warum antworte ich überhaupt nicht. Jetzt kann ich einfach so meine Taten dann begründen. Warum mache ich das? Warum mache ich das? #00:07:47-9#

I: hm (bejahend) #00:07:47-9#

B: Genau. Aber am Anfang war schon wirklich schwer. Oder die nehmen das einfach (unv.), da sitzten wir zum Beispiel so fünf, sieben Leute zusammen #00:07:59-6#

I: hm (bejahend) #00:07:59-6#

B: Ja der bestellt für sich einen.. ein Bier ein Bier ein Bier. Und dann kommt der Kiellner zu mir und sagt: Was hätten Sie gerne? Dann sagen die alle entweder Capuccino der Kaffe. Da lache ich damit. Aber die haben einfach so Abstand gemacht #00:08:15-7#

I: hm (bejahend) #00:08:15-7#

B: Wenn man nicht mittrinkt, dann (...) es gibt schon immer Abstand. (...) #00:08:22-1#

I: (...) Komisch. #00:08:22-1#

B: Es ist doch komisch, ja. Obwohl dass ich in einen Bar gearbeitet habe. Also ich habe hier in Sachs. Kennst du Sachs? #00:08:29-5#

I: Ja. #00:08:29-5#

B: Da habe ich einen Jahr gearbeitet #00:08:30-5#

I: //Du hast im Sachs gearbeitet? (lacht)#00:08:30-5#

B: Ja. Also der Ladenführer war mein Nachbar, also ist mein Nachbar bis jetzt. Und der hat mir die Arbeit angeboten, dann aht gesagt: 'Ja, du kannst (...) wenn du willst, dann kannst du da arbeiten.' #00:08:44-7#

I: Okay #00:08:46-3#

B: Ja ich hab als Türsteher ein Jahr gearbeitet. #00:08:49-7#

I: hm (bejahend) #00:08:50-4#

B: Ich hab mit den Kunden nichts zu tun, also immer #00:08:53-6#

I: Ja? #00:08:53-6#

B: Sondern nur ab halb zwei habe ich das zu tun, und dann vor die Tür. #00:09:02-9#

I: Ja. #00:09:02-7#

B: Dann mussten die alle leise sein. #00:09:01-3#

I: Ja. (...) (lacht) also ähm #00:09:10-0#

B: Also im Sachs habe ich. (lacht) #00:09:08-4#

I: //(lacht) #00:09:08-4#

B: Aber du warst niemals im Sachs. #00:09:11-4#

I: Also (...) pff, früher.. #00:09:11-4#

B: // ne nicht niemals, also als ich war. Also als ich gearbeitet habe. #00:09:13-0#

I: ähm #00:09:17-9#

B: Eigentlich es ist kein Bar. #00:09:19-7#

I: Nicht? #00:09:19-7#

B: Es ist kein Bar. Es ist einfach (...) also wenn man trinken will, dann geht nicht in Sachs. Wenn man saufen will, dann geht in's Sachs. #00:09:30-1#

I: (lacht)..Das stimmt(lacht) #00:09:30-1#

B: Es ist einfach so. Also ich hab so viel erlebt. So viele Sachen erlebt.(...) #00:09:41-2#

I: Ich komm mal wieder zurück zu meinem Leitfaden, Entschuldigung. #00:09:43-6#

B: Ja #00:09:43-6#

I: ähm, ah, ähm mit wem hast du hier ähm ah. Also du bist jetzt seit dreeinhalb Jahren hier. #00:09:54-7#

B: ja. #00:09:55-7#

I: Mit wem hast'n du normal hier in Deutschland Kontakt? Also.. Und hast du auch noch Kontakt nach Hause? Also, zu..#00:10:02-1#

B: //Ja, also Kontakt zu Hause habe ich schon, aber es ist einfach.. da war also (...) Zeit vor sechs Monaten waren, da konnte ich mit meinen Eltern nicht kontaktieren. #00:10:16-0#

I: hm (bejahend) #00:10:16-0#

B: Der Krieg war so dort, und das können wir einfach (...) die sind einfach verschwunden vor ungefähr sechs Monaten da hatten wir kontaktlos mit denen. Aber danach haben wir hald dann wieder Kontakt mit ihnen #00:10:27-4#

I: hm (bejahend) #00:10:27-4#

B: Das war ja also die große (...) die große Scheißezeit für mich. Also für uns alle drei. #00:10:35-3#

I: Das glaube ich. #00:10:35-3#

B: Ja. (...) Aber Kontakte hier in Deutschland.. Ich haben den alle. Also ich bin flexibel. #00:10:43-3#

I: hm (bejahend) #00:10:43-3#

B: Ich treffe mich mit denen alle. #00:10:47-8#

I: ähm #00:10:48-5#

B: Ja also ich hab kein Problem mit Fremde oder mit wem oder sowas, damit habe ich garkein Problem. #00:10:56-3#

I: hm (bejahend) #00:10:56-3#

B: Also viele Deutsche habe ich schon, also ich hab viele Kontakte mit den Deutschen auch #00:11:00-8#

I: hm (bejahend) #00:11:00-8#

B: Die sind ja älter. Also die sind ältere Leute, mehr als die junge Leute. Weil die Jungs, die, die erzählt haben, die, also wenn mit dem Alkohol da machen sie einfach Abstand. #00:11:14-8#

I: ehh.. #00:11:14-8#

B: Viele Kontakte habe ich mit denen überhaupt nicht. #00:11:18-6#

I: hm (bejahend) #00:11:18-6#

B: Also ich weiß es nicht, vielleicht habe ich mich zurückgehalten oder sowas, aber da habe ich schon ein paar  Situationen erlebt, und deswegen habe ich gesagt 'Mmh, jetz ist nicht.' #00:11:31-2#

I: (lacht)Auch die jungen sind nicht alle so. #00:11:31-2#

B: Ja die sind nicht alle so, ja. #00:11:36-8#

I: Verschiedene Finger einer Hand. #00:11:36-6#

B: Genau.(...) Ja also ich habe zu tun mit den älteren Menschen dann mehr #00:11:44-9#

I: hm (bejahend) #00:11:44-9#

B: Ja also die haben mehr Zeit. Also es kann sein auch die jungen Leute die haben nicht viel Zeit #00:11:52-6#

I: hm (bejahend) #00:11:52-6#

B: ..Und wenn du einfach mal also mit den jungen dann triffst und sowas, vielleicht treffen wir zusammen ein Monat oder zwei Monaten. Aber wenn zum Beispiel so eine Woche oder zwei Wochen oder einen Monat dann keinen Kontakt gibt #00:12:04-7#

I: hm (bejahend) #00:12:04-7#

B: Kein Anruf gibt, dann vergessen Sie dich einfach. #00:12:10-4#

I: ähm #00:12:10-4#

B: Auf die Seite. #00:12:12-0#

I: Ich hab nen paar Kontakte, ich glaub denen habe ich schon seit nem Jahr oder was nimmer zurückgeschrieben - aber das sind trotzdem immer noch Freunde von mir. #00:12:20-4#

B: Ja. #00:12:20-4#

I: Also(...) #00:12:22-0#

B: Ich weiß es nicht, also die Mentalität ist unterschiedlich. #00:12:24-6#

I: hm (bejahend) #00:12:24-6#

B: Die Mentalität ist unterschiedlich. Vielleicht also was ich bemerkt habe, also wenn ein Deutscher oder so jemand fremd kennen lernt #00:12:34-5#

I: hm (bejahend) #00:12:34-5#

B: Dann nimmt für sich schon Zeit und macht Abstand zuerst #00:12:39-7#

I: hm (bejahend) #00:12:40-3#

B: Bis er dann den Mensch kennen gelernt hat, also wenn(...)bis er den Mensch auch verstanden hat oder Ihn schon gut(...)kennt, dann ist er ein Freund von ihm. #00:12:53-4#

I: Ja. #00:12:53-4#

B: Das ist das Unterschied zwischen und, also uns in Syrien und daher auch. #00:12:59-4#

I: Ja bei uns ist das ein schleichender Prozess. #00:13:01-6#

B: Zum Beispiel wir unterscheiden nicht das (...) das ist ein Kollege, also Kollegen in der Arbeit oder das ist ein Freund von mir, das ist (...) ne. Die sind alle Freunde. #00:13:12-0#

I: Ah? #00:13:12-0#

B: Es gibt nicht da: ' Da ist mein Kollegen.' Dann haben wir zusammen was zu tun in der Arbeit, aber außer der Arbeit #00:13:21-7#

I: hm (bejahend) #00:13:21-7#

B: ..wir begrüßen uns überhaupt nicht. Eigentlich finden wir (...)  #00:13:24-3#

I: //Ja gut, das ist dann auch komisch #00:13:25-2#

B: also bei uns ist es einfach so: Die Freunde sind die Mitarbeiter oder die Kollegen die bei uns waren. #00:13:30-6#

I: hm (bejahend) #00:13:30-6#

B: Die Schul, also die die wir einfach zusammen in der Schule waren oder sowas. Das sind die Freunde. #00:13:38-5#

I: hm (bejahend) (...) Ich muss sagen, darüber hab ich mir noch keine Gedanken gemacht. #00:13:47-0#

B: Ja, also ich hab.. vielleicht merkt ihr das nicht #00:13:48-0#

I: hm (bejahend) #00:13:48-0#

B: Weil es ist einfach so, also ihr habt so erlebt, ihr seid so aufgewachsen, und deswegen es ist normal. #00:13:55-5#

I: ja. #00:13:55-5#

B: Aber wenn jemand fremd kommt und das sieht, dann sagt er vielleicht, also unterschiede (...) wir kommen einfahc wieder zu unterschiedliche Kultur, unterschiedliche (unv.).  #00:14:07-5#

I: Aber das find ich grad sehr interessant, dass ma nämlich (...) war mir nie wirklich bewusst. #00:14:13-5#

B: Ja, dass man einfach (...) Mein Arbeitskollege. Der ist wie ein Freund von mir. Also ihr habt so einfach (...) so Abstand. Ja es ist ein Kollegen in der Arbeit, ja der hat was mit mir zu tun in der Arbeit #00:14:27-8#

I: hm (bejahend) #00:14:27-8#

B: Draußen hab ich mit Ihm überhaupt nicht zu tun.(...) #00:14:32-4#

I: ähm #00:14:32-4#

B: Das macht ihr. #00:14:33-2#

I: Ja. #00:14:35-1#

B: Entschuldigung für das Wort 'Ihr' (lacht) #00:14:35-1#

I: Passt schon, passt schon. #00:14:36-2#

B: Und man kann einfach nicht verallgemeinern auch. #00:14:40-1#

I: Ja. #00:14:40-1#

B: Also mit meinen Worte, das.. (...) verallgemeinere überhaupt nicht, aber was ich gesehen habe. #00:14:48-1#

I: hm (bejahend) (...) Bist du dann mit (...) ähm, also du hast ja, was war's (...) Medizinisch -Tech.. Medizini.. Medizinisch -Technischer Assistent #00:14:58-3#

B: Ne, also ich bin Pharmazeutisch-Technischer Assistent. #00:15:00-2#

I: Ah okay #00:15:01-8#

B: Pharmazeutisch. Fast gleich. (lacht) #00:15:01-8#

I: Okay, ähm Also hast dann quasi mit deinen.. also mit den Leuten bist du dann auch ganz normal unterwegs. #00:15:11-3#

B: ähm(...) Nein. #00:15:13-3#

I: Ne? #00:15:15-1#

B: Ne. #00:15:16-4#

I: Okay. #00:15:18-2#

B: Nicht die alle. Also ich will das, aber die machen den Abstand. Und bis Sie dann uns kennen gelernt haben #00:15:23-3#

I: hm (bejahend) #00:15:23-3#

B: Und sie haben sich vielleicht mehr Mühe gegeben oder sowas bis sie dann mir akzeptiert. #00:15:30-4#

I: hm (bejahend) #00:15:31-0#

B: Dann bin ich einfach weg. #00:15:33-5#

I: Mmmh.. #00:15:33-5#

B: Da habe ich also zum BEispiel, also ja (...) Wie sagt man das? Die Spra.. Also die, die.. Sprache spielt auch eine Rolle #00:15:43-9#

I: hm (bejahend) #00:15:44-9#

B: Zum Beispiel, also die Leute die mit mir sind. Die Mitschüler, die mit mir sind #00:15:49-5#

I: hm (bejahend) #00:15:49-5#

B: Die reden nur auf bayrisch. #00:15:54-3#

I: Aber können sie's.. ähm können Sie auch Hochdeutsch? #00:15:56-5#

B: Die können schon. #00:15:58-3#

I: //Wollen nur nicht? #00:15:58-3#

B: Aber die reden nur bayrisch. Und wenn ich mit denen rede, dann.. Pff ja also wenn ich denen nicht verstehe und Sie mich nicht verstehen, da gibts keinen Kontakt dabei #00:16:10-7#

I: Ja. #00:16:10-7#

B: Also es gibt keinen Diskussion dazu. #00:16:16-2#

I: ähm ja. Glaubst.. Glaubst du, dass.. #00:16:16-9#

B: Und ich hab damals also jemanden von denen gefragt, ja am Anfang war wirklich schon schwer. Weil hab denen schon mal gefragt.. Die machen ja alle als Zusammenfassung von ein paar Fächer. #00:16:29-3#

I: hm (bejahend) #00:16:29-3#

B: Könnt ihr mal mir dann die Zusammenfassung schicken? Eine hat 'Ja' gesagt, 'das schicke ich dir heute Abend'. #00:16:37-4#

I: hm (bejahend) #00:16:37-9#

B: Abend, kommt garnix. Morgens, früh. Kommt garnix. Bin ich in die Schule. Kommt garnix. Da habe ich nicht nachgefragt. #00:16:50-2#

I: hm (bejahend) (...) Des is komisch. #00:16:53-0#

B: Ja. Vielleicht Sie hat das vergessen. #00:16:54-4#

I: Ja, hoffentlich #00:16:58-8#

B: Ja aber ich bin der Typ, der nicht nachfragt. #00:17:00-1#

I: hm (bejahend) #00:17:00-8#

B: Ich frage nicht nach. Weil ich habe einmal gefragt, und wenn man das will, dann macht das ür mich.  #00:17:08-8#

I: hm (bejahend) #00:17:08-8#

B: Und wenn das nicht will, dann  #00:17:11-8#

I: hm (bejahend) #00:17:11-8#

B: Frage ich nicht einfach wieder. Das mag ich überhaupt nicht, dass ich immer nachfrage (lacht). #00:17:17-5#

I: Kann ich verstehen.. ähm Aber glaubst du dass es grundsätzlich absichtlich passiert, also sowas ähm.. #00:17:30-2#

B: Am Anfang ja. #00:17:30-7#

I: Schon? #00:17:30-7#

B: Ja, am Anfang war schon. Mit Absicht. Ich weiß es nicht, vielleicht es war mein Gefühl ist falsch. Aber ich hab das Gefühl, dass einfach in Ihren Augen war: ' Es ist zu viel für einen Flüchtling, dass er dieses Beruf lernt.(...)' #00:17:51-1#

I: (...)Warum? #00:17:51-1#

B: Weil wir das nicht schaffen können.(...) (lacht) #00:17:58-2#

I: Ehh, okay #00:17:58-2#

B: So Ihren Glauben war. Oder die haben einfach kein Kontakt mit den Ausländer. #00:18:03-1#

I: hm (bejahend) #00:18:03-1#

B: Die haben mit den Ausländer garnix zu tun. Und deswegen sind nicht offensichtlich für den anderen. #00:18:10-9#

I: ähm, des hab ich auch schon öfter erlebt, also (...) ähm mit ähm mit Leuten, die grundsätzlich nie im Kontakt sind mit ähm mit Ausländern, oder auch nur mit Menschen von weiter weg aus Deutschland #00:18:28-1#

B: //Von weit weg, ja #00:18:28-1#

I: ähm(...) Die sind grundsätzlich am ähm voreingenommensten veranlagt. Also, die wollen dann auch keinen Kontakt. #00:18:46-0#

B: Genau. Die wollen einfach keinen Kontakt. Also die Leute die mit mir waren, die waren einfach (...) keine Ahnung. Also am Anfang waren wirklich für mich schon (...) dass die einfach dagegen sind. Wir waren vier Ausländer und 18 Deutsche. #00:19:00-3#

I: hm (bejahend) #00:19:00-9#

B: Und die meißtens waren auch Mädchen. Also ich hab nur ein (...) ein Junge gehabt. Also wir nur ein Junge in der Klasse und der zweite war ich. #00:19:08-5#

I: Also es waren zwei Jungs und #00:19:11-8#

B: Und 20 Frauen. #00:19:13-8#

I: Ui. #00:19:15-6#

B: Es war wirklich die Hammer. #00:19:17-5#

I: Was? #00:19:17-5#

B: Ja. Und deswegen(...) #00:19:20-1#

I: Glaubst du, dass es dann auch ähm (...) Geschlechterrollenspiele gibt oder (...) #00:19:28-2#

B: Nee, also die mit mir ist (unv.) also die spielt überhaupt nicht, keine, also bei mir spielt überhaupt keine Rolle #00:19:31-8#

I: Okay gut (...) Ne ich mein, dass irgend wie dann (...) keine Ahnung, Frauen dann mehr abblocken oder was weiß ich. #00:19:37-9#

B: Ne also die sind (...) also die sind mit Ihm dann im Kontakt mehr als ich (...) also mehr als bei mir. Es ist der Grund einfach, dass er immer Sie (unv.) versteht. #00:19:50-0#

I: hm (bejahend) #00:19:50-0#

B: Dass der einfach (...) denen versteht.  #00:19:52-2#

I: Ja. #00:19:52-2#

B: Aber ich hab Ab- und zu dann mit denen nicht gesprochen und da also ich hab am Anfang nur einfach gesehen: Die sprechen nur den Dialekt #00:20:01-4#

I: hm (bejahend) #00:20:01-4#

B: Und mit denen konnte ich keinen Anfangen, also keine Diskussion einführen. Und deswegen ist schon schwer, also die haben am Anfang versucht und sowas (...) Da war ich einfach in einer Situation, wo man garnix oder keine Lust hat #00:20:19-5#

I: hm? #00:20:19-5#

B: Und ich gehe nicht mit und deswegen die haben einfach danach gedacht, ja er will mit uns nicht mehr oder sowas. Deswegen die haben Abstand gemacht.  #00:20:28-4#

I: Das heißt so.. #00:20:28-4#

B: //Vielleicht war(...)war ich auch schuld, also (...) vielleicht war ich auch schuld #00:20:31-6#

I: //Kann man jetz nicht sagen #00:20:31-6#

B: Aber ich war in einer Situation wo ich einfach (...) nicht anders machen kann. #00:20:39-3#

I: hm (bejahend) #00:20:39-3#

B: Also ich bin wirklich also in.. so am Anfang der Ausbildung wo also es hat schon gut gelaufen, so Anfang der ersten zwei, drei Monaten #00:20:52-6#

I: hm (bejahend) #00:20:53-0#

B: Und ich hab mich nicht gewöhnt auf den (...) (unv.) System von der Schule. Also die deutsche System ist ganz anders als bei uns #00:21:01-9#

I: Das Schhulsystem? #00:21:02-7#

B: Das Schulsystem ist ganz anders. Es ist (...) #00:21:07-7#

I: Da würde ich jetz ganz gerne direkt mal nachfragen: ähm so wie ist das Schulsystem in Syrien, und wie ist es dann im Vergleich hier? #00:21:15-0#

B: Also unser System ist französisch. #00:21:19-0#

I: Ist französisch? #00:21:19-0#

B: Ist französisch. #00:21:19-0#

I: Ah! #00:21:19-0#

B: Ja. Was wir in Syrien so, pff, ja, geamcht haben oder sowas war einfach der Lehrer ist verantwortlich, dass wir 75\% von dem Unterricht lernen #00:21:34-4#

I: hm (bejahend) #00:21:34-4#

B: oder verstehen #00:21:33-5#

I: hm (bejahend) #00:21:33-5#

B: und dass wir einfach zu Hause die 25\% damit wir einfach komplett den Unterricht beherrschen können. #00:21:43-2#

I: hm (bejahend) #00:21:43-2#

B: Wir machen einfach den 25\% zu Hause. So, unser System sage ich einfach, also, es war ein bisschen dumm. Und nicht dumm. Daher zum Beispiel es gibt so immer recherchieren und nachfragen und immer so (...) ähm (...) wie sagt man (...) oder es ist, kommt auch darauf an auch welchen Beruf hast du denn? Zum Beispiel in Syrien als ich mein Abitur gemacht hab, also wir (...) wir (...) wir werden von 12 Fächer geprüft #00:22:19-2#

I: hm (bejahend) #00:22:19-5#

B: Und wir haben schon viele schwere (unv.) Fächer gemacht. Zum Beispiel als ich mein Abitur abgem (...) also (...) abgeschlossen hab, ich bin ja in Mathe und Physik und Chemie und Biologie, Sozialkunde, Arabisch, Französisch, Englisch und Religion. #00:22:37-0#

I: hm (bejahend) #00:22:37-0#

B: Ich bin in diese neun Fächer geprüft, um dann den Abschluss zu (...) also um dann mein Abitur zu bekommen #00:22:45-0#

I: hm (bejahend) #00:22:45-5#

B: Hier? Nee. Also es gibt hier nur 3 Fächer oder 4 Fächer, man wird dann für 4 Fächer dann sein Abitur machen #00:22:52-2#

I: ähm Abitur glaub ich warn (...) also ich hab's 2013 gemacht #00:22:57-0#

B: hm (bejahend) #00:22:57-0#

I: ähm des war Deutsch, Englisch, Mathematik und zwei Wahlfächer - also bei mir glaub ich waren #00:23:03-9#

B: //fünf Fächer. #00:23:03-9#

I: Genau ähm Ich glaub ich hatte Geographie und #00:23:07-3#

B: //Sprachen oder sowas oder?#00:23:08-0#

I: Ja #00:23:10-7#

B: Ne des war nicht auch die, die also was ich meine. Aber das System bei uns war einfach, dass wir (...) wir kommen einfach in die Schule #00:23:20-3#

I: hm (bejahend) #00:23:20-3#

B: Der Lehrer ist wirklich verantwortlich für uns #00:23:24-8#

I: hm (bejahend) #00:23:25-3#

B: Dass wir einfach die 75\% von der (...) von dem (...) von dem Stoff her dann verstehen, dann wieder nach Hause kommen, dann lernen wir den Rest. #00:23:35-3#

I: Dast du dann das Gefühl, dass.. #00:23:37-6#

B: Hier ist es(...) Entschuldigung. (...) Hier ist es einfach (...) keine Ahnung, also der Lehrer (...) Ihm ist das egal, ob wir das verstanden haben oder nicht, weil ich hab mit so, mit der unser Lehrerin auch gesprochen. Sie redet bayrisch. #00:23:55-9#

I: hm (bejahend) #00:23:55-9#

B: Ich hab mit Ihr schon gesprochen 'könnten Sie vielleicht das auf Hochdeutsch dann umstellen vielleicht oder so?' - 'Ja, wenn du das Gefühl hast, dass du einfach das nicht verstehst, mach einfach die Hände auf, also, Hände hoch, und dann fragst du mich nach.'
 #00:24:15-5#

I: //Mmmmmmmmh #00:24:15-5#

B:Okay, wie lange dauert das? Soll ich jede Minute dann die (...) die (...) mein Hand hoch gehen und dann 'Entschuldigung, ich hab noch eine Frage!' #00:24:26-1#

I: Ja..
 #00:24:26-1#

B: Da nerve ich einfach die alle Mitschüler! #00:24:26-5#

I: ja. #00:24:26-5#

B: Das habe ich gemacht einmal. #00:24:26-9#

I: Ja? #00:24:26-9#

B: Also in einem Unterricht. Ich bin einfach, also ich hab Sie ungefähr zwei, dreimal untergebr - also, unterbrechen müssen. Aber die alle nerven sich.. #00:24:40-0#

I: Ja #00:24:40-1#

B: .. dass ich das nicht verstanden habe oder sowas #00:24:43-4#

I: aber, das (...) das verstehe ich ned wirklich. Also des war ne Klasse mit 18 deutschen Schülern und vier syrischen, oder? #00:24:52-4#

B: Ja. Nicht syrischen, also sind unterschiedliche, ja. #00:24:54-3#

I: //also(...)also, okay. Aber (...) alle davon sprechen deutsch und die 18 davon sprechen bayrisch. (...) Das ist doch (...) ja. Das ist doch dann eigentlich (...) logisch, im Unterricht dann nur deutsch zu sprechen.  #00:25:16-1#

B: Ja. Die, die (...) Lehrer hat gemeint 'ja, also ich bin seit 20 Jahren hier als Lehrerin. #00:25:23-4#

I: hm (bejahend) #00:25:23-4#

B: Und so, mit (...) also mit (...) euch, euch also die meint mit euch hald dann die Ausländer, habe ich noch nicht erlebt, dass ich in einen, einem Unterricht Ausländer unterrichte. #00:25:37-6#

I: Das war Ihre erste #00:25:38-1#

B: // Sie hat sich gewöhnt auf Ihr Dialekt zu sprechen. #00:25:40-7#

I: Das war quasi Ihre erste Klasse, in der Sie auch Ausländer unterrichtet hat? #00:25:46-1#

B: Ja, obwohl das war vor uns auch war eine, eine, eine Irakerin. #00:25:52-6#

I: hm (bejahend) #00:25:53-1#

B: Sie hat auch die Ausbildung abgeschlossen. #00:25:55-2#

I: hm (bejahend) #00:25:55-2#

B: Ja, Sie war auch und sie hat (...) also sie hat (...) ihr wirklich tut sehr leid und (...) sie hat schon, wirklich viel gegeben. Also viel mühe gegeben #00:26:06-6#

I: hm (bejahend) #00:26:06-6#

B: Um Sie dann zu verstehen, aber Sie ist ein Mädchen. Wenn Sie nachfragt, dann antworten die alle(lacht). #00:26:16-6#

I: (lacht) #00:26:16-6#

B: Es ist einfach so. Und sie, sie kann einfach den Kontakt mit den Mädchen dann (...) #00:26:23-4#

I: Ja. #00:26:23-4#

B: Und Ihr (...) Es gab jemand der Sie auch geholfen hat und sowas. Ja bei mir es ist (...) es war einfach.. Ja, ich hab einfach Schuld bekommen. Als ich dann zuerst nachgefragt, ob die, ob Sie dann mir die Zusammenfassung mir schicken oder sowas, da hat niemand geschickt #00:26:42-7#

I: Ja. #00:26:42-7#

B: Da habe ich gesagt 'Ne'. #00:26:43-9#

I:  Ja, okay. Vorbei? #00:26:46-6#

B: Brauche ich euch nicht mehr #00:26:46-8#

I: Ja? (...) Schade. #00:26:52-2#

B: Ne ich habe Lust überhaupt nicht. #00:26:55-0#

I: hm? #00:26:55-0#

B: Also schade ist überhaupt nicht. Man lernt auch davon. Also davon. #00:26:58-5#

I: Ja, aber.. (...) das #00:27:00-3#

B: //Man lernt einfach davon #00:27:00-4#

I: Das is eine Lektion, die (...) ähm (...) die wär' auch in schöner gegangen.
 #00:27:08-7#

B: (...) Ja. (...) #00:27:16-9#

I: Nochn Kaffee? #00:27:16-9#

B: Nee, danke. #00:27:20-5# 