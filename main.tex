\documentclass[12pt,oneside]{article}

%%%%%%%%%%%%%%%%%%%%%%%%%%%%
%%   Zusaetzliche Pakete  %%
%%%%%%%%%%%%%%%%%%%%%%%%%%%%
\usepackage{acronym}
\usepackage{enumerate}  
\usepackage{a4wide}
\usepackage{fancyhdr}
\usepackage{graphicx}
\usepackage{palatino}
\usepackage{blindtext}
\usepackage{multirow}
\usepackage[toc]{appendix}

%folgende Zeile auskommentieren für englische Arbeiten
\usepackage[ngerman]{babel}

\usepackage[T1]{fontenc}
\usepackage[utf8]{inputenc}
\usepackage[bookmarks]{hyperref}
\usepackage[justification=centering]{caption}
\usepackage[style=chicago-authordate,natbib=true,backend=biber]{biblatex}
\usepackage{csquotes}
\bibliography{literatur}

%%%%%%%%%%%%%%%%%%%%%%%%%%%%%%
%% Definition der Kopfzeile %%
%%%%%%%%%%%%%%%%%%%%%%%%%%%%%%

\pagestyle{fancy}
\fancyhf{}
\cfoot{\thepage}
\setlength{\headheight}{16pt}

%%%%%%%%%%%%%%%%%%%%%%%%%%%%%%%%%%%%%%%%%%%%%%%%%%%%%
%%  Definition des Deckblattes und der Titelseite  %%
%%%%%%%%%%%%%%%%%%%%%%%%%%%%%%%%%%%%%%%%%%%%%%%%%%%%%

\newcommand{\JMUTitle}[9]{

  \thispagestyle{empty}
  \vspace*{\stretch{1}}
  {\parindent0cm
  \rule{\linewidth}{.7ex}}
  \begin{flushright}
    \vspace*{\stretch{1}}
    \sffamily\bfseries\Huge
    #1\\
    \vspace*{\stretch{1}}
    \sffamily\bfseries\large
    #2
    \vspace*{\stretch{1}}
  \end{flushright}
  \rule{\linewidth}{.7ex}

  \vspace*{\stretch{1}}
  \begin{center}
    \includegraphics[width=2in]{Unisiegel.png} \\
    \vspace*{\stretch{1}}
    \Large Bachelorarbeit  \\

    \vspace*{\stretch{2}}
   \large Lehrstuhl f\"{u}r Informationswissenschaft\\
    \large Universität Regensburg\\
    \vspace*{\stretch{1}}
    \large Betreuer:  #7 \\[1mm]
    
    \vspace*{\stretch{1}}
    \large Regensburg, den #6
  \end{center}
}


%%%%%%%%%%%%%%%%%%%%%%%%%%%%
%%  Beginn des Dokuments  %%
%%%%%%%%%%%%%%%%%%%%%%%%%%%%

\begin{document}

  \JMUTitle
      {Untersuchung von Informationslücken im Integrationsprozess von Geflüchteten in Deutschland}        % Titel der Arbeit
      {MAXIMILIAN SCHMIDHUBER}                  % Vor- und Nachname des Autors
      
      {Fakultät für Sprach-, Literatur- und Kulturwissenschaften}  % Name der Fakultaet
      {Regensburg 2019}                          % Ort und Jahr der Erstellung
      {24.05.2019}                              % Tag der Abgabe
      {Prof. Dr. David Elsweiler}               % Name des Erstgutachters
      {Prof. Dr. Bernd Ludwig}                  % Name des Zweitgutachters
      
  \clearpage

\lhead{}
\pagenumbering{Roman} 
    \setcounter{page}{1}

\tableofcontents
\clearpage

\addcontentsline{toc}{section}{\listfigurename}
\listoffigures

\addcontentsline{toc}{section}{\listtablename}
\listoftables
\clearpage

%%%%%%%%%%%%%%%%%%%%%%%%%%%%
%%  Kurzzusammenfassung   %%
%%%%%%%%%%%%%%%%%%%%%%%%%%%%
\markboth{Abstract}{Abstract}
\section*{Abstract}

\blindtext


%%%%%%%%%%%%%%%%%%%%%%%%%%%%
%%  Einstellungen  %%
%%%%%%%%%%%%%%%%%%%%%%%%%%%%
\cleardoublepage
\pagenumbering{arabic}  
    \setcounter{page}{1}
\lhead{\nouppercase{\leftmark}}

%%%%%%%%%%%%%%%%%%%%%%%%%%%%
%%  Hauptteil  %%
%%%%%%%%%%%%%%%%%%%%%%%%%%%%

\section{Einleitung} \label{einleitung}

10\% der Arbeit

Ende 2017 waren weltweit 68,5 Millionen Menschen auf der Flucht. Unter 'Auf der Flucht'  sind Menschen zusammengefasst, welche als interne Vertriebene, Gefl\"uchtete oder Asylsuchende verzeichnet sind. Dies ist die h\"ochste Zahl, die jemals von UNHCR verzeichnet wurde.\cite{uno2018flucht}
Die neuesten Zahlen vom Juni 2018 zeigen keine Ver\"anderung. \cite{uno2018flucht}
Das IPCC geht davon aus, dass sich diese Entwicklung in der n\"aheren Zukunft nicht ver\"andern wird. \cite{sr15ipcc}\newline
Die Zahl der Asylantr\"age in Deutschland ging allerdings von 2016 bis 2018 von 745.545 \"uber 222.683 auf 185.853 zur\"uck. \cite{statistica2019asyl} In der T\"urkei befanden sich im Jahr 2018 3.5 Millionen Gefl\"uchtete.\newline
\newline
Momentan gibt es Grund zur Annahme, dass Asylsuchende oft nicht ausreichend auf den Integrationsprozess vorbereitet sind. \cite{oduntan2017information} \cite{gillespie2016mapping}\newline

Im Rahmen der sogenannten Fl\"uchtlingskrise wurden in den Aufnahmezentren vom {Bundesamt f\"ur Migration und Fl\"uchtlinge} (BAMF) zahlreiche Neuanstellungen vorgenommen. Bis zum Mai 2017 wurden gem\"a\ss{} des BAMF 454 dieser Neuanstellungen nicht ausreichend trainiert. Dennoch wurde auch dieses Personal damit beauftragt, Asylrechtsentscheidungen zu treffen.\newline
Die Zahl der Untrainierten wurde bis zum Februar 2018 auf 36 reduziert, 769 der 2139 Angestellten hatten allerdings die Ausbildung noch nicht abgeschlossen.\newline
Damit gibt diese Situation Grund zur Annahme, dass Asylsuchende nicht ausreichend auf einige der kommenden Herausforderungen vorbereitet wurden. \cite{asylum2018bamf}\newline

Mittels semi-strukturierter Interviews werden die Erlebnisse der Teilnehmer unter deutscher Jurisdiktion ergr\"undet. Dabei sollens mittels Bates' \textit{socio-cognitive approach}\cite{bates2005introduction}, welcher von Dervin's sense-making methodology\cite{dervin2003sense} inspiriert wurde, Informationsl\"ucken aufgedeckt.\newline
Mit diesem konnte festgestellt werden, wie ein Publikum verschiedene Situationen im Lauf des Lebens logisch begr\"undet hat. Um dies zu erm\"oglichen, wurden die Probanden eingeladen, Erfahrungen und Schlussfolgerungen mit Hilfe des Interviewers so detailliert wie m\"oglich wiederzugeben.\newline
Gem\"a\ss{} Dervin wird dennoch immer davon ausgegangen, dass eine erstellte Kategorisierung falsch, irrelevant oder f\"ur andere Sitautionen unangemessen sein kann.\textit{\textbf{Um diese zu \"uberpr\"ufen, wird zum Dialog zwischen wissenschaftlichen Institutionen und dem allt\"aglichen Leben aufgerufen}} .\textit{Diese Arbeit kann Aufschluss auf die Anwendbarkeit der in Oduntan et al.'s Forschung erarbeiteten Kategorien f\"ur den deutschsprachigen Raum geben.}\newline

Ein Interview, bei dem der Sense-Making approach verwendet wird, kann auf dem Ansatz der micro-moment timeline basieren. Bei dieser werden die Interviewten gebeten, eine pers\"onliche Situation mit Bezug zum Forschungsfokus genau zu beschreiben (In diesem Fall: Eine Situation abrufen, in der eine Informationsl\"ucke Auswirkungen auf das Leben des interviewten hatte).
Die von Oduntan et al. festgestellten Kategorien, in denen Informationsl\"ucken geh\"auft auftraten, waren:
\begin{enumerate}
    \item Ablauf des Asylverfahrens\newline
    Auch ein anhaltender Einfluss auf die mentale Gesundheit sowie getroffene Entscheidungen nach der Ablehnung sind hier relevant.
    \item Juristische Komponente \newline
    Beispiel: War dem Gefl\"uchteten klar, dass im Fall einer negativen Asylentscheidung Berufung eingelegt werden kann?
    \item Wohnen \newline
    Wie viele Wohnungen hat der Proband nach seiner Flucht bewohnt, wie lange im Durchschnitt, aus welchen Gr\"unden erfolgten Umz\"uge? Momentane Zufriedenheit mit der Wohnsituation? Wie wurde die momentane Wohnung entdeckt?
    \item Bildung \newline
    beinhaltet Erfahrungen mit dem deutschen Bildungs - und Schulungssystem
    \item Sozial\newline
    Wie ist das soziale Umfeld aufgebaut? Wie steht der Proband in Beziehung zu verschiedenen Demographischen Gruppen?
    \item Informationsquellen\newline
    Welche Informationskan\"ale werden von den Probanden verwendet, um sich zu orientieren und offene Fragen zu beantworten?
\end{enumerate}
Besagte Situation wird in Time-Line steps beschrieben, d.h. was passierte als erstes, zweites, etc. Innerhalb eines Schrittes wird behandelt, welche Fragen sich zu diesem Zeitpunkt bildeten, welche Gedanken und Gef\"uhle den Probanden zu dem Zeitpunkt bewegten.\newline
Hierbei ist die Sense-making Metapher zu ber\"ucksichtigen, mit Hinblick auf Situation, Informationsl\"ucke, Br\"ucke und Resultat.\cite{dervin2003sense}\newline
Dies f\"uhrt zu weiteren Fragen:\newline
\begin{enumerate}
    \item Was f\"uhrte zu dieser Frage?
    \item Was hat Sie mit deinem Leben zu tun?
    \item Gesellschaft und Machtverh\"altnisse?
    \item Wurde die Frage beantwortet?
    \item Wie?
    \item Welche Hindernisse gab es?
    \item War die Antwort hilfreich?
    \item War die Antwort hinderlich?
    \item Auf welche Weise?
\end{enumerate}

%context → situation → information need (situational information behaviour) -> infogaps

\newline
Bei der Auswertung der Daten werden die Interviews zun\"achst transkribiert. In Oduntan et al.'s Arbeit wurden die Daten anschlie\ss{}end thematisch analysiert und die Erfahrungen auf Gemeinsamkeiten und Unterschiede \"uberpr\"uft. Mittels dieser Vorgehensweise konnten Verbindungen zwischen individuellen und situationellen Faktoren festgestellt werden, womit Informationsl\"ucken in den Erfahrungen ausgemacht werden konnten. Die verschiedenen Kategorien und Themenbereiche wurden gekennzeichnet.\cite{oduntan2017investigating} Diese Arbeit orientiert sich an Oduntan et al.'s Arbeit.
\newline
Ein Gro\ss{}teil der Asylsuchenden in Deutschland stammt aus dem Nahen Osten, etwa die H\"afte der Asylsuchenden im Jahr 2017 wurde in Syrien, dem Irak oder Afghanistan geboren\cite{asylum2017seekers}. Daraus ergeben sich potentielle Probleme, welche voraussichtlich am leichtesten mit Interviews angegriffen werden k\"onnen.
Zun\"achst ist die Sprachbarriere zu ber\"ucksichtigen: In einem Interview besteht f\"ur den Interviewenden die M\"oglichkeit, eine Frage nach eigener Einsch\"atzung genauer zu erkl\"aren. Bei zu gro\ss{}en Sprachproblemen kann mit einem Dolmetscher oder Sprachassistenzsystem gearbeitet werden.\newline
Die grundlegende Aufgabe dieser Arbeit ist das Aufdecken von Informationsl\"ucken, ein erkl\"arungs - und redeintensives Unterfangen. Es wird davon ausgegangen, dass die Bed\"urfnisse der Individuen stark variieren. Infolgedessen muss eine potentiell aufgedeckte Informationsl\"ucke durch wiederholtes Nachfragen abgesichert werden, um Missverst\"andnisse zu vermeiden.\newline
Der semi-strukturierte Ansatz wurde ausgew\"ahlt, um den Gespr\"achsfuss so wenig wie m\"oglich zu unterbrechen und gegebenenfalls das Gespr\"ach zu fokussieren.\newline
Mit einem Aufbau der Studie auf diese Weise ist die M\"oglichkeit gegeben, einen direkten Vergleich zu Oduntan et al.'s Arbeit und eventuellen weiteren Folgearbeiten aufzubauen.\newline
Ein anderer Ansatz, etwa ein Fragebogen, ist in dieser Hinsicht nicht Zielf\"uhrend: Dieser ist darauf ausgelegt, f\"ur eine bestimmte Zielpopulation standardisiert werden zu k\"onnen. Dabei werden mehr Informationen vorausgesetzt als derzeit zur Verf\"ugung stehen.\newline
Die qualitativen Daten, welche mit den Interviews generiert werden, k\"onnen allerdings \"ahnliche Arbeiten in Zukunft unterst\"utzen.\newline




____________________________

refinition refugee Hannah Arendt VS UN?

Hannah Arendt: 'in the first place, we don't like to be called "refugees". We ourselves call each other "newcomers" or "immigrants." [..] A refugee used to be a person driven to seek refuge because of some act committed or some political opinion held. Well, it is true we have had to seek refuge; but we committed no acts and most of us never dreamt of having any radical opinion. With us the meaning of the term “refugee” has changed. Now “refugees” are those of us who have been so unfortunate as to arrive in a new country without means and have to be helped by Refugee Committees.
[..]Yes, we were “immigrants” or “newcomers” who had left our country because, one fine day, it no longer suited us to stay, or for purely economic reasons. We wanted to rebuild our lives, that was all. In order to rebuild one’s life one has to be strong and an optimist. So we are very optimistic. [..]
We lost our home, which means the familiarity of daily life. We lost our occupation, which means the confidence that we are of some use in this world. We lost our language, which means the naturalness of reactions, the simplicity of gestures, the unaffected expression of feelings. We left our relatives in the Polish ghettos and our best friends have been killed in concentration camps, and that means the rupture of our private lives.
Nevertheless, as soon as we were saved—and most of us had to be saved several times—we started our new lives and tried to follow as closely as possible all the good advice our saviors passed on to us. We were told to forget; and we forgot quicker than anybody ever could imagine.
[..] Apparently nobody wants to know that contemporary history has created a new kind of human beings—the kind that are put in concentration camps by their foes and in internment camps by their friends.

[..]have had such curious notions as to believe that we are not only “prospective citizens” but present “enemy aliens.” In daylight, of course, we become only “technically” enemy aliens—all refugees know this. But when technical reasons prevented you from leaving your home during the dark house, it certainly was not easy to avoid some dark speculations about the relation between technicality and reality.
%Bürger zweiter Klasse?
[..]Once we could buy our food and ride in the subway without being told we were undesirable.[..]We wonder how it can be done; we already are so damnably careful in every moment of our daily lives to avoid anybody guessing who we are, what kind of passport we have, where our birth certificates were filled out—and that Hitler didn’t like us. 
%Rassismus auf der Straße
[..] since passports or birth certificates, and sometimes even income tax receipts, are no longer formal papers but matters of social distinction.
%Nicht in die Disco gelassen

The comity of European peoples went to pieces when, and because, it allowed its weakest member to be excluded and persecuted.

\section{Related Work}

-> was für information needs wurden in der normalen Literatur aufgedeckt?

Wie bereits von \cite{oduntan2017investigating} festgestellt wurde, ..

The identified information needs of general migrant
include: social, cultural knowledge and skill-based
knowledge (Fisher, Durrance, & Hinton, 2004); health
insurance, emergency care, vaccinations, dental, preventive
care, pregnancy, what to expect from health services,
school enrolment, housing and employment (Courtright,
2005). Silvio (2006) identified academic, apprenticeship,
health, employment, political and how to deal with
discrimination and Shoham & Strauss (2008) identified
housing, schooling, health, driving, banking, legal issues,
work and language. Lingel (2011) identified information
need around neighborhood, grocery shopping, local routes,
libraries and hospitals. These studies also identified a broad
range of information sources that included family and
friends (Shoham & Strauss, 2008) and social networks such
as ball parks, bike shops, day care centers, church, barber
shops, hair salons and garage (Fisher et al., 2004).

However, these information needs are at an individual-level
and are set within certain specific confined instances.
Individual-level information needs isolate the studied from
either others in a process and surrounding contextual factors
of situations. For instance, Fisher et al. (2004) studied the
use of library programs by the Queens Borough public
library and Courtright (2005) studied health information
seeking investigation of Latino newcomers to Southern
Indiana in the United States. Similarly, Shoham & Strauss
(2008) investigated the information needs of new
immigrants’ families from North America to Israel and


\section{research design}

\subsection{Informationen über die Teilnehmer}

Demographische Infos
Fluchtgrund
Fluchtroute
Asylstatus

\subsection{Methodik}



\section{Auswertung der Interviews}

Bei der Durchführung der Interviews wurde darauf geachtet, den Interviewten so weit wie möglich die Möglicheit zu geben, über sowohl Zeit als auch Ort der Interviews zu verfügen. Proband 1 (5. bzw 12.04.) entschied sich für ein Kaffee in zentraler Lage (draußen, nur Durchgangsverkehr, Nachbartische nicht besetzt. Neutral, aber dennoch relativ ungestört); das Interview fand im zweiten Anlauf statt.\newline
Das zweite Interview (23.04.) fand in der Wohnung des Interviewten statt; hier konnte schon im Voraus eine persönliche Beziehung auf Vertrauensbasis aufgebaut werden.
Interview nummer drei (02. 05.) fand im EJSA - Jugendcafé in Regensburg im Aufenthaltsraum statt; hier kam es öfter zu Störungen, und der Interviewte war in direkter Umgebung eines Freundeskreises. Dieser Ort wurde auf ausdrücklichen Wunsch des Interviewten gewählt.
Interview Nummer 4 (03.05.) fand in der Wohnung des Probanden statt. Gelegentlich besuchten dessen Mitbewohner das Interview, allerdings ohne zu Unterbrechen. 

 Das fünfte Interview fand in einem Nebenraum des EJSA - Jugendcafé statt.

IT1 :	46:58
IT2: 	67:05
IT3:	27:47
IT4:	35:05
IT5:	73:19
IT6: 	61:31

Average:			51:57.5
total				5:11:45
max/mindeviation	45:32

In Oduntan's Arbeit wurden Informationslücken bei der Reaktion auf die 
\begin{enumerate}
	\item[Ablehnung des Asylbescheids]
		
	\item[Gesetzlichen Grundlagen]
		solicitor
		appeals and court
	\item [Wohnen]
		content
		eviction
		allocation
	\item [Bildung]
		language
	\item [soziales Umfeld]
		extra-curricular
	\item [Informationsquellen]
		Friends
		Internet
		Interpreters
		Caseworkers	
\end{enumerate} 

gefunden.\newline

Da sich Oduntan's Arbeit auf abgelehnte Asylgesuche beschränkte, konnte schon früh davon asugegangen werden, dass sich bestimmte Bereiche zwischen Ihrer Arbeit und dieser Testgröße unterscheiden würden.

Bei den Interviews in dieser Arbeit wurden Informationslücken in folgenden Bereichen entdeckt:

\begin{enumerate}

	\item[Wohnung]
	\item[Bildung]
	\item[Sprachlich]
	\item[Medizinisch]
	\item[Juristisch]
	\item[Sozial]			%sozieale interaktionen und Rassismus
	
\end{enumerate}

Diese wurden mittels eines von Dervin informierten soziokognitiven ansatzes analysiert.

Dervin's Sense - making:

1. Was führte zu dieser Frage?
2. Was hat Sie mit deinem Leben zu tun?
3. Gesellschaft und Machtverhältnisse?
4. Wurde die Frage beantwortet?
5. Wie?
6. Welche Hindernisse gab es?
7. War die Antwort hilfreich?
8. War die Antwort hinderlich?
9. Auf welche Weise?

info gap according to dervin:
	questions, confusions
	muddles, riddles
	angst				
	
Situationen werden in der sense-making methodology in micro-moment timeline steps dargestellt.(cite dervin, p241) Bei jedem dieser Schritte werden die sense-making elemente untersucht:
\begin{enumerate}
	\item	Welche Fragen taten sich auf?
	\item	Welche Gedanken?
	\item	Welche Gefühle?
	\item 	Welche Emotionen?
\end{enumerate}
Jeder dieser Schritte wird dann durch die Situation - Gap - Metapher gejagt: 

situation, lücke, brücke, resultat.

Deshalb 'nur' 6 interviews. -> Gedanken und Emotionen werden selten mit jemandem besprochen, den man nicht kennt. (IT4, min27: Wie hat sich das angefühlt? -> keine Antwort.)

Da Oduntan ausschließlich mit Probanden mit bereits abgelehntem Asylbescheid gearbeitet hatte, wurde eine gewisse Diskrepanz in den Ergebnissen erwartet.\newline


\subsection{Bildung}

Bildung:
    IT1.1, min21:01: Anderes Schul/Bildungssystem in Syrien (franz.)
    Bayrisch (IT1)
    Wie bekomme ich eine AUsbildung?
    Wie gehe ich mit Rassismus am Arbeitsplatz um? (IT3, min21. vgl Rassismus)
    Mache ich nach der Ausbildungen den Meister dazu? (IT4, min12)
    Was mache ich, wenn ich außerhalb der Regelzeit einen Ausbildungs/Arbeitsplatz brauche? (IT5, min39)
    Wie lerne ich den Umgang mit einem Computer ohne Zugang zu einem? (IT5, min40ff)(min40:58 eigener              Lösungsansatz) -> Demotivation(leider ich kann nicht, min48) -> nicht bestanden

\subsubsection{Vorbildung}
\subsubsection{Bildung in Deutschland}
\subsubsection{Sprache}
Sprachlich:
    Deutsch (IT1.1: min8. min7:47: kann jetzt gegenargumentieren)
    Bayrisch (IT1.1 min15:43, vgl sozial)(IT4, min6+min16)(IT6, min28)
            Bayrischer Dialekt (IT5, min65f)(IT1, min16f)
    Wie kommuniziere ich am besten mit dem Arzt, ohne dessen Sprache zu sprechen? (IT2, min28, vgl medizinisch)(IT4, min12)
    Wie lerne ich die Landessprache? (IT3, min9:33)
    Wie kommuniziere ich ohne Kenntnis der Landessprache? (IT4, min
    Wie kommuniziere ich mit der Polizei ohne Kenntnis der Landessprache? (IT6, min14)

Bildung:
    IT1.1, min21:01: Anderes Schul/Bildungssystem in Syrien (franz.)
    Bayrisch (IT1)
    Wie bekomme ich eine AUsbildung?
    Wie gehe ich mit Rassismus am Arbeitsplatz um? (IT3, min21. vgl Rassismus)
    Mache ich nach der Ausbildungen den Meister dazu? (IT4, min12)
    Was mache ich, wenn ich außerhalb der Regelzeit einen Ausbildungs/Arbeitsplatz brauche? (IT5, min39)
    Wie lerne ich den Umgang mit einem Computer ohne Zugang zu einem? (IT5, min40ff)(min40:58 eigener              Lösungsansatz) -> Demotivation(leider ich kann nicht, min48) -> nicht bestanden
    
    
Sozial:
    Trinkkultur (IT1.1: min8:15, 11:14)
        Trinkkultur in Bayern / Alkoholkonsum (wurde nur in IT1 abgefragt.)
    Anderes Verhältnis zu Mitmenschen zwischen D und Syrien (min13:12, alle Freunde)
    Wie kommuniziere ich mit meiner Familie, ohne Sie in Gefahr zu    bringen? (IT2, min15)
    WIe kontaktiere ich meine Familie (IT1.1, min10:16)
    Bayrisch (IT1.1 min15:43, vgl sprachlich)
    Was gibt es in diesem Land für soziale Regeln? (IT3, min10:20 -> peinlich?)
    Wie lerne ich neue Leute kennen? (IT4, min17)
        In Deutschland ist es sozial akzeptiert, nachzufragen, und zu hinterfragen? (IT5, min54/55)
        
Rassismus:
    IT1.1: Wollen die mich dabei haben? (IT1, min17:51), wollen keinen Kontakt (IT1, min19)
    IT2: Wie verhalte ich mich, wenn Ich rassismus erlebe? (IT2, min57)( IT3, min21)
    Wie gehe ich mit Rassismus am Arbeitsplatz um? (IT3, min21. vgl Bildung)
    Basiert Rassismus teilweise auf MEdienberichten? (IT3, min22:17)(IT6 -> das mit dem unmotivierten Zahnarzhelfer)
    Was mache ich, wenn mir der Zutritt zur Disco verwehrt wird, weil die Aufenthaltsgenehmigung nicht als gültiges Ausweisdokument anerkannt wurde? (IT6, min21)
    
Wohnung:
    Wie bekomme ich eine Wohnung ohne permanente Aufenthaltserlaubnis? (IT4, min25)
    Wie finde ich eine Wohnung mit permanenter Aufenthaltserlaubnis? (IT5, min69)
    
Medizinisch:
    Wie kommuniziere ich am besten mit dem Arzt, ohne dessen Sprache zu sprechen? (IT2, min28, vgl sprachlich)(IT3, min20)
    Der Konsum von mehr Schmerzmitteln als von den Betreuer vorgesehen hat auch 
    negative Konsequenzen? (IT2, min29ff., min45)
    Muss ich meinen Arbeitgeber auf meine Schwerbehinderung hinweisen? (IT5, min36f)
    
    
Juristisch:
    Wie bekomme ich eine permanente Aufenthaltsgenehmigung? (IT4, min25/26; min29/30)
    Was gibt es in diesem neuen Land für Regeln? (IT3, min 9:33)
    Wie verhalte Ich mich bei einer Schlägerei?(IT3, min18f)
    Warum wurde mein Asylgesuch abgelehnt? (IT4, min28)
    Wie lang muss ich auf die Antwort des BAMF warten? (IT6, min4)
    
Sprachlich:
    Deutsch (IT1.1: min8. min7:47: kann jetzt gegenargumentieren)
    Bayrisch (IT1.1 min15:43, vgl sozial)(IT4, min6+min16)(IT6, min28)
            Bayrischer Dialekt (IT5, min65f)(IT1, min16f)
    Wie kommuniziere ich am besten mit dem Arzt, ohne dessen Sprache zu sprechen? (IT2, min28, vgl medizinisch)(IT4, min12)
    Wie lerne ich die Landessprache? (IT3, min9:33)
    Wie kommuniziere ich ohne Kenntnis der Landessprache? (IT4, min
    Wie kommuniziere ich mit der Polizei ohne Kenntnis der Landessprache? (IT6, min14)
    
Sonstige:
    Wie kann ic4h Termine pünktlich wahrnehmen (IT2, min62f.)
    Wie bekomme ich neue (Fußball-)Kleidung? (IT3, min10:29)
    (Infolücke: Als Lackierer keine weißen Klamotten anziehen. Wie bekommen? DIe haben's Ihm gesagt! IT4        min15/16)
    Was ist Bayern eigentlich? Wo bin ich hier? (IT6, min17)
    Wie kann ich günstiger Sport treiben? Fitness ist mir zu teuer. (IT6, min36f)
        Ramadan (IT4, min31f)
    
Zukunft nachfragen:
    Alkoholkonsum in Bayern wurde von IT1 genannt, wurde nicht weiter verfolgt
    IT1.2: min 00:27. Woher wusste er, dass er Revision einlegen kann?


subsection{Interview 1}

Demographisch

3 1/2 jahre in Deutschland.
23, männlich
asylstatus: anerkannt
Syrien

Fluchtgrund:
politische unterdrückung + polizeiliche willkür. (IT1.2 min2ff, ) ANgst vor Folter (in1.2 min11:39)
B: Weil, wenn ich dort bleibe, dann muss ich die Waffe tragen. (...) #00:08:53-0#
Mord auf offener STraße (it1.2 min14:50)

Fluchtroute: Boot

Kontakte


Medizinisch


Wohnung


Bildung
Bildungsgrad: Akademischer Abschluss in Syrien

Rassismus


Bayrisch


Inflormationslücken


Notizen

Setting: Wo? Umgebung?
Zeitpunkt: zwischen Tür und Angel?

4 Aspekte einer Nachricht: Fokus auf Inhalts - und Selbstoffenbarungsebene, weniger Beziehung

Treffpunkt und Zeitpunkt von den Interviewees selbst bestimmen lassen






Grund für Deutschland: Raus aus Syrien, wurde in Passaua uf dem Weg in die Niederlande von der Polizei angehalten.
Sozial: breites Netzwerk an Freunden, auch Deutsche (allerdings eher ältere, bei jüngeren Trinkkultur Problemfaktor).

Hat 'immer noch eine Lücke': Kultur und Tradition in Syrien komplett anders als in Deutschland.
Job: Nachbar ist Besitzer einer Bar, kam so an Türsteher - Job.
Ausbildung zum Pharmazeutisch - technischen Assistenten, da Pharmazie/Chemie - Studium nicht anerkannt.
Mentale Stressfaktoren:
Ein Kontaktverlust vor etwa 6 Monaten zum Zeitpunkt des Interviews zu den Eltern (Kriegszeug eben)

Dem Probanden war nach eigener Angabe klar, dass im Falle eines negativen Asylbescheids Revision eingelegt werden kann.

'[..] wenn ich dort bleibe, dann muss ich die Waffe tragen.'

Reise nach Europa: 'einfach' über die Grenze zur Türkei bei Aleppo, dann 2, 3 tage in der Türkei gelebt, dann einen Schleuser bezahlt und getroffen. Dann Übersetzen nach griechenland. Boot mit 6m Länge, 46 personen, mit Elektromotor. (Anderer interviewee: 6x4m, 60Menschen -> Klassenunterschiede?)
Übersetzen 2 1/2 Stunden gedauert
polizei bringt Ihn und sene beiden Brüder von einer Grenze zur nächsten. 


Probleme: 
    Deutschland hat Trinkkultur um Alkohol -> Schwieriger für nicht - Konsumenten, sich zu integrieren. Syrien hat das nicht, feiern ohne Alkohol im Normalfall.
    Auch der Konsum von Schweinefleisch wurde am Frande angemerkt, jedoch nicht mit der selben Gewichtung wie die zum Alkohol.
    
    Ein weiteres Problem in diesem Kontext ist die Sprachbarriere der relativ frisch angekommenen, die Probleme dabei aufweisen, sich in solchen Situationen zu rechtfertigen
    Sprachbarriere zweistufig: Jemand, der Deutsch kann, mag dennoch noch lange nicht mit der bayrischen Sprache zurecht kommen.
    
    Anderes sozialsystem Deutschland VS syrien: 
        D: Unterscheidung zwischen Kollegen, Bekannte, Freunden.
        S: Alles Freunde.
        
    Bildung: Anderes Schulsystem in Syrien: französisches Schulsystem, welches nach der Einschätzung des Interviewten den Lehrkörper in die Verantwortung nimmt, dem Schüler 75\% des Stoffs im Rahmen des Lehrplans beizubringen. In Bayern kam die Sprachbarriere hinzu: Die Lehrkraft sprach nur mit bayrischem Dialekt, dies führte zu Informationslücken. Der Aufforderung, sich bei
    Unverständnis des bearbeiteten Stoffs immer zu melden, nahm der Interviewte aus Angst vor weiterer sozialer Ausgrenzung nur in seltenen Fällen wahr.\newline
    In dieser Ausbildungsklasse befanden sich 18 deutsche und 4 ausländer; 20 Schüler waren weiblich und 2 männlich.
%Fluchtgrund: Festnahme und Erzwingen eines falschen GEständnisses (nur unterschreiben, nicht lesen) durch Polizeigewalt, 'Wenn wir euch auf dieser Straße wieder sehen sperren wir euch länger weg'
%               Anklagepunkte: Sexuelle Übergriffe, Störung der Anwohner, Sachbeschädigung, ..
% Anderer Grund: Installieren eines (fachlich unkundigen) regimetreuen Fakultätsrats, welcher potentielle Regimegegner ausmachen soll. Junge Menschen sollen in die Armee, nicht an die Uni -> Professoren teilen Ansicht.
%Demo an der Uni gegen das Regime wurde u.a. von Panzern zerschlagen
%Demonstranten verschwinden, werden gefoltert und teilweise erst in Leichensäcken an die Familien zurück übergeben.
%Polizei stellt Schläger in Ihren Dienst und ist Blind betreffend der Verbrechen, die diese begehen (Messern eines Oppositionellen vor einer Moschee)



B: Aber so üblich oder so, so groß wie hier in Deutschland, das habe ich nicht erlebt. Aber mir ist(...)ALso mir interessiert überhaupt nicht ob jemand trinkt oder nicht oder sowas. #00:06:38-2#

B: Mich interessiert einfach, dass jemand zu mir kommt und sagt: Warum? #00:06:42-2#

I: hm? #00:06:45-0#

B: Also warum(...)wir trinken schon, warum trinkst du nicht? Ich sag's ganz ehrlich du hast ja, also.. (unv.) Die Antwort kommt sofort: Du bist hier in Bayern. Du musst trinken! #00:06:58-0#

B: Du musst Schwein essen! #00:06:57-9#

B: Hää? Was hat das mit Integration zu tun? #00:07:06-9#

I: //(lacht) Passiert dir das öfter? #00:07:07-0#

B: Ja also für mich ist viele Geschichten entgegen also (unv.) am Anfang der (...)ja..(...) Am Anfang, so 2016, 2017.. #00:07:17-9#

B: ..War das für mich schon so hart, weil ich einfach nicht verstanden hab, weil mein Deutsch ist ganz schlecht, und deswegen ich habe nicht ganz (...) nicht alles verstanden #00:07:31-8#

I: hm (bejahend) #00:07:31-8#

B: Und (...) Und ich kann nicht antworten. #00:07:33-9#

I: Ja.. #00:07:33-9#

B: Also das ist die, die, die.. also, jetzt hab ich schon meine Gründe und warum antworte ich überhaupt nicht. Jetzt kann ich einfach so meine Taten dann begründen. Warum mache ich das? Warum mache ich das? #00:07:47-9#



\subsection{Interview 2}

Demographisch
19, männlich
Afghanistan
Schiite
Muttersprache Dari, kann auch Persisch. (Sind wie Bayrisch zu Deutsch)

Fluchtgrund


Fluchtroute


Kontakte


Medizinisch


Wohnung


Bildung
Bildung: keine Bildung bis zur Ankunft in Deutschland, lernte auch die Schrift seiner Heimat in Deutschland

Rassismus


Bayrisch


Inflormationslücken


Notizen

Setting: Wo? Umgebung?

Zeitpunkt: zwischen Tür und Angel?

4 Aspekte einer Nachricht: Fokus auf Inhalts - und Selbstoffenbarungsebene, weniger Beziehung

Treffpunkt und ZEitpunkt von den Interviewees selbst bestimmen lassen






Grund zur Flucht: Aber mein Mama hat gesagt: 'Du musst gehen. Deshalb du kannst kein andere Wahl jetz, wenn du beim zweiten Mal (...) beim zweiten Mal verhaftet wirst, dann (...) kriegst du echt (...) zwei, vielleicht zwei oder drei Jahre im
Gefängnisstrafen, also Schlagstrafen und Strafen Geld #00:06:36-3#


Flucht über Pakistan in den Iran, dann Iran in die Türkei. Dort mit einem Schlepper nach Griechenland übergesetzt. Griechenland nach Ungarn zu Fuß; auf der Route wurden die GEflücheten von Hilfsklräften mit Lebensmitteln versorgt. Ungarn nach Deutschland mit dem Zug

Wohnen: Alle bisher bezogenen Wohnungen wurden von den örtlichen Hilfsorganisationen gestellt. Der damals zuständige Vormund und die Betreuer der Wohngruppe halfen beim Aussuchen der Wohnung.\newline
Die erste Wohnung wurde aus medizinischen Gründen verlassen; alle Zimmer waren auf zwei Bewohner ausgelegt. Der Proband leidet an schwerer Migräne (Wurde auch mal für nen gehirntumor gehalten, min48) und hätte daraus resultierend ein Einzelzimmer benötigt, welches in diesem Wohnsystem jedoch nicht vorgesehen war. Deshalb wurde er an das Wohnprogramm einer anderen Sozialorganisation verwiesen (min29).

Medizinisch:
In der ersten bezogenen Wohnung gab es gemäß der Aussage des INterviewees wenige Betreuer, jedoch wurde er noch von seinem Vormund individuell betreut. Die Bewohner der Gruppe wurden zur Selbstständigkeit angehalten, i.e. selbstständige Besuche beim Arzt(min28)
Problem: Kein Übersetzer beim Arztbesuch; Kommunikation 'Mit Händen und Füßen' (min28). Führte zu Missverständnissen.
Rekonstruktion eines Arztbesuches:
Der Interviewte machte sich auf den Weg zum Arzt (Ob geschickt oder aus eigenem Antrieb unklar), ohne über suffiziente Deutschenntnisse oder einen Dolmetscher zu verfügen. Betreuer brachten Ihm allerdings einen Satz bei, um sein Leiden zu beschreiben. Bsp.: "Mir ist schlecht"(min29). Die Nachfrage des Arztes nach "wo?" wurde bereits nichtmehr verstanden (min29); Resultierte in einer potentiell Missverständlichen 'Konversation' mit Händen und Füßen.\newline
Weiteres: Der Interviewee leidet (vermutlich genetisch bedingt, min34) an schwerer Migräne und konsumierte schon seit Kinder/Jugendtagen Schmerzmittel (min29). Im Tagesbetrieb der ersten Wohngruppe bekam er jeweils eine Tablette Schmerzmittel vor dem Schlafen gehen.\newline
Als dieser Betreuer jedoch über einen längeren Zeitraum frei hatte, versorgte er den Betreuten mit 20 Tabletten des Schmerzmittels, um Ihn bis zu seiner Rückkehr zu versorgen (Ob ein Ersatzbetreuer für die Wohngruppe zuständig war oder das Aushändigen mit Anweisungen des Betreuers verbunden war hab ich vergessen zu fragen.)\newline
Der zu diesem Zeitpunkt 16-jährige nahm bei einem Migräneanfall an einem Abend drei Tabletten besagten Schmerzmittels, dann 'B: ähm mir war schwindlig, dann
ich (...) mir war schwindlig, dann ich bin einfach auf dem Bett gelegen und geschlafen bis nächstes Tag.  #00:29:28-7#\newline
Tags darauf hatte der interviewte einen Arzttermin, welcher nach einem kurzen Gespräch feststellte, dass binnen sieben Tagen ein Großteil der 20 Tabletten ohne Regulation konsummiert wurde. Daraufhin kontaktierte dieser den Vormund des Betreuten, besagte Betreuung der Wohngruppe, Heimleitung, das Jugendamt, einen Dolmetscher und andere zuständige Stellen.\newline
Der Interviewte erklärte den Anwesenden, dass die Kombination des schnarchenden Mitbewohners mit seiner schweren Migräne der Auslöser für sein Verhalten gewesen seien; Daraufhin wurde der Leiter der Einrichtung aufgefordert, dem Interviewten ein Einzelzimmer zu geben. Da dies jedoch nicht möglich war, wurde er an eine andere soziale Wohneinrichtung weitergeleitet. \newline
In der neuen Einrichtung wurde auch ein geregelter Konsum der verordneten Medikamente (min32) sowie geordnete Kommunikation mit dem behandelnden Arzt (min41) sichergestellt.

(Migräne in Verbindung mit Ohnmachtsanfällen, min32/33)

Nach eigener Aussage war Ihm zu diesem Zeitpunkt der Zusammenhang zwischen Gesundheitlichen Schäden und Tablettenkonsum nicht bekannt. (mehr dazu ab min42) Die Leber des Probanden wurden im Iran durch Tablettenkonsum geschädigt (Flecken auf der Leber, Schmerzen).
Stand jetzt: Nimmt im Monat ein bis zweimal Tabletten (min44)
'damals ich hatte kein Betreuerin. Ich musste einfach meine Schmerzen wegmachen.'
B: Ich habe nix, nicht gedacht: 'Wenn ich Tabletten nehme, ich mache meine Kopfschmerzen weg, sondern ich bekommen davon Leberschmerzen'. Ich habe nicht gewusst. #00:43:43-9#
    -> Informationslücke: Konsum von Medikamenten. Bildungsbedingt?
    Kontext: 
    B: Im Iran habe ich oft Tablette genommen, jeden Tag dreimal. (...) Musste ich, ich war beim Arzt, der hast du gesagt: 'Du musst Tabletten nehmen'. Also ich hatte Kopfschmerzen #00:44:26-6#

Proband hat als Bauarbeiter gearbeitet, um besagte Schmerzmittel zu finanzieren. Diese Arbeit resultierte in andauernden Rückenproblemen, welche in Deutschland zu einem Abbruch einer Ausbildung als Verkäufer führte.

Informationslücke: gegen Rassismus argumentieren können? (I.e. die klauen unsere Jobs/die kriegen Geld vom Staat in Arsch geschoben) (Gefühl der Hilflosigkeit verhindern-> min59 : Ich finde, vielleicht, der hat recht?)

Informationslücke: Verpasste Termine (min63): Ich habe nicht gewusst! Niemand hat mir gesagt! (Informationslücke auf Seite der Betreuer -> anderes Ordnunggsystem?)
%        ist es eine Informationslücke, wenn für einen bestimmte informationen auf dem Weg verloren gehen?
        Effekt: Arzttermin findet zwei Wochen später statt (Dervin Zeug)

Aufgrund schwerer traumatischer Erlebnisse auf der Fluchtroute: 'Wenn ich nach Afghanistan abgeschoben wieder abgeschoben werde, ich werde nie wieder nach Deutschland kommen. Ich will die wieder diese Grenze sehen. Das war furchtbar. Ich will dort hungrig bleiben, nicht sterben, aber nicht wieder (...) solche Grenze sehen. 
Boot: min 12

Ankunft in Regensburg am 19.09.2015 (so genau angeben?), der Sprachkurs Deutsch begann Tags darauf.

Ausländer im Iran stehen häufig vor dem Risiko, verhaftet und abgeschoben zu werden. Sie verfügen über keine gültigen Papiere und sind dementsprechend  zur Arbeit im informellen Sektor gezwungen.

Der Bruder des interviewten wurde, wie der Interviewte selbst, von der iransichen Regierung angehalten, in Syrien zu kämpfen. Dies hätte gemäß deren Aussage zu einer Bleibeerlaubnis (incl Pass) für Ihn und Seine Familie sowie der Bereitstellung monetärer Mittel geführt.
Der Kampf in Syrien gegen den IS wird gemäß der Aussage des Interviewten von den Schiiten als würdevoll gesehen, weshalb viele dazu geneigt sind, sich diesem anzuschließen. Andere wurden gezwungen.(min15)

Der Vater des Interviewten wurde gemäß eigener Aussage von den Taliban 'geschlachtet', als dieser fünf Jahre alt war.(min17): Und bei meiner Familie, meine Bruder Beispiel sollte hingehen. Aber der hat nicht (...) meine Mama hat nicht akzeptiert. Meine Mama hat gesagt: 'Ich hab einmal eure Vater gesehen, wie Taliban hat geschlachtet. Ich will nicht meine Kinder so sehen. (...) Und meine Mama hat gesagt, ich hatte solche Scheiße erfahrung, ich will nicht wieder diese Erfahrung 

Kontakt zur Familie: Kontakt vor einem Jahr, dann Brief v. Iran an Familie (min 15) -> abhören. (min15), min18. 
Letzter Kontakt vor zwei Monaten zur Mutter, Foto: Lag im Krankenhaus. (min34) 
Informationsstand zur Mutter im Iran vor vier Monaten (über Telefon der Nachbarin, min19/20); der Interviewte zeigte sich sichtlich bedrückt von einer
Flut, welche zum Zeitpunkt des Interviews im Iran und Afghanistan viele Menschenleben forderte.

Kontakt zum Bruder: wurde von der Frau des Bruders über FB kontaktiert (min38)

Kontakte in Deutschland: EJSA - Jugendcafé, Campus Asyl, Betreuer der WG, Ehrenamtliche Nachhilfelehrer. Kein Kontakt mehr zum ehem. Vormund

Bildung: Keine Vorbildung in Afghanistan oder dem Iran; der Interviewte lernte sowohl die lateinische als auch arabische Schrift in Deutschland.

Bayrischer Dialekt wurde nicht als Problem und dem Lernprozess hinderlich angesehen. (Wurde als freundlich und respektvoll wahrgenommen. min24/25)

Informationskanäle: Internet, Telefon

Asylstatus: Aufenthaltserlaubnis (bis Dez.)

Interviewee war zum Zeitpunkt der Ankunft in Deutschland nicht volljährig; ein Vormund kümmerte sich deshalb um alle Anfänglichen Aufgaben(min27).

Wie viel über die persönlichen Umstände des Interviewten erzählen? -> Vertraut dennoch in den deutschen Staat
Auf der Flucht: Türkische Polizei hat den ertrinkenden eines gekenterten Bootes auf der Fluchtroute meim Ertrinken zugeschaut, (min 12) dennoch Vertrauen in deutsches Rechtssystem

Potentielle Nachfragepunkte Interview 5: min25, Aber (...) Mein Chef war nicht gut. (...).
                -> potentielles follow-up.
                
\subsection{Interview 3}

Demographisch
Fluchtgrund
B: ähm, hm. (...) Ah, als ich nach Deutschland, bevor dass ich nach Deutschland komme, habe mir gedacht: 'Ja, wenn ich in Deutschland bin, oder in Europa' #00:25:04-6#

I: hm (bejahend) #00:25:04-6#

B: ähm, 'kann ich dann normal leben vielleicht?' Also, ich hab mir so gedacht am Anfang. Dass ich dort normal Leben kann.  #00:25:12-1#

Fluchtroute
Kontakte
Medizinisch
Wohnung
Bildung
Rassismus
Bayrisch
Inflormationslücken

Notizen
Erwähnt mehrmals, wie wichtig Regeln in D sind (zB min14:48)
schließt aus mangel an kommunikationsfähigkeit auf potentiell Ärger mit dem Gesetz?(min16:52)

Situation: Im Jugendcafé, am X.X.2019. Proband umgeben von Freunden und Bekanntschaften, die immer weider an den Tisch kommen. Wurde von Ihm explizit gewünscht, das da zu machen.

Demographisch:
männlich, 21
Geboren im Iran, Afghanische Familie (sieht sich selbst als Afghane, und wird im iran als afghane diskriminiert.)
4 Jahre in Deutschland, seit 2015
Bildung:(min4)	hat im Iran lesen und schreiben gelernt, wurde von Bekannten unterrichtet.
Fluchtroute: (Landroute?)
Fluchtgrund: Unterdrückung?
Asylstatus: Asylverfahren läuft noch (§25a), Abschiebeverbot wgn im Iran geboren, laufendes Verfahren wegen schwerer Körperverletzung)

Kontakte:   Soziale Einrichtung, EJSA - Jugendcafé dient als Ankerpunkt und soziale Stütze?
			Cousin (gemeinsam mit Ihm geflohen, min7)
			Fußballmannschaft (min7)
			Freundin, Ihre Familie
			Familie zu Hause (Internet: Facebook? min24)
			
Berichtet davon, Anfangs Probleme mit den Regeln, der Sprache, (Was ist peinlich?) gehabt zu haben. Hatte zwei Monate vor dem Start seines Deutschkurses komplett 'frei', er sieht dies als positive Erfahrung, welche eine Adaption an die neue Kultur ermöglichte. (min11: B: Genau. Dann haben wir mit durch (unv.) unso weiter haben wir immer ein bisschen mehr verstanden, Regeln und die Sprache undso. Das wird immer schwieriger dann,  #00:10:12-8#

I: hm (bejahend) #00:10:12-8#

B: wenn man nicht versteht und macht irgendwas, dann denkt man nicht ob das (...) schlimm ist undso ne? #00:10:16-6#)

Hebt die Lehrerin Seines Deutschkurses explizit als positiv hervor, hald Ihm, sich wohl zu fühlen und die Kultur zu verstehen.
Vertraut dem Deutschen Gesetzwerk und damit der Polizei; Iranische Polizei: Willkürlich, macht, was Sie will.(min15)

ab min 18: Schlägerei bei Fußballspiel gegen Regenstauf; Ist momentan wegen schwerer Körperverletzung angeklagt.

min22:17: äußeret Wunsch, dass Ausländer differenziert und individuell betrachtet werden (nicht nur als ausländer)

Notiz zu Angst vor der ABschiebung:
Abgeschobene, die dann starben. (min27f)
			
Context -> Situation -> Information need
Informationslücken:
Sprachbarriere nach der Ankunft (min20)
Rassismus (min21) (in Regensburg weniger, in Dresden mehr)

Potentielle Nachfragepunkte Interview 3:

\subsection{Interview 4}
(Nicht sicher, ob der sich so recht wohlgefühlt hat. Interview wurde im Wohnzimmer seiner WG durchgeführt; Platz/Zeit wurden von Ihm ausgewählt. Kurz vor Ramadan, deswegen jetzt noch schnell)

Setting

Demographisch
Fluchtgrund
Fluchtroute
Kontakte
Medizinisch
Wohnung
Bildung
Rassismus
Bayrisch
Inflormationslücken
Notizen

Demographisch:
männlich, 20
Afghanistan, Sunnite
Seit Anfang 2015 in Deutschland
Fluchtroute: Landweg (LKW in der Türkei bestiegen, in Regensburg wieder raus (min2))
Asylverfahren: Duldung (Seit Anfang 2018), laufendes Verfahren nach §25a). Interview beim BAMF:                        min26/27. 2x abgelehnt,
            


Kontakte:
Anfangs keine in Deutschland (min5)
Afghanistan: Onkel
Jugendcafé
Betreuer der WG

Wohnungssuche:
    Nicht möglich, ohne unbefristete Aufenthaltserlaubnis, bei laufendem Asylverfahren eine 'normale' Wohnung zu mieten(min25). Deshalb Sozialwohnung einer Örtlichen Vereinigung (St. VInzent)

Medizinisch:
wurde von den Behörden übernommen

Bildung:
2 Jahre Schulbildung + 1 Jahr Koranschule
Macht vielleicht Meister nach seiner Ausbilung in Deutschland? (min12)

Infolücken:
Sprachbarriere, anfangs Verständigund mit Händen und Füßen(min4). Behandlung der verletzten Beine        nach der Flucht ohne Kommunikationsmöglichkeit mit den behandelnden Ärzten. (Kann sich nicht mehr     an viel erinnern, min12)
sprachbarriere: bayrisch (machte deutsch lernen schwieriger). lösung: arbeit (min 6,'oh! so               funktioniert!' min16)
    (B: Und ich hab auch ein Tag zum Beispiel 'Habadere' gehört, das war mein erste Wort das ich gelernt hab. #00:05:25-3#)
    Hat bei einer vorherigen WG mit deutschen Kindern viel Deutsch gelernt
Wie Praktikum bekommen? (min14) -> In Zusammenarbeit mit den Betreuern und Lehrkräften. Vertrag wird     an die Bedürfnisse des GEflüchteten angeglichen (um die Schulzeit herum)
(Infolücke: Als Lackierer keine weißen Klamotten anziehen. Wie bekommen? Die haben's Ihm gesagt!         min15/16)
Warum reden die nicht mit mir? (min16/17) -> schon vorher in der Schule geklärt: B: Genau, davor haben     wir in die Schule auch gelernt, dass wenn bei deutsche arbeitet oder irgendwas macht, dann die        brauchen länger, dass die dich kennt #00:16:33-0#
Wie Leute kennen lernen? -> Praktikum gemacht, Sprache gelernt etc. min17
Wie bekomme ich eine permanente AUfenthaltserlaubnis? (hier: 25a), min 25/26); min29: Schule,             Ausbildung. Woher die Info? Viele Leute -> diverse Kontakte. NIcht im INternet (min30)
Ablehnung des Asylgesuchs (2x), weiß noch nicht, warum. (min28). War Ihm irgendwann zu viel, er hat's     dann anders probier. (Aber irgendwann habe ich alles gelassen. #00:27:56-3#)


Keine Erfahrungen mit Rassismus in Deutschland (min11)
B: Am Anfang war schon Probleme, aber ich hab das gemerkt, dass die (...) manche Leute so vorsichtig mit mir war. #00:15:55-2#
[..]hald, so, vorsichtig, nicht geredet und so einfach

Notizen:
Trip nach Hamburg mit dem Fahrrad (min19). Erst trainiert, dann losgefahren.
(min21: nicht gewillt, über Autoritäten in Afghanistan zu reden)
Hat gelegentlich als 'Dolmetscher' zwischen Betreuern und NEuankömmlingen fungiert (min23)
Unterschiede zu Heimat: Religiöse Feste (min31f)
    Ramadan von Mitarbeitern akzeptiert. min32

Nachfragepunkte:

\subsection{Interview 5}
Situation/Seiiting: 09.05.19, im Jugendcafé. Nebenraum des Jugendcafé. Interviewee zeigte sich offen und motiviert, wollte seine Geschichte erzählen. Zeit und Ort wurden vom Interviewee ausgewählt: jetzt, hier - vielleicht hinten im Nebenraum?
min54:      Interkulturelle Differenz Deutschland VS naher Osten (nicht verwerflich, nachzufragen)
min55:      hinterfragen der Prozesse -> deutsch, oder an junges Alter in Heimat gebunden?

Demographisch
Männlich, 19
Seit Mitte 2015 in Deutschland
Im Iran geboren; Eltern Afghanen -> Keine iranischen Ausweisdokumente. (min5)
Hat unbefristete Aufenthaltsgenehmigung (min70)


Fluchtgrund:
min8    Bürger zweiter Klasse:B: Wir dürfen nicht in Bäckerei arbeiten. Wir dürfen nicht im Krankenhaus arbeiten. Wir dürfen nicht, zum Besipiel, im ähm (...) Amt arbeiten.[nicht studieren, nicht lernen] #00:07:19-4#
B: Aber hier Gott sei Dank, wir dürfen schon in die Schule gehen. Wir dürfen schon Ausbildung machen. Wir dürfen schon (...) Welche, welche, keine Ahnung, welche Beruf (...) Welche, zum Beispiel welche (...) Wünsche haben wir? Hier, ich glaube (...) am besten. Das, das kann ich nicht einfach mehr, ähm, mehr Sätze dazu sagen. Ich weiß nicht, weil, ähm (...) Wenn du zum Beispiel im Iran bist #00:08:33-2#

Fluchtroute:
min13   Mai 2014 geflüchtet, 11 Monate unterwegs.
min14   Flucht via Schiff, Türkei nach Griechenland. 2 Wochen (10 Tage ohne Essen, mit Spenderniere           Meerwasser getrunken -> Infolücke..min15).
min19   700 Leute Schiff
min34   zu Fuß bis Österreich, dann mit Zug nach Deutschland.

Kontakte
Zwei unbekannte Onkel + zwei Tanten in Afghanistan, die gesehen wegen Gesetzeslage (B: (...) Muss man nochmal schwarz an die Grenze, ja. #00:05:58-7#). 
min 9   Cousin im Iran, Elektriker(privat finanziert). Arbeitet als Einzelhändler, darf nicht                 praktizieren( min 11)
Kollegen in der Arbeit
Betreuer + Mitbewohner in der WG
Jugendcafé
Sportclubs


Wohnung
min69   2 oder 3? wohnt jetzt in einer privaten Wohnung

Medizinisch
min1:   Hat 3 Jahre Dialyse hinter sich (3x wöchentlich, im Iran, start mit 7 oder 8(min3))
min2    Verkleinerte Niere
min3    2012 Nierentransplantation; geht alle 6 Wochen in D zum Arzt+ nimmt MEdikamente
min38   Macht sehr viel SPort in D

Bildung
B: Wir dürfen nicht studieren. Wir dürfen nicht lernen. Als ich krank war, ich ähm (...) als ich klein war, oder jünger war #00:07:26-5# B: Ich war nie in der Schule. #00:07:30-2#
min 8   erste besuchte Schule in Deutschland, mit 15 oder 16
min 8   6 monate Koranschule -> kennt arabische + persische Buchstaben
min 36  2 Jahre Integrationskurs, dann Quali, dann Ausbildung als Elektriker (Vorbildung im Iran, 2,5         Jahre, min55)
min40   Quali in D (nicht bestanden, vgl, Informationslücke)
min49   Beginnt im September Elektrikerausbildung, nachdem er ~6 Monate die Fachbegriffe gelernt hat.         Falls noch nicht bereit, nächsten September
min60   Weiterbildung im Jugendcafé -> Englisch; Schwimmkurs; Kletterkurs

Rassismus

Informationslücken
min36f   Firma sagen, ob Schwerbehindertenkarte vorhanden?
min39   'Was soll ich machen jetzt? -> Ausbildungsplatz war weg; alle Klassen waren voll
min40ff.     Umgang/Maschinenschreiben mit Computern ohne Zugang zu Laptops lernen? (min40:58 eigener              Lösungsansatz) -> Demotivation(leider ich kann nicht, min48) -> nicht bestanden

Bayrisch:
min65f      Kein Problem (nur mit bayerwald-dialekt, min67. wurde aber positiv aufgefasst)

Nachfragepunkte


Notizen
B: Wenn jemand hat (...) ähm, Gesundheit, und sagt: 'Hey, Gott! Warum gibst du mir nicht Geld?' Ich sage: 'Hey, Gott! Gott sei Dank, dass du mir eine, ähm, neue Niere mir gegeben. Und ich gehe nicht zum Dialyse. Ich brauche kein Geld. Ich brauche (...) eigentlich meine Körper einfach gesund wird.  #00:04:05-2#
B: //Schwarz aber sehr gefährlich. Die dürfen schießen an die Grenze #00:06:02-9#
will nie wieder regeln brechen: B: Ja, ähm, jetzt, in Deutschland, wenn ich würde eine, ich wollte eigentlich zum Beispiel (...) eine Straße zum Beispiel ist 100 Meter, ähm, 100 Meter weit weg. Es gibt eine Ampel. #00:15:21-8#
min35f. Optimismus: B: Ich sage nicht, ähm, hey, keine Ahnung: Scheiße Schule, oder Scheiße, keine            Ahnung, die Leute, oder (...) ähm Scheiße Arbeit oder scheiße den oder Scheiße den. Ich sag           garnix! Ich, ich chille! #00:34:20-3#
min51       Irakischer Arbeitskollege kann die Begriffe selber ned, die er Ihm (sofern Sie gepartnert             werden) erklären soll (Irak arabisch, Iran persisch -> kommunikation auf Deutsch. min53)

min55        anderes Verhältnis zu 'Weiß ich nicht, erklär mir das bitte' in Afghanistan
            -> hier einfach nachfragen, in Afghanistan/Iran nicht hinterfragen!

\subsection{Interview 6}

Setting: In der WG des Probanden auf seiner Couch; er kam gerade von der Arbeit heim. Nach einem Gebet begann das Interview. Interviewee brauchte etwas Zeit, um 'aufzutauen', zeigte dann aber Vertrauen.

demographisch:
21, männlich
Afghanistan, im Iran aufgewachsen (mit 8 in den Iran) -> kein Pass
seit 2015 in D
Duldung (25a läuft

fluchtgrund

fluchtroute
min11   Landroute, Türkei anch Österreich im LKW (zu wenig Essen an bord)

kontakte
Jugendcafé
Betreuer der WG
Arbeitskollegen
Freunde in der Stadt (D und Ausländer)
Familie im Iran, über's Telefon

medizinisch
min35   geregelter ablauf f. Boxen -> immer di+do um 17 Uhr.

wohnung
min29   wurde im Bettenlager(erste Wohnung, 20-30 bett-saal) von nem besoffenen RUssen verprügelt
        resultierte in EInzelzimmer, dann X 3,5-4 monate.
min32   dann nach st vinzent, mit dem Kumpel, mit dem er her kam. 1.5 Jahre
min33   dann in's betreute Wohnen, mit besagtem Kumpel (betreutes Wohnen -> keine permanente                  Genehmigung). Kumpel bekam ne permanente und hat jetz ne 'normale' Wohnung

bildung
min5    Ausbildung seit 5 monaten (Karosseriebauer)
min9    Geht man nicht in die Schule. Lernt man nur  in der Werkstatt. #00:08:23-4#
min9    keine Schulbildung in Persien, weder lesen noch schreiben
min10   3 Monate Deutschkurs (erster Tag als sehr hart empfunden)


rassismus
min6    vielleicht Rassismus? Zahntechnikerausbildung fällt weg, wil die keine Ausländer wollen.
        B: Ja. Dort habe ich schon zwei Wochen Praktikum gemacht, und (...) ja, da habe ich schon eine Stelle gefunden, und (...) ja. Und (...) dann (...) ich habe so  zu spät meine Bewerbung abgegeben und die haben eine andere genommen, und (...) der Junge hat schon zwei Monate, drei Monat gearbeitet, und einfach weg(...) gehaut. #00:06:30-7#
min45   alte Frau hat ihn auf offener Straße beschimpft und bespuckt. Alter Mann: Geh dahin wo du             herkommst! Nehmt unser Geld! Geht was arbeiten (war auf dem Weg zum EInkaufen)
        min47: ist nicht darauf eingegangen.
min48   deutscher Freund wird vor 25-30 jährigen angesprochen: 'Hey, was machst du mit dem Ausländer?'         ist nicht darauf eingegangen.
B: ich habe geasgt, wenn du gegen (...) gegenseitig was schimpfst, (...) den Typ, dann bist du auch, so eine (...) so eine Idiot. Dann (...) Wie unterscheidest du zwischen dir und Ihm? #00:49:58-0#


bayrisch
min25   hatte anfänglich Probleme mit bayrisch (wird va in der werkstatt gesprochen) -> mitarbeiter           haben's ihm beigebracht (min26 'die wollen einfach, dass ich bayrisch lerne) 



informationslücken
min4    wie lang muss ich auf's BAMF warten? -> ging nicht mehr zum BAMF, jetzt         §25a
min14   wie mit der Polizei kommunizieren? keiner spricht die Sprache des               anderen, kein Dolmetscher           anwesend. Google Translate! (Freund         konnte persisch lesen und schreiben)
min17   wollten nach Frankfurt zu nem Kumpel. Was ist Bayern? Regensburg?!
min21   mit temporärer aufenthaltserlaubnis zugriff zur Disco verwehrt -> Bürger         2. Klasse? fühlt sich         kacke an. Was machen in der SItuation?
min28   am Anfang: was war das auf bayrisch? -> schrauben falsch gebracht etc
min36   welchen Sport machen? Fitnessstudio is kacke, und alles is verdammt             teuer!(min37 ~100€ mtl.)
        -> selbst im Jugendcafé organisiert.

Notizen
min4    18 Monate keine Antwort vom BAMF auf unbefristete Aufenthaltegenehmigung
min6    wollte eigentlich nicht karosseriebauer werden, hatte schon vier Jahre          im Iran als Mechaniker         gearbeitet. wollte hier was neues machen.
min9    neue ausbildung direkt am ersten Tag des Probearbeitens bekommen
min14   erster kontakt: sehr hilfsbereiter muslimischer taxifahrer
min16   erster kontakt mit polizisten: waren nett, gaben Essen
min20   nur mit Bestechung ausm Knast wieder raus
min23   zist gut zu §25a informiert -> muss ausbildung nicht schaffen.
min39   ausprobieren des Jugendcafé: lernt deutsch, nette leute. er kommt wieder. auch mitgekocht,            aber jetzt überschneidung mit Boxen
min41   Ramadan -> geht nicht ins Jugendcafe (muss npoch für'n Tag darauf               kochen)
min49   Iran: Polizei kacke, Leute fast alle super.
min53   äußert bedenken, dass potentielle rassisten mit 'Heute hat eine                 Ausländer Scheiße gebaut.'            #00:52:31-1# konfrontiert wird. ->         keine individuelle Behandlung des Falles, zuerst der                            'Ausländerfilter'. Objektive AUseinandersetzung mit einem Verbrechen nur         ohne die                     'Ausländer-INfo'
min55   aufenthaltserlaubnis: teil des öffentlichen Lebens; geht zur Arbeit,            trifft Leute.
        Was, wenn die Abschiebung bevor steht? darf nicht in die disco, nicht ausreisen -> Psychicher stress. 'baust du irgendwann scheiße' min56
        min57: hat schon von abgeschobenem nacha fghanistan gehört
        %Interview 3 hatte doch 2, die nach der Abschiebung starben?
        
        B: Dann hörst du solche Geschichten, und dann hast du viel Angst. (...) #00:56:57-0#
        
        B: Die erzählen (...) irgendwelche Geschichte und sagen: 'Ja, du musst dort gehen, und du musst kämpfen, und ja' Möchten natürlich nicht gehen! 'Oh? ähm, du gehst in Gefängnis! (...) vier bis fünf Monaten. Dann, schieben wir dich nach Afghanistan danach.' (...) #00:59:55-5#


\subsection{notes}
Deutschland weniger als erklärtes Zielland, mehr als 'ich will hier weg, irgendwie nach Europa kommen' oder das Besuchen von Bekannten in Deutschland




\section{Diskussion der Ergebnisse}

\section{Fazit und Ausblick}

10\% der Arbeit

\clearpage
\lhead{}
\printbibliography
\addcontentsline{toc}{section}{\bibname}

%%%%%%%%%%%%%%%%%%%%%%%%%%%%
%% Eidesstattliche Erklärung
%% muss angepasst werden 
%% in Erklaerung.tex
%%%%%%%%%%%%%%%%%%%%%%%%%%%%
\input{Erklaerung.tex}

\newpage
%\appendix
%
\section{Interview 1, Teil 1}

I: (...)Ah, jetz (...) ähm /also es qird jetz quasi aufgenommen und dann später setz ich mich an' Computer, hör's ma nomml an und tipp's runter #00:00:13-2#

B: okay #00:00:11-9#

I: ähm Beziehungsweise wir ham auch ne Software ähm da muss ich mich aber erst noch ähm genau informieren wie des dann aussieht #00:00:23-1#

B: okay #00:00:23-1#

I: aber grundsätzlich die(..)meine Arbeit bezieht sich da drauf ähm im bei im Integrationsprozess ähm dann ähm Lücken zu finden also wann hast du zum Beispiel irgendwann mal so nen Erlebnis ghabt 'Oh! So hätte des funktioniert!' ähm so ja ist jetz vielleicht ne schlechte Erklärung. ähm ich weiß ned seit wann bist du denn in Deutschland? #00:00:51-2#

B: Also ich bin seit drei einhalb Jahr in Deutschland #00:00:53-2#

I: Seit drei einhalb Jahren. #00:00:54-2#

B: Seit drei einhalb Jahren. #00:00:55-3#

I: hm (bejahend). Du sprichst phänomenal gut Deutsch. #00:00:58-9#

B: Es ist nicht Wunder. (lacht) Es ist kein Wunder. Es gibt schon viele, die mehr als ich(...)Schöner als ich sprechen können auch. Es gibt Leute die einfach bayrisch gelernt haben. Also ich bin nicht der einzige deswegen (...) es ist kein Wunder also wenn man in einem Land lebt dann muss ich die Sprache beherrschen oder? #00:01:16-5#

I: hm (bejahend) Trotzdem du(...)du sprichst gut. ähm darf ich dich fragen wie alt du bist? #00:01:21-5#

B: also ich bin 23 #00:01:24-6#

I: du bist 23?! #00:01:22-9#

B: ja(...)sehe ich älter aus oder was? #00:01:26-9#

I: du siehst älter aus. ich bin 24. #00:01:26-8#

B: (lacht) echt jetzt? #00:01:27-8#

I: ja(lacht) #00:01:29-6#

B: ne ich bin immer noch(...)noch jung #00:01:31-5#

I: mhm #00:01:33-5#

B: ja. Vielleicht wegen dem Bart oder sowas dann sehe ich älter aus(lacht) #00:01:36-6#

I: Hab ich keinen anständigen Bartwuchs leider.(lacht) #00:01:37-7#

B: (lacht)ja. #00:01:41-7#

I: ähm warte(...)was war für dich so die Entscheidung ähm warum hast du dich entschieden dass Du nach Deutschland kommen willst? #00:01:55-8#

B: Also zuerst die Entscheidung war nicht dass ich nach Deutschland komme #00:01:57-8#

I: sondern einfach nur Europa? #00:01:56-3#

B: Es ist eifach dass ich raus von Syrien komme.hm  #00:02:01-1#

I: hm (bejahend) #00:02:00-6#

B: Egal wo ich den Weg mich nehmt.  #00:02:04-2#

I: hm (bejahend) #00:02:04-2#

B: Und als ich dann weg von Syrien war habe ich mich entschieden dass ich in die Niederlande dann (...) #00:02:09-0#

I: hm (bejahend) #00:02:09-0#

B: Also in die Richtung der Niederlande #00:02:11-8#

I: hm (bejahend) #00:02:11-8#

B: Aber daher hat die Polizei mich angehalten und ich darf nicht weiter fahren. #00:02:16-6#

I: ähm wo bist du da angehalten worden? #00:02:20-2#

B: In Passau. #00:02:20-2#

I: In Passau? #00:02:22-8#

B: In Passau. #00:02:22-8#

I: Okay. #00:02:22-8#

B: Da ist die Polizei und die ahben mich (...) also die haben mich (unv.) um die dann den Fingerabdruck zu machen #00:02:30-7#

I: hm (bejahend) #00:02:30-7#

B: Und ich hab gesagt, ich wollte dann weiter in die Niederlande fahren. #00:02:33-9#

I: Ja? #00:02:33-9#

B: Nein. Das machst du vielleich später. #00:02:38-3#

I: Okay.. #00:02:38-3#

B: Das war ja ein Dolmetscher dabei und ich hab mit dem geredet und ich wollte ja weiter (...) ne, leider funktioniert jetzt nicht.(...) Okay(...) #00:02:51-4#

I: hm (bejahend) #00:02:51-4#

B: Okay. Aber naja also ich bereue mich überhaupt nicht (lacht). Hier habe ich auch nicht #00:02:55-5#

I: //Herzlich willkommen.(lacht) #00:02:56-2#

B: Danke(lacht) #00:02:57-9# Ja, sozusagen habe ich hier bis jetzt mich schon gut integriert #00:03:07-0#

I: ähm #00:03:07-0#

B: Die Sprache gelernt.. #00:03:06-9#

I: Ja #00:03:06-9#

B: Und viele Freunde auch habe ich schon(...)ja(...)Aber man fühlt sich immer so dass man immer noch ein Lücke hat. Weißt du, es geht darum dass ich vielleicht in Syrien aufgewachsen bin, und dann bin ich nach Deutschland gekommen #00:03:23-2#

I: hm (bejahend) #00:03:23-2#

B: Weil es spielt schon große Rolle. Wenn man in einem Land aufgewachsen ist und dann kommt auf ein anderen Land und der hat sich gewöhnt wie er dort(...)was er macht und was für Ihn angeboten wurde, und hier dann etwas anderes. #00:03:42-9#

I: ähm #00:03:44-4#

B: Die Kultur spielt eine Rolle, die traditionelle spielt auch eine Rolle #00:03:49-5#

I: Wie unterschiedlich sind die beiden Kulturen? #00:03:53-0#

B: 180 Grad. #00:03:53-0#

I: Echt?(lacht) #00:03:53-0#

B: Ja. 180 Grad. #00:03:54-6#

I: ähm inwiefern? Also ich.. #00:04:00-1#

B: //INwiefern.. Also.. #00:04:00-1#

I: Ich war noch nie in Syrien #00:04:02-3#

B: Ja (...) in Syrien, also  ich sag's immer den(...)den dieses Sprichwort, für mich ist das immer so: Nicht alle Finger gleich. #00:04:13-6#

I: Ja.. #00:04:13-6#

B: Es sind nicht immer alle Finger gleich. Aber wo ich gelebt habe und wo ist mein Gebiet ist und wo ich aufgewachsen bin(...)da war für mich ja, wenn ich das vergleichen, da ist die (unv.). Aber jetzt, also vielleicht es ist am Anfang. Aber jetzt habe ich mich gewöhnt. #00:04:31-8#

I: Ja. #00:04:31-8#

B: Und es gibt immer noch schwere Punkte. #00:04:38-4#

I: Zum Beispiel? #00:04:38-4#

B: Ja, also strenger Muslim bin ich nicht, aber zum Beispiel ich trinke Alkohol. #00:04:42-5#

I: Hab ich letztes mal mitbekommen (lacht) #00:04:45-7#

B: Was hast du? #00:04:44-6#

I: ähm bei unserem ersten Termin #00:04:47-8#

B: Okay #00:04:46-3#

I: ähm wir hattens ja schonmal ausgemacht ghabt ähm das war ganz lustig dann du hattest ja dann gschrieben 'Oh scheiße, du warst gestern bis fünf was trinken #00:05:00-2#

B: Ne also ich war nicht beim trinken sondern das war eine Party. #00:05:06-8#

I: hm (bejahend) #00:05:06-8#

B: Eine Geburtstagsfeier #00:05:08-9#

I: hm (bejahend) #00:05:08-9#

B: Aber wir machen das(...) also wir feiern aber ohne zu Trinken. Es ist(...) es geht einfach. #00:05:15-2#

I: ähm ein Freund von mir, also ich bin draußen im Jugendcafé, vielleicht kennst du das? #00:05:21-4#

B: Wo ist das? #00:05:22-3#

I: ähm das ist.. #00:05:25-7#

B: Trefft ihr euch jeden Donnerstag.. #00:05:29-3#

I: Jeden Donnerstag, genau #00:05:29-3#

B: Okay, ja (lacht) #00:05:29-7#

I: also da hinten draußen, hinter'm Bahnhof. #00:05:32-5#

B: Ja, eine Freundin von mir die kommt immer. #00:05:36-4#

I: Ah? #00:05:35-1#

B: Ja. #00:05:35-1#

I: ähm also nen Freund von mir ist eben dort und der hat da auch seinen Geburtstag gefeiert #00:05:41-6#

B: hm (bejahend) #00:05:41-6#

I: ähm(...)und da is kein Tropfen Alkohol getrunken worden #00:05:49-2#

B: Braucht man nicht(lacht) #00:05:49-2#

I: Genau! #00:05:50-0#

B: Das braucht man nicht. #00:05:51-3#

I: Das war für mich tatsächlich komplett neu #00:05:54-5#

B: Ja. #00:05:55-1#

I: Eine Geburtstagsfeier, auf der nur getanzt wird. #00:05:57-6#

B: ja. #00:05:57-6#

I: UNd es war schön! Es hat Spaß gemacht! #00:05:57-6#

B: Ja. #00:06:00-2#

I: ähm(...)ja(...)Andere Kultur, aber es is.. #00:06:04-1#

B: Ja es ist andere Kultur. Also es ist nicht 'Schock' für mich also ich sag's nicht dass in Syrien wird nie jemals getrunken oder sowas #00:06:13-9#

I: hm (bejahend) #00:06:13-9#

B: Doch, schon! #00:06:14-3#

I: Ja #00:06:14-3#

B: Und es gibt schon, und es gib schon Laden für den Alkohol, wo man reinkommt und dann kauft für sich einfach was er will. #00:06:23-7#

I: hm (bejahend) #00:06:23-9#

B: Aber so üblich oder so, so groß wie hier in Deutschland, das habe ich nicht erlebt. Aber mir ist(...)Also mir interessiert überhaupt nicht ob jemand trinkt oder nicht oder sowas. #00:06:38-2#

I: Ja. #00:06:38-2#

B: Mich interessiert einfach, dass jemand zu mir kommt und sagt: Warum? #00:06:42-2#

I: hm? #00:06:45-0#

B: Also warum(...)wir trinken schon, warum trinkst du nicht? Ich sag's ganz ehrlich du hast ja, also.. (unv.) Die Antwort kommt sofort: Du bist hier in Bayern. Du musst trinken! #00:06:58-0#

I: Nein? #00:06:58-0#

B: Du musst Schwein essen! #00:06:57-9#

I: Ahh.. #00:06:57-9#

B: Wieso willst du dich integrieren? Hää? #00:07:02-4#

I: Was?(lacht) #00:07:02-4#

B: Hää? Was hat das mit Integration zu tun? #00:07:06-9#

I: //(lacht) Passiert dir das öfter? #00:07:07-0#

B: Ja also für mich ist viele Geschichten entgegen also (unv.) am Anfang der (...)ja..(...) Am Anfang, so 2016, 2017.. #00:07:17-9#

I: Ja? #00:07:17-9#

B: ..War das für mich schon so hart, weil ich einfach nicht verstanden hab, weil mein deutsch ist ganz schlecht, und deswegen ich habe nicht ganz (...) nicht alles verstanden #00:07:31-8#

I: hm (bejahend) #00:07:31-8#

B: Und (...) Und ich kann nicht antworten. #00:07:33-9#

I: Ja.. #00:07:33-9#

B: Also das ist die, die, die.. also, jetzt hab ich schon meine Gründe und warum antworte ich überhaupt nicht. Jetzt kann ich einfach so meine Taten dann begründen. Warum mache ich das? Warum mache ich das? #00:07:47-9#

I: hm (bejahend) #00:07:47-9#

B: Genau. Aber am Anfang war schon wirklich schwer. Oder die nehmen das einfach (unv.), da sitzten wir zum beispiel so fünf, sieben Leute zusammen #00:07:59-6#

I: hm (bejahend) #00:07:59-6#

B: Ja der bestellt für sich einen.. ein Bier ein Bier ein Bier. Und dann kommt der Kiellner zu mir und sagt: Was hätten Sie gerne? Dann sagen die alle entweder Capuccino der Kaffe. Da lache ich damit. Aber die haben einfach so Abstand gemacht #00:08:15-7#

I: hm (bejahend) #00:08:15-7#

B: Wenn man nicht mittrinkt, dann (...) es gibt schon immer Abstand. (...) #00:08:22-1#

I: (...) Komisch. #00:08:22-1#

B: Es ist doch komisch, ja. Obwohl dass ich in einen Bar gearbeitet habe. Also ich habe hier in Sachs. Kennst du Sachs? #00:08:29-5#

I: Ja. #00:08:29-5#

B: Da habe ich einen Jahr gearbeitet #00:08:30-5#

I: //Du hast im Sachs gearbeitet? (lacht)#00:08:30-5#

B: Ja. Also der Ladenführer war mein Nachbar, also ist mein Nachbar bis jetzt. Und der hat mir die Arbeit angeboten, dann aht gesagt: 'Ja, du kannst (...) wenn du willst, dann kannst du da arbeiten.' #00:08:44-7#

I: Okay #00:08:46-3#

B: Ja ich hab als Türsteher ein Jahr gearbeitet. #00:08:49-7#

I: hm (bejahend) #00:08:50-4#

B: Ich hab mit den Kunden nichts zu tun, also immer #00:08:53-6#

I: Ja? #00:08:53-6#

B: Sondern nur ab halb zwei habe ich das zu tun, und dann vor die Tür. #00:09:02-9#

I: Ja. #00:09:02-7#

B: Dann mussten die alle leise sein. #00:09:01-3#

I: Ja. (...) (lacht) also ähm #00:09:10-0#

B: Also im Sachs habe ich. (lacht) #00:09:08-4#

I: //(lacht) #00:09:08-4#

B: Aber du warst niemals im Sachs. #00:09:11-4#

I: Also (...) pff, früher.. #00:09:11-4#

B: // ne nicht niemals, also als ich war. Also als ich gearbeitet habe. #00:09:13-0#

I: ähm #00:09:17-9#

B: Eigentlich es ist kein Bar. #00:09:19-7#

I: Nicht? #00:09:19-7#

B: Es ist kein Bar. Es ist einfach (...) also wenn man trinken will, dann geht nicht in Sachs. Wenn man saufen will, dann geht in's Sachs. #00:09:30-1#

I: (lacht)..Das stimmt(lacht) #00:09:30-1#

B: Es ist einfach so. Also ich hab so viel erlebt. So viele Sachen erlebt.(...) #00:09:41-2#

I: Ich komm mal wieder zurück zu meinem Leitfaden, Entschuldigung. #00:09:43-6#

B: Ja #00:09:43-6#

I: ähm, ah, ähm mit wem hast du hier ähm ah. Also du bist jetzt seit dreeinhalb Jahren hier. #00:09:54-7#

B: ja. #00:09:55-7#

I: Mit wem hast'n du normal hier in Deutschland Kontakt? Also.. Und hast du auch noch Kontakt nach Hause? Also, zu..#00:10:02-1#

B: //Ja, also Kontakt zu Hause habe ich schon, aber es ist einfach.. da war also (...) Zeit vor sechs Monaten waren, da konnte ich mit meinen Eltern nicht kontaktieren. #00:10:16-0#

I: hm (bejahend) #00:10:16-0#

B: Der Krieg war so dort, und das können wir einfach (...) die sind einfach verschwunden vor ungefähr sechs Monaten da hatten wir kontaktlos mit denen. Aber danach haben wir hald dann wieder Kontakt mit ihnen #00:10:27-4#

I: hm (bejahend) #00:10:27-4#

B: Das war ja also die große (...) die große Scheißezeit für mich. Also für uns alle drei. #00:10:35-3#

I: Das glaube ich. #00:10:35-3#

B: Ja. (...) Aber Kontakte hier in Deutschland.. Ich haben den alle. Also ich bin flexibel. #00:10:43-3#

I: hm (bejahend) #00:10:43-3#

B: Ich treffe mich mit denen alle. #00:10:47-8#

I: ähm #00:10:48-5#

B: Ja also ich hab kein Problem mit Fremde oder mit wem oder sowas, damit habe ich garkein Problem. #00:10:56-3#

I: hm (bejahend) #00:10:56-3#

B: Also viele Deutsche habe ich schon, also ich hab viele Kontakte mit den Deutschen auch #00:11:00-8#

I: hm (bejahend) #00:11:00-8#

B: Die sind ja älter. Also die sind ältere Leute, mehr als die junge Leute. Weil die Jungs, die, die erzählt haben, die, also wenn mit dem Alkohol da machen sie einfach Abstand. #00:11:14-8#

I: ehh.. #00:11:14-8#

B: Viele Kontakte habe ich mit denen überhaupt nicht. #00:11:18-6#

I: hm (bejahend) #00:11:18-6#

B: Also ich weiß es nicht, vielleicht habe ich mich zurückgehalten oder sowas, aber da habe ich schon ein paar  Situationen erlebt, und deswegen habe ich gesagt 'Mmh, jetz ist nicht.' #00:11:31-2#

I: (lacht)Auch die jungen sind nicht alle so. #00:11:31-2#

B: Ja die sind nicht alle so, ja. #00:11:36-8#

I: Verschiedene Finger einer Hand. #00:11:36-6#

B: Genau.(...) Ja also ich habe zu tun mit den älteren MEnschen dann mehr #00:11:44-9#

I: hm (bejahend) #00:11:44-9#

B: Ja also die haben mehr Zeit. Also es kann sein auch die jungen Leute die haben nicht viel Zeit #00:11:52-6#

I: hm (bejahend) #00:11:52-6#

B: ..Und wenn du einfach mal also mit den jungen dann triffst und sowas, vielleicht treffen wir zusammen ein Monat oder zwei Monaten. Aber wenn zum Beispiel so eine Woche oder zwei Wochen oder einen Monat dann keinen Kontakt gibt #00:12:04-7#

I: hm (bejahend) #00:12:04-7#

B: Kein Anruf gibt, dann vergessen Sie dich einfach. #00:12:10-4#

I: ähm #00:12:10-4#

B: Auf die Seite. #00:12:12-0#

I: Ich hab nen paar Kontakte, ich glaub denen habe ich schon seit nem Jahr oder was nimmer zurückgeschrieben - aber das sind trotzdem immer noch Freunde von mir. #00:12:20-4#

B: Ja. #00:12:20-4#

I: Also(...) #00:12:22-0#

B: Ich weiß es nicht, also die Mentalität ist unterschiedlich. #00:12:24-6#

I: hm (bejahend) #00:12:24-6#

B: Die Mentalität ist unterschiedlich. Vielleicht also was ich bemerkt habe, also wenn ein Deutscher oder so jemand fremd kennen lernt #00:12:34-5#

I: hm (bejahend) #00:12:34-5#

B: Dann nimmt für sich schon Zeit und macht Abstand zuerst #00:12:39-7#

I: hm (bejahend) #00:12:40-3#

B: Bis er dann den Mensch kennen gelernt hat, also wenn(...)bis er den Mensch auch verstanden hat oder Ihn schon gut(...)kennt, dann ist er ein Freund von ihm. #00:12:53-4#

I: Ja. #00:12:53-4#

B: Das ist das Unterschied zwischen und, also uns in Syrien und daher auch. #00:12:59-4#

I: Ja bei uns ist das ein schleichender Prozess. #00:13:01-6#

B: Zum Beispiel wir unterscheiden nicht das (...) das ist ein Kollege, also Kollegen in der Arbeit oder das ist ein Freund von mir, das ist (...) ne. DIe sind alle Freunde. #00:13:12-0#

I: Ah? #00:13:12-0#

B: Es gibt nicht da: ' Da ist mein Kollegen.' Dann haben wir zusammen was zu tun in der Arbeit, aber außer der Arbeit #00:13:21-7#

I: hm (bejahend) #00:13:21-7#

B: ..wir begrüßen uns überhaupt nicht. Eigentlich finden wir (...)  #00:13:24-3#

I: //Ja gut, das ist dann auch komisch #00:13:25-2#

B: also bei uns ist es einfach so: Die Freunde sind die Mitarbeiter oder die Kollegen die bei uns waren. #00:13:30-6#

I: hm (bejahend) #00:13:30-6#

B: Die Schul, also die die wir einfach zusammen in der Schule waren oder sowas. Das sind die Freunde. #00:13:38-5#

I: hm (bejahend) (...) Ich muss sagen, darüber hab ich mir noch keine Gedanken gemacht. #00:13:47-0#

B: Ja, also ich hab.. vielleicht merkt ihr das nicht #00:13:48-0#

I: hm (bejahend) #00:13:48-0#

B: Weil es ist einfach so, also ihr habt so erlebt, ihr seid so aufgewachsen, und deswegen es ist normal. #00:13:55-5#

I: ja. #00:13:55-5#

B: Aber wenn jemand fremd kommt und das sieht, dann sagt er vielleicht, also unterschiede (...) wir kommen einfahc wieder zu unterschiedliche Kultur, unterschiedliche (unv.).  #00:14:07-5#

I: Aber das find ich grad sehr interessant, dass ma nämlich (...) war mir nie wirklich bewusst. #00:14:13-5#

B: Ja, dass man einfach (...) Mein Arbeitskollege. Der ist wie ein Freund von mir. Also ihr habt so einfach (...) so Abstand. Ja es ist ein Kollegen in der Arbeit, ja der hat was mit mir zu tun in der Arbeit #00:14:27-8#

I: hm (bejahend) #00:14:27-8#

B: Draußen hab ich mit Ihm überhaupt nicht zu tun.(...) #00:14:32-4#

I: ähm #00:14:32-4#

B: Das macht ihr. #00:14:33-2#

I: Ja. #00:14:35-1#

B: Entschuldigung für das Wort 'Ihr' (lacht) #00:14:35-1#

I: Passt schon, passt schon. #00:14:36-2#

B: Und man kann einfach nicht verallgemeinern auch. #00:14:40-1#

I: Ja. #00:14:40-1#

B: Also mit meinen Worte, das.. (...) verallgemeinere überhaupt nicht, aber was ich gesehen habe. #00:14:48-1#

I: hm (bejahend) (...) Bist du dann mit (...) ähm, also du hast ja, was war's (...) Medizinisch -Tech.. Medizini.. Medizinisch -Technischer Assistent #00:14:58-3#

B: Ne, also ich bin Pharmazeutisch-Technischer Assistent. #00:15:00-2#

I: Ah okay #00:15:01-8#

B: Pharmazeutisch. Fast gleich. (lacht) #00:15:01-8#

I: Okay, ähm Also hast dann quasi mit deinen.. also mit den Leuten bist du dann auch ganz normal unterwegs. #00:15:11-3#

B: ähm(...) Nein. #00:15:13-3#

I: Ne? #00:15:15-1#

B: Ne. #00:15:16-4#

I: Okay. #00:15:18-2#

B: Nicht die alle. Also ich will das, aber die machen den Abstand. Und bis Sie dann uns kennen gelernt haben #00:15:23-3#

I: hm (bejahend) #00:15:23-3#

B: Und sie haben sich vielleicht mehr Mühe gegeben oder sowas bis sie dann mir akzeptiert. #00:15:30-4#

I: hm (bejahend) #00:15:31-0#

B: Dann bin ich einfach weg. #00:15:33-5#

I: Mmmh.. #00:15:33-5#

B: Da habe ich also zum BEispiel, also ja (...) Wie sagt man das? Die Spra.. Also die, die.. Sprache spielt auch eine Rolle #00:15:43-9#

I: hm (bejahend) #00:15:44-9#

B: Zum Beispiel, also die LEute die mit mir sind. Die Mitschüler, die mit mir sind #00:15:49-5#

I: hm (bejahend) #00:15:49-5#

B: Die reden nur auf bayrisch. #00:15:54-3#

I: Aber können sie's.. ähm können Sie auch Hochdeutsch? #00:15:56-5#

B: Die können schon. #00:15:58-3#

I: //Wollen nur nicht? #00:15:58-3#

B: Aber die reden nur bayrisch. Und wenn ich mit denen rede, dann.. Pff ja also wenn ich denen nicht verstehe und Sie mich nicht verstehen, da gibts keinen Kontakt dabei #00:16:10-7#

I: Ja. #00:16:10-7#

B: Also es gibt keinen Diskussion dazu. #00:16:16-2#

I: ähm ja. Glaubst.. Glaubst du, dass.. #00:16:16-9#

B: Und ich hab damals also jemanden von denen gefragt, ja am Anfang war wirklich schon schwer. Weil hab denen schon mal gefragt.. Die machen ja alle als Zusammenfassung von ein paar Fächer. #00:16:29-3#

I: hm (bejahend) #00:16:29-3#

B: Könnt ihr mal mir dann die Zusammenfassung schicken? Eine hat 'Ja' gesagt, 'das schicke ich dir heute Abend'. #00:16:37-4#

I: hm (bejahend) #00:16:37-9#

B: Abend, kommt garnix. Morgens, früh. Kommt garnix. Bin ich in die Schule. Kommt garnix. Da habe ich nicht nachgefragt. #00:16:50-2#

I: hm (bejahend) (...) Des is komisch. #00:16:53-0#

B: Ja. Vielleicht Sie hat das vergessen. #00:16:54-4#

I: Ja, hoffentlich #00:16:58-8#

B: Ja aber ich bin der Typ, der nicht nachfragt. #00:17:00-1#

I: hm (bejahend) #00:17:00-8#

B: Ich frage nicht nach. Weil ich habe einmal gefragt, und wenn man das will, dann macht das ür mich.  #00:17:08-8#

I: hm (bejahend) #00:17:08-8#

B: Und wenn das nicht will, dann  #00:17:11-8#

I: hm (bejahend) #00:17:11-8#

B: Frage ich nicht einfach wieder. Das mag ich überhaupt nicht, dass ich immer nachfrage (lacht). #00:17:17-5#

I: Kann ich verstehen.. ähm Aber glaubst du dass es grundsätzlich absichtlich passiert, also sowas ähm.. #00:17:30-2#

B: Am Anfang ja. #00:17:30-7#

I: Schon? #00:17:30-7#

B: Ja, am Anfang war schon. Mit Absicht. Ich weiß es nicht, vielleicht es war mien Gefühl ist falsch. Aber ich hab das Gefühl, dass einfach in Ihren Augen war: ' Es ist zu viel für einen Flüchtling, dass er dieses Beruf lernt.(...)' #00:17:51-1#

I: (...)Warum? #00:17:51-1#

B: Weil wir das nicht schaffen können.(...) (lacht) #00:17:58-2#

I: Ehh, okay #00:17:58-2#

B: So Ihren Glauben war. Oder die haben einfach kein Kontakt mit den Ausländer. #00:18:03-1#

I: hm (bejahend) #00:18:03-1#

B: Die haben mit den Ausländer garnix zu tun. Und deswegen sind nicht offensichtlich für den anderen. #00:18:10-9#

I: ähm, des hab ich auch schon öfter erlebt, also (...) ähm mit ähm mit Leuten, die grundsätzlich nie im Kontakt sind mit ähm mit Ausländern, oder auch nur mit Menschen von weiter weg aus Deutschland #00:18:28-1#

B: //Von weit weg, ja #00:18:28-1#

I: ähm(...) Die sind grundsätzlich am ähm voreingenommensten veranlagt. Also, die wollen dann auch keinen Kontakt. #00:18:46-0#

B: Genau. Die wollen einfach keinen Kontakt. Also die Leute die mit mir waren, die waren einfach (...) keine Ahnung. Also am Anfang waren wirklich für mich schon (...) dass die einfach dagegen sind. Wir waren vier Ausländer und 18 Deutsche. #00:19:00-3#

I: hm (bejahend) #00:19:00-9#

B: Und die meißtens waren auch Mädchen. Also ich hab nur ein (...) ein Junge gehabt. Also wir nur ein Junge in der Klasse und der zweite war ich. #00:19:08-5#

I: Also es waren zwei Jungs und #00:19:11-8#

B: Und 20 Frauen. #00:19:13-8#

I: Ui. #00:19:15-6#

B: Es war wirklich die Hammer. #00:19:17-5#

I: Was? #00:19:17-5#

B: Ja. Und deswegen(...) #00:19:20-1#

I: Glaubst du, dass es dann auch ähm (...) Geschlechterrollenspiele gibt oder (...) #00:19:28-2#

B: Nee, also die mit mir ist (unv.) also die spielt überhaupt nicht, keine, also bei mir spielt überhaupt keine Rolle #00:19:31-8#

I: Okay gut (...) Ne ich mein, dass irgend wie dann (...) keine Ahnung, Frauen dann mehr abblocken oder was weiß ich. #00:19:37-9#

B: Ne also die sind (...) also die sind mit Ihm dann im Kontakt mehr als ich (...) also mehr als bei mir. Es ist der Grund einfach, dass er immer Sie (unv.) versteht. #00:19:50-0#

I: hm (bejahend) #00:19:50-0#

B: Dass der einfach (...) denen versteht.  #00:19:52-2#

I: Ja. #00:19:52-2#

B: Aber ich hab Ab- und zu dann mit denen nicht gesprochen und da also ich hab am Anfang nur einfach gesehen: Die sprechen nur den Dialekt #00:20:01-4#

I: hm (bejahend) #00:20:01-4#

B: Und mit denen konnte ich keinen Anfangen, also keine Diskussion einführen. Und deswegen ist schon schwer, also die haben am Anfang versucht und sowas (...) Da war ich einfach in einer Situation, wo man garnix oder keine Lust hat #00:20:19-5#

I: hm? #00:20:19-5#

B: Und ich gehe nicht mit und deswegen die haben einfach danach gedacht, ja er will mit uns nicht mehr oder sowas. Deswegen die haben Abstand gemacht.  #00:20:28-4#

I: Das heißt so.. #00:20:28-4#

B: //Vielleicht war(...)war ich auch schuld, also (...) vielleicht war ich auch schuld #00:20:31-6#

I: //Kann man jetz nicht sagen #00:20:31-6#

B: Aber ich war in einer Situation wo ich einfach (...) nicht anders machen kann. #00:20:39-3#

I: hm (bejahend) #00:20:39-3#

B: Also ich bin wirklich also in.. so am Anfang der Ausbildung wo also es hat schon gut gelaufen, so Anfang der ersten zwei, drei Monaten #00:20:52-6#

I: hm (bejahend) #00:20:53-0#

B: Und ich hab mich nicht gewöhnt auf den (...) (unv.) System von der Schule. Also die deutsche System ist ganz anders als bei uns #00:21:01-9#

I: Das Schhulsystem? #00:21:02-7#

B: Das Schulsystem ist ganz anders. Es ist (...) #00:21:07-7#

I: Da würde ich jetz ganz gerne direkt mal nachfragen: ähm so wie ist das Schulsystem in Syrien, und wie ist es dann im Vergleich hier? #00:21:15-0#

B: Also unser System ist französisch. #00:21:19-0#

I: Ist französisch? #00:21:19-0#

B: Ist französisch. #00:21:19-0#

I: Ah! #00:21:19-0#

B: Ja. Was wir in Syrien so, pff, ja, geamcht haben oder sowas war einfach der Lehrer ist verantwortlich, dass wir 75\% von dem Unterricht lernen #00:21:34-4#

I: hm (bejahend) #00:21:34-4#

B: oder verstehen #00:21:33-5#

I: hm (bejahend) #00:21:33-5#

B: und dass wir einfach zu Hause die 25\% damit wir einfach komplett den Unterricht beherrschen können. #00:21:43-2#

I: hm (bejahend) #00:21:43-2#

B: Wir machen einfach den 25\% zu Hause. So, unser System sage ich einfach, also, es war ein bisschen dumm. Und nicht dumm. Daher zum Beispiel es gibt so immer recherchieren und nachfragen und immer so (...) ähm (...) wie sagt man (...) oder es ist, kommt auch darauf an auch welchen Beruf hast du denn? Zum Beispiel in Syrien als ich mein Abitur gemacht hab, also wir (...) wir (...) wir werden von 12 Fächer geprüft #00:22:19-2#

I: hm (bejahend) #00:22:19-5#

B: Und wir haben schon viele schwere (unv.) Fächer gemacht. Zum Beispiel als ich mein Abitur abgem (...) also (...) abgeschlossen hab, ich bin ja in Mathe und Physik und Chemie und Biologie, Sozialkunde, Arabisch, Französisch, Englisch und Religion. #00:22:37-0#

I: hm (bejahend) #00:22:37-0#

B: Ich bin in diese neun Fächer geprüft, um dann den Abschluss zu (...) also um dann mein Abitur zu bekommen #00:22:45-0#

I: hm (bejahend) #00:22:45-5#

B: Hier? Nee. Also es gibt hier nur 3 Fächer oder 4 Fächer, man wird dann für 4 Fächer dann sein Abitur machen #00:22:52-2#

I: ähm Abitur glaub ich warn (...) also ich hab's 2013 gemacht #00:22:57-0#

B: hm (bejahend) #00:22:57-0#

I: ähm des war Deutsch, Englisch, Mathematik und zwei Wahlfächer - also bei mir glaub ich waren #00:23:03-9#

B: //fünf Fächer. #00:23:03-9#

I: Genau ähm Ich glaub ich hatte Geographie und #00:23:07-3#

B: //Sprachen oder sowas oder?#00:23:08-0#

I: Ja #00:23:10-7#

B: Ne des war nicht auch die, die also was ich meine. Aber das System bei uns war einfach, dass wir (...) wir kommen einfach in die Schule #00:23:20-3#

I: hm (bejahend) #00:23:20-3#

B: Der Lehrer ist wirklich verantwortlich für uns #00:23:24-8#

I: hm (bejahend) #00:23:25-3#

B: Dass wir einfach die 75\% von der (...) von dem (...) von dem Stoff her dann verstehen, dann wieder nach Hause kommen, dann lernen wir den Rest. #00:23:35-3#

I: Dast du dann das Gefühl, dass.. #00:23:37-6#

B: Hier ist es(...) Entschuldigung. (...) Hier ist es einfach (...) keine Ahnung, also der Lehrer (...) Ihm ist das egal, ob wir das verstanden haben oder nicht, weil ich hab mit so, mit der unser Lehrerin auch gesprochen. Sie redet bayrisch. #00:23:55-9#

I: hm (bejahend) #00:23:55-9#

B: Ich hab mit Ihr schon gesprochen 'könnten Sie vielleicht das auf Hochdeutsch dann umstellen vielleicht oder so?' - 'Ja, wenn du das Gefühl hast, dass du einfach das nicht verstehst, mach einfach die Hände auf, also, Hände hoch, und dann fragst du mich nach.'
 #00:24:15-5#

I: //Mmmmmmmmh #00:24:15-5#

B:Okay, wie lange dauert das? Soll ich jede Minute dann die (...) die (...) mein Hand hoch gehen und dann 'Entschuldigung, ich hab noch eine Frage!' #00:24:26-1#

I: Ja..
 #00:24:26-1#

B: Da nerve ich einfach die alle Mitschüler! #00:24:26-5#

I: ja. #00:24:26-5#

B: Das habe ich gemacht einmal. #00:24:26-9#

I: Ja? #00:24:26-9#

B: Also in einem Unterricht. Ich bin einfach, also ich hab Sie ungefähr zwei, dreimal untergebr - also, unterbrechen müssen. Aber die alle nerven sich.. #00:24:40-0#

I: Ja #00:24:40-1#

B: .. dass ich das nicht verstanden habe oder sowas #00:24:43-4#

I: aber, das (...) das verstehe ich ned wirklich. Also des war ne Klasse mit 18 deutschen Schülern und vier syrischen, oder? #00:24:52-4#

B: Ja. Nicht syrischen, also sind unterschiedliche, ja. #00:24:54-3#

I: //also(...)also, okay. Aber (...) alle davon sprechen deutsch und die 18 davon sprechen bayrisch. (...) Das ist doch (...) ja. Das ist doch dann eigentlich (...) logisch, im Unterricht dann nur deutsch zu sprechen.  #00:25:16-1#

B: Ja. Die, die (...) Lehrer hat gemeint 'ja, also ich bin seit 20 Jahren hier als Lehrerin. #00:25:23-4#

I: hm (bejahend) #00:25:23-4#

B: Und so, mit (...) also mit (...) euch, euch also die meint mit euch hald dann die Ausländer, habe ich noch nicht erlebt, dass ich in einen, einem Unterricht Ausländer unterrichte. #00:25:37-6#

I: Das war Ihre erste #00:25:38-1#

B: // Sie hat sich gewöhnt auf Ihr Dialekt zu sprechen. #00:25:40-7#

I: Das war quasi Ihre erste Klasse, in der Sie auch Ausländer unterrichtet hat? #00:25:46-1#

B: Ja, obwohl das war vor uns auch war eine, eine, eine Irakerin. #00:25:52-6#

I: hm (bejahend) #00:25:53-1#

B: Sie hat auch die Ausbildung abgeschlossen. #00:25:55-2#

I: hm (bejahend) #00:25:55-2#

B: Ja, Sie war auch und sie hat (...) also sie hat (...) ihr wirklich tut sehr leid und (...) sie hat schon, wirklich viel gegeben. Also viel mühe gegeben #00:26:06-6#

I: hm (bejahend) #00:26:06-6#

B: Um Sie dann zu verstehen, aber Sie ist ein Mädchen. Wenn Sie nachfragt, dann antworten die alle(lacht). #00:26:16-6#

I: (lacht) #00:26:16-6#

B: Es ist einfach so. Und sie, sie kann einfach den Kontakt mit den Mädchen dann (...) #00:26:23-4#

I: Ja. #00:26:23-4#

B: Und Ihr (...) Es gab jemand der Sie auch geholfen hat und sowas. Ja bei mir es ist (...) es war einfach.. Ja, ich hab einfach Schuld bekommen. Als ich dann zuerst nachgefragt, ob die, ob Sie dann mir die Zusammenfassung mir schicken oder sowas, da hat niemand geschickt #00:26:42-7#

I: Ja. #00:26:42-7#

B: Da habe ich gesagt 'Ne'. #00:26:43-9#

I:  Ja, okay. Vorbei? #00:26:46-6#

B: Brauche ich euch nicht mehr #00:26:46-8#

I: Ja? (...) Schade. #00:26:52-2#

B: Ne ich habe Lust überhaupt nicht. #00:26:55-0#

I: hm? #00:26:55-0#

B: Also schade ist überhaupt nicht. Man lernt auch davon. Also davon. #00:26:58-5#

I: Ja, aber.. (...) das #00:27:00-3#

B: //Man lernt einfach davon #00:27:00-4#

I: Das is eine Lektion, die (...) ähm (...) die wär' auch in schöner gegangen.
 #00:27:08-7#

B: (...) Ja. (...) #00:27:16-9#

I: Nochn Kaffee? #00:27:16-9#

B: Nee, danke. #00:27:20-5# \newpage
%\section{Interview 1, Teil 2}

B: Ich hab noch kurz, also, so halbe Stunde. #00:00:07-4#

I: Du des (...) halbe Stunde is gut #00:00:08-3#

B: //Ja wir können (unv.) weiter machen auch, also wir können dann ja an andrem Tag machen. Jetz hab ich Ferien auch. #00:00:10-6#

I: Gerne. #00:00:10-6#

B: Ja. Ja, dann machen wir es einfach. #00:00:14-4#

I: Okay #00:00:14-4#

B: Ja. #00:00:14-4#

I: ähm ja mach ma einfach mal die (...) Asylstatus: Du bist akzeptiert? Du bist anerkannt oder? #00:00:12-9#

B: Ja ich bin schon anerkannt, ja. #00:00:14-8#

I: Ähm, also eine der anderen Dinger is (...) wäre dir damals bewusst gewesen, dass wenn dein Status abgelehnt worden wäre, also (...) wenn du abgelehnt worden wärst, dass du da dann Revision einlegen kannst direkt. Hättest du des damals schon gewusst? #00:00:27-6#

B: Mmmmmh, ja. #00:00:29-3#

I: Okay #00:00:29-3#

B: Ja, hätte ich schon gewusst. #00:00:37-2#

I: Gut. ähm, wie hat das eigentlich funktioniert damals. ALso, du bist aus Syrien nach Europa gekommen.. #00:00:36-7#

B: Genau. Also ich bin einfach, also kurz bevor ich dann das Land verlassen habe, habe ich einfach (...) da war jetzt so kleine Situation, und zwar..#00:00:46-5#

I: (Verschüttet Kaffee) Ow, fuck. #00:00:46-5#

B: Was? Zu heiß? (lacht) #00:01:19-9#

I: Nee, ich (...) (unv.) (...) shit. #00:01:23-3#

B: Und des war einfach (...) also ich war damals auf die Straße einfach im Ramadan (...) kennst du den? Monat Ramadan.. #00:01:29-5#

I: //Kenn ich. #00:01:29-5#

B: Genau. Ich war damals mit mein Cousin einfach auf die Straße. Es war so um drei Uhr.. Nee, um zwölf Uhr war in der Nacht. #00:01:38-1#

I: hm (bejahend) #00:01:38-1#

B: Waren wir einfach in Aleppo unterwegs. Und wir haben nur einfach Kaffee getrunken weil wir einfach bis um drei Uhr auf die Straße bleiben damit wir wieder zurück nach Hause kommen. #00:01:50-4#

I: hm (bejahend) #00:01:50-4#

B: Es ist Ramadan, also man hat gegessen und dann sollte ein einfach, also (...) #00:01:54-9#

I: //hm (bejahend) #00:01:54-8#

B: unterwegs sein oder sowas. Ein bisschen spazieren gehen. Und dann hat mich die Polizei gehaftet *räuspert sich*. Und der Grund war, weil wir einfach auf die Straße waren. #00:02:07-1#

I: hm (bejahend) #00:02:07-1#

B: Wir dürfen nicht auf die Straße bleiben. #00:02:11-0#

I: Warum? #00:02:11-0#

B: Wir sind einfach 3 Tagen in GEfägnis gegangen, weil und ohne Grund dafür.  Und dann kommt den dritten Tag, dann kommt einfach zum abfragen, also der Direktor. Ne, nicht der Direktor, wie sagt man das? Den (...) den Chef von die Polizei? #00:02:33-9#

I: Polizeipräsident? #00:02:34-8# #00:02:35-9#

B: Ne, nicht der Präsident.. #00:02:36-2#

I:// ähm (...)ähm #00:02:36-4#

B: Ne, nicht den Präsident, sondern einfach in der Stadtion dann. #00:02:39-8#

I:// Der (...) Der Stadtionschef? #00:02:40-3#

B: Der uns Fragt einfach. Ja, die Stadtionschef. er hat uns gefragt (...) was haben wir dotr gemacht, und wie sind wir auf diese Straße gegangen? Und wir waren nicht (...) nicht ich und mein Cousin. Wir sind 30 Personen zusammen gehabt. #00:02:56-9#

I: hm (bejahend) #00:02:56-9#

B: Und wir haben nix zu tun. Also wir haben garnix gemacht. Wir sitzen einfach so (...) so ein Kreis, also großes Kreis, wo einfach sich die, die ganze Jungs dann sich treffen und sowas. Und das (...) niemand hat da was schlimmes getan oder sowas. #00:03:12-8#

I: hm (bejahend) #00:03:12-8#

B: Ja, fragt er uns, und was habt ihr dort gemacht und sowas? Dann kommen wir einfach zu Ihm und dann sagt er zu uns: 'Unterschreibt Ihr hier dann.' (...) Okay, dürfen wir das lesen? - Nein. #00:03:28-2#

I: Nur unterschreiben und nicht lesen? #00:03:30-0#

B: Nur unterschreiben. Ja (...) Wieso (...) dürfen wir oder nicht. Ne. Aber, ist das geschrieben was wir gesagt haben, oder ist das von vorher schon geschrieben? 'Wollt Ihr raus oder nicht?' So war die, die die Antwort. 'Wollt Ihr raus oder nicht?' Ja. 'Dann unterschreibt Ihr.' Kommen wir raus (...) nach zwei Tagen habe ich mich mit einem Richter, also mit einem Richter schon getroffen #00:04:01-6#

I: hm (bejahend) #00:04:01-6#

B: Und der war schon der (...) so, wie sagt man? Der Chef im Gericht. #00:04:06-0#

I: hm (bejahend) #00:04:04-7#

B: Da habe ich mit Ihm schon getroffen, der war so unser Nachbar. Dann habe ich, dann hat er mir gesagt: 'Was habt IHr gemacht? (...)' Was steht in unser Bericht? 'Ja, willst du das lesen?' Ja jese ich gern! 'Okay: Sexuelle Belastung.(...)ähm Störung der Mitbewohner (...)ähm Was ahben wir noch gemacht? Auf der Straße einfach Flaschen geworfen und Glas abgebrochen und ähm was war das? Es ist einfach (...) es ist einfach viel! Wir haben ja das garnix gemacht. Aber wir haben unterschrieben. #00:04:54-8#

I: (...) #00:04:54-8#

B: Das war der Grund (...) Ja und ich war auch an der Uni. #00:04:57-4#

I: hm (bejahend) #00:04:59-7#

B: Ich hab Chemie, also ich war in Chemie eingeschrieben. #00:05:01-6#

I: hm (bejahend) #00:05:02-3#

B: Und ich hab einfach das (...) Also ich stehe noch immer in Gefahr. Weil die haben schon uns gesagt: 'Wenn wir euch wieder sehen, und auf diese Straße, oder diesem Gebiet. Dann werdet Ihr ein Monat oder mehr auch verhaftet.' #00:05:20-4#

I: Warum wollten die euch von der Straße herunten haben? #00:05:23-6#

B: Einfach die wollen einfach keinen auf der Straße. Und die sind einfach neidisch und nicht neidisch, also es stört Ihnen, dass wir einfach nicht in die Armee gegangen, sondern in die, also an der Uni sind. #00:05:37-6#

I: Hmh? #00:05:37-6#

B: Weil, sie brauchen schon Leute um zu kämpfen. #00:05:41-8#

I: hm (bejahend) #00:05:41-8#

B: Und Sie brauchen kein Mensch dann zu Studieren. #00:05:44-8#

I: hm (bejahend) #00:05:44-8#

B: Jetzt sind die Universitäten, die in Syrien sind. Die sind fast leer von junge leute. #00:05:52-8#

I: hm (bejahend) #00:05:52-8#

B: Die sind fast leer. (...) Den, den Professor macht die, die Fragen, die Prüfungen schwerer. Damit wir einfach das nicht bestehen. #00:06:02-8#

I: Selbst die Professoren machen da mit? #00:06:05-4#

B: // Ja. Die sind auch damit. ALso, die haben immer noch Hand damit. Der, der, der Chef von der, der UNiversität, also es waren viele Demos, auch an der Uni. #00:06:15-2#

I: hm (bejahend) #00:06:15-6#

B: Und normalerweise in unserem Gesetzt steht ähm die Polizei darf, also darf kein (...) keine Waffe oder keine schwer Waffe an der Uni, also (...) an die UNi reinkommen. #00:06:29-9#

I: Ja, is gut so. #00:06:29-9#

B: Es darf keine. Aber: Als die Demo war. Es war die Hammer. Es war Panzer auch auf der, auf der (...) an der #00:06:39-3#

I: // Panzer?! #00:06:39-3#

B: Ja. Die Panzer waren auch einfach auf auf die Straßen. DIe sind unterwegs, immer. #00:06:45-2#

I: ähm wofür war die Demo? #00:06:49-9#

B: Dagegen. Also Revolution einfach. Dagegen, also gegen des, die Regierung und gegen das hald. #00:06:55-7#

I: ähm gegen Assad oder(...)? #00:06:55-2#

B: Und sind viele Menschen.. Ja. #00:06:57-4#

I: hm (bejahend) #00:06:58-9#

B: Also seine Regime. #00:06:59-9#

I: Ja. #00:07:01-5#

B: Und viele Menschen sind, also wurden verhaftet und niemand wusste dann (...) Wo sind die oder der Chef von der Universität also es gibt schon von jeden von jede Fakultät oder sowas da gibts nur ein Chef #00:07:14-4#

I: hm (bejahend) #00:07:14-4#

B: Zu Beispiel also vin Phatmazie und Medizin da gibts nur einen CHef von denen. #00:07:20-6#

I: hm (bejahend) #00:07:20-6#

B: Und der war selber, also der hat mit der Polizei gearbeitet. Also er war kein, kein, kein Professor oder kein Chef von der Uni sondern der war einfach ein (...) ein Scheiß (lacht) von die Polizeit. #00:07:34-7#

I: Hä? Der is da einfach installiert worden? #00:07:37-5#

B: Der sitzt da einfach, der sitzt dort #00:07:39-0#

I: Ja? #00:07:39-0#

B: Und wenn was von (...) wennw as passiert an der Uni oder sowas, dann kennt er einfach die Leute und schreibt er einfach die Polizei und dann ruft er einfach an, und die Name und die Name und die Name und die Name, die sprechen über den (...) über die Regierung. #00:07:53-7#

I: hm (bejahend) #00:07:53-7#

B: Nächsten Tag, siehst du die alle, die Schüler ähm die alle Studenten nicht. Wo sind die? Verschwunden. (...) Gehen die Eltern dann hin, fragen die nach: 'Die sind nicht bei uns.' Nach einem Jahr (...) die kriegen einfach die Körper von ihren, von Ihren Söhne oder von Ihren Tochter oder sowas. Die sind gestorben. Wie? Die haben nicht angehalten. Ja, so war es. So war es. Und wenn man, und wenn man dann an diese Situation so guckt, dann sagt man: Also lieber, dann ich einfach weg ham. #00:08:48-1#

I: hm (bejahend) #00:08:48-1#

B: Weil, wenn ich dort bleibe, dann muss ich die Waffe tragen. (...) #00:08:53-0#

I: hm (bejahend) #00:08:54-0#

B: Un mich zu schützen.  #00:08:56-0#

I: Ja. #00:08:56-0#

B: Und um meine Familie auch zu schützen. #00:08:58-4#

I: Ja. #00:08:58-4#

B: Weil solange ich dort bleibe, da habe ich keine Chance. Da muss ich einfach mich einmischen in diese Situation, da muss ich einfach den (...) den Dreck also mitmachen. Sonst lebe ich nicht. Also sonst bin ich immer unter dem Druck. Da also es ist (...) das war so, also wirklich (...) eklig. (...) Anfang 2013, Anfang 2014, da war wirklich so schwer, dass man auf die Straße unterwegs ist. Da ist meine Mutter also zum Beispiel, also wenn ich zur Schule gehe oder zur Uni gehe, da hat meine Mutter gebetet, dass ich wieder zurück komme. (...) Dass ich wieder zurück, dass ich wieder gesund wieder zurück komme. (...) #00:09:50-9#

I: Bei nem normalen Tag an der Uni? #00:09:54-4#

B: Ja. Weil man weiß überhaupt nicht was da an der Uni passiert. Obwohl dass du mitmachst oder nicht mitmachst, da bist du auch mitschuld, also da bist du einfach schuld. #00:10:05-0#

I: ähm #00:10:06-6#

B: Wenn deine, zum Beispiel wenn deine, deine Klasse oder deine (...) einfach (...) deine Studenten, die mit dir dann sind. Wenn es zum beispiel an der Chemie-Fakultä (...) Fakultät, gibt es heute einen Demo. Sind die alle Studenten #00:10:23-0#

I: Heute? #00:10:23-0#

B: Ja, also  #00:10:23-7#

I: //Ahso, ähm (...) metaphorisch#00:10:24-3#

B: Nur als Beispiel. Sind, ist heute eine Demo. #00:10:28-3#

I: hm (bejahend) #00:10:28-3#

B: Okay, dann sind die alle Studenten von Chemie schuld. Verhaften wir die alle. #00:10:36-2#

I: (...) #00:10:36-2#

B: Am nächsten Tag, wer kommt raus? Wer einfach Vitamin B bei die Regierung hat. Und die alle, die, die anderen zum Beispiel, die blieben so eine, eine Woche und dann den Schlag und den Elektro (...) ähm (...) Sachen. Es gibt viele auch, die einfach raus gekommen sind, und die haben, also die sind nicht mehr bewusst. also die haben einfach Ihren Konzentration verloren wegen den Strom #00:11:08-3#

I: Hmh?! #00:11:08-3#

B: Ja da sitzt eifach einen (...) auf einen Stuhl (...). Da wird den Stuhl dann mit dem Strom verbunden  #00:11:16-4#

I: Ein elektrischer Stuhl? #00:11:16-4#

B: Und du bist da.. Ja, elektrischen Stuhl. Und dann bist du so ein (...) so auch (...) mit dem (...) Gürtel oder sowas bist du so verbunden #00:11:28-2#

I: hm (bejahend) #00:11:28-2#

B: Und dann kannst du dich nicht bewegen und den Stuhl nicht stören. (...) #00:11:33-4#

I: (...) #00:11:33-4#

B: Es gibt viele die da.. also mit diese Situation (unv.) (...) die sind einfach am Stuhl gestorben. Die haben das nicht angehalten. #00:11:39-1#

I: Das sind dann die, die zu ihren Eltern wieder im Sack zurückkommen? #00:11:42-9#

B: Genau. #00:11:42-9#

I: (...) #00:11:46-4#

B: Und mit diese Sit.. Also wirklich, also mit diese Situation kann man nicht leben. #00:11:49-9#

I: Nein. #00:11:49-9#

B: Kann man nicht leben. Die viele haben mich gefragt: 'Ihr habt eure Eltern, und Ihr habt Ihre (...)ähm Frauen und Ihre sowas verlassen. #00:12:00-9#

I: Mhm #00:12:03-3#

B: Ja, also wenn das die Möglichkeit hätte, dann hätte ich auch meine Eltern mitgebracht. (...) Ja. Aber es gibt viele, die das nicht vertehen wollen, oder die das nicht verstanden haben und jetzt verstehen die einfach #00:12:20-3#

I: hm (bejahend) #00:12:20-3#

B: Warum sind wir Jungs nach (...) nach Europa gekommen.  #00:12:24-3#

I: hm (bejahend) #00:12:24-3#

B: Es gibt viele Leute jetzt, die gehen einfach wieder zurück.Weil die einfach (...) die haben gesehen also es gibt viele Leute die einfach Beruf haben seit Jahren. Und die werden hier nicht anerkannt. Weil Sie einfach keine Papiere haben. Weil Sie Ihre Papiere verloren haben. #00:12:41-7#

I:  ähm wie ist denn momentan die Situation in Syrien, hat sich die irgendwie verbessert? #00:12:48-4#

B: Ne. Also mit (...) mit uns, wenn ich jetzt zurückkehre.Da ist die Hammer. Da sehe ich meine Eltern überhaupt nicht. Da bin ich einfach auf die Grenze, also vom Flughafen wurde ich verhaftet, weil ich Verräter bin. Weil ich nicht zur Armee gegangen bin. (...)  #00:13:08-3#

I: (...)ähm(...) #00:13:13-6#

B: Es ist schon schwer zu verstehen, oder? #00:13:13-2#

I: (...)ähm(...) Ich sitz hier auf meiner Insel der Glückseligkeit und kann mir das überhaupt nicht vorstellen #00:13:22-3#

B: Ja #00:13:22-3#

I: was da überhaupt passiert.(...) #00:13:24-8#

B: Es ist (...) es ist viel passiert. Es war also (...) jeden Freitag war eine Demo #00:13:31-5#

I: hm (bejahend) #00:13:31-5#

B: Also wir kommen raus von der Moschee zu Beispiel #00:13:32-5#

I: hm (bejahend) #00:13:33-1#

B: Und, ja es gibt eine Demo. Damals, also vor meinen Augen. Da waren schon so schlechte Leute, die einfach immer betrunken waren und die immer nur ähm(...) mit den (...) mit den Drogen zu tun haben. Und die immer Probleme mit der Polizei haben. #00:13:52-1#

I: hm (bejahend) #00:13:52-6#

B: Die Polizei hat denen dann genommen #00:13:55-4#

I: hm (bejahend) #00:13:55-4#

B: und die sind einfach ähm Mitglied mit der Polizei geworden. #00:14:02-6#

I: Die haben bei der Polizei dann mitgemacht? #00:14:02-6#

B: Genau. Warum? Um die Leute, die die Demo machen da einfach zu (...) wie sagt man? #00:14:10-5#

I: ähm infiltriert? #00:14:10-1#

B: Ja. (...) Und es gab einfach wirklich also eine Situation die vor meinen Augen passiert. Die haben einer verhaftet. Die waren fünf leute. #00:14:23-0#

I: hm (bejahend) #00:14:23-0#

B: Und die haben nur (...) also die haben keine Waffen, sondern Messer. Und die fünf Messer kommt (...) in seinem, also seinem Körper. War vor der Moschee, die haben Ihm verhaftet und, und keiner kann dann dran kommen, weil (...) also (...) also rpber von den fünf Leuten waren einfach Leute, die mit, die mit Waffe waren. Man kann einfach nicht rein kommen. #00:14:50-2#

I: Ja. #00:14:51-1#

B: UNd die haben Ihn getötet. Ja. #00:14:55-7#

I: Die Polizei?! #00:14:55-7#

B: Nicht die Polizei, sondern die.. #00:14:57-7#

I: //Die, die für die Polizei.. Ohh.. #00:14:59-4#

B: Die, die für die Polizei gearbeitet haben. Und die Polizei steht da. Die sieht schon das. Aber die machen die Auge zu. Wir haben nicht gemacht! Also die Polizei sagt, wir haben das nicht ghemacht. aber wer das gemacht hat, die sind auch Mitglieder von euch, oder? Sie bekommen schun Gehalt von der Polizei. Ja also es ist vieles (...) viel zu viel. Viel zu viel zu erzählen. #00:15:27-6#

I: Ja. #00:15:28-1#

B: Das reicht keine Stunde.(lacht) #00:15:30-5#

I: Nein(lacht) #00:15:30-5#

B: Also es reicht kein Tag auch. #00:15:35-1#

I: Fuck.(lacht) #00:15:35-1#

B: Ja. Fuck. (...) #00:15:40-8#

I: ähm wie (...) Wie bist du dann nach Europa gekommen? #00:15:48-3#

B: Also ich bin einfach über die Grenze von Aleppo #00:15:50-5#

I: hm (bejahend) #00:15:51-7#

B: Da war so eine (...) ein Dorf dort, an der Grenze, genau von Aleppo, von der Türkei. Da bin ich einfach über die Grenze. Also, illegal auch. Über die Grenze reingekommen. #00:16:02-2#

I: hm (bejahend) #00:16:02-5#

B: Und dann von der Türkei hab ich so zwei oder drei Tagen dort gelebt, und dann habe ich so ein, also ein Schleuser  getroffen. #00:16:14-4#

I: hm (bejahend) #00:16:14-4#

B: Der für Ihn dann Geld bezahlt hab, und dann bin ich einfach nach Griechenland mit dem (...) #00:16:21-5#

I: Mit dem Boot. #00:16:21-5#

B: Mit dem Boot, ja. WIr waren so, fast vierzig (...) vierzig (...) ähm (...) 46 Personen waren wir auf dem Boot. Und dem Boot ist sechs Meter lang. Ja. #00:16:34-7#

I: 46 Personen auf dem sechs Meter Boot? #00:16:35-3#

B: Ja, auf dem sechs Meter Boot. #00:16:36-6#

I: ähm war des dann ein Holzboot oder..? #00:16:38-9#

B: Es war ja Holzboot. Ne, nicht Holz. War schon elektronisch. #00:16:42-5#

I: //ähm #00:16:42-5#

B: Ja. #00:16:45-3#

I: Okay. #00:16:45-3#

B: Es ist.. Also, wir haben (...) #00:16:47-3#

I: //Hört sich trotzdem nicht vertrauenerweckend an. #00:16:47-6#

B: Kannst du nicht vertrauen, aber da hast du keine andere Möglichkeit. #00:16:52-1#

I: hm (bejahend) #00:16:52-7#

B: (...) Und deswegen, also weil es so schwer ist(...) da haben wir, vielleicht, wir sind ja Jungs, vielleicht können wir schwimmen oder sowas machen oder sowas. Aber da waren Kinder und Frauen auch, da war und sowas (...). Da hast, das hat ungefähr zweieinhalb Jahr, also zweieinhalb Stunden gedauert, bis wir dann die Grenze von Griechenland erreicht haben. (...) Ja. #00:17:20-9#

I: (...) Shiit. #00:17:24-4#

B: Ja. Also wenn ich jetzt mich erinner, an die, also an den Weg. Da ist die Hammer. (lacht) #00:17:30-9#

I: (...) #00:17:30-9#

B: Ja. Also was wir auf dem Weg erlebt, wir waren doch VIP auch.  #00:17:40-4#


I: hm (bejahend) #00:17:40-9#

B: Da ist wo die Grenze geöffnet wurde, also (unv.) wo auch die Leute, also die Polizei von den Ungarn #00:17:50-0#

I: hm (bejahend) #00:17:50-0#

B: Die hat einfach die, die die Grenze geöffnet #00:17:53-2#

I: hm (bejahend) #00:17:53-7#

B: War früher noch (...) Leute die einfach in einem Wald gestorben oder (...) ja. Weil Sie einfach Hunger gehabt haben, doer weil Sie einfach von jemand gestohlt wurden, und die haben kein Geld mehr. Und die können einfach nicht in (...) in die Ungarn dann rein kommen. #00:18:13-8#

I: hm (bejahend) #00:18:13-8#

B: Weil wenn die Polizei denen sieht, dann (...) dann müssen Sie dort in Ungarn bleiben und sowas. Ja war wirklich, also wirklich sehr schwer, wo die Leute so fast 10km oder mehr auch am Tag geh (...) zu Fuß gehen müssen #00:18:30-1#

I: hm (bejahend) Das is viel. #00:18:33-3#

B: Ja. Also bei uns war wirklich VIP. Wir kommen an in einem Land, nimmt uns die Polizei von diesem Land (...) bis zur Grenze von dem nächsten Land. Und dann (...) am nächsten #00:18:44-2#

I: hm (bejahend) #00:18:44-2#

B: Also wir (...) es kann sein, dass wir in einem Land für ein Tag bleiben oder eine Nacht bleiben, und dann am nächsten Tag lang wieder, anderem Land (...) bis wir dann nach Deutschland erreichen. Also wir (...) also wir dann nach Deutschland angekommen sind.  #00:19:01-3#

I: ähm aber deine (...) deine Flucht war dann Mitte zwanzigfuchtzehn oder? #00:19:08-4#

B: Mitte Zwanzig..? #00:19:10-2#

I: ähm 2015. #00:19:13-7#

B: Ja, das war im Oktober.  #00:19:13-9#

I: Oktober 2015. #00:19:13-9#

B: Also Ende September, Anfang Oktober bin ich angekommen. (...) #00:19:19-5#

I: (...) Seitdem hat sich einiges verändert. #00:19:25-2#

B: Ja. #00:19:26-3#

I: (...) #00:19:34-8#

B: Können wir vielleicht unterbrechen? #00:19:36-4#

I: Ja. #00:19:36-4#\newpage
%\section{Interview 2, Teil 1}

I: ähm das is jetz erstmal (...) erstmal sind so die Standardfragen #00:00:09-2#

B: hm (bejahend) #00:00:09-2#

I: Also, zum Beispiel: Wie alt bist du? #00:00:12-4#

B: 19. #00:00:14-7#

I: Du bist 19. #00:00:15-1#

B: hm (bejahend) #00:00:15-1#

I: ähm dann (...) Okay dass du männlich bist glaub ich ist relativ offensichtlich (lacht) #00:00:20-6#

B: (lacht) Ja.#00:00:21-7#

I: ähm du kommst aus Afghanistan #00:00:25-3#

B: ähm ich komme aus Iran aber ursprünglich komme ich aus Afghanistan #00:00:30-6#

I: Genau, also (...) ursprünglich kommstu du aus Afghanistan aber du bist aus dem Iran dann hierher geflohen oder? #00:00:38-0#

B: ähm nicht so genau. #00:00:40-2#

I: WIe genau istn des gelaufen? #00:00:40-2#

B: Genau (...) ich bin in Afghanistan geboren. #00:00:44-3#

I: hm (bejahend) #00:00:44-3#

B: Als ich fünf Jahre alt war dann meine Mutter mit meinen Geschwistern nach Iran gefl(...) also, nach Pakistan geflüchtet. Dann war, gab's dort keine Sicherheit, dann die mussten nach Iran kommen. #00:00:57-3#

I: hm (bejahend) #00:00:57-7#

B: Dann wir sind nach Iran geflüchtet. Dann ich bin dort aufgewachsen. Also ich habe nicht meine Muttersprache gelernt sondern Persisch #00:01:07-1#

I: hm (bejahend) #00:01:07-5#

B: Also Persisch und Dari sind nicht gleich, so ein bisschen unterschiedlich. #00:01:15-7#

I: So wie Deutsch und Bayrisch dann oder.. #00:01:17-3#

B: //Genau. Ja, so #00:01:17-7#

I: Ah okay. #00:01:19-1#

B: Hmm, (...) dann Beispiel ich hatte auch Problem. Ich kann nicht erinnern, aber meine Familie, mein Freund hat zu mir gesagt: 'Als du klein warst, du bist in Geschäfte gegangen und dann du wolltest Kaugummi kaufen. Auf unsere, auf Dari, Kaugummi heißt [Sayic], auf Persisch heißt [Adansi]. Dann ich hab [Asayir] genannt, die haben nicht verstanden.  #00:01:45-0#

I: hm (bejahend) #00:01:45-0#

B: Manchmal schauen verrascht, so.. (...) #00:01:47-5#

I: hm (bejahend) #00:01:47-5#

B: Und deswegen wir hatten Problem, bis ich die Sprache gelernt habe. (...) #00:01:56-9#

I: Aber deine Familie konnte schon persisch oder? #00:02:01-3#

B: Ne, garnicht. #00:02:01-3#

I: Also Ihr habt alle dann im Iran quasi Persisch gelernt? #00:02:07-0#

B: Außer mein Mama. #00:02:07-0#

I: (Entpackt Erdbeeren) #00:02:10-9#

B: ähm #00:02:10-9#

I: Manche müss ma wegschmeißen. #00:02:12-4#

B: Ja. (...)(...)(...) Dann  #00:02:24-7#

I: Wasch ma's noch eben ab? #00:02:26-4#

B: Ich wasche es, warte #00:02:26-9#
 //Pause\newpage
%\sectio{Interview 2, Teil 2}

I: Jetz funktioniert's.  #00:00:06-5#

B: Ja, und (...) Was war dein Frage? #00:00:07-9#

I: ähm (...)  wir waren dabei, wie du dann nach Deutschland gekommen bist? #00:00:13-7#

B: Aha (...) dann, ich bin, ich war zum 14 - jähriger war ich im Iran. #00:00:21-5#

I: hm (bejahend) #00:00:23-2#

B: Dann die Iranische Polizei haben mich verhaftet #00:00:25-8#

I: hm (bejahend) #00:00:27-4#

B: Dann (...) mich in's Gefängnis gebracht. Und ich war sieben, acht Tage - ich kann nicht erinnern, ich habe vergessen, schon lange her. Dann, die haben mich nach Afghanistan geschickt, ich war auch verletzt.  #00:00:26-5#

I: Was für eine Verletzung hattest du?  #00:00:28-8#

B: Also, als ich im Gefängnis war, dann die Polizei ist in der Nacht gekommen - ich weiß nicht, wie viele waren. Also das war so im Keller, so dunkel. Wir hatten keine (unv.), sondern nur so eine, so eine kleinen, so eine kleine Tür, wo man schwer reinkommt und rausgehen, sowas. #00:00:50-2#

I: hm (bejahend) #00:00:50-7#

B: Dann die haben mich geschlagen, in's Gesicht, überall. Dann nächste Tag die haben mich nach Afghanistan geschickt. (...) Ich war auf der Grenze, zwei Tage, dann ich hab von der anderen Aghaner ein Telefon geliehen, damm meine Mama angerufen. #00:01:05-7#

I: hm (bejahend) #00:01:05-7#

B: Dann sie hat zu mir gesagt:(...) 'geh nicht weiter, sondern ich nach Kabul oder nach Mazar-e einfach kommen. Also versuch mal wieder nach Iran kommen. Dann wir sind gekommen. (...) Dann wir eigentlich bin von Afghanistan bis Deutschland gekommen. (...) #00:01:25-8#

I: Okay. #00:01:26-9#

B: Weil ich im Iran war, die haben mich nach Afghanistan geschickt, da muss ich von Afghanistan wieder versuchen nach Europa kommen. #00:01:32-3#

I: ähm wie hast du's dann geschafft? War das über ähm über Land oder über's Boot? #00:01:39-0#

B: Also von Afghanistan bis Pakistan wir sind so ein bisschen zu Fuß und (...) ein bisschen mit dem Auto gekommen. #00:01:49-2#

I: hm (bejahend) #00:01:49-2#

B: Und so (...) wir haben versucht von Afghanistan direkt nach Iran kommen

I: hm (bejahend) #00:01:55-8#

B: Auf der Grenze war sehr gefährlich, die Polizei haben einfach erschossen. #00:02:00-6#

I: Hmh? #00:02:00-6#

B: Ja. Erschossen - viele Leute sind verletzt geworden, gestorben. Also damals (unv.) Kinder und Familie, Männer, waren richtig viele Leute. Jetzt ist auch so; jetzt gerade ist auch so. Wenn man auf die Grenze dann man sieht Beispiel wie viele Leute gestorben sind oder wie viele verletzt sind. Manche sind verletzt, die bleiben zwei, sieben Tage so verletzt die erbluten bis sie tot werden. Also niemand kann denen Beispiel zum Krankenhaus sowas bringen, weil Polizei erschießen. Die wollen nicht Beispiel die Afghanen nach Iran fluchten. #00:02:54-2#

I: Die wollen die nicht reinlassen? #00:02:54-2#

B: Sie müssen, also sie (...) sie machen Ihre Aufgabe, also Sie einfach erschießen, so den Leute. #00:03:04-3#

I: Aber du meintest doch mal, dass ähm(...) die Iraner ursprünglich mal Afghanen rekrutieren wollten, um gegen den IS zu kämpfen. #00:03:18-3#

B: Das war ja schon aber die haben(...) #00:03:20-6#

I: //Das ist jetz nichtmehr dann? Also das war dann nicht mehr? #00:03:23-9#

B: Doch, das war schon. Die haben schon zu mir gesagt, bevor, bevor ich in's Gefängnis. Bevor Sie mich in's Gefängnis bringen, Sie haben zu mir gesagt, Sie haben mir ein Blatt gegeben (...) und damals konnte ich nicht so gut Persisch lesen, die haben mich erklären,  gesagt *räuspert sich*: 'Wenn du (...) nach Syrien zum Kämpfen gehst, dann (...) du kriegst Geld und deine Familie haben Sicherheit, du danach, nach den zwei, drei Jahre kriegst du iranische Pass oder Ausweis. Du kannst besser leben, du kannst (...) Beispiel: Eine Wohnung mieten, sowas. Musst nicht schwarz leben.' Dann ich habe nicht akzeptiert. (...) ähm dann die haben mich einfach geschickt nach Afghanistan. #00:04:09-7#

I: hm (bejahend) #00:04:09-7#

B: Also, wie gesagt:ich war auch (...) ein paar Tage im Gefängnis (lacht). Dann (...) #00:04:16-8#

I:// Fuck. #00:04:16-8#

B: Wir wollten vom (lacht) Iran direkt nach (...) von Afghanistan direkt nach Iran kommen, dann nicht geschafft. Dann wir mussten wieder von Afghanistan nach Pakistan fahren #00:04:29-7#

I: hm (bejahend) #00:04:29-7#

B: Mit dem Auto. Wie gesagt, wir sind so ein bisschen zu Fuß gegangen. #00:04:33-9#

I: hm (bejahend) #00:04:33-9#

B: Wir mussten auf den Berge zu Fuß gehen, danach (...)ähm ein Auto auf uns gewartet. Wir waren (...) viele Leute in einem Auto. (unv.). Ich war da hinten vom Auto, ich weiß nicht, was heißt..? #00:04:48-7#

I: Kofferraum? #00:04:47-3#

B: Ja. Wir waren, ich weiß nicht, vier Leute? Fünf Leute?  #00:04:54-5#

I: Im Kofferraum? #00:04:54-5#

B: Mmmh.(lacht) #00:04:59-3#

I: Ooh. (lacht) Wie lange ging das? #00:05:01-0#

B: Ich glaube mehr als acht Stunden. #00:05:04-9#

I: Unbequem. #00:05:06-3#

B: Mmh. Danach meine ganze Körper hat eingeschlafen. #00:05:08-8#

I: Mmmh. #00:05:12-5#

B: Das ist, wenn man so (...) eine Stunde, zwei Stunden sitzt, dann man steht auf, dann die Beine funktioniert nichtmehr. #00:05:22-0#

I: Ja. #00:05:22-0#

B: Bei mir war ganz grob aus dem Kopf. (lacht) #00:05:23-9#

I: (lacht) #00:05:23-9#

B: Ich musste mich wie ein (...) wie eine Schlange auf dem Boden ziehen. #00:05:29-8#

I: Baaaah.. #00:05:29-8#

B: War aber.. (...) Wir hatten Angst, weil da hinten Polizei waren. Wenn die uns wieder verhaften wollten (...) Also (...) #00:05:38-3#

I: Da ist es besser als das Gefängnis. #00:05:40-7#

B: Mmh, dann vielleicht würden noch tiefer sein. #00:05:45-4#

I: hm (bejahend) #00:05:46-1#

B: Dann, wir sind nach Iran (...) geschafft von Afghan (...) von Afghanistan nach Pakistan geschafft, dann von Pakistan nach Iran geschafft. (...) Dann ich habe mich mit meiner Mama in Teheran getroffen #00:05:59-4#

I: hm (bejahend) #00:05:59-4#

B: Sie war nicht zu Hause, weil (...) ich hatte Angst. Ich (...) Depressivität nehmen.(...) Dann (...) meine Mama hat zu mir gesagt, also ich wollte nicht, echt, ich wollte nicht nach Deutschland kommen.  #00:06:13-0#

I: Ja. #00:06:13-0#

B: Meine Mama hat (...) ich meine nicht nach Deutschland, nach Europa. Aber mein Mama hat gesagt: 'Du musst gehen. Deshalb du kannst kein andere Wahl jetz, wenn du beim zweiten Mal (...) beim zweiten Mal verhaftet wirst, dann (...) kriegst du echt (...) zwei, vielleicht zwei oder drei Jahre im Gefängnisstrafen, also Schlagstrafen und #00:06:36-3#

I: hm (bejahend) #00:06:36-3#

B: Strafen Geld. Dann ich hab (...) gedacht, ja die Mama ist, die Mama hat recht dann. Am besten (...) fahre ich weiter. Dann (...) wir sind vom Iran nach (...) in die Türkei gekommen, also auf der Grenze aber richtig (...) gefährlich. Ich hab (...) zwei tote Leute gesehen. War furchtbar. #00:07:00-5#

I: An der Grenze? #00:07:00-5#

B: (...) Mmh. Ich weiß nicht, also der war unkennbar. Also (...) das, also das Gesicht war ganz kaputt. (lacht nervös) Wie ein Obst abgelaufen, sowas? #00:07:13-6#

I: hm (bejahend) #00:07:13-6#

B: Wo.. furchtbar, ich kann nicht ob (...) es war in der Nacht, und (...) wir sind geschafft (...) dann von der Türkei bis (...) Griechenland mit dem Boot gekommen. #00:07:29-6#

I: hm (bejahend) #00:07:30-0#

B: Also, über den Meer. #00:07:31-9#

I: hm (bejahend) #00:07:31-9#

B: (...) Dann von Griechenland bis (...) Hungary? zu Fuß gekommen #00:07:44-6#

I: Von Griechenland bis Ungarn zu Fuß.  #00:07:46-2#

B: Ja, ganze Weg zu Fuß. (...) #00:07:48-4#

I: Das ist ein weiter Weg. Wie lange hat'n des gedauert? #00:07:49-8#

B: (lacht) Weiß ich nicht. Viele Leute waren, das war eine Massenflucht.  #00:07:56-8#

I: Aber (...) sag mal, auf dem (...) Weg von Griechenland nach Ungarn, das sind so viele Leute, wie kriegt ma da was zu Essen und alles? #00:08:09-6#

B: Im Hungary (...) im Hungary wir haben von Deutschland Essen bekommen. #00:08:15-5#

I: Echt? #00:08:15-5#

B: Ja. Wir haben gesehen, viele Leute arbeiten freiwillig, Sie hatten so deutsche (...) wie heißt das (...) deutsche (...) #00:08:25-0#

I: Fahnen? #00:08:25-0#

B: Ja, genau, deutsche Fahne, die hatten da (...) #00:08:28-6#

I: hm (bejahend) #00:08:28-6#

B: die Fahnen, die (...) die haben auch deutsch gesprochen. Wir haben nicht verstanden. Die haben Englisch gesprochen, und (...) aber manche, also Sie miteinander die haben deutsch gesprochen, und mit uns Englisch.  #00:08:41-2#

I: hm (bejahend) #00:08:41-8#

B: Und bis Griechenland, um Griechenland wir haben von Griechenland Hilfe bekommen #00:08:46-1#

I: hm (bejahend) #00:08:46-1#

B: (...) Und auch die Deutsche waren in Griechenland dann. (...) ähm von (...) Österreich, von Hungary bis Österreich, wir sind mit dem Zug gefahren, dann wir waren erst drei (...) (unv.) sieben? Sieben Tage, ich glaube.. #00:09:07-1#

I: hm (bejahend) #00:09:07-1#

B: Ne, ich glaube weniger. Weiß ich nicht, vergessen - zu lange her. Dann wir sind von Griechenland bis Deutschland mit dem Zug wieder gekommen. Aber (...) von Afghanistan (...) bis Griechenland, die Grenzen waren Katastrophe. Als ich auf dem Wasser war, im ähm, im Meer #00:09:35-1#

I: hm (bejahend) #00:09:35-1#

B: Wir waren zwei Boote, und aus Plastik. Die waren nicht Sicherheit, Beispiel: Die war sechs Meter lang und vier Meter breit. Wir waren 60 Leute auf dem (unv.)(lacht). #00:09:44-4#

I: Was?! #00:09:53-1#

B: (Isst Erdbeere) Danke, ne? #00:09:53-1#

I: Ja gerne! Ich probier's auch mal (...) (isst Erdbeere) Hoffentlich schmecken die nicht nur nach Wasser.(lacht) #00:09:58-7#

B: Nur ein bisschen. (...) Aber schmeckt, danke. #00:10:08-0#

I: Gerne. Mal schaun, wir haben auch Erdbeerpflanzen zu Hause #00:10:13-4#

B: hm (bejahend) #00:10:13-4#

I: Dann bring ich dir mal welche mit #00:10:14-8#

B: Danke. (...) Dann (...) Grenze von (...) Afghanistan/Iran war am gefährlichsten. Es war sehr gefährlich. (...) #00:10:30-7#

I: hm (bejahend) #00:10:30-7#

B: Aber (...) wir waren im Meer, dann (...) zwei Boote. Einer war Araber, die war (unv.) ,waren auch viele Familien. #00:10:46-4#

I: hm (bejahend) #00:10:46-8#

B: Dann, ich glaube Ihre Boot waren kaputt. Dann die sind ähm, das Boot hat sich umgedreht. #00:10:55-9#

I: Das ist gekentert? #00:10:57-6#

B: Dann viele Leute ertrunken. Aus dem drei Leute sie haben geschafft wieder auf den Boot kommen. Familien sind gestorben, aber die (...) das war in Türkei. Aber die (...) Türkei haben nicht geholfen. Garnicht. Wir haben die Polizei (unv.)(...), die Polizei haben nur angeschaut. #00:11:17-7#

I: Die haben zugeschaut? #00:11:17-7#

B: Nur zugeschaut. Nur zugeschaut. Garnix gemacht. Wir waren dabei. Und wir hatten auch Angst, also wir haben auch geweint und geschrien (...). Nicht für den, nicht für die Araber, sondern für uns (lacht). Wir waren auch auf dem Boot im Meer. Aber zum Glück wir waren vierundf.. ähm 45 Minuten (...) im Wasser, dann wir haben geschafft. Also es war sehr gefährlich. Von, von Griechenland ist (...) Deutschland war nicht so gefährlich, aber von Afghanistan. Afghanistan, Pakistan, Iran. Die waren am gefährlichsten Grenze. #00:12:03-5#

I: (...) #00:12:03-9#

B: Sehr, sehr gefährlich. #00:12:05-8#

I: hm (bejahend) #00:12:05-8#

B: Ich hab (...) viele tote Leute gesehen, viele Verletzte gesehen. Manchmal ich denke, das war wie ein Albtraum oder wie einen Film. #00:12:17-8#

I: Zum Glück bist du jetzt hier. #00:12:18-9#

B: (lacht) Viele Leute haben geschafft. Wenn ich nach Afghanistan wieder abgeschoben werde, ich werde nie wieder nach Deutschland kommen. Ich will nie wieder diese Grenze sehen. Das war furchtbar. Ich will dort hungrig bleiben, (lacht) nicht sterben aber nicht wieder (...) solche Grenze sehen (lacht). Weißt du, als ich (...) also *räuspert sich* daheim versucht von Afghanistan nach Iran direkt kommen, gerade habe ich schon gesagt, und ich versuche jetzt richtig erklären, was ich genau gesehen habe. #00:12:56-5#

I: Du, musst du nicht. Also, wir, wir können auch wo anders weiter machen. #00:13:00-8#

B: Also (...) #00:13:02-3#

I: Das musst du wirklich ned alles wieder nacherleben. Meinetwegen können wir auch ganz woanders ansetzen wenn du willst. #00:13:09-1#

B: Also die (...) was ich erlaub (...) erlebt habe, das ist (...) wie ein Bild, immer, vor den meinen Augen, weißt du? #00:13:17-0#

I: hm (bejahend) #00:13:17-0#

B: Ich sehe immer (lacht) manchmal, in der Nacht, ich meine die, ähm ich lege mich auf den Bett, dann ich will über gute Sachen denken, dann die kommen einfach die (lacht) die schlechte Sachen, ich will nicht denken, aber (...) Aaaah! #00:13:30-7#

I: Ow, fuck(...) #00:13:35-0#

B: (lacht) Es ist so. #00:13:40-5#

I: ähm, lass was anderes ähm, lass wo anders hinspringen oder? #00:13:42-2#

B: Okay, dann du kannst andere Frage stellen. (lacht) #00:13:44-8#

I: ähm #00:13:46-3#

B: Umso besser(lacht) #00:13:46-3#

I: Erstmal, ähm (...) seit wann bist du in Deutschland, und genau in Regensburg? #00:13:57-8#

B: Ich bin direkt nach Regensburg gekommen. #00:14:01-1#

I: Du bist direkt nach Regensburg gekommen? #00:14:02-8#

B: Ja. #00:14:02-8#

I: Und das war vor (...) drei Jahren gell?  #00:14:05-0#

B: Dreieinhalb Jahren. #00:14:05-8#

I: Dreieinhalb Jahren. #00:14:06-0#

B: 19. 09. 2015 (...) Ich bin in Regensburg angekommen, also in Deutschland. #00:14:13-6#

I: hm (bejahend) #00:14:13-6#

B: Dann (...) #00:14:16-8#

I: ähm #00:14:16-8#

B: zweite Tag, wir haben mit (...)ähm Deutschkurs begonnen. Sprachkurs. #00:14:22-6#

I: Also du bist hier angekommen, und direkt am Tag drauf is der Deutschkurs losgegangen. #00:14:23-8#

B: // Ja, 20.09. wir haben mit dem Sprachkurs begonnen. Also, das damals einei halbe Stunde. #00:14:35-1#

I: Anderthalb Stunden am Tag? #00:14:35-1#

B: hm (bejahend) #00:14:35-1#

I: Hmja, hat sich ausgezahlt (lacht) #00:14:36-1#

B: (lacht) Ja. Nicht so gut geworden, aber ja. Es könnte besser werden. #00:14:43-9#

I: Mit der Zeit wird das  #00:14:46-5#

B: Mh. #00:14:46-5#

I: Funktioniert ja jetz schon ganz gut! #00:14:47-4#

B: So ein bisschen (lacht) #00:14:49-7#

I: Ich meine, wir können uns unterhalten oder? (lacht) #00:14:51-7#

B: Mh. Also ich verstehe mehr als ich spreche. Ich verstehe mehr. Ich versuch's auch ohne Fehler spüren, aber (...) so schwierig. #00:15:03-3#

I: Das kommt mit der Zeit. #00:15:03-3#

B: Hoffe ich (lacht) #00:15:07-8#

I: ähm sag mal mit ähm also momentan hast du keinen Kontakt zu deiner Familie zurück oder? #00:15:16-3#

B: Ne. Als ich ähm nach Deutschland kam, dann ich hatte schon Kontakt bis ein Jahr, dann (...) meine Familie haben zu mir gesagt 'wir haben eine Brief bekommen von iranische Politik, also iranische Staat' #00:15:32-1#

I: hm (bejahend) #00:15:32-1#

B: Die haben gemeint, also damals war Krieg im Syrien #00:15:37-5#

I: hm (bejahend) #00:15:37-8#

B: Meine Mutter hat gemeint, also wir haben am Telefon gesprochen, Sie hat gemeint, (...) im Brief steht, wir wohnen seit mehr als zehn Jahren im Iran schwarz (...)  #00:15:52-2#

I: hm (bejahend) #00:15:52-7#

B: Und entweder wir haben kein Ausweis, wir müssen wieder zurück nach Afghanistan oder (...) Ich habe einen große Bruder, der ist zwei, drei Jahre größer als ich. #00:16:02-0#

I: hm (bejahend) #00:16:02-5#

B: Entweder mein Bruder muss nach Syrien gehen, kämpfen, dann Beispiel meine Familie kann im Iran bleiben und die bekommen Geld. Eigentlich, meine Mama ist sehr religiös, mein Bruder (...) und wir sind Schiiten (...) und meißten Schiiten nach Syrien zum Kämpfen gegangen wegen (...) Ihrer Würde oder (...) sie haben geglaubt, wir kämpfen für unsere Religion, nicht Beispiel für Iran. #00:16:38-6#

I: hm (bejahend) #00:16:38-6#

B: Manche, also, manche sind freiwillig hingegangen, manche haben Geld gekriegt. Manche mussten gehen. (...) Es war so (...) verschiedene Gründen. #00:16:50-5#

I: Ja. #00:16:50-5#

B: Und bei meiner Familie, meine Bruder Beispiel sollte hingehen. Aber der hat nicht (...) meine Mama hat nicht akzeptiert. Meine Mama hat gesagt: 'Ich hab einmal eure Vater gesehen, wie Taliban hat geschlachtet. Ich will nicht meine Kinder so sehen. (...) Und meine Mama hat gesagt, ich hatte solche Scheiße erfahrung, ich will nicht wieder diese Erfahrung (...) #00:17:12-8#

I: Ja #00:17:14-8#

B: Dann mein Bruder (...) dann meine Familie mussten (...) also, die Mama hat gesagt, wenn ich dich rufen an, dann die kontrollieren grad die ganzen Telefonen. Wenn die Leute telefonnieren, dann die kontrollieren, wissen wie viel IS im Iran sind.  #00:17:32-9#

I: hm (bejahend) #00:17:33-4#

B: Im Iran gibts schon viele IS. Die #00:17:35-5#

I: Oh. #00:17:35-5#

B: wollen Anschlagen noch oder so #00:17:38-1#

I: Und deswegen werden die Telefone abgehört? #00:17:40-1#

B: Ja. #00:17:41-7#

I: hm (bejahend) #00:17:42-8#

B: Und meine Mama hat Angst Beispiel, und die (...) kontrollieren auch, kontrollieren auch die Afghaner keine Ausweis haben, die kontrollieren die Telefonnate dann #00:17:51-3#

I: hm (bejahend) #00:17:51-3#

B: Wenn meine (...) viele Afghaner haben so (...) die haben gefunden, Beispiel die Afghaner hatten keinen Ausweis, die haben mit den Kindern gesprochen, dann (...) die Polizei hatten die Name und die Telefonnummer im Internet und die haben, ähm die haben nachgeschaut, die haben gesehen, die Familien wohnen seit schon 20, 30, 15, 17 Jahre schwarz. Entweder sie müssen zum Krieg oder zurück nach Afghanistan #00:18:16-0#

I: hm (bejahend) #00:18:16-3#

B: Viele Afghane sind nach Afghanistan geflüchtet(...) und meine Mama hat Angst die müssen umziehen irgendwo anders. Dann wir hatten eineinhalb Jahre keine Kontakt #00:18:29-5#

I: hm (bejahend) #00:18:29-5#

B: Dann Sie haben sich wieder gemeldet, dann wir haben zwei, dreimal miteinander gesprochen. (...) Dann meine Mama hat wieder aufgehört, also wir haben wieder nicht Kontakt gehabt bis vor 4 Monaten, wir haben wieder gesprochen. (...) Sie hat zu mir gesagt, momentan (...) also vielleicht bald wird besser. Vielleicht wir könnenn mehr unterhalten.  #00:18:54-3#

I: (...)Das wäre gut. #00:18:55-0#

B: Und, ich wünsche mir auch (...) ich hoffe (...) #00:19:00-0#

I: ähm sie ruft dich dann quasi auf deinem Handy an, oder? #00:19:02-6#

B: Ja.  #00:19:03-8#

I: Okay #00:19:03-8#

B: Die haben nicht glaube meine Nummer, ich habe nicht Nummer von denen.  #00:19:07-1#

I: hm (bejahend) #00:19:08-0#

B: Sie haben schon zu mir angerufen, aber nicht von Ihrer Nummer, sondern von mein Nachbarin. #00:19:11-5#

I: hm (bejahend) #00:19:12-0#

B: Ich habe immer Kontakt mit mein Nachbarin, die wohnt im Iran #00:19:15-5#

I: hm (bejahend) #00:19:15-5#

B: Der (...) also ich nenne Sie immer Tante, weil Sie (...) hat von uns viel gemacht (...) weil Sie für uns viel gemacht hat (lacht) #00:19:25-0#

I: Mhm. #00:19:27-3#

B: Ja. (...) #00:19:31-1#

I: Ja (...) Hoffentlich wird das bald besser. #00:19:36-2#

B: Ich hoffe. Trinkst Tee? #00:19:38-3#

I: Ich hab noch, danke! #00:19:38-3#

B: Hmh. #00:19:42-3#

I: ähm mit wem hast du dann hier in ähm in Deutschland Kontakt? Also ich mein, okay, die Leute aus'm Jugendcafé natürlich, und #00:19:56-8#

B: Ja #00:19:56-8#

I: Hallo, ich #00:19:56-8#

B: (lacht) ja.  #00:20:00-7#

I: ähm wer ist denn sonst noch so dabei? #00:20:06-3#

B: Hmh (...) Schon viele Leute kenne ich. #00:20:11-5#

I: hm (bejahend) #00:20:11-5#

B: Ich habe Kontakt, Beispiel: Mein Schwimmlehrerin, mein Schwimmlehrer, die sind beim (...) Mhh (...) Campus Asyl? #00:20:24-8#

I: Ah! Mhm. #00:20:28-8#

B: Vielleicht kennst du auch denen? #00:20:28-8#

I: ähm, mit denen habe ich geschrieben, aber ich muss (...) getehen, dass, ähm dass ich da momentan noch nicht geantwortet hab #00:20:40-3#

B: hm (bejahend) #00:20:40-3#

I: Muss ich noch machen (lacht) (...) ähm Ja. Shit. Muss ich noch machen. #00:20:50-2#

B: Und kenne ich viele andere Leute, die haben mich echt viel geholfen. Beispiel Thomas, macht seit drei Jahren #00:21:00-6#

I: hm (bejahend) #00:21:00-6#

B: Also gibt mir seit drei Jahren Nachhilfelehrer. #00:21:04-3#

I: Der macht vor allem Mathe gell? #00:21:04-3#

B: hm (bejahend), also Mathe, Deutsch, (...) ähm #00:21:11-2#

I: Jetz können wir dann mal Englisch anpacken. #00:21:10-6#

B: Mhhja genau (lacht) #00:21:14-0#

I: (lacht) #00:21:14-0#

B: Und (...) der macht auch einfach Ehr (...) Ehrenamtlich, oder? #00:21:18-2#

I: Der is Ehrenamtlich, genau. ähm(...) #00:21:24-1#

B: Vor drei Jahren, (...) der hat mit mir begonnen  mit dem Lernen. Damals konnte ich nicht Deutsch reden. Also, so, ein bisschen.  #00:21:33-8#

I: hm (bejahend) #00:21:33-8#

B: So (...) habe ich Ausländerdeutsch geredet. Ich gehen, du gehen (lacht) #00:21:39-2#

I: (lacht) Ja gut aber, ähm, ich denk mal da hast du dich dann jetz schon deutlich verbessert oder? #00:21:47-2#

B: Bisschen. (lacht) #00:21:52-5#

I: Ich denk das ist ganz passabel so. (lacht) #00:21:54-1#

B: (lacht) (...) Wie gesagt, es (...) es könnte besser werden. Aber ich versuche es gerade. #00:22:01-2#

I: Das kommt alles mit der Zeit. #00:22:02-8#

B: hm (bejahend) #00:22:03-7#

I: Hast du die Asterixe schon gelesen? #00:22:05-5#

B: So, ja (...) Ich glaube, ich hab ein Buch fertig gemacht, aber (...) wie gesagt, gibt's viele von den neue Worten. #00:22:13-2#

I: ähm also weil wenn du (...) ähm wenn du was anderes lesen willst, dann sag auch Bescheid. Also, wenn dir das ned gefällt (...) #00:22:23-4#

B: Ja, danke #00:22:23-4#

I: Dann such ich irgendwas anderes, und ansonsten, wenn du's fertig gelesen hast, dann bring ich dir neue, da gibts mehr. #00:22:28-7#

B: Ja, okay. Danke. #00:22:28-7#

I: Gerne. (...) ähm Ja. #00:22:33-3#

B: Und, neue Frage? (lacht) #00:22:35-9#

I: ähm okay, ähm (...) Bildung: Du hast in Afghanistan und im Iran warst du nie in der Schule gell? #00:22:45-4#

B: Ne.  #00:22:48-6#

I: Also erst ab hier. #00:22:49-5#

B: Ja, seit dreieinhalb Jahre besuche ich Schule. #00:22:53-0#

I: hm (bejahend) #00:22:53-0#

B: Mache ich Schule. Vorher habe ich nicht besucht. #00:22:57-9#

I: ähm Bezüglich der Schule: ähm da hab ich nen (...) Interview ghabt mit jemandem aus Syrien #00:23:09-2#

B: hm (bejahend) #00:23:10-2#

I: Der meinte, dass seine Lehrer alle bayrisch reden würden und dass er Sie nicht verstehen würde. Is das bei dir auch? #00:23:18-6#

B: Bei mir war so in der Ausbildung. #00:23:22-8#

I: Echt?
 #00:23:24-5#

B: Mein Lehrer hat auf bayrisch gesprochen, ich habe nicht verstanden, dann ich hab gesagt: 'Können Sie bitte, also versuchen Sie bitte auf , auf Deutsch sprechen.' Der sagt: 'Das ist nicht mein Probem, das ist dein Problem. Du musst verstehen.' Das war so. (lacht) #00:23:40-5#

I: Hmh..? (...) Okay. #00:23:43-2#

B: (lacht) Aber der war nett, der hat immer Spaß gemacht mit uns. Aber wegen Sprache der hat gesagt 'Ist nicht mein Problem.' Der konnte nicht Hochdeutsch sprechen. #00:23:51-5#

I: Ach, er konnt's nicht? #00:23:53-0#

B: Er hat gesagt 'Das ist dein Problem. Du bist in Deutschland, du musst die Sprache lernen.' #00:23:58-2#

I: Okay. (lacht) #00:24:02-5#

B: (lacht) Aber der war nett. #00:24:02-5#

I: Ach? Das (...) das ist schön. #00:24:02-3#

B: Der hatte immer zu uns Respekt. #00:24:06-8#

I: hm (bejahend) #00:24:07-4#

B: Außerdem die Sprache war nicht seine Schuld, war unsere Schuld.(lacht) #00:24:08-0#

I: (lacht) #00:24:10-3#

B: Trotzdem ich habe eine vier geschrieben. #00:24:14-8#

I: (lacht) #00:24:16-1#

B: Aber (...) Mein Chef war nicht gut. (...) #00:24:24-5#

I: (...) Warte, mach ma einfach mal die nächste. ähm, also du bsit momentan ein anerkannter ähm Geflüchteter oder? Also dein Asylstatus ist (...) 'angenommen', oder? #00:24:40-9#

B: Ja, so (...) DU kannst hier (unv.) (Zieht Aufenthaltserlaubnis aus dem Geldbeutel) #00:24:48-5#

I: Aufenthaltserlaubnis? #00:24:49-8#

B: Mhm. #00:24:51-8#

I: Ah. (...) Ist ein interessantes Ding. Warte (...) Moment, ich hol mal eben meinen Perso. (...) #00:25:13-0#

B: Ganz unterschiedlich. #00:25:18-8#

I: (...) Ja, aber (...) warte. (...) Ach tatsächlich, der (...) Auch das Wackelbildchen hier drunter ist anders #00:25:32-5#

B: Ja. Ich denke (...)(lacht) #00:25:44-7#

I: Sogar mit deiner (...) mit der zuständigen Behörde mit dazu.  #00:25:52-1#

B: hm (bejahend) #00:25:52-1#

I: Ach warte, das steht bei mir auch drauf. (...) Meh, ja okay, steht bei mir das selbe drauf. #00:25:58-8#

B: (lacht) #00:25:58-8#

I: ähm (...) Aber das heißt, dein Status war von Anfang an anerkannt. Wo hast du den Antrag dann gestellt ghabt? #00:26:12-0#

B: Ich weiß es nicht, meine Vormund hat gemacht. Damals hatte ich Vormundin #00:26:17-2#

I: hm (bejahend) #00:26:18-1#

B: Sie hat alles für mich gemacht und (...) weiß es nicht. #00:26:23-4#

I: Hast du mit der jetz noch Kontakt? #00:26:25-7#

B: ähm ne. #00:26:25-6#

I: Okay. #00:26:25-6#

B: Sie hat, also (...) sich schon beschäftigt. #00:26:34-7#

I: hm (bejahend) #00:26:34-7#

B: Sie macht gerade eine andere Arbeit. #00:26:34-5#

I: Ah. (...) Mmh, und des is jetz hier auch (...) is jetz die zweite Wohnung, in der du hier lebst oder? #00:26:46-5#

B: hm (bejahend) #00:26:46-5#

I: Also, erst war's ein Stock weiter unten, und jetzt hier? #00:26:50-9#

B: // ähm ich war ein Jahr drei Monate in Don Bosco. #00:26:55-7#

I: Achso. #00:26:56-2#

B: Habe dort gewohnt. Dann *räuspert sich* wir hatten keine Betreuer. (...) Eigentlich, ich musste bis 21 dort bleiben. ähm Als ich in Deutschland gekommen bin, dann wir waren selbe ständig. Wir hatten Vormundin aber wenige Betreuer. Wir mussten selbstständig sein und selbst zum Arzt gehen.  #00:27:16-8#

I: hm (bejahend) #00:27:17-3#

B: Und manche Sachen selber erledigen. Wir waren Don Bosco Gruppe 4. #00:27:20-0#

I: ähm hat das funktioniert? #00:27:23-2#

B: ähm nicht so genau. Weiter nur Problem. Dann (...) #00:27:29-4#

I: Was ist denn da schief gelaufen? #00:27:31-2#

B: ähm ich bin öfter zum Arzt gelaufen, und wir hatten keine Übersetzer, keine Betreuer. Und der hat, Beispiel wenn ich Kopfschmerzen hatte, der hat nicht verstanden was ich meine. Der hat einfach meinen Fuß angeschaut oder meinen (...) #00:27:45-2#

I: Du bist zum Arzt gegangen, und du konntest aber damals noch kein Deutsch, und es war kein Dolmetscher dabei? #00:27:50-9#

B: Ne. Und ich hab einfach, wir haben so einen Satz gelernt, wir haben nur gelernt: 'Mir ist schlecht.'  #00:28:01-9#

I: hm (bejahend) #00:28:02-4#

B: Dann, der sagt 'wo?' Ich habe nicht verstanden. Der hat einfach mit seinem angefasst, Beispiel hier, hier, hier, dann ich musste einfach zeigen Beispiel hier oder (...) das war echt schwierig. #00:28:16-1#

I: hm (bejahend) #00:28:16-5#

B: Das war echt so hart, harte Zeit aber (...) Gott sei Dank, es ist schon vorbei (lacht). Und (...) ich hatte Migräne, jetzt habe ich auch, dann musste ich immer Tabletten nehmen #00:28:28-9#

I:  hm (bejahend) #00:28:30-2#

B: Bis ich schlafen konnte. Dann unsere Betreuerin hat immer zu mir gesagt: 'Bevor ich nach Hause gehen (...) ich gebe dir eins, eine Tablette, dann kannst du nehmen, dann schlafen.' (...) Und (...) wir haben, also ich habe meine bekommen, danach der sagt ich mache (...) ich weiß nicht, wie viele Tage Urlaub, dann du kriegst von mir 20 Tabletten. Dann ich hab bekommen, damals war ich 16, 16-jähriger, dann ich hab an einem Nacht, in der Nacht hatte ich so stark Kopfschmerzen, dann ich habe eine genommen. Hat nicht geholfen.  #00:29:11-4#

I: hm (bejahend) #00:29:11-4#

B: Dann ich habe wieder eine genommen. Wieder nicht geholfen. #00:29:11-4#

I: (...) #00:29:11-4#

B: Ich habe drei Tabletten genommen, die waren 600. #00:29:13-9#

I: Oooh. #00:29:15-8#

B: Danach (...) #00:29:15-8#

I: Ibuprofen 600? #00:29:16-1#

B: Weiß ich nicht, das war das aber. War 600, darauf habe ich gesehen dann. #00:29:21-5#

I: Mhm. #00:29:21-5#

B: ähm mir war schwindlig, dann ich (...) mir war schwindlig, dann ich bin einfach auf dem Bett gelegen und geschlafen bis nächstes Tag.  #00:29:28-7#

I: Das sind sehr starke Schmerzmittel. #00:29:32-3#

B: Ja, und nächste Tag, ich hatte einen Termin beim Arzt. #00:29:34-7#

I: hm (bejahend) #00:29:34-7#

B: Ich habe 20 Tabletten bekommen, in (...) sieben Tage ich habe fast alles genommen. (lacht) #00:29:44-3#

I: (...) #00:29:50-9#

B: Dann ich hatte einen Termin bei Hautarzt, dann (...) ähm (unv.), dann die hat zu mir gesagt: 'Wie gehts dir momentan? Wie gehts mit deinen Tabletten? Kriegst du nicht, wie viele nimmst du am Tag Tablette?' Dann ich habe gesagt, meine Betreuerin ist im Urlaub, ich habe fast 20 Tablette in sieben Tage genommen. Dann (...) dieser Arzt hat meine Vormundin angerufen, meine Betreuerin, der Chefin, (...) und der Chef, und Jugendamt. Alle sind gekommen, udn Übersetzer. Wir waren ungefähr neun Leute. #00:30:29-3#

I: Uff. #00:30:29-3#

B: Vom Jugendamt, Chef von unsere Heim, Betreuerin, Sie musste vom Urlaub kommen. #00:30:36-1#

I: Oh, Scheiße. #00:30:36-8#

B: UNd meine Vormundin (lacht), meine Betreuerin. Wir waren alle in einem Raum.  #00:30:40-0#

I: Ahh. (...) #00:30:44-8#

B: Dann hat zu mir gesagt: 'Wo wohnst du?' Ich habe (unv.), also, (...) Ich wohne in Don Bosco, und zwar in Gruppe vier. Wir (...) zwei Leute wohnen in einem Zimmer, und der schnarcht jeden Tag (lacht) jede Nacht #00:30:57-5#

I: (lacht) #00:30:59-0#

B: Muss ich immer so stark Tabletten nehmen, bis ich so tief schlafen, nicht zuhören, ne? #00:31:02-3#

I: Ja. #00:31:02-3#

B: Dann, mich nicht aufwecken. Dann hat zu meinem Chef gesagt: 'Entweder du gibst ihm ein Einzelzimmer, oder (...) er muss von Don Bosco raus.' Dann die haben gesagt: 'Wir können nicht Einzelzimmer geben, weil hier (...) niemand EInzelzimmer hat.' #00:31:23-0#

I: hm (bejahend) #00:31:23-0#

B: Deswegen die haben mich in (unv.) geschickt. Und so war, ich habe Einzelzimmer bekommen, zum Glück dann. Ich habe meine Tabletten regelmäßig bekommen, nicht zwei, drei, sondern immer eine. #00:31:36-9#

I: Ist die Migräne seitdem besser geworden dann? #00:31:40-4#

B: Ja. Früher hatte ich so stark, letztes Jahr mir war (...) immer schwindelig. Ich durfte nicht alleine in die Stadt gehen. #00:31:48-0#

I: Was? ähm (...) Bist du dann auch mal umgekippt oder..? #00:31:55-5#

B: Öfter. #00:31:56-0#

I: Öfter?! #00:31:56-0#

B: Zweimal die Betreuerin und mich vom, von der Straße abgeholt. (lacht) #00:32:03-9#

I: Oi. #00:32:07-1#

B: Dieses Jahr, vor zwei Wochen hatte ich wieder einen Unfall. Wegen Schwindligkeit.  #00:32:09-7#

I: Vor zwei Wochen erst? #00:32:14-5#

B: hm (bejahend). Dann ich war im Krankenhaus, ungefähr vier Stunde, dann die haben gesagt: 'Du hast nix. Geh nach Hause.' Dann bin ich nach Hause gekommen.(lacht) (...) Haben gesagt, das ist einfach (...) ich weiß nicht, der sagt, dass nicht Krankheit einfach (...) irgendwas mit dir (...) nicht stimmt. (lacht) #00:32:35-6#

I: Also, du hast bis jetzt keine Diagnose dazu? #00:32:39-0#

B:// Aber ist nicht schlimm. (...) Was ist das? #00:32:43-3#

I: ähm, die Ärzte wissen nicht genau, was dir fehlt, oder wissen's schon?  #00:33:07-0#

B: Sie haben fast ganze Gruppe untersucht, aber wir (...) du bist gesund ansich. (lacht) #00:32:54-7#

I: ähm (...) #00:32:57-1#

B: Hast du nur Migräne, dann kannst (...) das ist Erbe. Von meine Mama. Die hat auch Migräne. #00:33:02-2#

I: Okay. (...) #00:33:02-2#

B: Und meine Mama hat auch Herzschmerzen, ich hoffe, dass ich nicht die selbe bekomme.  #00:33:12-0#

I: Sind die schlimm? #00:33:15-8#

B: Als ich in Iran war (...) sie hatte viele Probleme. (...) Aber ich weiß nicht, ob sie (...) am Leben ist oder nicht. Letztes mal, ich habe ein Foto bekommen vor zwei Monaten, sie (...) lage im (...) Krankenhaus. (...) Weiß ich nicht, also (...) ich hoffe (...) sie ist am Leben. Ich will Sie wieder sehen. Schauma mal. Das macht mein Schicksal. #00:33:49-2#

I: ähm (...) Hast du (...) seitdem noch irgendwie Kontakt mit deiner Nachbarin? #00:33:54-2#

B: (unv.) Ja so, ich rufe Sie immer an. Also, nicht immer, jedoch zwei, drei mal. #00:34:02-4#

I: Und die weiß auch nicht, was los ist? #00:34:03-6#

B: Ne, sie hat zu mir gesagt, wenn du nach Iran geflüchtet hast, dann können wir zusammen finden. (...) Und vielleicht im Sommer fliege ich nach Iran, vielleicht nicht. Ich weiß nicht. #00:34:17-3#

I: hm (bejahend) #00:34:18-0#

B: Aber ich werde Sie finde. Schauma mal. #00:34:21-4#

I: Du willst dich im Sommer auf die Suche machen, oder wie? #00:34:27-2#

B: Vielleicht. Ich warte auf meinen Reisepass. (...) Wenn ich bekomme, dann ich versuch's denen finden und (...) aber (...) und mein große Bruder wohnt seit zweieinhalb Jahren in Amerika. #00:34:46-3#

I: In Amerika? #00:34:48-1#

B: hm (bejahend). In New York. (...) #00:34:58-0#

I: ähm wird er (...) auch bei der Suche mitmachen, oder (...)? #00:35:03-5#

B: Ja, der macht auch. Der hat gesagt, vielleicht fliegt in drei, oder zwei Monaten nach Iran. #00:35:14-7#

I: Also mit (...) mit Ihm bist du jetzt quasi #00:35:18-7#

B: Kontakt? #00:35:20-3#

I: Im kontakt. #00:35:20-3#

B: hm (bejahend) Also die war (...) der war auch bei meiner Familie vor drei Jahren #00:35:27-3#

I: hm (bejahend) #00:35:27-3#

B: Und (...) mein Bruder musste (...) von meiner Mama, meiner Familie getrennt werden. Dann, Sie haben sich getrennt. Dann, mein Bruder mit seiner Familie von Polizei verhaftet und nach Afghanistan geschickt. Dann, die waren in Afghanistan, dann (...) so der hat zu mir erzählt. Der hat zu mir gesagt: ' Eine Freund hat zu mir gesagt, gehen wir nach Pakistan, wir können uns melden.' Dann (...) also (...) die UNO hat, ähm gibt uns Geld, bis wir in Pakistan bisschen Essen Geld kriegen #00:36:13-9#

I: hm (bejahend) #00:36:13-9#

B: Bis wir am Leben bleiben, dann (...) wir machen Interview in Pakistan bei amerikanische Soldat #00:36:21-9#

I: hm (bejahend) #00:36:22-3#

B: Dann machen (...) vielleicht wenn wir Chance haben, dann wir machen für zwei Monaten, wir bekommen einen Anwalt, dann können wir nach Amerika fliegen. Dann hie haben das gemacht und geschafft. Und mein Bruder von Pakistan nach Amerika geflogt. Geflogen. Ist. Ja, der ist nach Afghanistan geflogen, dann der wohnt, dann wir hatten keinen Kontakt. (...) Vor (...) fünf? Monaten ich habe in Facebook mit jemanden geschrieben. Dann die hat gesagt: 'Bist du Mahdi?' Ich sage: 'Ja'. Dann die sagt: 'Ich bin Frau von deinem Bruder'. Sage ich 'was'? #00:37:01-5#

I: Ui. #00:37:01-5#

B: Sagt er (...) #00:37:05-9#

I: Wie, wie bist, wie bist, hast du Sie erwischt? #00:37:10-4#

B: War furchbar, ich haben nicht gekannt. Weil, ich habe gedacht, mein Bruder ist im Iran, der kann nicht so einfach anrufen, mit mir reden. #00:37:17-0#

I: hm (bejahend) #00:37:17-0#

B: Aber das war, wir haben in Facebook geschrieben. #00:37:19-8#

I: hm (bejahend) #00:37:19-8#

B: Dann die hat gesagt: 'Ich habe deinen Namen in Facebook angegeben, dann ich habe dich gesehen deswegen ich habe dir geschrieben. Dein Bruder ist in der Arbeit, der kommt am Abend. Aber wir können, aber du kannst mich gleich anrufen'. Dann ich habe Ihr angerufen, wir haben gesprochen. (...) Dann Sie hat so einfach über sich erzählt wie wir nach Amerika gekommen sind, war sehr schwierig so, so, so. #00:37:48-2#

I: hm (bejahend) #00:37:48-8#

B: Und (...) mein Bruder hat zu mir gesagt: 'Wenn geht, ich will dich nach Amerika sehen zu mir'. Dann ich habe gesagt: 'Ne, ich bleibe in Deutschland'. Ich will nicht nach Amerika. Vielleicht mache ich Urlaub, aber ich will nicht dort wohnen #00:38:05-3#

I: hm (bejahend) #00:38:05-7#

B: Mh. (...) (unv.) Gefällt mir in Deutschland. Weil, hier man hat (...) weil hier hat man mehr Sicherheit hat, mehr Freiheit. #00:38:21-4#

I: Als in Amerika? #00:38:23-0#

B: Hier. #00:38:23-0#

I: hm (bejahend) #00:38:23-4#

B: In Amerika, jeder Leute hat ein Waffe. #00:38:27-4#

I: Stimmt. #00:38:27-4#

B: Gefährlich. #00:38:31-5#

I: Hier ist es nur auf der Autobahn gefährlich. #00:38:32-4#

B: Ja. (lacht) (...) Nächste? #00:38:39-6#

I: ähm gerne, ja. (...) Nächstes mal bring ich dir'n Tee mit, den trink ich dir ja immer weg #00:38:50-3#

B: Nee, passt! Ich hab viele gekauft, das kostet nicht so viel. Ich glaube. (lacht) #00:38:58-8#

I: (Macht Handgeste) Das ist eine italienische Handgeste.(lacht) #00:39:04-1#

B: Das ist, ähm (...) 60 Cent? Nicht so viel. #00:39:06-3#

I: Danke!  #00:39:09-1#

B: Ich gehen auch immer Tee trinken. Ich kaufe immer viele. Ist garkein Problem. #00:39:14-8#

I: Ich muss mir angewöhnen, mehr Tee zu trinken und weniger Kaffe. #00:39:23-2#

B: Ja #00:39:23-2#

I: Das wäre gesünder. #00:39:24-6#

B: Ja, Kaffee ist nicht so gut. Ich habe heute eine Tasse getrunken (lacht). #00:39:30-5#

I: ähm ich hab ein bisschen mehr. (...) Heute geht sogar, ich glaube es sind nur fünf. Nee warte, sieben. #00:39:45-4#

B: Sieben Tassen? Unglaublich. #00:39:56-2#

I: ähm, aber hat's noch mehr so Situationen gegeben, wie zum Beispiel das mit dem Arzt? Also, dass du (...) vor allem am Anfang (...)ähm nicht die Möglichkeit hattest, mit dem, mit dem du reden musstest, reden zu können? Also, dass hald zum Beispiel kein Dolmetscher dabei war oder irgend sowas in die Richtung? #00:40:22-5#

B: War echt schwierig. #00:40:24-7#

I: Ist das öfter passiert? #00:40:24-7#

B: Ja. Ein halbjahr ist das passiert. Dann, ich bin her gekommen, dann zum Glück meine, immer meine Betreuerin mit mir zum Arzt gegangen sind. Meine Betreuerin war dabei und eine Dolmetscher. #00:40:40-8#

I: ähm #00:40:40-8#

B: Hier war besser als Don Bosco. #00:40:43-0#

I: hm (bejahend) #00:40:43-0#

B: Auch Essen, Platz, (...) und (...) wir, wir hatten so, ich hatte mehr, ähm, Möglichkeiten als in Don Bosco. Don Bosco, die Leute waren sonst nett #00:41:00-9#

I: hm (bejahend) #00:41:00-9#

B: Die waren nett, aber (...) wir hatten wenige Möglichkeiten. #00:41:04-8#

I: Also das Haus ansich war nicht so gut, ähm, nicht so gut wie hier? #00:41:09-1#

B: Also doch, wie gesagt wir waren zwei Leute in einem Zimmer. #00:41:13-8#

I: hm (bejahend) #00:41:14-3#

B: Der war auch krank #00:41:17-3#

I: hm (bejahend) #00:41:17-3#

B: Der hat die ganze Nacht geschnarcht und so. #00:41:20-7#

I: Mh #00:41:22-5#

B: Ja, das (...) war schon richtig laut. #00:41:25-3#

I: Das ist ne (...) schlechte Kombination wenn er sehr laut schnarcht und du Migräne hast. #00:41:29-1#

B: hm (bejahend) Ich musste immer zwei Tablette nehmen bis ich so tief schlafen kann. #00:41:34-4#

I: hm (bejahend) #00:41:34-9#

B: Danach, ich hab viele Tabletten genommen, als ich auch im Iran war. Dann, ich hab immer Schmerzen gefühlt, so hier, diese Seite so, auf dem (...) so diese Seite, linke Seite. Dann, #00:41:49-7#

I: // Auf der linken Seite? #00:41:49-7#

B: Ja. Dann, ich war beim Arzt. Dann die haben gesagt auf deine, auf deine Leber gibt's zwei, dre Flecke #00:42:00-7#

I: hm (bejahend) #00:42:01-2#

B: Die sind wegen Tabletten. Du hast viele genommen. Die sind direkt auf die Leber gegangen. #00:42:05-9#

I: Mm (...) hm (bejahend) #00:42:05-9#

B: Und hat gesagt, musst du jetzt (...) ähm, mehr Wasser trinken, oder Tee trinken, und nicht Tabletten nehmen. Wenn du nimmst, es wird größer, und noch starke schmerzen. Hat gesagt, wenn so wird, müssen wir operieren. #00:42:22-4#

I: hm (bejahend) #00:42:22-4#

B: Dann ich habe Angst gehabt, scheiße! Dann ich hab weniger Tabletten genommen. ähm mehr Tee und Wasser getrunken, und ich war jetztes Jahr wieder beim Arzt, der hat gesagt, is besser geworden. #00:42:36-9#

I: Ist besser geworden? #00:42:37-8#

B: Ja, ein bisschen. Er sagt, vielleicht dauert zwei, drei Jahren, bis ganz weg sind, aber musst du bis dahin (...) wenn du Tabletten nimmst, nicht nehmen, aber ich will jeden Monat eine. Nicht mehr. Also ich habe in diesem Monat zwei Tabletten genommen #00:42:54-7#

I: ähm Schmerzmittel wieder oder? #00:42:56-9#

B: Ja, Schmerzmittel. Manchmal ich habe sehr starke Kopfschmerzen, ich kann nicht mit Wasser oder Tee wegmachen, sondern Tablette. (...) #00:43:06-3#

I: Aber du hast jetzt Tabletten, die dir direkt helfen. #00:43:14-4#

B: Das ist (...) mein Tablette diese früher habe ich bekommen. #00:43:14-9#

I: hm (bejahend) #00:43:14-9#

B: Genauso das. Aber, wie gesagt, ich nehmne jetzt jeden Monat zweimal, vielleich einmal, vielleicht nicht. #00:43:23-2#

I: Das ist sehr viel weniger als 20 in einer Woche. #00:43:28-6#

B: hm (bejahend). Aber wie gesagt, damals hatte ich keine Betreuerin, ich musste einfach meine Schmerzen wegmachen #00:43:32-3#

I: hm (bejahend) #00:43:33-2#

B: Ich habe nix, nicht gedacht: 'Wenn icht Tabletten nehme, ich mache meine Kopfschmerzen weg, sondern ich bekommen davon Leberschmerzen'. Ich habe nicht gewwusst. #00:43:43-9#

I: hm (bejahend) #00:43:43-9#

B: Es würde noch schlimmer sein, also tiefer. Aber zum Glück, nicht so passiert. #00:43:53-2#

I: Wie lang hat das gedauert, bis (...)ähm bis das festgestellt wurde, also bis du quasi so viele Medikamente genommen hast damals? #00:44:03-5#

B: Letztes Jahr. #00:44:05-5#

I: Letztes Jahr? #00:44:05-5#

B: hm (bejahend) #00:44:05-5#

I: Also (...) ähm  #00:44:11-5#

B: Zweieinhalb Jahre. #00:44:13-6#

I: (...) #00:44:14-7#

B: Im Iran habe ich oft Tablette genommen, jeden Tag dreimal. (...) Musste ich, ich war beim Arzt, der hast du gesagt: 'Du musst Tabletten nehmen'. Also ich hatte Kopfschmerzen #00:44:26-6#

I: hm (bejahend) #00:44:27-0#

B: Und damals habe (...) habe ich auch ab - und zu geweint, wegen Kopfschmerzen war so stark. Und (...) das war so eine lustige Geschichte: Ich hatte so stark Kopfschmerzen, ich war, ich glaube (...) 13-jähriger, sowas. #00:44:43-1#

I: hm (bejahend) #00:44:43-5#

B: Dann, wir waren beim Arzt *räuspert sich* (...) Also mein Mama hat gesagt, zu mir gesagt: 'Müssen wir zum Arzt gehen'. Dann wir sind zum Arzt gegangen, dann *räuspert sich* der hat ähm der hat, der Arzt hat zu mir gesagt: 'was ist los bei dir?' Ich sag: 'Ich hab so starke Kopfschmerzen, wenn so stark wird, ich muss weinen. Dann (...) ich muss weinen und (...) irgend, manchmal ich schlagen meinen Kopf an die Wand. Dann der sagt: 'vielleicht hast du im Kopf etwas'. Dann, die haben so, Foto gemacht, ich weiß nicht, was heißt auf Deutsch? #00:45:18-0#

I: ähm (...) nen ähm (...) Röntgen, eine Röntgenaufnahme. #00:45:26-1#

B: Genau. Ja, sowas. Dann, hat zu meine Mama gesagt: ' Nächste Woche kommen Sie bitte hier'. #00:45:38-4#

I: Machen wir's gleich auf Englisch, das ist X-Ray #00:45:36-0#

B: X-Ray ist? (lacht) #00:45:37-2#

I: Entschuldigung. (lacht) #00:45:39-7#

B: Dann, der Arzt hat zu meine Mama gesagt, nächste, nächste Woche kommen zu mir, dann er sagen, welche Problem dein SOhn hat. Bis dahin, wir bekommen ein Ergebnis. #00:45:51-4#

I: hm (bejahend) #00:45:51-4#

B: Dann, ich bin wieder mit meine Mama zum Arzt gegangen, also nächste Woche. #00:45:57-3#

I: hm (bejahend) #00:45:57-8#

B: Dann (...) der Arzt hat zu meine Mama gesagt: ' Deine Sohn hat (...) Tumor. Im Kopf'. Dann hat gesagt: 'Er kann sechs Monaten leben, dann (...) wird gestorben'. (...) Dann, hat meine Mama gesagt: 'Lass Ihn, was er machen will. Der ist nur sechs Monate am Leben'. Dann, meine Mama hat echt richtig viel geweint.
 #00:46:27-0#

I: Verständlich (...) #00:46:27-0#

B: Hm, dann (...) war, ich hab schon alles geholt, dann (...) meine Mama hat, und meine Bruder ist zu mir gekommen, meine Schwester haben zu mir gesagt: 'Was willst du machen? Du weißt alles, was, was wird bei dir?' #00:46:43-2#

I: hm (bejahend) #00:46:43-2#

B: Du weißt, du kannst nur sechs Monaten leben, dann nicht mehr. Sagt, mach, was du willst. Was willst du essen? Wo willst du gehen. Sagt, wir sind immer bei dir. Ich hab zu denen gesagt: 'Ich habe keine Angst vorm Sterben.' Habe gesagt: 'Ich bin schon vorbereitet. Ich habe garkeine Angst. Ist mir echt scheißegal, was Ihr, was Ihr (...)' #00:47:06-2#

I: Du warst damals 13? #00:47:08-6#

B: hm (bejahend)(lacht) #00:47:11-1#

I: Mhm (...) #00:47:14-1#

B: Dann, drei Monaten so gelaufen. Ich war auch dünner geworden, ich hatte schon Angst, aber ich wollte nicht zu meiner Mama, zu meinem Bruder sagen: 'Ich habe Angst.' Weinen undso #00:47:24-5#

I: Ja. #00:47:26-0#

B: Dann, die werden auch trauriger. Deswegen ich habe gesagt: 'Ich werde sowieso tot'. Damals habe ich auch nicht so gut verstanden, was bedeutet sterben. (lacht) #00:47:32-9#

I: hm (bejahend) #00:47:34-1#

B: Ich habe nicht so gut verstanden. Ich habe gedacht: 'Man schläft, dann wird fertig.' Dann (...) nach den drei Monaten (...) da, eine Iraner, also der wohnt in Deutschland, in München. #00:47:52-9#

I: hm (bejahend) #00:47:52-9#

B: Der ist Arzt #00:47:52-9#

I: hm (bejahend) #00:47:52-9#

B: Der ist Arzt, der wohnt in München. Ich glaube in Deutschland geboren ist, aber der ist Iraner. #00:47:58-9#

I: hm (bejahend) #00:47:59-2#

B: Dann, der hat, der hat eine Wohnung im Iran. Eine Villa. Wo wir wohnen. #00:48:07-8#

I: Da habt Ihr gewohnt, in seiner Villa? #00:48:10-7#

B: Ja. #00:48:10-7#

I: hm (bejahend) #00:48:10-7#

B: Dann, eine, ich glaube, war Freund von mein Bruder, hat gesagt: 'Ich habe eine, ähm, Chef. Der wohnt in Deutschland, der ist Arzt. Vielleicht, der versteht besser als diese Arzt zu dem deine Mama gegangen ist.' Dann wir haben mit Ihm ausgemacht, gesprochen, dann der hat gesagt: 'Ich bring dich nach Teheran, und ich schau genau an, was bei dir ist. Vielleicht ist es nicht Tumor.' Er sagt: 'Wenn du Tumor hast, dann du musst bis jetzt ganz dünn werden, du könntest nicht deinen Kopf bewegen. #00:48:44-3#

I: hm (bejahend) #00:48:44-3#

B: Bis jetzt müsstest du im Bett nur bleiben. Er sagt: '(unv.) Ich glaube nicht, dass Tumor.' #00:48:48-5#

I: hm (bejahend) #00:48:49-1#

B: Er hat so gesagt. Dann, er hat wieder mit diese Dinge fotografiert und (...) ein paar Test genacht (...) Und hat zu mir gesagt: 'Ich sage nächste Woche euch Bescheid.' Dann sagt: 'Wenn Tumor ist, dann müssen, also ich kann Dich nach Deutschland bringen.' Der sagt: 'Ich habe dort eine (...) Krankenhaus, so.' Der war echt nett. (...) Dann nach einer Woche, hat zu meinen Mama angerufen, hat gesagt: 'ähm, ich habe eine gute Nachricht für Sie.' Dann meine Mama hat gesagt: 'Kommen Sie bitte zu uns, wir haben für Sie gekocht.' Dann, der ist zu uns gekommen, der hat so mit Süßigkeit gebracht. Und, bei uns ist so, wenn jemand eine Kranke besucht, dann bringe dem hald Süßigkeit. Oder Obst. #00:49:47-1#

I: hm (bejahend) #00:49:47-7#

B: Es ist einfach, ähm (...) unsere (...) Kultur #00:49:53-6#

I: hm (bejahend) #00:49:54-1#

B: Dann, (...) der ist zu uns gekommen, hat geagt: 'Deine Sohn hat keine Tumor, sondern nur eine starke Migräne. Muss Tabletten nehmen, und auch öfter spazieren gehen. Und Wasser trinken. Hat ähm für, für mich so eine (...) Tabletten geschrieben, hat gesagt: 'Du musst dene nehmen.' #00:50:22-1#

I: ähm, verschrieben. #00:50:26-7#

B: Verschrieben. Hat geagt, die, die Tablette sind richtig teuer, aber (...) #00:50:28-6#

I: hm (bejahend) #00:50:28-6#

B: Hat gesagt: 'Ich kann euch ein bisschen helfen, aber nicht so viel.' Nur zum ersten Mal hat geholfen, also ein bisschen bezahlt. Danach, ich musste auch viel arbeiten. Ich hab echt richtig viel gearbeitet, bis ich selber meine Tabletten bezahle. Ich wollte nicht meine Mama und meinen Bruder bezahlt. Ich wollte selber bezahlen. #00:50:50-7#

I: hm (bejahend) #00:50:50-7#

B: Ich habe schon gesehen, meine Mama arbeitet fleißig, mein Bruder. Ich wollte nicht die ganze schwere Sachen einfach von meine Mama, Bruder lassen. Ich hab (...) Ich hab mich festgestellt: 'Ich muss arbeiten. Ich muss (...) mein Tabletten selber bezahlen. Nicht mein Mama und mein Bruder. Ich habe so versucht. Dann ich habe gearbeitet, und immer für meine Tablette habe ich selber bezahlt. Ich war auch oft im Krankenhaus, zwei Wochen, eine Woche. (...). Und (...) und danach (...) ich den Rücken nicht mehr (unv.) gehabt, in der Arbeit schwere Sachen tragen. #00:51:40-3#

I: Du hast da auf Baustellen gearbeitet, gell? #00:51:41-4#

B: Ja. Meine, also, ich glaube es heißt Tandome oder so? Es ist (...) eine Säule sowas? #00:51:53-9#

I: Wirbelsäule? #00:51:53-1#

B: Ja, so. Das hat, also halb ist zerrissen. #00:51:56-6#

I: Hmh? #00:51:58-2#

B: Also, kaputt gegangen. Der Arzt hat geagt: 'Wenn du (...) ein, zwei Monate arbeitest, dann wird die ganze Säule zerrissen. Der sagt: 'Halbes ist schon kaputt. Das ist (unv.), also, hat gesagt: 'Halbes ist schon kaputt.' Der hat so ein Foto uns gezeigt #00:52:17-1#

I: ähm, die Dinger zwischen den Wirbeln inder Wirbelsäule? #00:52:20-1#

B: Also, der gibts die zwei Dinge, die festhalten die Wirbelsäule. Diese Dinge, die #00:52:26-0#

I: Ah, ja! Die Muskeln da. #00:52:30-4#

B: Diese Muskel war, einer war, also diese Seite war richtig verletzt. #00:52:34-2#

I: hm (bejahend) #00:52:35-1#

B: Dann ich musste sechs Monaten zu Hause Pause machen, bis ich vierzehnjäriger geworden bin. #00:52:40-8#

I: Was musstest du auf der Baustelle bitte machen? #00:52:43-6#

B: Ich hab (...) viel gearbeitet. Zement getragen. Stein. Schwere Sachen #00:52:56-4#

I: (...) #00:52:56-4#

B: Also wie gesagt, ich musste meine Tabletten selber bezahlen, ich wollte nicht, mein Mama und mein Bruder bezahlen. #00:52:59-8#

I: hm (bejahend) #00:53:00-4#

B: Sie waren echt unter Druck. #00:53:02-6#

I: hm (bejahend) #00:53:02-9#

B: Mein Bruder hat richtig fleißig gearbeitet. Für unsere, also, Lebensmittel, damit wir selber kaufen können. Dass wir nicht zu den anderen gehen, Geld leihen oder so. Und (...) ja, dann (...) ich weiß nicht, ich bin letzten, ich bin hier gekommen. Und dieses Jahr habe ich wieder bekommen. Als ich in der Ausbildung war. (lacht) #00:53:29-0#

I: Das ist wieder passiert? #00:53:29-0#

B: Ja. (lacht) #00:53:32-3#

I: Mhh. #00:53:33-7#

B: Deswegen ich habe meine Ausbildung aufgehört. #00:53:36-1#

I: Mhh. #00:53:36-1#

B: Aber, jetzt ist wieder besser geworden. #00:53:38-9#

I: Aber, zwischendurch: Was war das eigentlich für eine Ausbildung? #00:53:42-3#

B: Verkäufer. #00:53:44-7#

I: Verkäufer. #00:53:44-7#

B: Ja, ist besser geworden, ähm, ich versuche ab nächstes Jahr wieder eine Ausbildung mache. Ich kann nicht für immer Schule machen. #00:53:52-1#

I: ähm, mit diesen Ruückenmuskeln: Du fängst jetzt morgen den Schwimmkurs an, ne? #00:54:00-8#

B: Ja. #00:54:01-6#

I: ähm für genau diese beiden Rückenmuskeln, also für diese Rückenmuskeln #00:54:07-7#

B: hm (bejahend) #00:54:07-7#

I: ähm (...) Rückenschwimmen. Des soll extrem gut dafür sein. #00:54:13-6#

B: Ich war beim Physio, also, Physiotherapie. #00:54:15-4#

I: hm (bejahend) #00:54:16-2#

B: DIe haben zu mir gesagt: 'Du hast viele Muskulatur im Rücken.' #00:54:19-6#

I: hm (bejahend) #00:54:20-1#

B: Der sagt: 'Du musst viel laufen gehen, bis die locker werden.' #00:54:25-5#

I: Bis sie locker werden? #00:54:25-5#

B: (lacht) #00:54:27-4#

I: ähm, das heißt, du sollt weniger Muskeln am Rücken haben oder wie? (lacht) #00:54:31-1#

B: (lacht) Ich war (...) zwei Monaten beim (...) ich war zwei Monaten beim Physiotherapie, also hast (...) #00:54:39-8#

I: Ja? #00:54:39-8#

B Jeden Woche zweimal. #00:54:41-1#

I: Ja. #00:54:41-9#

B: Die mein Rücken nach der Ausbildung ich musste hingehen. Jetzt die hat immer zu mir gesagt: Du musst mehr laufen gehen, bis deine Muskulatur locker werden. Und ich habe auch so gehört. Ich hab auch viel Laufen gegangen. (...) Und ich glaube ich muss nächsten Monat wieder wegen meinem Rücken hingehen und zeigen. (...) *räuspert sich* Und mein Arzt hat gemeint, (...) ich hab am ganzen Körper sehr starke Muskulatur. #00:55:15-1#

I: Mh ja. #00:55:16-2#

B: Am ganze Körper. #00:55:16-2#

I: Stimmt, hast du. also (...)(lacht) #00:55:16-2#

B:Ja, der hat zu mir gesagt: 'warum so ist?' Ich sage: 'Weil ich als zwölfjähriger in die Arbeit gegangen.' (lacht) #00:55:26-7#

I: (...) #00:55:26-7#

B:  Ich habe als zwölfjähriger wie ein Fitness gegangen. (lacht) #00:55:33-3#

I: Oww. #00:55:35-5#

B: Aber, es is schon vorbei. Schon lange. (...) Zum Glück. #00:55:40-6#

I: Zum Glück. #00:55:40-6#

B: Ja. Jetzt (...) mache ich dieses Jahr Schule. Vielleicht nächstes Jahr ? Danach eine Ausbildung. #00:55:48-8#

I: Nächstes Jahr machen wir'n Quali. #00:55:53-1#

B: Hoffe ich. (...) Ich versuche es. #00:55:57-5#

I: Kriegen wir hin! #00:56:02-1#

B: Ich hoffe. Danke. #00:56:05-8#

I: Ja, bitte! ähm machen wir mal wieder hier ein bisschen weiter. ähm (...) hast du (...) hier, seit du in Deutschland angekommen bist, ähm (...) Rassismus erlebt, in irgeneiner Art und Weise? #00:56:22-3#

B: hm (bejahend), zwei, dreimal. #00:56:22-2#

I: Inwiefern? Also, einmal hast ja gemeint, das mit dem Fahrkartenkontrolleur zum Beispiel.  #00:56:29-4#

B: Ja. #00:56:31-2#

I: hm (bejahend) #00:56:32-2#

B: Und einmal ich war so im Geschäft, Norma.  #00:56:36-7#

I: hm (bejahend) #00:56:36-7#

B: Ich war an der Kasse. Und (...) wir waren viele LEute, und der Mann war nicht so ualt und auch nicht so jung, also so zwischen 50, 55 #00:56:48-6#

I: hm (bejahend) #00:56:48-6#

B: Der hat einfach, angefangen mit dem Sprechen so laut, hat gesagt: 'Die (...) Flüchtlingen sind nach Deutschland gekommen, die (...) können nicht arbei(...) in der Arbeit. Die kriegen von der Stadt Geld, und Sie kriegen unsere Geld, wir geben gerade der Stadt Geld an die (...) die sind faul, und dann aber (...) aber damals war ich in der Ausbildung, das war (...) schon lange her #00:57:24-5#

I: hm (bejahend) #00:57:25-0#

B: Ich hab zu Ihm gesagt, dass das nicht stimmt. Der hat gesagt: 'Wer bist du?' Ich sage, ja, dass das nicht stimmt. Ich mache gerade Ausbildung auf dem meine, also, auf dem Vertrag steht 850€ aber ich kriege 150€. Ich habe zu Ihm gesagt, er sagt: 'Ja, aber du hast drei Jahre das Geld gekriegt. Du gehst zur Schule, du sollst arbeiten.' Sage ich: 'Ja, sie haben recht, aber (...) aber wir lernen gerade die Sprache, ähm sag wir werden später in die Arbeit gehen, Sie sind fast in Rente, Sie bekommen später Ihre Geld wieder zurück.' Und zwei, drei Leute waren, die waren auch Ausländer, zu mir gesagt: 'Einfach halt die Klappe, geh weg von Ihr, warum redest du mit dem, das bringt nix.' Dann, ich hab (...) Ich hab gesehen, die zwei haben recht. Warum ich muss mit dem reden? Das bringt nix. Dann ich bin einfach (...) von Geschäft rausgekommen, nach Hause, ich war (...) so ein Tag, zwei Tage, war ich so ein bisschen traurig. Danach ist, habe ich vergessen alles. Ja. #00:58:34-6#

I: Ja. #00:58:35-6#

B: Ich finde, vielleicht der hat recht? Weiß ich nicht. #00:58:36-4#

I: Nee! #00:58:38-5#

B: (lacht) #00:58:39-8#

I: Das ist tatsächlich nen (...) Trugschluss, also, ähm, er sieht, er kriegt wenig Geld, und er sieht auch: Du kriegst Geld vom Staat. #00:58:52-9#

B: hm (bejahend) #00:58:53-8#

I: Das ist so das, dass er mitbekommt. ähm(...) Das ist allerdings nen relativ kleiner Pool, also (...) Dem Staat ansich steht so viel mehr Geld zur Verfügung. #00:59:08-0#

B: hm (bejahend) #00:59:08-9#

I: ähm, das ist quasi (...) Die zwei Parteien, die eh kein Geld haben, gegeneinander auszuspielen, ähm und währenddessen können andere, die über deutlich mehr Geld verfügen, tatsächlich machen, was Sie wollen. ähm(...) So (...) Das würde jetz ein bisschen lang dauern, aber (...) grundsätzlich (...) Der Mann sollte mehr Geld verdienen. Und dir steht das Geld, das du bekommst, und vor allem deine Ausbildung und deine Schulbildung sthet dir voll und ganz zu. Das ist dein gutes Recht. UNd, also ganz ehrlich: ähm(...) Deutschland braucht dich und andere Leute, die ähm die zuziehen, weil: Du siehst den alten Mann? Der ist momentan 55, in 15, 20 Jahren, der kriegt zwar seine Rente noch dazu, aber es ist nicht sonderlich viel Geld, und irgendwann kann er nichtmehr für sich sorgen. Dann ist er zum Beispiel in nem Pflegehein, und in nem Pflegeheim (...) will keiner arbeiten, weil's momentan genau wie bei Ihm so aussieht: ähm Man kriegt kein Geld und ma hat zum Beispiel beschissene Arbeitszeiten. ähm(...) Und ähm, also vielleicht is es dir nen bisschen aufgefallen: Deutschland wird (...) älter. #01:01:04-3#

B: hm (bejahend) #01:01:04-9#

I: Also es werden weniger junge Menschen geboren, als alte sterben. #01:01:12-1#

B: hm (bejahend) #01:01:12-9#

I: Das heißt, im Laufe der Zeit wird Deutschland immer älter werden (...) weswegen das Land darauf angewiesen ist tatsächlich, dass Leute rein kommen, um zu helfen. Und wenn du jetz zum Beispiel hier ne Ausbildung machst, das ist nen absoluter Glücksfall. #01:01:34-0#

B: hm (bejahend) #01:01:34-0#

I: Also ähm also(...) Lass dir gesagt sein: Der Mann absolut nicht recht. #01:01:41-7#

B: Okay (lacht) #01:01:41-7#

I: (lacht) Ganz ehrlich, wir können froh sein, dass du hier bist. #01:01:46-4#

B: Okay. #01:01:48-2#

I: (lacht): Jetz lassen des Interview mal sein und machen noch nen bisschen was für dich in der Schule wenn du noch Lust hast? #01:01:59-2#

B: (lacht)(schüttelt Kopf) #01:02:01-6#

I: (lacht) Ich pausier das jetz auf jeden Fall mal. (...)(...)(...)(...)(...) Ähm, nochmal: Du, also, du (...) wenn du verpasst manchmal Termine, die für dich ausgemacht worden sind #01:02:24-4#

B: (hm (bejahend) #01:02:24-4#

I: ähm weil dir nicht gesagt worden ist, dass die Termine überhaupt statt finden? #01:02:27-9#

B: Ja, die, manchmal die werden vergessen. Beispiel: heute. Sie (...) sie hat zu mir gesagt: 'Hast du heute, warst du heute beim Termin?' Ich sage: 'Welchem Termin?' 'Beim Stadt!' Sage ich so: 'Wann?' Sage: 'Heute um neun Uhr.' Ich sage: 'Nee. Ich habe nicht gewusst. Niemand hat zu mir gesagt.' 'Oooh. (lacht) Sorry, wir haben vergessen!' #01:02:46-4#

I: (lacht) #01:02:47-1#

B: Oder wenn ich Beispiel irgendwo draußen bin, dann die rufen mich: 'Mahdi, wo bist du?' 'Huh?' 'Du hast gerade einen Termin!' 'Ja okay, ich komme gleich!' #01:02:55-7#

I: (lacht) #01:02:55-7#

B: Die werden vergessen, oder die sagen zu spät. #01:02:58-5#

I: ähm aber sind das dann die Betreuer, die's dir zu spät sagen, oder? #01:03:06-5#

B: Also, man, hier gibt's (...) 21 Leute. #01:03:11-3#

I: hm (bejahend) Also hier im Haus. #01:03:11-8#

B: Ja. Meine ich den erste, ähm, zweite Stock und dritte Stock #01:03:16-1#

I: hm (bejahend) #01:03:16-1#

B: Also ich meine nur für uns. Dann (...) die bringen manchmal durcheinander. #01:03:22-4#

I: hm (bejahend) #01:03:23-8#

B: Beispiel Leute müssen, verliert die Sachen, (...) Termin ausmachen, und Ärztin anrufen, Chef anrufen, meißte machen hier Ausbildung. Deswegen die rufen die Chef an: 'Heute der ist krank, oder der nicht kommen heute. Der kommt. Der ist gekommen. Der war in der Arbeit oder so, so, so. WIrd viele Fragen gestellt. #01:03:44-5#

I: hm (bejahend) #01:03:44-5#

B: Oder, Beispiel: Wenn die von der Ausbildung reinkommen zu Hause, der sagt: 'Mein Chef hatte keine gute Laune oder ich habe meine Kollegen gestritten, oder meine Kollegin, Beispiel, hat mich geschimpft. So, gibts viele Gründe. #01:04:01-7#

I: hm (bejahend) #01:04:01-7#

B: Deswegen vielleicht Sie haben mehr Stress. Aber (...) Die haben öfter schon vergessen, mir Termine sagen. (lacht)  #01:04:10-7#

I: Hmm (lacht) #01:04:10-7#

B: Das Sie haben Beispiel zwei Tage später gesagt. Haben. #01:04:15-1#

I: Ups. #01:04:16-2#

B: Nach die zwei Tagen sie sagt: 'Hast du gestern, vorgestern einen Termin.' Ich habe nicht gewusst. 'Ja okay, wir machen einen neuen Termin aus, aber es dauert wieder zwei Wochen bis du diese Termin bekommst. #01:04:26-7#

I: Ahhhh #01:04:29-1#

B: Deswegen ich bin nicht sicher, weil das öfter passiert. #01:04:35-2#

I: Okay (...) Ja gut, aber das passiert auch mal. #01:04:39-3#

B: hm (bejahend) #01:04:39-3#

I: Danke!\newpage
%\sectio{Interview 4}

 #00:00:02-2#

I: ..wie des funktioniert. Ah, ja, jetz läufts. Erstmal hi. #00:00:08-0#

B: Hi. #00:00:10-4#

I: (lacht) ähm also ich würd jetz einfach mal so meine Fragen hier runter gehen #00:00:14-4#

B: Ja. Wenn du magst. #00:00:17-4#

I: ähm (...) Wie alt bist du?  #00:00:20-9#

B: ähm ich werde jetzt 21. Ich bin 20 Jahre alt. #00:00:21-4#

I: Du bist 20? #00:00:22-3#

B: Ja. (lacht) #00:00:23-4#

I: Du siehst älter aus. #00:00:23-3#

B: Ich sehe sehr sehr alt aus. (lacht) #00:00:24-8#

I: (lacht) #00:00:27-0#

B: Hatte so viele schlechte Sache da. #00:00:29-9#

I: Echt? #00:00:29-9#

B: Ja, genau. #00:00:31-8#

I: Fuck. ähm, okay - du bist ganz offensichtlich männlich. ähm (...) Aus welchem, wo kommst du ursprünglich her? #00:00:43-6#

B: Ich komme aus Afghanistan. #00:00:45-1#

I: Du kommst aus Afghanistan. #00:00:46-0#

B: Genau. #00:00:47-2#

I: Okay. Und wann bist du nach Deutschland gekommen? #00:00:52-2#

B: Ich bin Anfang 2015 nach Deutschland gekommen. #00:00:54-7#

I: Anfang 2015? #00:00:57-6#

B: Genau. #00:00:59-0#

I: ähm, (...) bis ähm wie bist du nach Deutschland gekommen? #00:01:05-8#

B: Also, ich bin über die Türkei (...) Türkei nach Deutschland bin ich in eine LKW gekommen. #00:01:12-1#

I: Okay. #00:01:13-9#

B: Genau. #00:01:13-9#

I: ähm also (...) quasi über die Landroute über Ungarn dann? #00:01:19-5#

B: ähm das weiß ich nicht ähm, irgendwann war ich in (...) Regensburg #00:01:23-5#

I: Echt? #00:01:22-9#

B: Ja.  #00:01:25-2#

I: Also du bist quasi in der Türkei in den LKW gestiegen #00:01:27-9#

B: Ja. #00:01:28-8#

I: Und bist in Regensburg dann wieder ausgestiegen. #00:01:29-8#

B: Wieder ausgestiegen, ja. #00:01:32-0#

I: Okay #00:01:32-8#

B: Ja. #00:01:34-2#

I: ähm was isn dann passiert? #00:01:39-3#

B: Dann ähm (...) ich war ganz verletzt, meine Füße und sowas, ich konnte garnicht #00:01:41-9#

I: //was? #00:01:41-9#

B: Ja. So, keine Ahnung, mein Fuß war so dick, weil ganze Blut und in LKW war wenig Platz #00:01:47-6#

I: hm (bejahend) #00:01:48-1#

B: Und (...) ich bin zu Fuß hald Stadt gekommen, zum, wo die (...) Pakistanische Geschäft ist. Bin dann gestanden, und ich hab so Leute gesagt wo hald Polizei oder so ist #00:02:02-4#

I: hm (bejahend) #00:02:02-4#

B: Und dann Polizei ist zu mir gekommen und die haben mich geholt #00:02:04-6#

I: hm (bejahend) #00:02:04-6#

B: Und danach die haben zu mich zum Heim geschickt #00:02:08-8#

I: Bist du dann noch ins Krankenhaus gekommen oder so? #00:02:10-7#

B: Genau, dann die haben mich genommen zum Krankenhaus, die haben Blut genommen und alles geschaut, dann musste ich länger warten, bis halt meine Füße besser wird undso. #00:02:21-0#

I: // Was ist da passiert? (...) #00:02:20-5#

B: Habe ungefähr drei Monate gewartet.  #00:02:49-5#

I: Was ist mit deinen Füßen passiert, wenn ich fragen darf? #00:02:24-4#

B: Ja, ich bin so gesessen in LKW, und die ganze Blut war hald hier, so, das war so dick. Dann musste ich ganze Nacht, meine Betreuer und alle, Massagen hald und dass die Blut oben und unter geht. #00:02:37-9#

I: Wie lang bist du so gesessen? #00:02:40-8#

B: Da war so zwei Monate, ein Monat konnte ich nicht gehen, musste ich mit Dings gehen dann. #00:02:44-6#

I: Au. #00:02:44-6#

B: Ja.  #00:02:49-6#

I: Okay (...) ähm konnt (...) konntest du schon Deutsch als du angekommen bist? Also beziehungsweise, wie hat das funktioniert? #00:02:59-7#

B: // Ja ich konnte gar (...) ne ich konnt egarkein Deutsch. Nicht Englisch, nix. Ich hab nur irgendwie mit Hände geredet irgendwie #00:03:06-9#

I: // Okay. (...) hm (bejahend) #00:03:06-7#

B: Ja. #00:03:09-0#

I: Es hat irgendwie funktioniert. #00:03:10-2#

B: Irgendwie hat's funktioniert, es war schwer aber (...) konnte nix machen (lacht) #00:03:14-9#

I: Ja. #00:03:15-8#

B: Ja. #00:03:17-2#

I: ähm bist du in Afghanistan zur Schule gegangen? #00:03:22-6#

B: Ich bin zwei Jahre gegangen, ja. Und ein jahr so Koranschule hald. #00:03:25-9#

I: ähm an ner Koranschule. #00:03:27-9#

B: Ja, ein Jahr. #00:03:28-8#

I: Okay #00:03:28-8#

B: Und zwei Jahre Normale Schule #00:03:30-0#

I: Okay #00:03:30-0#

B: Ja. #00:03:30-0#

I: ähm (...) das (...) wie unterschiedlich sind die Schulen dort zu jetzt ähm, zum Beispiel der Schule in der du jetzt bist? Is des #00:03:46-6#

B: Was meinst du? #00:03:46-6#

I: Ist des zum Beispiel ein anderes System dass du was lernst, oder (...) ähm unterscheidet sich das irgendwie? #00:03:54-7#

B: Achso unterscheidet das was du in normale Schule hald, so zum BEispiel Sozialkunde machst und Mathe und sowas.  #00:03:59-5#

I: hm (bejahend) #00:03:59-5#

B: Und in Koranschule kann nur, liest du das weil dass a, auf arabisch ist. #00:04:04-1#

I: hm (bejahend) #00:04:04-9#

B: Und dann liest du hald nur auf arab, ähm, arabisch. #00:04:05-9#

I: M~hm. #00:04:07-6#

B: Genau, deswegen. Das ist der große Unterschied. Sonst is (...) nicht unter. #00:04:11-9#

I: Okay. #00:04:11-9#

B: Ja. #00:04:11-9#

I: ähm, hattest du, ähm, am Anfang dann irgendwelche Probleme, dass du dich hier zurecht findest? #00:04:21-9#

B: Oh, in Deutschland? #00:04:23-3#

I: Ja. #00:04:23-3#

B: Am Anfang schon Probleme. Keine SPrache, keine Leute, nix. #00:04:28-9#

I: hm (bejahend) #00:04:28-9#

B: Dann, ich war verletzt. #00:04:29-6#

I: Ja. #00:04:30-6#

B: Ja. #00:04:31-5#

I: ähm, hattest, also du hast dann hier vermutlich nen Deutschkurs besucht oder? #00:04:40-4#

B: Ja. Ich bin so, ich hab Sprachkurs besucht vier Monate am Anfang, und danach so, die haben mich zur Mittelschule geschickt. #00:04:47-9#

I: Okay. #00:04:49-2#

B: Genau. #00:04:49-2#

I: ähm (...) hat's dir Probleme bereitet, dass, also in ähm (...) Regensburg wird ja relativ oft auch, ähm oft auch bayrisch geredet. Also ich glaub (...) bei mir hört man's auch nen bisschen raus (...) ähm, hat dir das Probleme bereitet? So dass Bayrisch ja doch so nen bisschen nen anderes Sprachmodell ist wie jetz Hochdeutsch? #00:05:12-1#

B: Schon, schon, manchmal haben wir in dem, im Deutschkurs oder so wenn wir mit dem Lehrer geredet haben, haben wir schon ein bisschen Deutsch gehört #00:05:21-5#

I: hm (bejahend) #00:05:21-5#

B: Undn ich hab auch ein Tag zum Beispiel 'Habadere' gehört, das war mein erste Wort das ich gelernt hab. #00:05:25-3#

I: Habadere? (lacht) #00:05:27-8#

B: Habadere, Genau. Aber ja, sowas. Aber ein bisschen hatten wir, ein bisschen Probleme mit (...) schnell reden und (...) ähm mit bayrisch hald. #00:05:38-4#

I: hm (bejahend) (...) aber jetz funktioniert's oder? #00:05:38-4#

B: Jetz funktioniert's schon, ja, weil ich auch in der Arbeit bin, die reden auch bayrisch. #00:05:41-0#

I: hm (bejahend) #00:05:41-0#

B: Und in die Schule undso es ist kein Problem. #00:05:44-6#

I: ähm was genau für ne Arbeit machstn du jetz? #00:05:46-7#

B: Also ich mache Ausbildung als Fahrzeuglackierer. #00:05:49-5#

I: hm (bejahend) #00:05:49-5#

B: Ja. #00:05:51-3#

I: Cool. ähm, wo hier? #00:05:53-5#

B: ähm, bei (...) hinter bei Handelskammer, bei (unv.)-Straße in der hald. #00:06:00-9#

I: (...) okay, ich weiß ned wo das is. #00:06:00-7#

B: Ja, das is (...) das is der Bus Nummer 10 Richtung Zuckerfabrik und dann die letzte Bushalte Bushaltestelle hald steige ich aus. #00:06:09-4#

I: hm (bejahend) (...) Ah, da hinten. #00:06:10-8#

B: Da hinten, ja. #00:06:11-9#

I: Ja. (...) ähm, hast du (...) jetzt momentan noch Kontakt nach Afghanistan? #00:06:24-5#

B: Ich hab noch mit meine Onkel Kontakt. Sonst habe ich keinen Kontakt #00:06:29-4#

I: Okay #00:06:30-6#

B: Und (...) Freunde hald, sonst #00:06:32-1#

I: hm (bejahend) #00:06:32-1#

B: Nicht viele. #00:06:32-5#

I: ähm und mit wem hast du hier in Regensburg jetz so zu tun? #00:06:36-2#

B: Mit meinem Kumpel, Khsafi, Imdad, und mit denen dass ich mit dene Kicker spiele. Die haben, mit die haben Kontakt. ICh spielen Cricket und (...) ja. #00:06:48-7#

I: Ja, cool #00:06:51-2#

B: Ja. (lacht) #00:06:52-9#

I: ähm (...) du, wie weit ist denn dein Asylverfahren schon? #00:07:01-7#

B: Asylverfahren (...) eigentlich nichts, ich hab Duldung. #00:07:06-6#

I: Hm? Du (...) du hast ne Duldung, okay. #00:07:08-6#

B: Ja, ich haben eine Duldung. #00:07:11-2#

I: Gut, ähm (...) und ändert sich da dann noch irgendwas oder (...)? #00:07:16-5#

B: Ja, ich habe eine Antrag geschrieben von (...) 25a) #00:07:21-9#

I: hm (bejahend) #00:07:22-7#

B: und das läuft noch. Ich weiß nicht mehr wie das geht. #00:07:26-7#

I: Gut. #00:07:26-7#

B: Ja. #00:07:28-4#

I: (...) Ja, mal schaun was da dann raus kommt. #00:07:29-7#

B: Genau, also was da raus kommt, ja. #00:07:31-3#

I: ähm (...) Ihr habt jetz hier in der WG auch immer noch einen Betreuer oder? #00:07:35-7#

B: Ja. #00:07:38-3#

I: Okay. Kommt Ihr mit dem zurecht?  #00:07:48-4#

B: Schon, schon, super. Sie ist nett  #00:07:41-9#

I: //passt. #00:07:41-9#

B: was wir brauchen, SIe ist immer für uns da. #00:07:45-5#

I: (...) Das ist cool #00:07:46-8#

B: Genau. Alles gut. #00:07:49-0#

I: ähm (...) welche Wohnungen hastn du dann seit du hier in Regensburg angekommen bist (...) welche Wohnungen hastn du dann ghabt? Ihr warst erst in ner (...) #00:08:01-1#

B: Ja, erste war ich in (...) ähm (...) O M 1? #00:08:07-8#

I: hm (bejahend) #00:08:08-9#

B: Ja, glaube ich (...) Also erste war ich da. Dann zweite war ich bei deutsche Kinder. #00:08:14-4#

I: hm (bejahend) #00:08:14-9#

B: ach, ähm sieben war deutsche Kinder, sechs und ein war ich dabei. In (unv.) eins, und danach bin ich hierher gezogen. #00:08:23-9#

I: ähm, hat (...) da hab dich vorhin so'n bisschen gehört, dass es da n bisschen Konflikte gegeben hat teilweise. Was is denn da passiert? #00:08:33-6#

B: Ja (...) Eigentlich, bei mir war gut, aber bei Ihm weiß ich nicht #00:08:37-2#

I: Achso #00:08:37-2#

B: Ja, genau, wir waren verschiedene Gruppe. #00:08:38-9#

I: hm (bejahend) #00:08:39-5#

B: Ich wareich war meine in (unv.)mit Deutsche, war alles gut. #00:08:41-0#

I: hm (bejahend) #00:08:41-0#

B: Bin gelaufen. Ich hab mit denen viel geredet, viel gemacht (...)  #00:08:47-9#

I: hm (bejahend) #00:08:47-9#

B: Damit ich hald Deutsch lerne. #00:08:48-3#

I: hm (bejahend) #00:08:48-9#

B: Und jeder Person musste hald pro Woche einmal kochen und einkaufen undso. Das war super. Gut gemacht. #00:08:57-9#

I: Also, ähm ihr hattet da so'n richtiges WG-Leben? So, jeder is mal zum Einkaufen gegangen, und (...) wie viele Leute waren des dann in der WG? #00:09:08-1#

B: Waren wir zu siebt. Sieben Personen waren wir da. #00:09:14-3#

I: Das heißt, jeder war einen Tag für Einkaufen und Kochen zuständig. #00:09:15-5#

B: Genau, ich war zum Beispiel Montag, andere war Dienstag, andere Mittwoch und so weiter.(...) #00:09:22-3#

I: Das (...) hört sich tatsächlich nach Spaß an. #00:09:23-0#

B: Schon, schon. Es war gut. (lacht) #00:09:24-2#

I: (lacht) #00:09:26-5#

B: Hatte Spaß gemacht. #00:09:27-4#

I: Cool. (...) ähm (...) Hast du dann im Jugendcafé auch mal gekocht ode so? #00:09:38-5#

B: Im Jugendcafé hab ich noch nichtgekocht, weil ich nicht kochen kann. (lacht) #00:09:39-3#

I: Nicht? #00:09:41-4#

B: Nein. #00:09:42-8#

I: Hmm (...) Okay, ich schau mal hier weiter. ähm (...) Gut. (...) Hast du (...) hier in (...) also, seit du hier angekommen bist, irgendeine Art von Rassismus erlebt? #00:10:01-3#

B: Ich habe nicht erlebt. #00:10:03-2#

I: Echt? #00:10:03-2#

B: Ja. #00:10:05-6#

I: Cool. (...) Freut mich für meine Stadt (lacht) (...) Also überhaupt nicht, nie? #00:10:13-7#

B: Nein. DIe, viele Leute haben so (...) Demo gemacht undso. Und mich interessiert einfach das nicht. Ich bin wie ich bin und andere Leute mir (...) nur Spaß machen. Demo und (...) keine Ahnung.  #00:10:24-8#

I: Gesunde Einstellung. #00:10:26-1#

B: Ja. Und ich habe auch keine Zeit, dass ich mit denen in Demo gehe undso, so #00:10:30-8#

I: hm (bejahend) #00:10:30-8#

B: Ja, die machen, was die will. (...)  #00:10:34-5#

I: Das heißt, ähm (...) dir gefllt s ganz gut so wie's jetzt grad läuft oder? #00:10:39-0#

B: Ja, in Regensburg gefällts super. Und Leute nett, alles perfekt. Arbeit hab ich, Schule hab ich. Was will ich noch? (...)Ja. #00:10:49-9#

I: ähm (...) was machst du, wenn du mit der Ausbildung fertig bist? #00:10:54-6#

B: ähm, ich bin bei Firma. Wenn ich übernommen bin, dann arbeite ich da zwei, drei Jahre. Und danach (...) weiß ich nicht.  #00:11:03-6#

I: hm (bejahend) #00:11:03-6#

B: Wenn ich gut bin, vielleicht mache ich Meister. #00:11:05-8#

I: hm (bejahend) #00:11:05-8#

B: Oder (...) dann arbeite ich hald da. (lacht) (...) Aber weiß ich nicht (...) sicher. (...) #00:11:14-6#

I: ähm (...) hattest du ähm (...) also ähm, ich jetz mal wieder so n bisschen zurück an Anfang, also nachdem du hier angekommen bist. ähm da (...) bist du zuerst von der Polizei aufgenommen worden und ins Krankenhaus gebracht, dass sich die um deine Beine kümmern. #00:11:42-6#

B: Ja. #00:11:42-6#

I: Und das hat auch alles reibungslos funktioniert? Also (...) #00:11:47-4#

B: Keine Ahnung. ICh wusste damals nicht irgendwas ist funktioniert. #00:11:49-2#

I: hm (bejahend) (...) Okay. #00:11:52-1#

B: Ja ich bin Krankenhaus die haben mich gebracht undso (...) und dann (unv.) hab ich hald vergessen. Schon lange. #00:11:58-3#

I: Ja. ähm und im Nachhinein hast du auch nix mitbekommen, dass (...) dass irgendwie irgendwas aus der Reihe gelaufen wäre? #00:12:10-2#

B: Nein. Dann war ich so Große Heim, da bei Dings. #00:12:14-4#

I: hm (bejahend) #00:12:14-4#

B: Und, die haben mich Don Bosco geschickt. #00:12:18-2#

I: hm (bejahend) #00:12:18-2#

B: Dann war ich eine Woche da. #00:12:18-4#

I: hm (bejahend) #00:12:18-4#

B: Danach war ich in (unv.)heim. War ich in der (unv.) #00:12:21-2#

I: okay. #00:12:23-2#

B: Sonst weiß ich nicht. (lacht) #00:12:23-9#

I: Hmm. (...) Warte (...) Also, ähm worum's grundsätzlich in meiner Arbeit geht ist so (...) ähm, diese Momente ähm, wenn du (...) was machen musst, und dann hast du dir später gedacht, so: 'Ah! Oh, so hätte das funktionieren sollen!' Hast du das auch zum BEispiel in der Arbeit nie gehabt? Wie funktioniert denn des da, so die Zusammenarbeit mit deinen Mitarbeiter. Also, mit den anderen bei dir in der Firma. Wie funktioniert denn des? Also so, von dem Moment an wo du da reingekommen bist. #00:13:13-6#

B: An die, am Anfang war dich (...) Praktikant. #00:13:17-5#

I: hm (bejahend) #00:13:17-5#

B: Ich hab Praktikum gemacht. Und, ja, ich wusste auch nicht, wie du gesagt, (...) wie man zum Beispiel (...) eine Flüchtlinge eine Ausbildungsstelle bekommt oder so. #00:13:27-6#

I: hm (bejahend) #00:13:27-6#

B: Dann aheb ich Praktikum gemacht, und (...) #00:13:29-7#

I: ähm wie hast du denn die (...) Ausbilungsstelle, die Stelle dann bekommen, des Praktikum? #00:13:34-3#

B: Ja, ich war im B1 #00:13:37-6#

I: hm (bejahend) #00:13:38-1#

B: ähm (...) (unv.) hald, in die Schule. #00:13:41-7#

I: hm (bejahend) #00:13:42-0#

B: Und ich hab mit meinem Lehrer zusammen gesucht, wo hald Lackiererei, er sucht so (...)  #00:13:46-8#

I: Ah. #00:13:48-2#

B: Wo Lackiererei ist oder so #00:13:48-9#

I: hm (bejahend) #00:13:48-9#

B: Im Computer, dann haben wir Lackiererei gesucht und wir hald gefunden, telefonniert, und wir kommen so hin #00:13:55-6#

I: hm (bejahend) #00:13:55-6#

B: Mein Chef geredet und dann sind wir hingegangen wegen Praktikum, ich und mein Betreuerin #00:14:01-2#

I: hm (bejahend) #00:14:01-6#

B: Und dann haben wir gefragt, ob ich Praktikum bekomme, aber dass ich am, pro Woche zweimal kommen kann, weil drei Tage habe ich (unv.) und Schule, und zwei Tage kann ich zur Arbeit kommen. #00:14:14-3#

I: hm (bejahend) #00:14:14-8#

B: Mein Chef hat gesagt: 'Ja.' Und dann haben wir zwei Monate Vertrag geschrieben, pro Woche zweimal, und dann habe ich ein Praktikum gemacht und danach haben die gesagt: 'Ja, du kannst schon ab September hald hier Ausbildung machen.' #00:14:26-1#

I: Cool. #00:14:27-1#

B: Ja. #00:14:28-7#

I: (...)ähm wie hat's dann ausgesehen, nachdem du im BEtrieb drin warst? ähm, des (...) du bist quasi reingekommen und des war (...) ma hat dir dann direkt gezeigt, was du machen kannst, und (...) oder wie hat des da dann ausgesehen? #00:14:47-7#

B: Ja, ich bin reingekommen. #00:14:49-8#

I: hm (bejahend) #00:14:49-8#

B: Und ich hatte hald so (...) normale Kleidung an, weiß. #00:14:53-4#

I: hm (bejahend) #00:14:53-4#

B: Wie Doktor hald. #00:14:55-1#

I: hm (bejahend) #00:14:55-6#

B: Und dann die haben zu mir gesagt: 'zuerst tauschst du deine Kleidung.' #00:14:59-6#

I: hm (bejahend) #00:14:59-6#

B: 'Wir haben in Werkstatt anderes Kleidung. Ziehst du die da an.' Und dann habe ich auch gecheckt: 'Oh, so funktioniert!' #00:15:05-5#

I: hm (bejahend) #00:15:05-9#

B: Und dann die haben gesagt hald, da ist Kleber, da ist das, weil ich nicht wusste, wie das heißt und wie das funktioniert. #00:15:13-0#

I: hm (bejahend) #00:15:13-5#

B: Und dann hab (...) hab verstanden, was die wollen in der Arbeit. #00:15:17-2#

I: Also die haben die danna uch quasi (...) direkt die, ähm, deutschen Worte für die Sachen, die du brauchst, dann beigebracht. #00:15:24-9#

B: Genau, die haben beigebracht. #00:15:24-9#

I: Und die, die warn dann quasi alle auch direkt freundlich oder? #00:15:30-5#

B: Die waren am Anfang schon freundlich, und immer noch jetz. #00:15:33-9#

I: Cool #00:15:35-1#

B: Ja. #00:15:36-1#

I: ähm (...) Es freut mich irgendwie, dass das alles so reibungslos funktioniert. (...) hat's ähm (...) aber es (...) hat dann nie irgendwelche Probleme gegeben oder? #00:15:49-5#

B: Am Anfang war schon Probleme, aber ich hab das gemerkt, dass die (...) manche Leute so vorsichtig mit mir war. #00:15:55-2#

I: Inwiefern? #00:15:58-4#

B: Naja, so (...) nicht geredet undso. Weil die nicht wusste, wie ich bin und (...) #00:16:02-8#

I: hm (bejahend) #00:16:04-0#

B: Wie du reingekommen bist undso. #00:16:06-1#

I: Also die warn am Anfang einfach noch skeptisch oder (...)? #00:16:10-5#

B: Ja, so, hald, so, vorsichtig, nicht geredet und so einfach. #00:16:13-8#

I: hm (bejahend) #00:16:14-8#

B: Angst gehabt? Ich weiß nicht, so (...) #00:16:16-2#

I: ähm (...) bei, (...) ähm, bei uns Deutschen, des dauert des manchmal bisschen bis wir auftauen. #00:16:23-3#

B: Genau, davor haben wir in die Schule auch gelernt, dass wenn bei deutsche arbeitet oder irgendwas macht, dann die brauchen länger, dass die dich kennt #00:16:33-0#

I: hm (bejahend) #00:16:33-0#

B: und dir vertraut undso. #00:16:34-0#

I: hm (bejahend) #00:16:34-4#

B: Und das war genau so, dass ich dann Praktikum gemacht habe, und danach habe ich langsam Leute kennen gelernt und geredet undso. #00:16:42-2#

I: hm (bejahend) #00:16:42-2#

B: Jetzt ist, habe ich kein Problem, und dann geht's so. #00:16:44-6#

I: (...) Cool. ähm (...) Hast du, ähm (...) Hast du außerhalb von Regensburg, hast du dir Deutschland auch schon nen bisschen angeschaut? #00:17:00-3#

B: Schon, da war ich eine mal in (...) Hamburg und in München. So, zwei Städte hab ich angeschaut. #00:17:08-5#

I: Was hältst du davon. #00:17:11-9#

B: Ja, Hamburg war super. Viele Leute, schöne Stadt. Und München (...) gefällt mir nicht, weil, das ist irgendwie so (...) komisch. (lacht) #00:17:19-0#

I: Seh ich tatsächlich ganz ähnlich. #00:17:21-0#

B: Ja. #00:17:23-5#

I: Ich weiß (...) ich weiß ned, ähm (...) München muss (...) bis, ähm, vor 20 Jahren, 20 - 30 Jahren was, muss das ne wunderschöne Stadt gewesen sein. (...) ähm, aber ich weiß ned, ich find's mittlerweile irgendwie unsympathisch. Aber is auch nur so, meine persönliche Einschätzung. Keine Ahnung. #00:17:45-6#

B: Ja. Mir ist auch nicht gefallen. #00:17:48-2#

I: hm (bejahend) Aber Hamburg will ich mir umbedingt mal anschaun. #00:17:55-0#

B: Ja, ich war Hamburg, ich war in 2016. #00:17:55-0#

I: hm (bejahend) #00:17:55-0#

B: Und dann war ich Regensburg nach Hamburg mit dem Fahrrad unterwegs. Mit andere Leute.  #00:18:05-6#

I: hm (bejahend) #00:18:02-5#

B: Und so, ja genau. Waren dann da, haben Urlaub gemacht, dann zurück mit Auto gefahren hald. #00:18:07-9#

I: Ihr seid von Regensburg nach Hamburg mim Fahrrad gefahren? #00:18:11-8#

B: Mim Fahrrad gefahren, ja. #00:18:12-9#

I: Waas? Sehr cool. (lacht)  #00:18:16-9#

B: Ja.  #00:18:18-1#

I: Wie seid Ihr da drauf gekommen? #00:18:18-1#

B: Ja, ich war hald (...) ähm mein Betreuerin kennt (...) ein Person und der war in (...)ähm eine Gruppe hald und der war Betreuer und der hat mit andere Jungs auch geredet, dass die auch mitfahren sollen. Und die haben mich gefragt, ob ich auch mitkomme undso. Dann haben hier in Regensburg zuerst trainiert und (...) vorbereitet, danach waren wir hald unterwegs. #00:18:41-5#

I: Wie viele Leute wart Ihr dann? #00:18:44-8#

B: Waren wir so acht Personen waren. (...) #00:18:49-3#

I: ähm, wie lang seid Ihr dann gefahren? #00:18:52-7#

B: Sind wir (...) ungefähr sechs Tage gefahren. #00:18:53-6#

I: Von Regensburg nach Hamburg in sechs Tagen? #00:18:58-1#

B: In sechs Tagen, ja. In der Nacht haben wir geschlafen, und am Tag dann eben Fahrrad gefahren. #00:19:01-5#

I: (...) Ich muss gestehen ich weiß nicht wie viele Kilometer das sind? #00:19:08-0#

B: Das war ungefähr 780 so, Kilometer. #00:19:14-5#

I: 780 Kilometer in sechs Tagen? #00:19:14-5#

B: So ungefähr glaube ich, ja. #00:19:18-7#

I: Das heißt das sind dann (...) pro Tag deutlich über 100 Kilometer. #00:19:22-3#

B: Über 100 Kilometer. 140, 130. So sind wir gefahren. Manchmal auch bisschen, ja, kürzer #00:19:30-5#

I: hm (bejahend) #00:19:30-5#

B: Wenn regnet undso 80 so. Aber ungefähr sind wir 700. #00:19:34-6#

I: Sportlich. #00:19:34-6#

B: Am Anfang schon. (lacht) #00:19:37-5#

I: (lacht) (...) ähm, warte aber des war 2016, da war dann des mit deinen Beinen auch noch garned so lange her oder? #00:19:47-0#

B: Ja meine Beine waren nur wegen Blut, also nicht verletzt oder so. Da ist alles gesund geworden. #00:19:51-7#

I: Achso. #00:19:52-4#

B: Ja. #00:19:52-4#

I: Und dann (...) absolut keine Probleme mehr seitdem. #00:19:55-0#

B: Keine Probleme seitdem. #00:19:58-7#

I: Passt. (...) ähm (...) also die, ahja eins hab ich noch: also in Afghanistan ist ja (...) die Lage relativ angespannt, also auch zur, ähm Polizei jetz denke ich mal. Also ich hab's jetzt von Syrien und im Iran mitbekommen, dass da auf jeden Fall, ähm (...) die Polizei und die Regierung macht quasi was Sie will. Also so hab ich das jetzt mitbekommen. ähm (...) Ich weiß ned, welche Erfahrungen du da gemacht hast? Aber (...) Hat des dann irgendwelche Auswirkungen drauf, wie du zum Beispiel die Polizei, die draußen auf der Straße rumfährt, ähm (...) ob du denen dann vertraust oder, hat des da irgendwelche Auswirkungen drauf? #00:20:57-7#

B: ähm (...) Das habe ich garnicht gemerkt, wie das da sein soll. Keine Ahnung. #00:20:59-8#

I: Echt #00:21:01-0#

B: Ja. #00:21:03-1#

I: Okay. #00:21:03-1#

B: Ich weiß nicht. (...) #00:21:08-2#

I: (...) Ja, gut. Das heißt bei dir ist tatsächlich die Integration ziemlich reibungslos gelaufen oder? #00:21:21-4#

B: Keine Ahnung, da war so jung, ich weiß es nicht. Polizei undso. Hat kein Kontakt, nix. #00:21:25-1#

I: N-ne, ich mein jetz hier in Deutschland. #00:21:27-2#

B: In Afghanistan auch nicht! Keine Ahnung. (...) #00:21:32-9#

I: (...)ähm (...) Aber, du bist (...) ähm Bist du, seit du hier nach Deutschland gekommen bist, irgendwann auf, ähm (...), irgend nen Problem gestopen, was weiß ich, ähm, dass du zum BEispiel zum Arzt musstest und es war kein Dolmetscher da zum BEispiel? Irgendwas in die Richtung. (...) #00:22:18-8#

B: Ja zum Arzt ich bin schon oft gegangen. Und das war kein Dolmetscher #00:22:21-6#

I: hm (bejahend) #00:22:21-6#

B: Ich habe selber geredet. (...) #00:22:26-8#

I: (...) hm (bejahend) Okay. #00:22:29-1#

B: Ja. Von Heim war ich mit im Heim auch, ich bin manchmal mit Jungs und mit die andere Gruppe gegangen #00:22:34-5#

I: hm (bejahend) #00:22:34-8#

B: Betreuer hat dann mich mitgebracht, wegen hald Übersetzen undso. Ich konnte ein bisschen Deutsch und die anderen sind neu gekommen #00:22:41-6#

I: hm (bejahend) #00:22:41-6#

B: Und die konnten auch nicht. (Begrüßung eines Mitbewohners) Ich hab hald denen auch geholfen indso, Sprache undso. #00:22:52-5#

I: Also dann hast du quasi (...) ähm den Job des Dolmetschers übernommen. #00:22:56-2#

B: Ja. Nur zwei, drei Worte hald. (lacht) #00:23:00-0#

I: Ja (lacht) #00:23:00-0#

B: Nicht viel (lacht) #00:23:00-2#

I: Aber immerhin so, dass es irgendwie funktioniert hat. #00:23:03-4#

B: Ja, irgendwie hat's funktioniert(lacht) #00:23:05-6#

I: Ja, cool! (Begrüßung eines Mitbewohners) (...) ähm wie habt ihr denn die Wohnung hier gefunden? #00:23:27-3#

B: Ah, die ist von St. Vinzent. #00:23:29-7#

I: Die gehört zu St. Vinzent #00:23:28-9#

B: Die alle gehören St. Vinzent, ja. #00:23:30-1#

I: Okay. #00:23:30-7#

B: Die haben uns hald her geschickt. #00:23:33-1#

I: ähm (...) Also ihr habt quasi mit den Leuten bei St. Vinzent geredet, ähm dass Ihr ne Wohnung sucht und ob die irgendwas wissen? #00:23:41-6#

B: Ja. #00:23:42-9#

I: Und (...) die meinten dann: 'Jo, wir haben hier die Wohnung, schaut euch die doch mal an'? #00:23:48-9#

B: Also es war so damals, wir, ich wollte auch umziehen, und (...) Die wollten auch umziehen. Und St. Vinzent hat gesagt: 'Wenn jemand keine Aufenthalt hat, und dann darf man nicht umziehen.' #00:24:01-5#

I: hm (bejahend) #00:24:01-5#

B: Und die haben hald uns genommen und in diese Wohnung geschickt. (...) #00:24:07-0#

I: ähm also man darf ohne Aufenthaltserlaubnis nicht umziehen? #00:24:11-4#

B: Nur, nich umziehen und darf man nicht hald Wohnung mieten oder so. #00:24:15-5#

I: hm (bejahend) (...) ähm und die Aufenthaltserlaubnis habt Ihr momentan (...) doch die habt Ihr schon, oder? #00:24:24-0#

B: Nein, wir haben hald so , drei Jahre oder ein Jahr hatten wir nicht. #00:24:30-3#

I: hm (bejahend) #00:24:30-3#

B: So, zum Beispiel weil man kann entweder ein Jahr oder drei Jahre Aufenthalt hat, dann, dann bekommt man keine Wohnungserlaubnis #00:24:36-2#

I: Mmh #00:24:36-2#

B: Und wir haben hald sechs Monate und ich habe selbst , weiß nicht, drei Monate, und ich darf überhaupt nicht umziehen. #00:24:44-5#

I: Und des wird dann quasi immer nach Ablauf wieder verlängert, oder wie läuft das dann? #00:24:50-9#

B: Genau. Wieder anch drei Monate dann gehe ich hin, und muss ich verlängern lassen. #00:24:57-6#

I: hm (bejahend) ähm und die permanente Aufenthaltserlaubnis könntest du kriegen,w enn das rechtliche erledigt ist oder? Also das ist das, wo du meintest, dass das schon initiiert worden is oder? #00:25:13-0#

B: ähm ich hab zweimal Ablehnung und jetz habe ich hald diese Antrag 25a) und das läuft noch.  #00:25:19-2#

I: hm (bejahend) #00:25:19-8#

B: Weiß ich nicht mehr. #00:25:21-0#

I: ähm wie ist denn das mit der Ablehnung damals gelaufen? #00:25:25-3#

B: ähm ich war in Interview im BAMF #00:25:26-6#

I: hm (bejahend) #00:25:27-0#

B: Und die haben hald alles gesagt, nein, abgelehnt. Dann ahbe ich nochmal geklagt #00:25:31-6#

I: hm (bejahend) #00:25:31-6#

B: UNd ich hab nochmal Einladung bekommen, Interview, ich bin zu Gericht gegangen. #00:25:35-4#

I: hm (bejahend) #00:25:36-2#

B: Und die haben noichmal hald abgelehnt. Und danach habe ich gesagt 'jetz mache ich nichtmehr.' #00:25:42-6#

I: Also, und dann hattest du keine Lust mehr oder (...) ? #00:25:49-8#

B: Ja. Keine Lust, und ich konnte nicht warten. #00:25:52-2#

I: hm (bejahend) #00:25:53-0#

B: Und sonst (...) genau. #00:25:56-2#

I: (...) Ähm, magst du mir des noch nen bisschen genauer erzählen, wie des um die Ablehnung (...) ähm wie des rundrum ausgeschaut hat? #00:26:05-2#

B: ähm das weiß ich nicht. #00:26:06-2#

I: Okay. Wann war das denn? #00:26:06-9#

B: Das war in 2017 glaube ich. #00:26:12-7#

I: 2017? #00:26:12-7#

B: Also, nicht sicher, aber sowas war, ja. #00:26:15-7#

I: hm (bejahend) #00:26:15-7#

B: Und Anfang 2018 habe ich Duldung hald, ungefähr ein Jahr und ein paar Monate. #00:26:22-5#

I: Okay. #00:26:23-9#

B: Ja. #00:26:24-8#

I: ähm wie viele Ablehnungen ahst du da dann bekommen? #00:26:35-4#

B: Zwei. #00:26:35-4#

I: Zwei Ablehnungen. #00:26:35-4#

B: Ja. #00:26:35-5#

I: ähm (...) Wie hat sich das angefühlt? (...) #00:26:41-3#

B: (...)  #00:26:44-4#

I: (...) Okay. #00:26:44-4#

B: Ja. (...) #00:26:53-2#

I: (...) Ähm was (...) was dnekst du, warum (...) ähm entschuldige, dass ich da jetzt drauf rein bohre, aber das passt leider zu meinem Thema. ähm, oder soll ich das lassen? #00:27:15-5#

B: Wie du willst. #00:27:15-5#

I: Okay (...) also, wenn du keine Lust mehr hast dann sag's einfach. #00:27:18-9#

B: Okay. #00:27:20-4#

I: ähm (...) aber, run dum die Ablenung, was denkst du denn, warum's dazu gekommen ist? Und (...) ob du da dran was ändern hättest können. #00:27:37-1#

B: Ich weiß nicht, warum. Ich habe auch nicht genau verstanden. #00:27:40-7#

I: hm (bejahend) #00:27:41-1#

B: Und (...) ich weiß es nicht. Ich habe alles ganz vergessen. Ist schon lange her, seit zwei Jahre undso. Ich weiß nicht warum. Aber irgendwann habe ich alles gelassen. #00:27:56-3#

I: hm (bejahend) #00:27:56-7#

B: Ich konnte da nichtmehr warten, auch nochmal Klagen undso, dann habe ich Dulung. Lieber Duldung, besser. Muss wieder meine Ausbildung machen, muss ich viel lernen. Sprache lernen. Neue Leute kennen lernen. #00:28:08-4#

I: hm (bejahend) #00:28:09-0#

B: Das war mir wichtig, nicht andere. #00:28:10-9#

I: Ja. #00:28:12-0#

B: Ja. #00:28:12-8#

I: ähm und wenn, wenn du jetz dann die (...) ähm Ausbildung erledigt hast, dann könntest du auch ne permanente Aufenthaltsgenehmigung bekommen oder?
 #00:28:23-4#

B: Ich glaube schon, ja. Wenn ich dann sage, dass ich Hauptschulabschluss hab, und  #00:28:29-2#

I: hm (bejahend) #00:28:29-2#

B: Ausbildung geschafft hab und alles #00:28:30-3#

I: hm (bejahend) #00:28:30-8#

B: Dann haeb ich hald Arbeit #00:28:32-8#

I: hm (bejahend) #00:28:33-7#

B: Das Vertrag, dann vielleicht (...) Weiß ich auch nicht sicher. #00:28:38-6#

I: ähm hast du das dann von deinen Betreuern so mitbekommen, dass das so läuft, oder über wen? Oder war das über's BAMF oder (...) ? #00:28:50-8#

B: Wegen Ausbildung meinst du? #00:28:51-1#

I: ähm ja. Also, dass zu dum Beispiel dann(...) erstens, dass es die Möglichkeit zu ner Ausbildung gibt, und wenn du die erledigst und dann ne Schulbildung hinter dir hast, dass du dann gute Karten für ne permanente Aufenthaltserlaubnis hast. #00:29:05-6#

B: Ja das habe ich von viele Leute gehört #00:29:08-1#

I: hm (bejahend) #00:29:08-1#

B: Haben Lehrer, Betreuer, Chef (...) und selber #00:29:15-0#

I: hm (bejahend) #00:29:15-0#

B: ähm und ich habe Leute auch gesehen, dass die hald Ausbildung gemacht haben und danach kamen, die haben auch ein Jahr oder so, hald so Erlaubnis und Karte bekommen. #00:29:24-4#

I: Cool. ähm (...) Hast du auch im, ähm, im Internet irgendwas zum deutschen Asylverfahren gelesen oder so? Oder irgendwo anders? #00:29:36-8#

B: Nein. #00:29:38-0#

I: Nö? Okay. (...) ähm hat's was gegeben, so in der ähm Kultur, also ich denk mal dass Afghanistan und Deutschland (...) sind ja relativ unterschiedliche Kulturen im großen und ganzen würd ich sagen oder? #00:30:06-4#

B: Ja. #00:30:06-4#

I: ähm (...) was sind Unterschiede, die dir aufgefallen sind? #00:30:17-5#

B: (...) Ja, große Unterschied ist zum Beispiel: Weihnachtsfeier, wir bei uns, ist keine Weihnachtsfeier. #00:30:26-4#

I: hm (bejahend) #00:30:26-4#

B: Und bei uns ist zum Beispiel nach Ramadan so kleine Fest undso #00:30:30-8#

I: hm (bejahend) #00:30:31-3#

B: und #00:30:33-4#

I: //ähm wann (...) Entschuldigung, zwischendurch: Wann geht Ramadan jetz los? #00:30:36-3#

B: Ab nächste Woche Montag. #00:30:38-4#

I: Ab nächsten Montag läuft das? #00:30:39-9#

B: Ja. #00:30:39-9#

I: Danke dass du davor jetzt noch zeit gehabt hast. #00:30:41-5#

B: (lacht) #00:30:42-4#

I: (lacht) #00:30:43-0#

B: Und noch nicht so offen unter Ramadan, dann ein, ein Monat (...) und danach haben wir hald drei Tage, zwei Tage, so Fest hald. #00:30:48-2#

I: hm (bejahend) #00:30:48-7#

B: Wir keins (unv.) hatten wir gehabt und hier, mussten wir arbeiten leider. (lacht) #00:30:54-6#

I: hm (bejahend) #00:30:57-0#

B: Und nach diese Fest haben dann zwei Monate, nach zwei Monate, dann haben wir große Fest gehabt. Des die unterschiedliche (unv.). #00:31:04-7#

I: ähm, also du bist ja jetzt schon a paar Jahre hier, hat des bisher (...) immer funktioniert in der Arbeit, dann in der Zeit des Ramadan? Weil, ich mein des muss ja extrem stressig für dich sein. #00:31:18-9#

B: Eigentlich schon, schon funktioniert, ja. #00:31:20-6#

I: Echt? #00:31:20-6#

B: Ja, schon.  #00:31:22-3#

I: Weil ich mein du darfst, ähm, während'm Ramadan unter Tags ja weder essen noch trinken oder? #00:31:28-6#

B: Schon, schon. #00:31:30-6#

I: Schon? #00:31:30-6#

B: Ja, ich kann nicht trinken, nein. #00:31:30-7#

I: Genau, meine ich. ähm (...), und des is aber immer auch so von deinen Kollegen auch akzeptiert worden und (...) ähm das is hald einfach so. #00:31:44-7#

B: Ja, die sagt nichts. Mein Chef sagt: 'Deine Religion und deine is anderes, und meine Arbeit ist was anderes. Egal was du machst, aber ich brauche Arbeit für dich.' #00:31:52-4#

I: hm (bejahend) #00:31:52-4#

B: Genau. Von dir. #00:31:55-5#

I: Passt. #00:31:55-5#

B: Ich arbeite und mache Ramadan, kein Stress. Echt, alles Gut. Genau. #00:32:01-2#

I: ähm, sind dir noch andere Unterschiede aufgefallen? #00:32:05-9#

B: Das ist auch große Unterschiedlich dass ich mache Ramadan und die andere nicht (lacht) #00:32:09-1#

I: Jaa (lacht) (...) Kannst sie ja vielleicht überzeugen, dass Sie auch mitmachen. #00:32:15-6#

B: nein, das kann ich nicht. (lacht) #00:32:17-8#

I: (lacht) #00:32:17-8#

B: Jeder macht was er will. #00:32:21-2#

I: Ja. #00:32:21-2#

B: Ich kann nicht. #00:32:22-1#

I: Ja. ähm, aber abgesehen davon, dass jetz zum Beispiel hier wird ja Weihnachten oder Ostern oder was weiß ich gefeiert, ähm und (...) in Afghanistan dann der Ramadan zum Beispiel, ähm (...) Gibt's andere Unterschiede, die dich (...) sagen wir mal, verwundert haben, als du hier hergekommen bist? (...) #00:32:47-2#

B: Keine Ahnung. Vielleicht gibt es schon am Anfang, aber ich weiß es nichtmehr. Ich bin schon seit fünf Jahre ungefähr da #00:32:53-3#

I: hm (bejahend) #00:32:53-3#

B: Bin jetzt daran gewohnt #00:32:56-1#

I: Ja. #00:32:56-1#

B: Und viele weiß ich nicht, ich weiß nicht. #00:32:58-3#

I: Also es ist jetz (...) nichts quasi hängen geblieben, dass du (...) das dich so sehr verwundert hat, ähm dass es (...) hier komplett anders läuft als in Afghanistan, dass das bis jetzt hängen geblieben wäre oder so. #00:33:18-1#

B: Ne, eigentlich nicht, nein. #00:33:18-5#

I: Okay. (...) warte (...) ähm (...) Also, und die (...) du warst ja insgesamt drei Jahre in der Schule in Afghanistan, ne? #00:33:39-4#

B: Ja. #00:33:40-5#

I: ähm (...) was ist denn da so abgedeckt worden? ähm, waren des dann so (...) Grundsachen, so lesen und schreiben, und rechnen und dann eben die Koranschule? Oder was hast du da damals gemacht? #00:33:59-4#

B: ich weiß es nicht genau. Da war ich glaube wir haben so Mathe und Sozialkunde undso gemacht. #00:34:03-6#

I: Hmh? #00:34:05-1#

B: ähm, in die Schule. Aber das ist schon lange her. Ich weiß nicht. #00:34:09-5#

I: Okay. #00:34:10-8#

B: Ja. Das ist echt lange her. #00:34:11-8#

I: hm (bejahend) #00:34:11-8#

B: (...)Ich mach mal eben ne Pause. /Kurzes Gespräch abeits #00:34:40-3#

I: Also in welcher Bevölkerungsgruppe ist du denn eigentlich in Afghanistan? #00:34:42-0#

B: ähm Ich bin Sunnite hald. #00:34:47-3#

I: Okay, Danke! ähm, dann (...) ürd ich jetz sagen mal, lass ma's für heut, ähm wenn mir noch was einfallen würde, kann ich mich dann wieder bei dir melden? #00:35:01-2#

B: Hm, ja, wenn ich Zeit habe. ABer ich glaube nicht. (lacht) #00:35:03-5#

I: Okay(lacht) Ja gut, jetzt kommt Ramadan. #00:35:05-6#

B: Genau, Ramadan, muss ich viel lernen, alles viel Stress, ja. #00:35:14-2#

I: Also, Danke. #00:35:15-0#

B: Bitteschön.\newpage
%\section{Interview 5}

 #00:00:03-8#

I: Gut (...) Anscheinend, ja. #00:00:03-9#

B: Ja. #00:00:06-1#

I: Also, erstmal Danke dass zu Zeit hast! #00:00:08-5#

B: Bitte! Gerne (lacht) #00:00:09-7#

I: ähm (...) Dann würd ich jetz einfach mal mit dem Standardzeug anfangen wenn du Lust hast? #00:00:16-0#

B: Jo. #00:00:16-0#

I: ähm (...) wie (...) du bist offensichtlich männlich. ähm (...) wie alt bist du denn? #00:00:22-2#

B: Ich bin 19. #00:00:24-6#

I: Du bist 19? #00:00:24-6#

B: Ja. #00:00:24-6#

I: Ay. (...) Komm ich mir direkt alt vor, ich bin 24. Du siehst älter aus #00:00:31-3#

B: (lacht) #00:00:33-4#

I: Du siehst älter aus #00:00:33-4#

B: Ja, ähm, eigentlich ähm (...) ich kann auch Foto zeigen, weil ich ab drei Jahre Dialyse gehabt. #00:00:41-3#

I: Ach du scheiße. #00:00:41-3#

B: Und wenn ähm, wenn die Leute bei Dialyse #00:00:44-9#

I: hm (bejahend) #00:00:44-9#

B: Und ich muss ähm (...) dreimal in Woche, ich hab drei Jahre gehabt. Dialyse. #00:00:51-6#

I: Ja #00:00:51-6#

B: Wenn ich gehe oft dort dreimal in Woche #00:00:54-5#

I: hm (bejahend) #00:00:54-5#

B: Dann eigenlich muss die ganze Blut wieder abwaschen und geht Maschine und wieder rein und wenn kommt wieder zurück, meißte die Leute machen Dialyse, die #00:01:04-9#

I: hm (bejahend) #00:01:04-9#

B: die Haut von denen wird ein bisschen auch so wie (...) ältere #00:01:10-5#

I: hm (bejahend) #00:01:10-5#

B: und wird ein bisschen dunkel. #00:01:12-4#

I: hm (bejahend) #00:01:12-4#

B: Weißt du, weil die , ähm, eigene Blut geht in Maschine und  #00:01:17-4#

I: Ja #00:01:19-0#

B: Eigentlich, ich meine, du verlierst so (...) Blut von diese Schlauch und Maschine undso. Deswegen ich schaue auch so. Ja. #00:01:28-6#

I: Das hast du drei Jahre gemacht? #00:01:30-6#

B: Ja, drei Jahre gehabt, ja. #00:01:31-6#

I: Oi. Ähm, (...) Wo, wenn ich fragen darf? #00:01:35-6#

B: Im Iran. #00:01:38-1#

I: Im Iran? #00:01:38-1#

B: Ja. #00:01:38-1#

I: Darf ich fragen warum? #00:01:40-4#

B: Weil eigentlich meine, ähm, eigentlich meine eigene Niere war klein #00:01:44-9#

I: hm (bejahend) #00:01:45-5#

B: Die haben nicht aufgewachsen. #00:01:46-9#

I: hm (bejahend) #00:01:47-4#

B: Als ich geboren bin. #00:01:50-8#

I: hm (bejahend) #00:01:50-8#

B: War so, zu klein, und bis zwei, drei Jahre alt hatte ich schon so (...) ein bisschen schon aufgewachsen. #00:01:56-2#

I: hm (bejahend) #00:01:56-2#

B: Die waren richtig funktioniert. Danach, die haben nicht aufgewachsen. Dann, ich bin, ich glaube mit sieben oder acht ich habe angefangen so mit Dialyse #00:02:05-3#

I: hm (bejahend) #00:02:05-6#

B: Dann, drei Jahre habe ich gehabt, und 2012 habe ich schon transplantiert. Eine neue Niere ich hab bekommen. #00:02:13-5#

I: Sehr gut. #00:02:13-5#

B: Ja. #00:02:15-0#

I: Und jetz geht's dir gut. #00:02:17-3#

B: Natürlich. Ja. (lacht) #00:02:20-5#

I: Okay. (...) ähm, seit wann bist'n du dann in Deutschland jetz? #00:02:25-0#

B: ähm, seit eigentlich Mitte 2015. #00:02:27-7#

I: Mitte 2015. #00:02:29-1#

B: Ja, ungefähr im Mai war das. (...) Ich bin nicht ganz sicher #00:02:33-4#

I: Ja (...) ja. ähm (...) und jetz auch zum Beispiel des mit der Niere, des hat dann hier keine Probleme gemacht mehr oder so? #00:02:41-5#

B: ähm (...) ne, zur Zeit nicht. Also es funktioniert, ja (...) alles. Ja, ich gehe monatlich, oder (...) jede sechs Woche ich gehe zum Arzt #00:02:51-8#

I: hm (bejahend) #00:02:53-8#

B: Dann, die machen Untersuchung. Untersuchen. Und (...) die schauen, wie läuft mit meiner Niere. #00:02:58-8#

I: Okay. #00:02:58-8#

B: Aber trotzdem ich kriege schon Medikamente. #00:03:01-1#

I: Ah ja. Ja, ich muss sagen, da weiß ich nix drüber. #00:03:05-2#

B: Ja, nee passt schon! Alles gut (lacht). #00:03:06-0#

I: Aber (...) zum Glück funktioniert des. #00:03:13-0#

B: Ja! Ich bin so froh, weil, ähm, ich meine nicht als ich krank, ähm als ich, ähm (...) klein war, oder Junge war #00:03:18-6#

I: hm (bejahend) #00:03:18-6#

B: Dann ähm (...) ich meine nicht, oft ich war krank. Ich merkte nicht, eigentlich ich war krank. Jetzt auch genauso wie früher #00:03:28-0#

I: hm (bejahend) #00:03:28-5#

B: Ich merk garnix ich bin krank. #00:03:29-3#

I: Ja. #00:03:30-2#

B: Ja. Und ähm, aber trotzdem ich war so (...) oft im Krankenhaus. Und ich, ich hab so oft im Krankenhaus geschlafen, und #00:03:40-8#

I: //hm (bejahend) #00:03:40-8#

B: War richtig schwiertig. Und jetzt, ich merke (...) was bedeutet diese Gesundheit. #00:03:48-4#

I: Ja. #00:03:49-6#

B: Wenn jemand hat (...) ähm, Gesundheit, und sagt: 'Hey, Gott! Warum gibst du mir nicht Geld?' Ich sage: 'Hey, Gott! Gott sei Dank, dass du mir eine, ähm, neue Niere mir gegeben. Und ich gehe nicht zum Dialyse. Ich brauche kein Geld. Ich brauche (...) eigentlich meine Körper einfach gesund wird.  #00:04:05-2#

I: Ja. #00:04:05-9#

B: Ich brauch garnix. #00:04:09-5#

I: ähm da gibts nen schönen Satz, ich weiß aber ned von wem der is (...) ähm: 'Gesundheit ist eine Krone, die nur die Kranken sehen können.' #00:04:19-2#

B: Ja. #00:04:19-2#

I: ähm (...) Ich weiß (...) find ich nen schönen Satz, is bei mir hängen geblieben. ähm (...) Ja. ähm, aber du bist dann aus'm Iran, oder? #00:04:30-7#

B: ähm, ne, ähm (...) eigentlich, ich bin schon im Iran geboren #00:04:34-4#

I: hm (bejahend) #00:04:34-9#

B: Aber meine Eltern sind aus Afghanistan.  #00:04:35-9#

I: hm (bejahend) #00:04:36-8#

B: Und (...) ich weiß nicht, ähm (...) weißt du oder nicht? Leute aus Afghanistan: Die kriegen keine, ähm, richtige Dokumente vom Iran. #00:04:48-8#

I: Des hab ich mitbekommen. #00:04:48-6#

B: Ja? Es ist so. #00:04:53-0#

I: ähm (...) #00:04:52-1#

B: Wir haben keine Pässe, ich meine, keine richtige Ausweiß undso. #00:05:02-9#

I: Ja. #00:05:02-9#

B: Ich kann auch sagen, wir sind nicht ungültig im Iran #00:05:05-8#

I: Ja. #00:05:06-6#

B: Meine Eltern seit, ähm (...) fast (...) 40 Jahre oder 45 Jahre die sind im Iran. #00:05:14-1#

I: Und die sind seit 45 Jahren, leben die dort #00:05:17-3#

B: Ja #00:05:17-3#

I: UNd sind trotzdem noch illegal dort? Also, schwarz quasi? #00:05:21-7#

B: Naja (...) #00:05:21-7#

I: Also ohne, ohne, ohne gültige Papiere #00:05:21-1#

B: //Ja, ohne, ohne, ohne gültige Papier. Wir kriegen kein Papier. Wir haben keine (...) wir dürfen nicht in unsere Heimatland reisen. Ich hab (...) ich bin 19 Jahre alt #00:05:32-4#

I: hm (bejahend) #00:05:32-4#

B: Und (...) ich hab zwei ähm, Tante habe ich nicht gesehen, die sind in Afghanistan. #00:05:38-4#

I: hm (bejahend) #00:05:38-8#

B: Ich hab nie gesehen. Und zwei Onkel. Und ich kenne denen nicht. #00:05:47-2#

I: Wow. #00:05:47-2#

B: Ja. #00:05:47-2#

I: Vermut(...) Wenn du vermutlich nach Afghanistan gereist wärst um die zu besuchen, dann hättest du nicht mehr zurück gedurft, oder? #00:05:54-0#

B: (...) Muss man nochmal schwarz an die Grenze, ja. #00:05:58-7#

I: Ja. Schon, aber (...) #00:06:00-8#

B: //Schwarz aber sehr gefährlich. Die dürfen schießen an die Grenze #00:06:02-9#

I: //Ganz genau! Ja. #00:06:05-6#

B: Die Iran hatte schon erlaubnis #00:06:09-6#

I: hm (bejahend) #00:06:10-0#

B: wenn die (...) wenn die Leute schwarz einfach über die Grenze #00:06:11-3#

I: hm (bejahend) #00:06:11-3#

B: kommen oder fahren oder gehen #00:06:15-3#

I: Ja #00:06:15-3#

B: Dann die dürfen schießen eigentlich. (...) Und (...) der (...) mein Papa hatte gesagt also (...) ähm, Afghanistan ist unsicher #00:06:23-5#

I: Ja. #00:06:23-5#

B: Dann (...) können wir nicht gehen. Also (...) ich und meine ähm, Geschwester #00:06:30-5#

I: hm (bejahend) #00:06:30-5#

B: Die sind alle im Iran geboren. Gebort oder? #00:06:34-4#

I: Geboren. #00:06:35-3#
 
B: Geboren, ja. Und ähm (...) wir sind ähm (...) keine von ähm, niemand von meine Geschwester oder ich selbst ähm (...) Afghanistan noch nicht gesehen. #00:06:46-6#

I: Wow. #00:06:48-0#

B: Ja. Es ist so. Aber wir sind, ähm (...) ja, die Leute im Iran.  #00:06:54-5#

I: hm (bejahend) #00:06:54-5#

B: DIe MEnschen, die sagen halso so: 'Ihr seid ja Iraner undso, aber die Politiker sind richtig (...) keine Ahnung, ähm, Scheiße oder so? (lacht) #00:07:03-4#

I: Ja doch, schon. (lacht) #00:07:03-8#

B: ähm (...) Es gibt keine Bedürfnis, zum Beispiel im Restauran arbeiten. #00:07:09-7#

I: hm (bejahend) #00:07:10-2#

B: Wir dürfen nicht in Bäkerei arbeiten. Wir dürfen nicht im Krankenhaus arbeiten. Wir dürfen nicht, zum Besipiel, im ähm (...) Amt arbeiten. #00:07:19-4#

I: //Wo dürft Ihr arbeiten? #00:07:20-5#

B: Wir dürfen nicht studieren. Wir dürfen nicht lernen. Als ich krank war, ich ähm (...) als ich klein war, oder jünger war #00:07:26-5#

I: hm (bejahend) #00:07:26-5#

B: Ich war nie in der Schule. #00:07:30-2#

I: ähm #00:07:30-2#

B: Ich war nie in der Schule. #00:07:31-2#

I: Bist du dann später zur Schule gekommen? #00:07:33-8#

B: Ne, ähm, ich war nur hier in Deutschland. #00:07:34-5#

I: Also erst seit du hier in Deutschschland bist warst du in der Schule #00:07:35-7#

B: //Ja. (...) ja. Ich bin mit, ähm (...) eigentlich 15 oder 16 Jahre alt ich, ähm, ich bin schon angefangen zum Schule gegangen. #00:07:43-9#

I: Ja. #00:07:44-9#

B: Bin ich geganen zum Schule. #00:07:46-1#

I: Wow. #00:07:46-1#

B: Ich hab, ich kann nicht auf persisch lesen und schreiben.  #00:07:50-6#

I: hm (bejahend) #00:07:51-1#

B: Ich kann so wenig. Ich war nur sechs Monate in Koranschule. #00:07:54-7#

I: hm (bejahend) #00:07:55-0#

B: Das ist nur, ich kenne buchstabiere auf Persisch oder Arabisch #00:08:00-4#

I: hm (bejahend) #00:08:00-4#

B: So (...) Aber nicht so gut. Wenn ich schreibe, manchmal, dann schreibe ich falsch. #00:08:05-5#

I: (...)Natürlich #00:08:06-7#

B: Aber hier Gott sei Dank, wir dürfen schon in die Schule gehen. Wir dürfen schon Ausbildung machen. Wir dürfen schon (...) Welche, welche, keine Ahnung, welche Beruf (...) Welche, zum Beispiel welche (...) Wünsche haben wir? Hier, ich glaube (...) am besten. Das, das kann ich nicht einfach mehr, ähm, mehr Sätze dazu sagen. Ich weiß nicht, weil, ähm (...) Wenn du zum Beispiel im Iran bist #00:08:33-2#

I: hm (bejahend) #00:08:33-2#

B: Dann dürfst (...) immer, wenn du wolltest studieren, es gibt eine Grenze. #00:08:38-3#

I: Ja. #00:08:38-3#

B: Darfst du nicht studieren. #00:08:40-0#

I: Ja.  #00:08:42-2#

B: Wenn du wolltest zum Beispiel in deine Berufswünsche gehen, dürftest du nicht #00:08:44-9#

I: hm (bejahend) #00:08:44-9#

B: Mein Cousin hat jetzt, ähm, er ist ähm, letzte (...) Jahr #00:08:51-0#

I: hm (bejahend) #00:08:51-5#

B: In 2018 hat er seine Abschlussprüfung als ähm, Elektriker #00:08:58-6#

I: hm (bejahend) #00:08:58-6#

B: Er ist eigentlich Ingenieur. #00:09:00-6#

I: Ja. #00:09:01-5#

B: Ja. UNd er darf nicht arbeiten im Iran. (...) Er hat alles privat gezahlt, sein Papa. Als er in Grundschule war bis Universität #00:09:14-3#

I: Ja. #00:09:14-3#

B: Bis, ähm, bis zwölfte Klasse er war ungültig. Er hatte keine Zeugnis bekommen. Nur hat er die Chef von Schule Geld schwarz gegeben, damit er konnte schon lernen. #00:09:28-2#

I: Ja. #00:09:28-2#

B: Danach (...) hat er bei, ähm, wie nennt man? Beim, ähm, Abschlussprüfung? Oder ne, beim, ähm (...) Wenn man an die Uni geht. Wie nennt man? #00:09:38-8#

I: ähm, die (...) Die Einstiegsprüfung?  #00:09:43-3#

B: Ja, ähm (...) #00:09:43-3#

I: ähm (...) #00:09:43-3#

B: Also, ja. Hat er auch schon geschrieben #00:09:44-2#

I: hm (bejahend) #00:09:44-2#

B: ähm, er hatte, über Afghanistan hat er gemeldet. Er war auch wie ich, er war nie in Afghanistan. #00:09:51-2#

I: Ja.  #00:09:53-9#

B: Er war nur, über Afghanistan hat er dort angemeldet #00:09:55-6#

I: hm (bejahend) #00:09:55-6#

B: Formular ausgefüllt, und er durfte mit Iraner Prüfung schreiben. #00:10:01-1#

I: Okay. #00:10:01-1#

B: Ja, der hatte auch schon geschrieben, und hat auch schon bestanden, ich weiß nicht mit welcher Note, aber (...) hat schon bestanden. #00:10:07-1#

I: Ja. #00:10:07-1#

B: Ja. Es ist (...) #00:10:10-7#

I: Er hatte des schon, und durfte trotzdem nicht arbeiten? #00:10:10-4#

B: Ja. Er, er war auch schon ähm (...) viereinhalb Jahre an der Uni.  #00:10:17-8#

I: War schon vireinhalb Jahre an der Uni? #00:10:19-8#

B: Ja. Hatte auch, ich meine ähm, ich weiß nicht, auf deutsch nennt man Licence oder? #00:10:25-2#

I: Lizenz. #00:10:26-2#

B: Ja, Lizenz. Hatte Lizenz und Elektriker. #00:10:29-0#

I: Ja. #00:10:30-0#

B: Ja, er (...) #00:10:29-3#

I: //Also er hatte schon alles fertig #00:10:31-6#

B: Ja, ja. #00:10:33-4#

I: (...) #00:10:33-4#

B: Er ist jetzt zu Hause, und er geht manchmal zu, einfach ganz normale Arbeit, manchmal er verkauft so, wie in Bazaar. #00:10:42-6#

I: hm (bejahend) #00:10:42-6#

B: Verkauft so, solche Gläser, oder keine Ahnung (...) so wei Teekanne oder Wasserkanne oder (...) Solche Sachen. Er geht in Bazaar. So wie Flohmarkt oder solche, so. #00:10:55-2#

I: ähm, also er ist noch im Iran? #00:10:56-1#

B: Ja, ja. Schon. #00:10:56-7#

I: Aber ich mein (...) Da machen die doch die eigene, Ihre eigene Wirtschaft kaputt tatsächlich mit sowas. Das is dumm! #00:11:04-7#

B: Wie, die Politiker? Aber ich habe auch schon eine Video von Ihm #00:11:09-3#

I: Ja? #00:11:11-0#

B: Er hatte elber eine, ähm, Schloss (...) #00:11:13-0#

I: hm (bejahend) #00:11:13-5#

B: Und er hat, wie nennt man (...) von Bank, wenn man, es gibt so eine, diese mit Beton, wenn man die, ähm (...) #00:11:21-4#

I: Tresor? #00:11:20-8#

B: Wie? #00:11:23-0#

I: Tresor? #00:11:23-0#

B: Tresor! Hatte den Schloss von Tresor als Elektriker. #00:11:26-3#

I: Ja? #00:11:26-8#

B: Hatte schon gebaut #00:11:29-1#

I: Ja? #00:11:29-1#

B: Und ähm, konnte man eigentlich (...) wenn du zum Beispiel ähm, beim Gesprächsanlage #00:11:35-0#

I: hm (bejahend) #00:11:35-0#

B: ähm, das ist von ähm, wenn, wenn du nicht zu Hause bist #00:11:38-4#

I: Ja? #00:11:39-3#

B: Weißt du, diese Gesprächsanlage, wenn jemand Klingelt #00:11:41-7#

I: Ja #00:11:41-7#

B: Und du bist nicht zu Hause, und das klingelt in deine Handy #00:11:44-8#

I: Ja #00:11:44-8#

B: Dann du konntest mit deine Handy, wenn du schaust: 'Ja, bitte?' Dann ähm, entweder deine Freund oder Freundin, sagst du, also ich bin da, und sagst du, ich bin beim Einkaufen, dann konntest du mit deine Handy einfach ähm, ein PIN geben und dein Tür aufmachen. Und gehst du im ähm (...) der Freund oder jemand hinter dem Tür, der konnte rein gehen. #00:12:09-4#

I: Des is gut! #00:12:09-4#

B: Ja. Und ich hab schon sein Video, aber mein Handy ist, ähm, rüber. #00:12:12-6#

I: Wes will ich danach sehen! Des hört sich, hört sich interessant an! #00:12:13-4#

B: Das is 16 min und ich müsste eigentlich nur mal, weil ich hab alle gelöscht. Ist schon in mein Handy, aber ich kann nochmal, ähm, runterladen. #00:12:21-0#

I: Das würd ich gerne sehen! #00:12:23-9#

B: Ja, dann mach ich! #00:12:23-9#

I: Danke! #00:12:25-3#

B: Aber jetzt reden wir, oder nicht? #00:12:26-9#

I: Ja, stimmt, jetzt mach ma erstmal da weiter. #00:12:28-9#

B: Ja. #00:12:28-9#

I: ähm (...) dann würd ich jetzt mal so zeitlich nochmal bisschen zurück gehen #00:12:37-0#

B: Ja #00:12:37-0#

I: ähm (...) Wie bist du denn nach Europa gekommen #00:12:43-4#

B: Uff. (lacht) Das ist, (lacht), das ist ein (unv.). Also ähm, ich bin(...) 2014 bin ich ähm (...) von Zuhasue weg. #00:12:55-1#

I: ähm #00:12:56-3#

B: Also, bin ich weggegangen. #00:12:56-7#

I: Also eineinhalb Jahre früher als du angekommen bist? #00:12:59-9#

B: ähm, im Mai 2014 bin ich geflüchtet. #00:13:05-1#

I: hm (bejahend) #00:13:05-1#

B: Bis Mai 2015 bin ich ein Jahr gedauert. #00:13:10-8#

I: Fuuck. #00:13:12-6#

B: Ja, es ist eigentlich elf Monate. Aber trotzdem, ja passt. #00:13:14-5#

I: Das war ein langer Weg, bis du hierher gekommen bist. #00:13:18-6#

B: Ja, das ist (...) war wirklich schlimm und (...) ähm, war wirklich, ich weiß nicht, wie soll ich sagen. Ich war immer mein Rucksack dabei. #00:13:29-4#

I: hm (bejahend) #00:13:29-4#

B: Ich hatte immer mein Rucksack und manchmal ich war drei Tage zu Fuß #00:13:37-3#

I: hm (bejahend) #00:13:37-3#

B: Manchmal ich war vier Tage zu Fuß, ohne Essen. #00:13:41-2#

I: (...) #00:13:41-2#

B: Ich war in Schiff zwei Woche. Türkei bis Griechenland. #00:13:44-6#

I: Zwei Wochen auf dem Schiff?! #00:13:46-6#

B: Zwei Woche. Ja, wir haben Zehn Tage ohne Essen. #00:13:49-0#

I: Was?! #00:13:49-0#

B: Ja, ich kann auch in Youtube unsere Schiff zeigen. #00:13:52-5#

I: Ja? #00:13:53-7#

B: Ja. #00:13:55-0#

I: Des dann bitte gerne auch später. #00:13:57-0#

B: Ja, dann machen wir. #00:13:57-0#

I: Ja. #00:13:57-0#

B: Ich war Zwei, ähm, zwei Woche in Wasser #00:14:01-4#

I: Ja #00:14:02-3#

B: Wir hatten nur fünf Tage Essen, dann zehn Tage (...) ohne Essen. #00:14:09-6#

I: Was ist da schief gelaufen? #00:14:09-6#

B: ähm, der Motor war kaputt. Wir waren schon eigentlich schwarz #00:14:12-7#

I: hm (bejahend) #00:14:13-5#

B: Und, ähm, wir waren schon schwarz, und wir haben gedacht, also das dauert nicht lange. Vielleicht ein Tag oder zwei Tag. Wir haben manchmal Schokolade gegessen oder so. #00:14:22-6#

I: Ja #00:14:23-5#

B: Aber später dann (...) #00:14:26-6#

I: Zehn Tage komplett ohne Essen (...) #00:14:26-9#

B: Ja, und, ähm, eineinhalb Tage ohne Wasser. Wirklich, ohne (...) trotzdem, niemand hatte Wasser (...) Wir haben von Meer Wasser genommen #00:14:37-9#

I: (...) #00:14:38-7#

B: Dann, ich hab schon ein bisschen getrunken, wegen mein Niere #00:14:41-5#

I: Jaaa (...) #00:14:41-5#

B: Es braucht eigentlich schon Wasser. Und ich hab getrunken und (...) ich war noch schlimmer. #00:14:47-8#

I: Ja. #00:14:49-9#

B: Das ist, ähm, ich ähm, ich hatte so schwindlich hab ich bekommen, und (...) ich wollte eigenltich kotzen oder unterbrechen undso (...) dann (...) war so schlimm. War so schlimm, ja. #00:15:00-7#

I: (...) Zum Glück musst du das nie wieder machen.  #00:15:05-1#

B: Ja, ähm, jetzt, in Deutschland, wenn ich würde eine, ich wollte eigentlich zum Beispiel (...) eine Straße zum Beispiel ist 100 Meter, ähm, 100 Meter weit weg. Es gibt eine Ampel. #00:15:21-8#

I: hm (bejahend) #00:15:22-2#

B: Dann gehe ich 100 Meter. Würde ich nicht einfach von Straße einfach durchlaufen, auf die andere Seite. #00:15:28-2#

I: Ja. #00:15:29-1#

B: Ja. Dann, ich macn nie wieder, in meinem Leben, Nie! Schwarzfahren. #00:15:34-1#

I: Ja? #00:15:37-5#

B: Es ist so. Ich weiß, bei euch ist, ähm (...) jemand muss eigentlich solche Schwierigkeit sehen, damit ähm merkst du was ich sage. #00:15:48-5#

I: Ja. #00:15:48-5#

B: Weißt du was ich meine? #00:15:48-5#

I: Ja. #00:15:48-5#

B: Das ist (...) man muss eigentlich so, in diesem, diesem (...) Ort oder in diesem ähm, wenn du hungrig bist oder keine Ahnung, ich weiß nicht. Wie soll ich sagen? (...) Es ist wirklich Wahnsinn. Sehr schwer. #00:16:04-3#

I: hm (bejahend) #00:16:04-8#

B: Ja. UNd ähm, ich hatte auch noch Probleme: Ich hatte keine Medikamente dabei. Wegen mein Niere. #00:16:11-4#

I: Shit. Des komplette Jahr über? #00:16:13-7#

B: ähm, ne, ne. Nicht komplette Jahr. #00:16:16-2#

I: Ok. #00:16:16-2#

B: Ich hatte am Anfang, bevor ich, ähm #00:16:18-7#

I: Ja? #00:16:18-7#

B: Bevor ich flüchte, oder? #00:16:22-1#

I: hm (bejahend) Bevor du geflchtet bist. #00:16:24-9#

B: Ja, bevor ich geflüchtet bin, dann ich hab für drei Monate hab ich schon meine Medikamente mitgenommen. #00:16:31-2#

I: hm (bejahend) #00:16:31-6#

B: Dre Monaten. Dann, ähm, ich habe mitgenommen, und (...) es war (...) wahnsinnig schwierig. Das war wahnsinnig schwierig. UNd das war immer (...) ich konnte nicht, ich kann keine Englisch #00:16:48-0#

I: hm (bejahend) #00:16:48-0#

B: Ich war immer auf dem Weg ohne Freund, ohne (...) Ich hab immer gedacht, wenn ich hier sterben, ich weiß nicht, Griechenland, Mazedonien, Serbien, keine Ahnung, Österreich. Oder Türkei oder andere. Dann wie zum Beispiel wie kann ich bevor ich sterbe, wie kann ich zum Beispiel mein Familie bescheid sagen? Weißt du, du hast Angst, dass du alleine bist. #00:17:14-2#

I: Ja. #00:17:15-4#

B: Und niemand, keine Freund, keine Bekannt. Keine, keine Afghaner war mit mir. Mit mir unterwegs. #00:17:22-7#

I: hm (bejahend) #00:17:23-8#

B: Ja. Weil Syrien eigentlich war Krieg, und meißte Leute die waren auf diese schwarz, ähm, geflüchtet. #00:17:32-4#

I: Die meißten waren Syrer? #00:17:35-2#

B: Die meißten waren aus Syrien. #00:17:35-4#

I: hm (bejahend) #00:17:35-8#

B: Und deswegen, ja, war ein bisschen schwierig. #00:17:37-8#

I: (...) #00:17:39-2#

B: Ja, 11 Monaten. Ja. #00:17:44-5#

I:(...) Wow. ähm, (...) wie habt Ihr's dann geschafft, dass Ihr nach Griechenland gekommen seid? Ich mein, der Motor war kaputt. #00:17:53-6#

B: Ja, Motor war kaputt. Dann ähm, wihr haben eigentlich ähm, was? acht, neun Tage wir haben schon gewartet. Und die haben gesagt, also, heute, morgen, heute, morgen. Und alle haben so, also, wenn sind, wenn wir sind nicht da, dann sagen bitte Bescheid, ob (...) ähm bei uns waren die kleine Kinder. Wirklich ungefähr (...) acht Monate? Neun Monate? Ein Jahr? Ich miene, wir waren 700 Leute. #00:18:25-0#

I: 700?! #00:18:25-0#

B: (...) 700 Leute. #00:18:33-2#

I: (...) Das war nen verdammt großes Schiff, dass da kaputt gegangen ist. #00:18:35-6#

B: Ja. #00:18:36-5#

I: Ow. #00:18:40-1#

B: Aber wir waren nicht oben, wir waren im Keller. Wir waren ganz unten, im tief, im Keller. #00:18:43-1#

I: hm (bejahend) #00:18:43-6#

B: Dann ähm (...) ähm ich weiß nicht, muss (...) das kann ich auch schon sagen: Es gab kein Toilette. Und (...) Man konnte nicht, ähm, oben rausgehen. #00:18:58-2#

I: Warum nicht? #00:18:58-2#

B: Weil, ähm, zum Beispiel wenn viele Leute auf dem Schiff laufen #00:19:05-0#

I: Ja? #00:19:05-0#

B: Dann, es gibt die  Flugzeuge, die fliegen oben, und die schauen ähm, entweder die Schiffe sind kaputt, oder die, die haben Kontakt mit alle Schiffe. Egal ob aus, aus Europa, egal ob Asien (...) und die machen Fotografieren. #00:19:22-2#

I: hm (bejahend) #00:19:22-2#

B: (...) Und ähm, deswegen wir dürfen nicht oben laufen. Und, wir waren im 15 Tage im Keller. Und, tschuldigung, ich kann auch, ähm, (...) es gab nur eine kleine, diese Tuch, ähm, hier in der, an der Ecke #00:19:39-2#

I: Ja? #00:19:39-2#

B: Dann, auch Frauen. Auch Kinder. Auch, ähm, Erwachsene. Die haben dort auch Kacke gemacht, und keine Ahnung, die haben dort ähm (...) gepissen nennt man oder? #00:19:53-0#

I: ähm (...) #00:19:54-4#

B: Gepissen, und (...) das ist, diese unsere Nase, unsere Nase war Stoff nur als Bakterie. (...) Also nur. 15 Tage. Die Leute haben dort gepisst, einfach die hinten den Schiff gelaufen, die Leute haben dort geschlafen. Wir waren Gott sei Dank ganz hinten dem Schiff. Und die Leute hier, die geschlafen #00:20:17-3#

I: // krank geworden, oder? (...) #00:20:18-8#

B: Einfach (...), Ja, die waren wirklich, ähm, nach einund, ähm, eineinhalb Monaten, zwei Monaten, waren in Griechenland im Krankenhaus. Und ich war auch selber, ähm, zwei, drei Tage im Krankenhaus. #00:20:32-2#

I: hm (bejahend) #00:20:33-2#

B: Danach bin ich in, ähm, Gefängnis. Die haben mir gebracht. Ich war eineinhalb Monate im Gefängnis. In Griechenland. #00:20:38-7#

I: Ja. #00:20:40-1#

B: Damit, wir waren einfach schwarz. #00:20:41-2#

I: ähm, also das heißt, dass (...) Schiff ist irgendwann entdeckt worden, und Ihr seit davon runter gerettet worden? #00:20:49-4#

B: Ja. ähm, die, die Polizei von Griechenland hat uns gefunden #00:20:51-6#

I: hm (bejahend) #00:20:51-6#

B: Haben nochmein ein Schlauch, oder ein (...) Seele? Seele nennt man oder? So, so festgemacht unsere Schiff #00:21:00-0#

I: Seil. #00:20:58-1#

B: Seil, ja. Seil. ähm, die haben unsere Schiff festgemacht #00:21:04-4#

I: hm (bejahend) #00:21:04-4#

B: Und hat ähm, hat geschoben? #00:21:09-2#

I: Gezogen? #00:21:09-2#

B: Gezogen, ja. Gezogen. In Griechenland #00:21:13-6#

I: Ja. #00:21:14-3#

B: Ja. UNd dann, ähm, wir waren in Griechenland. Dann, erst einmal ein Woche wir waren in eine Fußballhalle oder Hallefußball? Dann #00:21:23-9#

I: hm (bejahend) #00:21:23-9#

B: ein Woche, danach ich war einmal im Krankenhaus, nur, ähm, zum Untersuchen #00:21:28-4#

I: Ja. #00:21:28-4#

B: ähm, zwei, drei Stunden. Danach, ähm, ich war in Gefängnis. #00:21:33-6#

I: Hast du danach wieder Medikamente wenigstens bekommen? #00:21:36-7#

B: Ich hatte schon ein bisschen dabei #00:21:38-3#

I: Ja? #00:21:39-8#

B: Danach, ich hab ähm, ich hab ähm gesagt, also mit dem Ärzte hab ich auch schon gesprochen #00:21:45-4#

I: hm (bejahend) #00:21:45-8#

B: Also zuerst, wir durften nicht zum Ärzte gehen #00:21:47-8#

I: Ja. #00:21:47-8#

B: Ich hab, ähm, im, im Woche, es gibt eine (...) eine Gefängnis, ungefähr (...) 3000? 4000? Leute waren dort. #00:21:59-3#

I: hm (bejahend) #00:21:59-7#

B: Und, ähm, ein, ein oder zwei Ärzte die kommen schon dort, und die gucken einfach: 'Ah, ich habe Kopfschmerzen!' Dann du kriegst eine (...) Tablette, ganz normal. Das ist von ganz normale Krankheit #00:22:10-4#

I: hm (bejahend) #00:22:10-8#

B: Und ich war dort, und ich habe gesagt, also ich hab Niere transplantiert, und ich nehme Medikamente. Die haben Sie? Ja ok, wir schreiben, musst in Kranken, ähm, Krankenhaus gehen. Ja, oder fahren. Dann (...) ich war nächste Woche dort. Und ich habe so, also: 'Hey, ich war nicht im Krankenhaus. Ich bin jetzt, mir geht's schlecht! Und ich kann nicht (...)' #00:22:34-2#

I: hm (bejahend) #00:22:34-2#

B: Die haben gesag: 'Ja ok, ich schreibe nochmal.' Ich habe gesagt: 'Bitte, schreiben Sie!' Und (...) nochmal haben, eine Abend war, ich war beim Uniklinikum. War kurze Zeit, zwei, drei Stunden. Arzt hatte auch (...) Griechenland mit mir geredet. Und ich wusste nicht. ich konnte nicht Englisch auch. #00:22:54-0#

I: Ja. #00:22:55-1#

B: Kein Dolmetscher, keine (...) keine Afghaner zum Beispiel sagt: 'Also, ich wohne seit drei Jahre hier.' Ich meine ein Person #00:23:02-9#

I: Ja. #00:23:02-9#

B: Damit ähm, ich konnte denen erklären, was passiert mit mir. #00:23:09-0#

I: Ja. #00:23:09-0#

B: Dann einfach so, und nochmal bin ich in Gefängis gekommen. Mit drei Polizei. Und meine Hand war fest, nur wegen ich war schwarz in Griechenland. ähm, es gab, keine Ahnung, Millionen Leute in ähm, in Griechenland #00:23:25-5#

I: Und du bist festgenommen worden? #00:23:26-1#

B: Ja. UNd ich hab gesagt, also ähm, ich, ich ähm, ich weg. Ich gehe nicht weg. Weil, ich brauche Arzt. Ich muss zum Arzt gehen. Dann, eine war ähm, zwei wie Security oder so, eine war auch re- linke Seite und eine war auf rechte Seite, eine war hinten. Und meine Arm war fest. Die haben gesagt, also ich war auch, die wollen eigentlich ähm den ähm, mein Blut abnehmen. #00:23:53-1#

I: hm (bejahend) #00:23:53-1#

B: Und die haben gesagt, also, die Krankenschwester hatte gesagt, machen Sie, ähm, seine Hände auf, und Arzt zu mir geagt, dann darfst du nicht ähm, auf Griechisch gesagt, aber ich hab nicht, man merkt #00:24:05-8#

I: hm (bejahend) #00:24:05-8#

B: dass über den Kontakt ich brauch oder so. #00:24:09-0#

I: Ja. Körpersprache #00:24:10-2#

B: Körpersprache, ja.  #00:24:11-3#

I: hm (bejahend) #00:24:11-9#

B: Und, wir haben hald, war wirklich schlimm. (...)  #00:24:17-5#

I: ähm, wie bist Du dann nach Deutschland gekommen? #00:24:19-3#

B: Ich, ähm, ich bin eigentlich dreimal von Griechenland nach, ähm, Mazedonien (...) dreimal zwei Tage zu Fuß. Das heißt sechs Tage ich war nur zu Fuß #00:24:34-0#

I: Ja #00:24:34-0#

B: Im Wald.  #00:24:35-3#

I: Oh? #00:24:36-5#

B: Ja. Dann, nochmal die Polizei hat uns ähm (...) festgenommen, wieder in Griechenland. Nochmal in die, ähm, Mazedonien. Dann wieder in Griechenland. Dann ich war nochmal, zum vierten Mal bin ich geschafft. #00:24:52-5#

I: hm (bejahend) #00:24:52-7#

B: ähm, also nochmal in Mazedonien. Im, ähm, Hauptstadt von Mazedonien heißt Budapest? Oder ich weiß es nicht (...) ähm, ich war schon dort #00:25:05-0#

I: Ich weiß jetz nicht wie's heißt #00:25:05-0#

B: Ja. Dann ähm, bin ich noch ein Monat, ich war im Gefängnis. Nur wegen schwarz. #00:25:12-1#

I: hm (bejahend) #00:25:13-0#

B: Das war nur ich hab Pech gehabt. Ich hab paar Leute zum Beispiel, auch Freunde von mir, die waren drei Monate, zwei Monate, so (...) #00:25:21-0#

I: hm (bejahend) #00:25:21-6#

B: Ich war nur, elf Monate. #00:25:21-8#

I: hm (bejahend) #00:25:22-3#

B: Ich war drei Monate im Türkei. Dreieinhalb Monate in, ähm, Griechenland, und eineinhalb Monate, oder ähm, oder mehr in Mazedonien. Ich war einenhalb, einen Monat oder mehr als einen Monat ich war nur im Gefängnis. Dort, ich hab zwei, dreimal habe ich geschlagen (...) geworden, oder wie nennt man? #00:25:44-7#

I: Bist du geschlagen worden. #00:25:44-9#

B: Ja. #00:25:48-8#

I: hm (bejahend) #00:25:48-8#

B: Und ich war beim Polizei und im Gefängnis, ich habe gesagt, also ich bin krank, und (...) ich brauche Medikamente. Die haben mir geschlagen: 'Hey, gehst du weg!' Und: ' jeder kommt zu uns uns sagt ich bin krank, ich bin krank.' Ich habe gesagt, dass, wirklich, ich habe meine Niere transplantiert. Er hat gesagt: 'Also, kommt (...) alle Leute und die sagen: 'Wir sind krank! Wir sind krank!'. Warum kommt Ihr einfach nach Europa? Was, was wolltet Ihr von Europa? Und, einfach so, geschenkt' und keine Ahnung, so (...) beim Arzt, ich war in , in, im Mazedonien. Mazedonien. Ich war auch beim Arzt. In Mazedonien, die haben ähm, eigentlich zwei offizielle Sprache. Türkisch und (...) Mazedonien, ganz normal #00:26:36-0#

I: hm (bejahend) #00:26:36-3#

B: Ich konnte schon ein bisschen Türkisch, weil ich war drei Monate in Türkei. #00:26:40-1#

I: hm (bejahend) #00:26:40-1#

B: Und ich hab, mit Arzt hab ich schon geredet, also ich bin krank, und ich hab (...) Niere transplantiert. und dann hat der gesagt:  'Wie viel Geld hast du?' Ich hab gesagt: 'Ungefähr 50, 60€.' #00:26:51-7#

I: hm (bejahend) #00:26:52-2#

B: Sagt er: 'Nein, das geht nicht. Also, gehst du nochmal.' #00:27:00-7#

I: (...) #00:27:00-7#

B: Und, ich bin wieder im Gefängnis dann, ich war (...) was? eineinhalb Woche, dann wir waren bei Gericht, warum wir sind schwarz gefahren. Dann (...) Ich war mit eine Familie, die waren schon drin #00:27:12-9#

I: hm (bejahend) #00:27:13-4#

B: Die waren Araber #00:27:14-3#

I: hm (bejahend) #00:27:14-3#

B: Und die waren schon drin. Und, weil die haben kein Dolmetscher auf Persisch, dann ich bin einfach draußen gewartet. Und ich hab so, also, ich muss zum Gericht, ähm, zu Richter muss ich sagen, also ich bin krank! Ich kann nicht in Gefängnis! Ja und bin ich einfach gegangen zu, ähm, Gericht #00:27:35-4#

I: hm (bejahend) #00:27:35-8#

B: Uund (...) zum Richter, tschuldingung. #00:27:36-9#

I: Ja. #00:27:36-9#

B: Dann, ich hab gesagt, also, ähm, war ein Frau, und ich hab gesagt: 'Ich bin krank.' Auf Englisch. Ich habe schon ein bisschen gelernt. 'I am sick' und 'I am (...)' keine Ahnung, dann 'Kidney transplantiert' und so, so. Ein bisschen hab ich auf Englisch mit denen geredet.  #00:27:55-7#

I: Ja. #00:27:55-7#

B: Ich konnte schon ein bisschen. Dann hat die so: 'Hä? Hast du transplantiert, und (...) bist du schon in Gefängnis?' Ich habe gesagt: 'Ja, leider.' Und ähm, 'wenn geben Sie mir keine Medikamente ich kann schon ein Jahr in Gefängnis bleiben. Aber mit Medikamente.' Sonst konnte ich nicht. #00:28:13-5#

I: hm (bejahend) #00:28:13-5#

B: Dann ähm, (...) ode rentweder ich mache Selbstmord? Oder wie nennt man, so? #00:28:19-4#

I: Dann bringst du dich um? #00:28:21-9#

B: Ja, 'Ich bringe mich um!' Und hatte gesagt: 'Nein, du kommst auf jeden Fall, also ich schreibe jetzt, ja du bist krank!' Und so, Sie hat geschrieben. Und wir waren fast noch ein Woche dort. Fast ein Monat. Danach ähm(...) von Gefängnis bin ich raus gekommen. Nochmal, ähm, geflüchtet, nach Serbien. #00:28:44-7#

I: Diesmal nach Serbien. #00:28:45-4#

B: Ja, von Mazedonien nach Serbien. #00:28:48-2#

I: hm (bejahend) #00:28:48-5#

B: Dann ich war auch schon dort, und ich wusste nicht eigentlich, wo muss ich hingehen. Ich war ganz alleine. Alleine, alleine! Und dann ich, ähm, Sprache undso, ich hatte ungefähr (...) 20€. Dann (...) meine Familie konnten auch nicht zu mir schicken. Weil ich hatte keinen Pass. Kann man nicht mit Western Union schicken undso. #00:29:12-4#

I: Hattest du noch Kontakt zu deiner Familie dann? #00:29:16-2#

B: Ne. (...) Dann, ich ähm, ich hab, zwei, drei Afghaner hab ich in Serbien gesehen.  #00:29:23-9#

I: hm (bejahend) #00:29:23-9#

B: Dann einer aus Marokko glaube ich. Ich hab ähm, ud ähm (...) ich weiß nicht, ein Land auf ähm, auf ähm (...) auf Deutsch (...) kennst du Al-Jazāʾir? (->Algerien) #00:29:37-8#

I: Al Jazeera? #00:29:39-1#

B: Ja, Al Jazeera! Kennst du? #00:29:41-0#

I: Der Nachrichtensender? #00:29:41-0#

B: Ja, ja, ja. #00:29:41-8#

I: Ja, kenne ich. #00:29:42-7#

B: ähm, das ist eine Land, und die Leute aus Al-Jazāʾir #00:29:47-7#

I: hm (bejahend) #00:29:47-7#

B: Und ich hab denen auch gefragt: 'Kann ich mit, ähm, eure Handy anrufen?' #00:29:52-3#

I: hm (bejahend) #00:29:52-3#

B: Die haben 'Ja' gesagt. Und ich hab mit meine Familie hab ich Kontakt gehabt. #00:29:55-4#

I: Das ist sehr nett von denen. #00:29:57-0#

B: Ja, war wirklich. Und ich hab mit paar Afghaner hab ich schon geredet, und (...) ich war fast drei Woche, ich war trotzdem im Hafenstadt. Aber ich hatte kein Geld, und ich war wirklich manchmal zwei Tage (...) oder drei Tage habe ich nix gegessen. Und ich konnte, ich habe keine (...) keine so öffentlichen Mensch, und ich komme zu dir und ich sage: 'Hey, ich bin hungrig! Kannst du Geld mir geben?' Dann, ich war wirklich drei, vier Tage ich war hungrig. #00:30:26-5#

I: Ja. #00:30:27-4#

B: Aber trotzdem bin ich nicht zu Menschen gegangen und ich sage, ähm, kannst du mir Geld geben oder kannst du mir Essen geben oder so? #00:30:32-9#

I: //hm (bejahend) #00:30:32-9#

B: Und ich habe nochmal ein Afghaner gefunden, und ich hab gesagt: 'Also ich bin ein Woche hier und(...) ich finden niemanden, ich weiß nicht, was soll ich machen?' Ich hab draußen geschlafen. (...) Und(...) ähm wirklich, in Serbien war nur regnen und kalt. Richtig kalt. #00:30:34-9#

I: hm (bejahend) #00:30:34-9#

B: Dann (...) bevor ich nach #00:30:40-5#

I: //Da war's dann schon Winter mittlerweile oder? #00:30:40-5#

B: Ja. ja. #00:30:43-1#

I: hm (bejahend) #00:30:43-1#

B: Dann (...) in Mazedonien hab ich auch (...) bis hier (Handgeste) war Schnee. #00:30:47-8#

I: Baaah #00:30:49-0#

B: Dann, ich war einmal drei Tage (...) habe ich schon gelaufen (...) danach ähm, wenn die Leute aus Syrien sehen, die könne schon zum Polizei gehen, und, ähm, die zeigen seine Papiere #00:31:06-2#

I: hm (bejahend) #00:31:06-6#

B: 'Ich bin aus Syrien'. Dann (...) die würden nicht Abschieben nach ähm Griechenland. Also wenn bin ich Afghaner, wenn ich sage, also ich bin Afghaner, dann die nehmen mich, und die schieben mir wieder nach Afghanistan. #00:31:22-3#

I: Ja. #00:31:22-3#

B: ähm nach ähm (...) Griechenland. #00:31:26-5#

I: Ja. #00:31:26-5#

B: Dann (...) ich war zwei Tage alleine #00:31:31-6#

I: hm (bejahend) #00:31:32-0#

B: und war Schnee bis hier (Handgeste), war halbes (...)ähm halbe Meter. Ich war zwei Tage, und ohne Essen. Ohne garnichts im Wald. Das was war, ey, ich kann nicht wirklich sagen. Dann, einfach bin ich gegangen. Ich hab gedacht, ähm, gedacht: 'Bin ich (...) wie im ähm (...) Horrorfilm' #00:31:53-2#

I: Ja. #00:31:53-2#

B: Ich hab gedacht, jetzt kommt ein, ähm, Wolf oder ein Löwe undso. Ich, ich ähm, hörte schon in Nacht. #00:32:01-5#

I: Ja. #00:32:01-5#

B: Die: 'Whooo'. Dann.. (...) #00:32:06-0#

I:// Ein Scheißgefühl oder? #00:32:07-0#

B: Jaa (lacht) das war oder (...) dann bin ich auf dritte Tag. Ich habe gesagt: 'Jetzt ist scheißegal.' #00:32:15-3#

I: Ja. #00:32:16-0#

B: Dann bin ich einfach oben gegangen. Dann, ähm, war ein Autobahn, habe ich schon gesehen, dann ich war oben und ich hab, ähm, (...) Einfach oben. Dann, ich habe gesagt: 'Mir ist wurscht was passiert.' Entweder Polizei kommt oder (...) jemand kommt. Ich gehe einfach. Ich war bei eine (...) Trenesstation. Bei eine (...) Deutsch habe ich vergessen. (lacht) Bei eine (...) Stadtion, bei eine (...) #00:32:43-3#

I: ähm, trainstadtion, also ähm (...) Bahnhof. #00:32:49-8#

B: Ja, Bahnhof. #00:32:49-5#

I: hm (bejahend) #00:32:49-9#

B: Und, ähm, bin ich nochmal einfach gesagt: 'Also, ich brauche ein bisschen warm werden. Ein bisschen.' Ich hatte schon drei Jacken, dann zwei Hose #00:33:02-4#

I: (...) #00:33:02-4#

B: und (...) ich war nie warm. Nie. Und mein Schuhe, so kalt. Fast meine Hose, war alle nass. #00:33:13-5#

I: hm (bejahend) #00:33:13-5#

B: Nass. #00:33:13-5#

I: Baaah (...) #00:33:13-5#

B: Ja, so. Und, bin ich einfach weiter, weiter, weiter. Bis hier. #00:33:21-7#

I: Du bist dann quasi nach Deutschland gegangen? (...)  #00:33:28-4#

B: Also, bis ähm, von Österreich #00:33:30-9#

I: Ja?
 #00:33:32-3#

B: Bis ähm, Deutschland #00:33:33-6#

I: Ja. #00:33:33-7#

B: Bin ich einfach mit dem Zug ge, ähm, gefahren. #00:33:37-2#

I: Also du bist bis Ästerreich gegangen, und dann  (...) endlich mim Zug hierher dann. #00:33:43-2#

B: Ja. Nur ÖSterreich bis Deutschland bin ich ähm, nur mit Zug. Sonst alles einfach zu Fuß. #00:33:52-5#

I: Wow. #00:33:52-5#

B: Das ist (...) ja, es ist. Also jetzt, jetzt (...) ich mach mit meinem Leben einfach Spaß. #00:34:03-5#

I: hm (bejahend) #00:34:04-2#

B: Weißt du was ich meine? Ich chille einfach.(lacht) #00:34:07-3#

I: (lacht) Ja? #00:34:08-4#

B: Ich sage nicht, ähm, hey, keine Ahnung: Scheiße Schule, oder Scheiße, keine Ahnung, die Leute, oder (...) ähm Scheiße Arbeit oder scheiße den oder Scheiße den. Ich sag garnix! Ich, ich chille! #00:34:20-3#

I: Die ganze Scheiße hast du schon hinter dir, jetzt ist alles schön? #00:34:23-3#

B: Ja. ja! #00:34:23-3#

I: (lacht) #00:34:23-8#

B: Wenn, wenn wäre kalt. Wenn wäre regnet. Wenn (...) Sonne, wenn (...) keine Ahnung, ich weiß nicht was kommt. #00:34:31-6#

I: hm (bejahend) #00:34:31-6#

B: Ich sage: 'Ja, das ist (...) ich hab schon ähm, jetzt ähm, Wohnung!' Ich ähm sage: 'Hey, (unv.)! Du warst (...) elf Monate auf dem Weg. #00:34:43-5#

I: hm (bejahend) #00:34:44-4#

B: Du hast immer gewünscht. Und jetzt wenn du arbeitest, zum Beispiel acht Stunden im Tag. Ist wurscht! #00:34:51-8#

I: hm (bejahend) #00:34:51-8#

B: (klatscht) Dann gehst du am Abend nach Hause und schläfst und kochst du, was du willst!' #00:34:53-9#

I: Ja. #00:34:55-8#

B: Dann (...) #00:34:58-7#

I: (lacht) #00:35:00-7#

B: Chillig. #00:35:00-7#

I: Phew. #00:35:00-7#

B: Ja, es ist so! #00:35:02-0#

I: Ja. #00:35:02-8#

B: Also, ähm, mitte 2015 #00:35:13-6#

I: hm (bejahend) #00:35:14-5#

B: Ich war, ähm,  zwei Jahre beim Integrationskurs #00:35:17-2#

I: hm (bejahend) #00:35:17-2#

B: Dann ich hab schon Quali gemacht #00:35:19-7#

I: ähm in dem Kurs? #00:35:20-8#

B: Ne, ich ähm, ich hatte schon B1. #00:35:25-2#

I: hm (bejahend) #00:35:25-6#

B: Danach habe ich ein Quali ähm, nach dem B1 #00:35:28-9#

I: hm (bejahend) #00:35:29-5#

B: Habe ich eine Ausbildung bekommen #00:35:31-2#

I: //Was denn für eine? #00:35:31-2#

B: Als, ähm, als ähm (...) Elektriker #00:35:34-6#

I: hm (bejahend) #00:35:34-6#

B: Als Energietechnikgebäude #00:35:37-0#

I: ähm wie hastn die Ausbildung bekommen? #00:35:39-0#

B: ähm (...) also ich, ich hab, ich, ich hab schon im Iran als Elektriker gearbeitet. #00:35:44-3#

I: Ah, okay. #00:35:45-2#

B: Und ich hab schon Praktikum gemacht bei verschiedene Firma #00:35:47-8#

I: hm (bejahend) #00:35:48-3#

B: In Regensburg #00:35:49-5#

I: hm (bejahend) #00:35:49-5#

B: Dann ich hab schon eine Ausbildung bekommen #00:35:52-4#

I: hm (bejahend) #00:35:52-4#

B: Das war eine kleine Firma, und ich wusste nicht, muss ich zu Firma sagen, dass ich eine Schwerbehindertenkarte habe. #00:35:58-6#

I: hm (bejahend) #00:35:59-0#

B: Das ist 50\%.  #00:36:03-1#

I: hm (bejahend) #00:36:03-6#

B: Dann (...) ähm, bevor ich unterschreibe, dann ich hab zu (...) zu Chef hab ich gesagt: 'Also ich hab schon eine (...) Schwerbehinderung Ausweis.' Ich hab schon Schule angemeldet, ich war zwei Woche in der Arbeit beim Ausbildung, meine ich. #00:36:17-2#

I: Ja. #00:36:17-2#

B: Also, zwei Wochen war ich in der Arbeit, und ich hab gesagt: 'Ja, machen wir irgendwann den ähm, (...) den (...) ähm Vertrag unterschreiben?' - 'Ist wurscht, ist kein Problem!' (...) Dann ich hab Schule angemeldet, ich war ähm, zwei Tage in die Schule #00:36:35-4#

I: hm (bejahend) #00:36:35-8#

B: Und, bin ich dort angemeldet und ich war zwei Tage in der Schule, dann bevor ich unterschreiben, dann ich hab gesagt so: 'Hey, ich hab Schwerbehinderungausweis.' Und er hat gesagt: 'Echt?' Und ich hab geagt: 'Ja.' Und er hat gesagt: 'Also leider kann ich nicht ähm, dich nehmen.' #00:36:55-2#

I: Was? #00:36:56-8#

B: Er hat gesagt: 'Ich habe Angst, nur wegen deiner Schwerbehinderungausweis. Wenn morgen kann ich nicht ähm, zum Beispiel ähm(...) wenn du morgen krank wirst, dann, was soll ich machen?' Ich hab gesagt: 'Also ich bin körperlich fit. Ich kann schon arbeiten. Ich bin (...) ich, ich weiß nicht, wie soll ich sagen? Ich, ich spiele jetzt in drei Vereine! #00:37:18-6#

I: Ja. #00:37:18-6#

B: Ich, ich spiele Fußball. Und ich, ich spiele zwei Vereine Volleyball. #00:37:24-6#

I: (Pfeifft) #00:37:25-9#

B: ja. Und (...) ich bin (...) als normaler Mensch, ich merke, ich bin noch gesünder. #00:37:31-8#

I: Ja. #00:37:31-8#

B: Weil ich mache jeden Tag Sport, ich mach alles. Und ich spiele jetzt beim Bezirksliga. Beim, ähm, Sportclub. #00:37:36-8#

I: hm (bejahend) #00:37:37-3#

B: Dann hat er gesagt: 'Nein, ich denke nur ähm, von später, 10 Jahre. Wenn du krank wirst.' Ich habe gesagt: 'Ja, wenn Sie wollen nicht mich haben, dann tschüss, passt schon.' #00:37:51-1#

I: Ja. #00:37:51-1#

B: Dann bin ich nochmal (...)ähm mit dem Aus(...)wahl eigentlich September. Alle Schüler war, ähm, alle Klasse war voll. #00:37:59-3#

I: Ja. #00:38:00-0#

B: Alle Klassen sind voll. #00:38:00-2#

I: Ja. #00:38:01-4#

B: Dann ich hab gesagt: 'Was soll ich machen jetzt?' #00:38:03-8#

I: Wo bist du dann hin? #00:38:05-7#

B: ähm, dann, ich war so Zuhause und mein, ähm, ich hab (...) mit Frau (unv.), die (...) Ich war bei Frau (unv.) in Wohngruppe. Oben. Beim, ähm, betreutes Wohnen. #00:38:41-7#

I: ähm, hier direkt im Haus. #00:38:42-9#

B: Da wo (unv.) wohnt, ja. #00:38:45-0#

I: Ah ja. #00:38:45-0#

B: Ja. #00:38:46-3#

I: Ja. #00:38:47-4#

B: Dann ähm, ich hab ähm (...) die Chefin von dem Haus gefragt, also was soll ich machen? Sie hat gesagt: Machst du nochmal B1. #00:38:59-8#

I: hm (bejahend) #00:39:00-5#

B: Ich habe gesagt: 'Hä? Has ist komisch! Ich war schon ein Jahr bei B1 Kurs!' #00:39:05-2#

I: hm (bejahend) #00:39:05-6#

B: Sie hat gesagt: 'Ich weiß nicht, was du mach (...) mach Praktikum.' Ich habe gesagt: 'Ein Jahr lang?!', frage ich. 'Ohne Geld?' Das ist (...) #00:39:16-5#

I: hm (bejahend) #00:39:16-5#

B: Und (...) Sie hat gesagt: 'Ich weiß nicht!' Ich hab einfach so (...) Ja, Schule habe ich schon gegangen. Dann, ich wollte schon eigentlich Quali machen. Ich hab schon Quali gemacht, aber leider ich hab nicht bestanden. Ich hab nur ein Schulabschluss hab ich schon gekriegt. #00:39:31-3#

I: Ja. #00:39:32-4#

B: Ja. Bevor (...) den Prüfung ich hatte so viele Stress (...) mit Chefin im Haus, und noch mit paar in, mit Lehrer und so Lehrerin. #00:39:42-1#

I: Was war denn da? #00:39:45-5#

B: Ja (...) (unv.) ich, ich konnte nicht einfach so (...) wie soll ich sagen? Ich hab(...) mit denen gesprochen, ich hab gesagt: 'Also, ich war nicht in der Schule' und ähm, bei mir Mathematik ist (...) ein bisschen schwer. #00:40:01-7#

I: Ja. #00:40:01-7#

B: Dann ähm, zum Beispiel (...) mit Computer beim ähm(...) nicht Technik, sondern beim, ähm, Information oder so? #00:40:11-4#

I: Informatik? #00:40:11-4#

B: Informatik! Dann mussten wir eigentlich im, mit Computer Bewerbung schreiben undso. #00:40:17-2#

I: Ja. #00:40:17-2#

B: Dann, ich hab ähm, mit Frau (unv.) habe ich schon geredet. Ich hab gesagt: 'Also ich brauche ein Laptop. Damit zu Hause ein bisschen Schreiben lernen' und, weil Prüfung haben wir 45 min, da müssen wir alles fertig machen. #00:40:28-2#

I: hm (bejahend) #00:40:28-7#

B: Ein Bewerbung schreiben und eine Kochprojekt müssen wir machen, ähm schreiben, weil zum Beispiel was müssen wir kochen, erklären, warum (...) und eine trad(...) traditionale Essen undso. #00:40:41-0#

I: hm (bejahend) #00:40:41-4#

B: Sie hat gesagt: 'Also ich kann nicht von jede kommt her und sagt: 'Ich brauch das'. Ich kann nicht einfach vorbereiten.' #00:40:46-7#

I: Ja. #00:40:47-5#

B: Ich hab gesagt: 'Sie müssen nicht für mich selbst kaufen.' #00:40:51-0#

I: Ja. #00:40:51-0#

B: Das ist, Sie kaufen eine von Gruppe (...) wenn ich brauch heute, dann benutze ich. Wenn nicht, dann gebe ich zum Beispiel die andere Person weiter. #00:40:58-2#

I: Ja. #00:40:58-2#

B: Hat sie Gesagt: 'Nein.' Nicht Nein, 'Ja, mache ich! Ich schreibe beim Campus Asyl, ich schreibe beim (...) BFZ, ich schreibe beim BSZ.' Das ist die, die, die Leute, die haben zum Beispiel paar Laptop oder Computer, vielleicht kriegen wir geschenkt oder vielleicht nicht. #00:41:15-1#

I: Ja. #00:41:15-1#

B: Und (...) ich weiß nicht, das ist (...) vielleicht manche Leute denken, ja, ich bin dumm. Aber ich bin nicht dumm. #00:41:20-1#

I: Ne. #00:41:21-1#

B: Ich war nicht in die Schule. Das ist (...) #00:41:23-9#

I: Das ist ein riesen Unterschied. #00:41:24-6#

B: Ja, und (...) ich bin mit dem Quali angefangen. Ich war nicht schlecht. Ich bin, ich bin (...) in Sport habe ich Note eins gekriegt, beim ähm Kochen habe ich Note eins (unv.), wirklich zwei. Wir haben, ähm, Projekt gemacht und ich hab wirklich, hab garnix aufgeschrieben. Ich hab, ähm, über die (...) über die Steuer #00:41:17-0#

I: hm (bejahend) #00:41:17-0#

B: ähm habe ich schon erklärt. Habe ich ein Projekt gemacht. #00:41:20-5#

I: Über die Steuererklärung? #00:41:22-2#

B: //Es ist echt egal, zum Beispiel, ähm, es gibt ähm, sechs Klasse Steuer. #00:41:25-7#

I: Ja? #00:41:25-7#

B: Wenn welche, welche Stufe zum Beispiel die Leute sind. Single, die Leute sind (...) ähm heiraten, die Leute Partner haben undso. Ich hab schon (...) #00:41:45-5#

I: Da weißt du mehr über Steuern als ich(lacht) #00:41:49-0#

B: Ah, naja.(lacht) Dann (...) also, wir müssen auch im Computer aufschreiben. Und im Chip einfach alles, ähm, speichern und dort müssen wir zeigen wie ein, ähm, ich hab vergessen die Name? (...) Sollte man vor dem Lehrer einfach, weißt du, das ist ähm (...) vorstellen. Vorstellen. Vorstellen. #00:42:10-6#

I: Ja.  ähm, also du hast des vor'm Lehrer vorgestellt? #00:42:12-6#

B: Ja! Und ich habe gesagt, ich habe schon paar geschrieben. #00:42:17-8#

I: hm (bejahend) #00:42:18-5#

B: Ich hab nur paar Blätter am Anfang zum Beispiel über (...) über die Steuer, also das, das, das (...) und (...) es gibt sechs Klasse Steuer, und es gibt, ähm, über Versicherung musste ich auch erklären. #00:42:31-9#

I: hm (bejahend) #00:42:32-4#

B: Zum Beispiel Rentenversicherung, die Pflichte. Unfallversicherung, Arbeitsversicherung, die sind, ähm, richtige. Und ich hab schon erzählt, ein bisschen. #00:42:42-7#

I: Ja. #00:42:42-7#

B: Habe Blätter, hab ich schon in Chip, hab ich schon gespeichert. Und ich hab, zu Lehrer hab ich gesagt: 'Also, ich war nie in der (...) in der Schule im Iran. Und (...) ich hab, mit dem Zeit hab ich nicht geschafft.' #00:42:58-2#

I: Ja? #00:42:58-2#

B: 'Also, mit mündlich kann ich schon Ihnen erzählen.' #00:43:02-7#

I: Ja. #00:43:02-7#

B: 'Sonst, wegen Computer, habe ich leider nicht in, ähm, Chip gespeichert.' #00:43:07-7#

I: Ja. #00:43:07-6#

B: Dann Sie: 'Ja, okay, erzählst du.' Und ich hab schon ein bisschen erzählt. Und die haben gesagt: 'Ja okay, wir wissen, dass du weiß #00:43:15-3#

I: hm (bejahend) #00:43:15-3#

B: dass ähm, (unv.), aber wir wollen auch ähm, mit dem, diese Chip, dass du zeigen im, mit Biro. Bim. Beamer. #00:43:25-6#

I: Beamer? #00:43:25-6#

B: Ja. Auf den Wand du musst zeigen, das ist Fotos. Zum Beispiel, es gibt sechs, ähm, Klasse Steuern. Und es gibt Versicherung, und so und so und so. (...) Ich hab schon, ja, bei Geschicht und Mathe, ich war wirklich so(...) einfach böse auf (...) weil, ich konnte schon schafen. Wirklich, ich hab so fleißig gelernt. Unglaublich. Ich hatte jeden Tag Nachhilfe. #00:43:49-2#

I: hm (bejahend) #00:43:50-2#

B: Dann, bevor den Prüfung, dann (...) ich hatte keinen Bock. #00:43:53-9#

I: hm (bejahend) #00:43:54-2#

B: Und ich hab gesagt: 'Nein.' Zum Beispiel ich brauche Laptop, ich hab am Anfang des Jahr hab ich gesagt, also, unsere Lehrerin hat gesagt, ähm: 'Ihr braucht Laptop!' #00:44:04-8#

I: hm (bejahend) #00:44:05-1#

B: 'Also, jetzt müsst ihr mit dem Laptop langsam anfangen zum schreiben und lernen. Ihr habt nur 45 Minuten Zeit und müsst Ihr schreiben.' #00:44:15-3#

I: hm (bejahend) #00:44:15-3#

B: Und (...) ähm, die Verena kennst du auch schon? Sie war auch, ähm, Verena, die Betreuerin von (unv.). #00:44:23-6#

I: ähm(...) #00:44:24-6#

B: Verena. #00:44:26-1#

I: Kann sein, ja. #00:44:26-1#

B: Ja, ja. Sie war, ähm, bevor die Spiegel ähm, festmachen, Sie war auch schon da. #00:44:31-7#

I: Ach ja, die, ja. #00:44:32-1#

B: Die Mädel. #00:44:33-6#

I: Ja. #00:44:33-6#

B: Und (...) SIe hatte auch einmal, zweimal Ihre Laptop geholt. #00:44:37-8#

I: hm (bejahend) #00:44:38-3#

B: Und ich hab gesagt, mit zwei, dreimal geht nicht. #00:44:40-1#

I: Das reicht nicht, ja. #00:44:42-0#

B: Das ist (...) ich kann nicht speichern, ich kann nicht ähm, zum Beispiel, wie kann man den Buchstabe hier klein machen, groß machen, und(...) zum Beispiel, wie kann man drucken? Wie kann man (...) Oh, das ist ein Computer, wenn du kannst, ähm(...), das ist, ähm, für dich ist kein Thema! #00:44:57-2#

I: Wenn ma's gewöhnt ist is es ganz einfach, abr du brauchst des, du brauchst des am Anfang, damit du dich dran gewöhnen kannst. #00:45:05-9#

B: //Ja. (...) Also ich hab nur auf der, ähm, Laptop, ähm, ERB. A und C und D und (...) Hey, es ist so: #00:45:15-8#

I: //Ja? #00:45:17-1#

B: Und ich hab gesagt, wirklich, das ist sehr nett von dir, wenn du deinen Laptop bringst, aber jetzt zu  spät. #00:45:26-1#

I: Ja. #00:45:26-1#

B: Ich hab einfach gegangen in die Schule bei Prüfung, beim ähm(...) oh, ich hab den Namen schon wieder vergessen (lacht). Beim, ja, ich hab, ähm ich war schon dort. Die alle Fragen von Mathe. Alle #00:45:42-6#

I: hm (bejahend) #00:45:42-6#

B: Die waren (...) alle in mein Kopf. Das hätte ich nur Probleme, wegen Zuhause, wegen mein Lehrerin. Und ich konnte nicht schreiben. Ich wusste schon #00:46:22-6#

I: Mmh #00:46:22-6#

B: Wie, zum Beispiel, wie Pi man rechnet. #00:46:25-4#

I: hm (bejahend) #00:46:26-0#

B: Wie man zum Beispiel beim Geometrie  #00:46:28-5#

I: hm (bejahend) #00:46:28-9#

B: Ich bin, ich war wirklich in der Schule der beste beim Geometrie, ganze Jahr! #00:46:33-7#

I: Ja. #00:46:34-9#

B: Ich hab immer noch, das war  (...) eins, zwei, eins. Bevor ein Monat bevor anfangen zum, ähm, (...) ähm Prüfung, wir haben eine (...) Vorbereitung von Prüfung haben wir schon gemacht. Dann (...) ich habe immer vier. Vier. #00:46:50-5#

I: Ow. #00:46:50-5#

B: Und, die haben gesagt: 'Hey!' Meine Lehrer hatten mit mir gestreiten, 'Was ist los mit dir? Also jetzt, Prüfung, musst du anfangen!' Ich habe gesagt: 'Leider, ich kann nicht.' #00:47:00-3#

I: hm (bejahend) #00:47:00-3#

B: Ich, mit dem, mit dem (...) Kopf bin ich nicht bereit. #00:47:01-8#

I: Ja. #00:47:01-8#

B: Dann (...) Ja. #00:47:04-4#

I: (...) #00:47:05-0#

B: Leider nicht geschafft. #00:47:06-4#

I: Shit. #00:47:06-4#

B: Ja, mit, ähm, Deutsch als zweite Sprache hätte ich ähm(...) drei glaube ich? #00:47:11-4#

I: hm (bejahend) #00:47:11-4#

B: Sonst Mathe und ähm(...), Mathe und Geschichte habe ich nicht bestanden. #00:47:16-7#

I: hm (bejahend) #00:47:17-1#

B: Ja, das ist(...) #00:47:18-7#

I: Ja, dann reichts ned. #00:47:19-4#

B: Ja, das ist (...) muss 3,0 sein, damit du Quali bestehst. #00:47:25-0#

I: (...) ähm #00:47:30-2#

B: Aber Haubtsache ich hab zu viel (unv., Geld?) (lacht) #00:47:28-6#

I: Jetzt hast du's auf jeden Fall im Kopf #00:47:31-5#

B: Ja, schon. #00:47:33-6#

I: Und(...) #00:47:33-6#

B: Und danach habe ich auch, ähm, B2 habe ich auch schon gemacht. (...) Das ist ein Niveau von ähm deutsche Sprache. #00:47:40-6#

I: Ja, ich weiß. ähm, du sprichst auch gut #00:47:44-1#

B: //Also, wenn man ähm, wenn man B2 macht, dann darf eigentlich ähm, an der Uni gehen. Wenn man zum Beispiel ich hab in meine Heimatland ich studiert #00:47:53-7#

I: Ja. #00:47:53-7#

B: Wenn du hast B2 #00:47:53-7#

I: Ja. #00:47:53-9#

B: ähm, Zertifikat, dann dürftest du an der Uni gehen. #00:47:59-2#

I: Ja? #00:47:59-2#

B: Ich hab, ich hätte keine Studie(lacht) ähm, ich habe mich in meiner Heimatland studiert, dann ich (...) ja, das ist nur einfach B2 Niveau. (...) Deutsch. #00:48:09-5#

I: Okay. (lacht) #00:48:11-5#

B: Mhm. (lacht) So (...) #00:48:13-4#

I: ähm, und wie hast du denn dann die andere Ausbildung bekommen? #00:48:17-8#

B: Also, jetzt ähm mache ich nicht Ausbildung #00:48:20-6#

I: hm (bejahend) #00:48:20-6#

B: jetzt mache ich arbeiten als, ähm(...) nochmal als Energietechnikgebäude, als Elekriker. #00:48:27-8#

I: ja. #00:48:28-8#

B: Dann (...) wir haben so gesprochen mit Chef #00:48:32-4#

I: hm (bejahend) #00:48:32-9#

B: Dass ich zum Beispiel ein Jahr ganz normal als Mitarbeiter #00:48:37-6#

I: hm (bejahend) #00:48:37-6#

B: dort arbeite. Danach muss ich, wenn ich kann die Material von Elektriker, die Namen von dene. #00:48:45-6#

I: hm (bejahend) #00:48:46-2#

B: Danach dürfte ich anfangen zum Ausbilden. Vielleicht sechs Monate, bis September #00:48:52-3#

I: hm (bejahend) #00:48:52-3#

B: Wenn September also ich sage: 'Also ich bin bereit!' #00:48:56-3#

I: Ja. #00:48:56-3#

B: Und ich kann schon, dann ab September. #00:48:59-4#

I: Ja. #00:48:59-4#

B: Wenn ich sage: 'Nein (...) ähm, ich bin noch nicht bereit.' Dann darf ich bis nächste September normal arbeiten, ich hab ein Vertrag von zwei Jahre. #00:49:08-0#

I: hm (bejahend) #00:49:08-4#

B: Danach, wenn (...) ich bin bereit, ich kann zum (...) zu Chef gehen, und sage: 'Also ich bin bereit.' Dann kann ich arbeiten. Weil die Material und Elektriker, es ist richtig schwer. Das ist ein (...) andere Fachsprache. #00:49:21-3#

I: Ja. ähm(...), aber, des heiß du lernst jetz die ganzen Begriffe, und dann machst du des? #00:49:28-0#

B: Ja. Ja. Weil, ähm weißt du: (...) ähm wenn du, wenn du hingehst als Elektriker, machst du Ausbildung. Das heißt, für dich ist nicht schwer. Weil du kannst schon die Sprache. #00:49:39-2#

I: hm (bejahend) #00:49:39-2#

B: Und du kannst zum Beispiel, was bedeutet Spitzzange, was bedeutet Zange, und solche Sachen. Weißt du schon. #00:49:45-1#

I: Ja. #00:49:45-1#

B: Aber bei mir, wenn ich weiß nicht. #00:49:47-1#

I: //Das sind komplett (...) #00:49:49-1#

B: Bei mir werden noch schwieriger. #00:49:46-4#

I: Das sind komplett neue Begriffe für dich. #00:49:48-9#

B: Ja. Und (...) ich muss zuerst die Name von, ähm, Material lernen. #00:49:54-7#

I: hm (bejahend) #00:49:54-7#

B: Und auf der andere Seite in die Schule #00:49:56-0#

I: hm (bejahend) #00:49:56-0#

B: kommt Physik, ich (...) ich kenne nicht Physik. Was bedeutet Physik? Ich kenne nicht zum Bei(...) ähm, sozial schon, aber  wenn du die Technik zum Beispiel für Elektriker(...), ähm Biologie haben wir keine: Physik, Chemie glaube ich haben wir. #00:50:12-1#

I: hm (bejahend) #00:50:12-1#

B: Dann Technik, dann (...) ich weiß nicht, was kommt noch? Viele Fächer, wenn Elektriker. #00:50:17-7#

I: hm (bejahend) ähm, sind die Leute in deiner neuen Arbeit, sind die nett? #00:50:24-2#

B: Ja. (...) ja, ein bisschen. (...) Also, wir sind, ähm, ungefähr, ähm, 16 Leute.  #00:50:35-7#

I: hm (bejahend) #00:50:36-6#

B: Dann, es gibt ähm(...) eine Afgh(...) ähm, zwei Afghaner. Einer hatte Ausbildung gemacht. Aus meine Stadt. Er ist aus meine Stadt. #00:50:43-3#

I: //Ah! (...) Ah, schön!(lacht) #00:50:44-2#

B: Ja, das ist so, und ähm, (...) noch eine aus dem Irak. #00:50:31-8#

I: hm (bejahend) #00:50:32-5#

B: Er ist schon ein bisschen anstrengend, ähm(...), er ist seit acht Jahre in Deutschland. Für Ihn auch ist schwierig. Er ist (...) ich glaube 40 Jahre alt #00:50:43-0#

I: hm (bejahend) #00:50:43-5#

B: Für Ihn auch ein bisschen schwierig, dass die (...) ich arbeite mit Ihm. Ich weiß, dass (...) wirklich, deutsche, deutsche Sprache ist nicht so einfach. Und (...) er arbeitet seit fünf Jahre mit diesen Firma, da wo ich gehe. Und ähm, er konnte nicht gut mir erklären. Dann ich hab seit (...) ein Woche oder so, mehr, ähm ich hab mit Ihm gearbeitet. Es ist ein bisschen, ähm, anstrengend. Ich kann (...) er kann nicht mir erklären. #00:51:07-6#

I: hm (bejahend) #00:51:08-0#

B: Wenn ich frage nach: 'Was bedeutet? Was bedeutet?' #00:51:09-1#

I: Also er kennt die Worte selber nicht? #00:51:11-2#

B: Ja. #00:51:11-2#

I: hm (bejahend) #00:51:11-2#

B: So wie, zum Beispiel: 'Mach diese und diese bis diese' #00:51:15-4#

I: Aber nen Großteil der anderen Leute in der Firma spricht deutsch oder? #00:51:17-6#

B: Ja, ja! Die ähm, meißte sind deutsche. ja. Wir haben ungefähr acht Leute, die sind Gesellen. #00:51:22-7#

I: hm (bejahend) #00:51:22-7#

B: Und ähm, zwei Meister. #00:51:26-5#

I: hm (bejahend) #00:51:26-5#

B: (...) Meister, ja. Und (...) ja. Sind (...) drei, vier auch als Mitarbeiter. #00:51:22-8#

I: Wär's denn nicht einfacher, wenn du (...) wenn du quasi dein Partner (...) also das ist quasi dein Partner zum Arbeiten oder? #00:51:27-0#

B: Ja, jede geht mit eine, ähm, ja. Also, jeden Tag wir treffen uns, ähm, in den Firma #00:51:32-6#

I: Ja. #00:51:32-6#

B: Dann die teilen auf. Also die sagen: 'Also du (...) mit dem. Du (...) mit dem.' Undso. #00:51:38-8#

I: Ah! #00:51:41-5#

B: Also ich darf nicht, ähm, als, ähm, Hilfe, ich darf nicht entscheiden, dass ich mit du gehen. #00:51:46-6#

I: hm (bejahend) #00:51:47-0#

B: Weißt du, das ist wenn Chef sagt, also, ähm, (unv.), du bist mit ähm, (unv.). #00:51:52-8#

I: hm (bejahend) #00:51:53-3#

B: So, du bist mit, ähm, ja. #00:51:52-8#

I: Aber des wird quasi spontan immer für die Baustelle entschieden? ähm, oder immer (...) wie lang ist denn der Zeitraum, wo du dann deinen Partner quasi hast? #00:52:04-0#

B: (...) Vielleicht heute mit Dir, morgen mit (unv.). #00:52:06-6#

I: Okay. #00:52:06-6#

B: Oder übermorgen. Es ist so. Vielleicht halbe Stunde mit dir oder zwei Stunde, drei Stunde. #00:52:11-5#

I: hm (bejahend) #00:52:12-0#

B: Weil, Elektriker immer wir sind unterwegs. Machen wir eine Steckdose hier kaputt #00:52:16-5#

I: hm (bejahend) #00:52:17-0#

B: Und reparieren dann auf der andere Baustelle. (...) Weiß du was ich meine? #00:52:21-1#

I: Ja. #00:52:21-1#

B: Es ist (...) es ist, jeden Tag bsit du unterwegs. #00:52:23-8#

I: Ja, ja. Und wie funktioniert's mit den anderen so? #00:52:29-0#

B: Mit den anderen hab ích kein Problem. Das ist (...) mit ähm, die, die, die Person aus dem Iran. Das ist auch kein Problem. Das ist nur, meine ich, nur wegen, ähm, Verständnis. #00:52:41-0#

I: Ja. #00:52:42-0#

B: Das ist ein bisschen schwer. #00:52:43-4#

I: ähm, im Irak wird auch ne andere Sprache gesprochen oder? #00:52:48-5#

B: Ja, die sprechen Arabisch. Wir sprechen Persisch. #00:52:50-6#

I: hm (bejahend) #00:52:51-1#

B: Wir verstehen uns, wir reden auch Deutsch eigentlich. #00:52:54-4#

I: Ja. #00:52:55-2#

B: Der kann schon Deutsch, aber, ähm(...) er kann nicht zum Beispiel diese Fachbegriffe der Sprache. #00:53:02-7#

I: hm (bejahend) Ja, ja natürlich. #00:53:07-3#

B: Wenn, wenn zum Beispiel ich sage: (...)'Gipsen' oder  (...) zum Beispiel 'Zuleitung', dann sagt: 'Was?' Nicht Gipsen! Der lächelt mich an, sagt: 'Haha! Das ist nicht, ähm, Gipsen! Angipsen!' und ich: 'Ja, okay.' Er merkt nicht, das ist zum Beispiel ich bin (...) ein Monate bei denen. Ich bin nur (...) oder fast ein Monate. Und (...) das ist von ein Monate war acht Tage ähm, Feier(...) nicht Feiertag, sondern ähm, Wochenende #00:53:38-7#

I: hm (bejahend) #00:53:38-7#

B: Acht Tage. Und zwei Tage war Ferien, drei Tage Ferien. Und ich bin ungefähr zehn oder 15 Tage mit dene gearbeitet. #00:53:46-1#

I: Ja. #00:53:46-1#

B: Er muss mir erklären, also (...): 'Das ist Angipsen', nicht 'Haha! Verstehst du nicht, was?' (...) #00:53:55-4#

I: Das ist (...) gnaa #00:53:57-1#

B: ähm nein, ich verstehe ihn! #00:53:58-1#

I: Ja? #00:53:58-1#

B: Weil, in unsere Heimatland, es ist so! #00:54:01-8#

I: Achso? #00:54:03-9#

B: Ja. In, ähm, in asiatische Länder, es ist so! Ich meine, bei mir würde auch genauso, so sein. Aber ich merke schon. Ich bin jetzt in Deutschland, ich weiß, wenn du nicht verstanden, dass du nochmal fragen. #00:54:18-3#

I: Und das darf ma im arabischen ned? (...) #00:54:22-2#

B: (...) Bei uns zum Beispiel, ähm, wenn wir verstehen schon euch Sprache, dann wir sagen: 'Ey, mach diese Kabel von hier bis hier!' Dann fertig. Wenn machst du fertig, dann (...) ja, sagst du: 'Was soll ich machen nochmal?' Dann mach den Kabel  bis hier, dann fertig. Ich meine (...) du fragst immer nach, und fragst du nicht (...) warum machen wir die Kabel bis hier? #00:54:48-0#

I: Ja? #00:54:48-0#

B: Weißt du, aber hier in Deutschland, ich frage: 'Warum?' #00:54:52-1#

I: Ja. #00:54:52-1#

B: In unsere Heimat, ich hab schon zweieinhalb Jahre als Elektriker gearbeitet. #00:54:59-4#

I: hm (bejahend) #00:54:59-8#

B: Ich habe niemals gefragt: 'Warum?' Und ich hab einfach gearbeitet, und ich kann jetzt einfach arbeiten. (...) Ich kann arbeiten. #00:55:11-3#

I: Ja. #00:55:11-3#

B: Ja, es ist so. Und (...) jetzt, zum Beispiel, ich frage nach! 'Hey, warum', zum Beispiel, 'von hier nach da'? Oder: 'Warum von hier?' Sagt: 'Also, das ist von Lampe. Und das ist', zum Beispiel, 'von Gesprächsanlage.' Oder von ähm, keine Ahnung, Thermostat. Oder von dem. Der. Die. Und sagt: 'Spülmaschine.' Ja, so, so, ja.! Ich, ich, ich kenne mich schon! #00:55:34-9#

I: hm (bejahend) #00:55:35-3#

B: Mit dem, ähm, Elektriker. #00:55:36-9#

I: Ja. #00:55:36-9#

B: Nur wegen auf Deutsch wissen. Wenn deine Herd oder deine Spülmaschine kaputt, dann muss ich sagen: 'Ist Spülmaschine.' Ich kann nich sagen: 'ja, den Kabel kaputt.' #00:55:49-2#

I: ja. #00:55:49-2#

B: Oder den kabel kaputt. (...) #00:55:51-2#

I: (...) ähm, wie meinst du? #00:55:51-5#

B: ähm, zum Beispiel, wenn zum Beispiel ähm(...) ich bin ganz normaler Elektriker in Deutschland #00:55:59-0#

I: Ja. #00:55:59-0#

B: Dann (...) ähm, wenn ich komme zu dir, du bist Kunde. Und (...) ich muss di erklären, warum die Kabel ist kaputt. #00:56:06-5#

I: Achso, ja. #00:56:07-2#

B: Ja. #00:56:09-8#

I: Ja. #00:56:09-8#

B: Wenn ich kann nicht Deutsch, und ich weiß nicht was bedeutet auf Deutsch. Ich weiß nur in meinem Kopf. Dann, das ist schwierig zwischen Ich und Du. #00:56:15-8#

I: hm (bejahend) Ja. #00:56:17-1#

B: Jemand muss erklären, warum das Kabel kauptt. #00:56:21-0#

I: Natürlich. #00:56:21-0#

B: Ja. und deswegen, ich weiß, warum kaputt ist #00:56:26-0#

I: hm (bejahend) #00:56:26-0#

B: Aber ich brauche die Fachsprache. #00:56:26-6#

I: Du, du kannst (...) #00:56:26-5#

B: Deswegen ich frage nach. #00:56:28-3#

I: Ja, natürlich. #00:56:29-0#

B: Warum? #00:56:30-6#

I: Ja.  #00:56:30-7#

B: Kaputt ist. #00:56:30-7#

I: Ist auch gut so. #00:56:32-2#

B: Ja, es ist so! Die, ich meine, nur, ich frage oft #00:56:36-0#

I: hm (bejahend) #00:56:36-0#

B: Warum ist so? Warum ist so? #00:56:38-6#

I: So lernst du's! #00:56:38-9#

B: Ja! Es ist so! Ja. #00:56:41-9#

I: ähm(...), aber hast du des dann auch öfter gehabt, dass (...) ähm, keine Ahung, dass die Kulturen zum Beispiel unterschiedlich waren? Also, das Arabische (...) das Arabische und das Deutsche. So grundlegend (...) #00:57:01-3#

B: //Zwischen Deutsche und (...) ähm (...) Persisch oder was meinst du? Weil ich bin Perser. #00:57:04-7#

I: ähm, ja (...) also, ja. Stimmt, das ist Persien. #00:57:08-1#

B: Ja. Zwischen, ähm, Deutsche und persisch meinst du? Oder zwischen Arabische Länder und Deutsche? #00:57:12-7#

I: Zwischen ähm Deutschen und dem Persischen, also dem Iran jetzt zum Beispiel. #00:57:18-6#

B: Es gibt schon, ja. natürlich #00:57:21-7#

I: hm (bejahend) #00:57:21-7#

B: Es gibt schon.  (...) Zum Beispiel, ich hab auch selber gearbeitet. (...) Ich hab hier, (unv.), in, ich meine nicht (unv.), sondern viele Frage in mein Kopf, als ich zum Beispiel meine Eltern gefragt: 'Warum ist das so?'  #00:58:15-0#

I: hm (bejahend) #00:58:15-0#

B: Sagt: 'Das ist haram!' Ich weiß nicht: 'warum dürfen wir nocht Freundin haben?' #00:58:26-7#

I: hm (bejahend) #00:58:26-7#

B: Die sagen: 'Haram!'  #00:58:29-9#

I: hm (bejahend) #00:58:29-9#

B: Die können nicht mir erklären (...) warum. Haram. #00:58:33-6#

I: Ist einfach so? #00:58:34-8#

B: Die sagen: 'Haram!' Weil die haben (...) und eure Eltern auch schon gehört. Meine Eltern (...) und meine Opa, Oma gehört. Die können nicht erklären. Aber hier, wenn in Deutschland, wenn ich sage, zum Beispiel: 'Warum darf man nicht schwarz fahren?' Dann erklärst du! #00:58:48-8#

I: Ja. #00:58:48-8#

B: Deswegen! Dann sag ich: 'Okay.' Ich frag hier mehr 'Warum' als ich in  Iran war #00:58:56-7#

I: (lacht) #00:58:57-1#

B: Ja, das ist, finde ich, es ist so! Muss ich wissen. #00:58:58-3#

I: Ist gut so! #00:58:58-3#

B: Weil (...) ja, ja! Das ist (...) wenn man zum Beispiel eine Sprache lernt oder (...) eure Kultur wissen (unv.)(...) #00:59:07-8#

I: Frag nach! #00:59:07-8#

B: Ja, ja.  #00:59:12-0#

I: (...)ähm, moment dann schau ich dann doch nochmal nach. ähm, mit wem hast du denn (...) hier mittlerweile so insgesamt Kontakt, und hast du auch jetzt wieder Kontakt in den Iran? #00:59:24-3#

B: ähm, ich hab schon eigentlich, ähm, mit 'hier' meinst du mit die Leute hier oder (...)? #00:59:30-6#

I: Mit den Leuten hier in Deutschland. #00:59:30-6#

B: Also(...) ähm, ich mag wirklich Deutsch lernen #00:59:35-9#

I: hm (bejahend) #00:59:35-9#

B: Das ist wie mein Muttersprache. #00:59:37-8#

I: hm (bejahend) #00:59:38-3#

B: Ich mag. Und ähm(...) deswegen (...) ich, ich (...) ich kontaktiere mit alle. Ich meine, alle einfach. Ich, ich mag auf Deutsch reden und reden und reden. #00:59:53-5#

I: Sehr gut. #00:59:54-9#

B: Und ähm, ich gehe auch zum Beispiel, ich hab ähm Freunde beim Fußball. Ich hab Freunde beim Volleyball. (...) Ja, und deswegen, einfach mag ich, ähm, deutsch lernen. Und wenn ich komme hier (EJSA Jugendcafé) auch. Dann wollte ich nur einfach mit euch einfach reden und (...) ähm, ein bisschen Spaß haben. Spielen einfach. #01:00:21-2#

I: Finde ich super. #01:00:21-2#

B: Ja. Ja. Also ich (...) ich arbeite schon, ähm, ich stehe auf um fünf Uhr. Ich gehe in die Arbeit, dann (...)ähm halb sechs muss ich im Hauptbahnhof sein. #01:00:34-8#

I: hm (bejahend) #01:00:34-8#

B: Dann fahre ich zum Firma. #01:00:36-8#

I: hm (bejahend) #01:00:36-8#

B: Dann (...) mache ich acht Stunde Arbeit, bis 16 Uhr. Dann bin ich 17 Uhr zu Hause. Dann nochmal, um mich hier (...) und jetzt bin ich da. Ja. ich mache Spaß einfach. Also, ähm(...) ich mag, ich mache Montag hier Englischkurs bei Frau (unv.) #01:01:00-2#

I: hm (bejahend) #01:01:00-6#

B: Und leider manchmal ich, ich schaffen nicht, nur wegen (...) Zeit. Dann bin ich hier unterwegs. #01:01:07-2#

I: Immer kann's nicht klappen. #01:01:08-8#

B: Ja, dann ähm, Dienstag. Mache ich beim ähm, Volleyball. Ich gehe zum Volleyball. Mittwoch, habe ich schwimmen Kurs. Donnerstag komme ich hier, Freitag habe ich Klettern und Schwimmen. Nach der Arbeit. #01:01:24-6#

I: Du machst verdamtm viel Sport. Und das alles nach der Arbeit. #01:01:28-1#

B: Ja, nach der Arbeit. #01:01:28-1#

I: Wow.(lacht) #01:01:29-0#

B: Ja, es ist so. (...) Gestern war Mittwoch oder? #01:01:34-1#

I: ähm, ja. #01:01:33-8#

B: Gestern, ich war nach der Arbeit an der Uni. Ich hab schon eine Stunde habe ich ähm, ahbe ich Schwimmen gelernt.  #01:01:42-3#

I: hm (bejahend) #01:01:42-3#

B: Dann bin ich wieder nach Hause. (...) Dann, ich war bei meinem Freund bis (...) ähm, fast zwölf Uhr. Danach habe ich geschlafen. Und um fünf Uhr wieder zum Arbeit. #01:01:50-3#

I: (lacht) #01:01:51-6#

B: Ja, es ist so. (...) Ja. Aber ich, ich meine (...) ich bin wirklich, körperlich bin ich fit. Gott sei Dank.#01:01:56-1#

I: Bist du, ja. #01:01:57-1#

B: Mache ich mehr Sport als andere Leute. #01:02:01-3#

I: hm (bejahend) #01:02:02-0#

B: Ja. #01:02:03-4#

I: Sehr viel mehr Sport. #01:02:04-3#

B: Ich mag Sport. #01:02:06-5#

I: hm (bejahend) #01:02:06-5#

B: Ich mag wirklich Sport. Ich war, egal, bei mir ist wurscht. Laufen, Fahrrad fahren, (...) Basketball, Volleyball, Fußball. (...) Ja. #01:02:16-8#

I: ähm ich bin ab und zu Schwimmen und ein bisschen Klettern. Da (...) stehe ich schon deutlich hinter dir, also (...) #01:02:22-8#

B: (lacht) #01:02:24-1#

I: (lacht) #01:02:24-6#

B: Nene, also, ich mach ähm(...) jede ähm Freitag? #01:02:28-7#

I: hm (bejahend) #01:02:28-7#

B: nach der Arbeit ich gehe zu Kletterzentrum in Lappersdorf. #01:02:32-6#

I: ähm, das DAV Zentrum? #01:02:34-9#

B: Ja, ja. #01:02:34-9#

I: Ah, da war ich bisher noch nie. Wie ist des denn? #01:02:38-5#

B: Ja, es gibt auch dort eigentlich Höhe klettern, das ist ungefähr, ähm, zwölf bis 15 Meter. Das ist hoch klettern. #01:02:45-9#

I: Kannst du sichern? #01:02:46-4#

B: Ja. #01:02:48-2#

I: Das kann ich noch nicht. Könne nwir vielleich mal zam klettern gehen! #01:02:53-2#

B: Gerne. Also du kannst auch schon ähm(...) jede Freitag (unv.) kommt die deutsche Leute auch schon dort. #01:03:00-7#

I: hm (bejahend) #01:03:00-7#

B: Die machen ähm(...) über die Campus Asyl. Du musst nicht zahlen #01:03:05-2#

I: hm (bejahend) #01:03:06-1#

B: Dann du kommst. Einfach du musst dort anmelden, dann gibst du einmal 20€, dann ganze Saison darfst du schon bei uns klettern. Aber Freitag. #01:03:17-5#

I: Jeden Freitag quasi? #01:03:19-0#

B: Ja. #01:03:19-0#

I: Oh! #01:03:20-3#

B: Ja. Das ist ein Verein und es gibt Kräne #01:03:24-3#

I: hm (bejahend) #01:03:24-3#

B: Dass du zum Beispiel mit denen lernst du, und Sicherheit machst undso. #01:03:29-4#

I: (...) Das merk ich mir. Also, des Programm is quasi jeden Freitag durchgehend? Immer im DAV Zentrum. #01:03:35-2#

B: Ja. ähm, es gibt, halb vier müssen, müssen wir dort sein. #01:03:39-7#

I: hm (bejahend) #01:03:40-2#

B: Halb vier, Freitag. #01:03:43-3#

I: Ja. #01:03:43-3#

B: Danach, ähm, fangen wir an. Bis, ähm, 18 Uhr. Zwei Stunden. #01:03:47-5#

I: zwei Stunden sind mehr als genug. #01:03:49-6#

B: Ja, ja. #01:03:50-0#

I: ähm #01:03:52-1#

B: Also letzte Woche, ich hab schon probiert. #01:03:53-2#

I: hm (bejahend) #01:03:53-6#

B: Mit ähm, mit schwarze Zeile. Schwarze (...) diese Dinger. Steine. #01:03:59-5#

I: Ja. #01:04:00-3#

B: Das war, ähm(...) war ähm(...) Fünf Plus. Sechs Minus. Kennst du auch die Nummer von denen? #01:04:07-3#

I: Kenne ich. #01:04:07-3#

B: Ja, dann war (...) sieben. #01:04:09-5#

I: hm (bejahend) #01:04:09-5#

B: Und war sieben Plus. Acht Minus. und ich hab den Sieben, Sieben ganz normales Schwarz, ahbe ich genommen. #01:04:19-2#

I: (pfeifft) Du kannst schon Siebener? #01:04:21-1#

B: Ja, ich habe letzte Woche mit Sieben angefangen. Also ich bin seit ähm(...), zwei Woche beim Klettern. Ja. Jede Freitag. #01:04:29-4#

I: Aber warte, mach ma da jetz wieder mal nen cut, weil des finde ich zwar total interessant, das is nur leider komplett irrelevant. Da reden wir danach weiter. #01:04:40-4#

B: Okay. #01:04:40-4#

I: (lacht) (...) ähm (...) Eins ist mir letztens bei nem anderen Interview aufgefallen, der meinte, dass er in der Schule Probleme gehabt hat, weil (...) die Schüler und die Lehrer ähm(...) relativ viel Bayrisch geredet haben. Also (...) #01:04:55-8#

B: Beim Ausbildung meinst du? #01:04:59-6#

I: ähm(...) Ja? #01:05:02-7#

B: Beim Ausbildung es ist schon so. #01:05:03-6#

I: hm (bejahend) #01:05:04-0#

B: Weil ähm, viele(...) deutsche Leute machen Ausbildung. #01:05:11-3#

I: Ja. #01:05:11-3#

B: Und wir sind schon in Bayern. #01:05:11-3#

I: Ja. #01:05:12-2#

B: Und ähm, es gibt schon Probleme. Zum Beispiel als Schreiner, meißte sind aus Bayern, und die reden richtig bayrisch. Zum Beispiel als Elektriker, ich war nur zwei Tage, un dich kenne auch ein Person, der (...) der hatte schon richtig bayrisch  gelernt, aber ich hab schon verstanden (...) was meint er. #01:05:32-6#

I: hm (bejahend) #01:05:32-6#

B: Ich kann, ich meine, nicht 100\%, sondern ich kann schon schätzen wenn jemand redet. #01:05:37-9#

I: Ja. #01:05:39-7#

B: bei mir war gut, aber ich kenne auch schon paar Leute, die reden (...) muss man richtig konzentrieren, damit die verstehe. #01:05:47-9#

I: hm (bejahend) Ich geb mir Mühe.(lacht) #01:05:50-4#

B: (lacht) Ja, es ist so. Es ist (...) #01:05:52-9#

I: Ja.  #01:05:54-4#

B: Ein bisschen schwierig. #01:05:57-7#

I: (...)ähm, hast du da persönlich auch Probleme gehabt damit? #01:06:00-8#

B: ne, ich nicht. Ich nicht. ähm, ich war beim Integration Kurs #01:06:05-3#

I: hm (bejahend) #01:06:05-3#

B: Die waren alle ganz normal, Hochdeutsch geredet. #01:06:08-5#

I: hm (bejahend) #01:06:08-9#

B: Dann (...) beim Quali ich war auch schon, die waren auch ganz normal. #01:06:15-2#

I: Ja. #01:06:16-5#

B: Ganz normal, die waren auch deutsch, die haben schon geredet. und ähm(...) nur beim Praktikum einmal ich war (...)ähm beim Zinngeber? Es gibt ein Elektrofirma. #01:06:28-8#

I: Ja? #01:06:30-0#

B: Dann ich hab schon ähm, dort habe ich Praktikum gemacht. Auf die Baustellen war die (...) paar Leute aus Bayrischer Wald. #01:06:34-7#

I: hm (bejahend) #01:06:34-7#

B: Die haben gesagt: '(unv.)'. Ich habe gesagt: 'Wie bitte?' '(unv.)' #01:06:40-5#

I: (lacht) #01:06:40-5#

B: Ich sage: 'Wie bitte? Ich habe nicht verstanden.' '(unv.)'  #01:06:45-5#

I: (lacht) #01:06:45-5#

B: Und mein, ähm, mein Chef hatte zu Ihm gesagt: 'Rede mit Ihm Hochbayrisch.' Nicht Hochdeutsch. #01:06:52-7#

I: (lacht) #01:06:52-7#

B: Dann so: '(unv.)' Ich sage: 'Wie bitte?' 'Woher kommstu?' Ich sage: 'Achja, okay. Ich komme aus Afghanistan.' (lacht) #01:06:58-6#

I: (lacht) #01:06:59-5#

B: Hochbayrisch war: 'Woher kommstu?' Ich sage; 'Ich komme aus Afghanistan.'(lacht) #01:07:05-2#

I: (lacht) #01:07:06-0#

B: Ja, das ist(...) Hochbayrisch war schon gepasst, verstanden (...) was meint er (lacht) #01:07:09-0#

I: Ja.(lacht) #01:07:10-8#

B: War auch okay.(lacht) #01:07:14-2#

I: Aber die waren nett oder? #01:07:14-2#

B: Jajaja! Der hatte mit mir, der wollte mit mir Kontakt haben! #01:07:18-2#

I: Ja. #01:07:18-2#

B: Und er hatte, er woll(...) er wollte mit mir reden, und er '(unv.)'. Ich habe gesagt: 'Hey, ich verstehe nicht!' #01:07:21-2#

I: (lacht) #01:07:23-2#

B: Ja, nach dem vierte Mal, als ich gesagt hab, also: '(unv.)'. Ahja okay. Ich komme aus Afghanistan. (...) #01:07:32-3#

I: Nett. (lacht) #01:07:34-2#

B: Ja.(lacht) Hat er geasgt: 'Musst du mit Ihm Hochbayrisch reden. Er ist, er ist nicht, so Deutscher, normal er versteht nicht.' Und dann '(unv.)' 'Jaa ich komme aus Afghanistan.' #01:07:46-1#

I: (lacht) #01:07:48-3#

B: Jawoll! Eigentlich war ein bisschen anstrengend. Aber  (...) normalerweise geht's. Wir arbeiten schon noch auf die Baustelle, aber (...) amche Leute, die merken schon, wir verstehen nicht, zum Beispiel Bayrisch. und die reden dann normal. #01:08:00-9#

I: Ja. #01:08:04-2#

B: Aber jetzt ich verstehe, ich meine, ich verstehe ganze Bayrisch. (...) Wir merken schon. #01:08:09-7#

I: hm (bejahend) #01:08:10-6#

B: Wenn die redet. '(unv.)' 'Ah ok. Machen wir.'(lacht) Aber wirklich nicht. Wenn die reden, wirklich nicht. #01:08:23-1#

I: hm (bejahend) #01:08:23-1#

B:  Aber die Fachsprache.Fachbegriffe. Und die kriegst du jetzt mit der Zeit alle mit oder? #01:08:28-8#

I: Ja, ja. (lacht) (...) Also, (unv.). ähm(...) Vie viele (...) In wie vielen Wohnungen warst du denn seit du nach Deutschland gekommen bist? #01:08:39-6#

B: Wie viele? #01:08:39-6#

I: Wohnungen. #01:08:39-6#

B: ähm, eigentlich, ähm(...) ein. Zwei. Drei. Und jetzt eigene Wohnung. Drei, weil ähm(...) die (...) Flüchtlinge heim #01:08:54-3#

I: hm (bejahend) #01:08:54-3#

B: ähm(...) Ganz normale WG oder? #01:08:57-6#

I: Ja. #01:08:57-6#

B: Mit paar Leute. #01:08:57-6#

I: hm (bejahend) #01:08:57-6#

B: Und jetzt habe ich eine privat gekriegt. #01:09:00-7#

I: Jetzt? Das heißt, dann bist du mittlerweile ähm #01:09:07-7#

B: Eigene Wohnung. #01:09:07-7#

I: ähm aber dass du ne eigene Wohnung kriegst ähm, dafür brauchst du ja auf jeden Fall ne Aufenthaltsgenehmigung. #01:09:13-7#

B: Ja, habe ich schon. #01:09:14-8#

I: Also, dein Asylverfahren ist fertig durchgelaufen #01:09:16-8#

B: ja. #01:09:17-5#

I: Hast du dann was permanentes quasi jetzt? #01:09:21-5#

B: Ja. #01:09:21-5#

I: Cool. #01:09:22-0#

B: Jo. Is super. #01:09:25-3#

I: Wie lange hat'n des gedauert? #01:09:25-3#

B: Bei mir eigentlich ähm(...) zweieinhalb schon gedauert, und ich bin seit, was? Ein Jahr oder eineinhalb Jahre bin ich, ähm, ein Jahr ich hatte schon Aufenthalt. #01:09:39-6#

I: hm (bejahend) #01:09:39-6#

B: Ja. #01:09:39-6#

I: ähm, hast du am Anfang dann auch Ablehnungen bekommen? Oder hat das sofort funktioniert? #01:09:44-8#

B: Nee, bei mir (...)ähm Bevor ich in (unv.) gehen #01:09:50-2#

I: (unv.) #01:09:50-6#

B: Dauert, ähm, eineinhalb Jahre. #01:09:53-2#

I: hm (bejahend) #01:09:53-6#

B: Ich war nicht beim (unv.). #01:09:55-7#

I: hm (bejahend) #01:09:55-7#

B: Danach, ich meine beim Gericht. #01:09:59-1#

I: ja. #01:10:00-3#

B: Danach, ich war beim Gericht. (es klopft). Dann (...) (Essen ist fertig) #01:10:07-5#

I: Oh! Dann ähm(...) dann packen wir's, Hm? #01:10:13-3#

B: Ja. #01:10:13-3#

I: Danke! #01:10:13-3#

B:Bitte, gerne! #01:10:13-2# \newpage
%\section{Interview 6}

 #00:00:02-8#

I: Ok. (...) Dann läuft des jetz. Erstmal danke, dass du heute Zeit hast! #00:00:09-0#

B: Danke. (lacht) Gerne #00:00:11-1#

I: ähm(...) Dann würd ich jetz erstmal die, ähm(...) allgemeinen Sachen machen. Also: Du bist ganz offensichtlich männlich. #00:00:21-2#

B: hm (bejahend) #00:00:22-4#

I: ähm(...) wie alt bist du denn? #00:00:27-2#

B: Ja ich bin schon 21. #00:00:30-5#

I: Du bist 21? #00:00:30-5#

B: Ja.(...) #00:00:34-1#

I: (lacht)Ich bin 24. #00:00:34-7#

B: Ach du bist 24? #00:00:37-1#

I: Ja, ja. #00:00:37-1#

B: Naja, schaust du nicht (unv.). #00:00:41-7#

I: Hab einein Jungbrunnen. (lacht) #00:00:44-4#

B: (lacht) Ok. Ja. #00:00:49-1#

I: ähm(...) #00:00:50-8#

B: Du hast keine Bart, aber ich habe schon. (lacht) #00:00:52-6#

I: (lacht)Der wächst bei mir ned so wirklich, der will einfach ned. (lacht) #00:00:55-6#

B: Ja ich weiß schon, bei manche ist schon, ja. #00:00:58-4#

I: ähm(...) du bist (...) Aus welchem Land bist du denn? #00:01:03-4#

B: ähm, aus Afghanistan. #00:01:05-8#

I: (...), auch in Afghanistan geboren. #00:01:08-3#

B: Ja. Ich bin schon in Afghanistan geboren, aber (...) aufgewachsen im Iran. #00:01:17-5#

I: De(...) des haben sehr viele genau so erlebt, oder? #00:01:19-5#

B: Genau. Als Kind hatte ich schon (...) schwieriger. Zuhause, Probleme. Ja, ich, und meine Eltern. Ja, das ist wahr. Das hatten meine Eltern gehabt, und (...) als Kind (...) konnte ich nicht (...) dort leben. Deswegen ich bin mit acht Jahren auf den (unv.).(...) #00:01:44-4#

I: (...)ähm(...), also, grundsätzlich noch: Du bist Moslem, oder? #00:01:49-9#

B: Ich bin Moslem, ja. #00:01:49-3#

I: ähm(...) welcher, ähm, Bevölkerungsgruppe gehörst du denn an? Also, bist du Schiite, Sunnite(...)? #00:01:57-2#

B: Nein, ich bin Sunnite. #00:01:59-2#

I: Du bist Sunnite? #00:01:59-2#

B: hm (bejahend) #00:02:01-0#

I: ähm(...) seit wann bist du denn in Deutschland? #00:02:10-0#

B: (...)2015 war(...)ähm(...) weiß es nichtmehr, welcher Monat. Aber 2015 war. #00:02:20-3#

I: 2015. #00:02:23-1#

B: hm (bejahend) 2015. #00:02:24-1#

I: ähm(...)und(...) dann noch des letzte zu dem Allgemeinen jetz: Was hast du denn für nen Asylstatus? Also (...)ähm, hast du ne Aufenthaltsgenehmigung, oder(...)? #00:02:40-8#

B: Nein. Ich habe keine Aufenthaltsgenehmigung. Hier normal, Dultung. #00:02:46-8#

I: Ne Duldung. #00:02:47-2#

B: Eine Duldung. #00:02:47-2#

I: Also (...) das, dass alle paar Monate wieder erneuert werden muss, oder? #00:02:55-7#

B: Früher war schon. Weil ich hab (...) ich habe schon einmal Interview gehabt, und ich das is schon (...) abgelehnt. #00:03:01-3#

I: hm (bejahend) #00:03:01-3#

B: Von der BAMF. Und dann (...) ich hab wieder zweimal geklagt, und (...) 18 Monate hat gedauert. Keine Antwort bekommen.  #00:03:15-5#

I: Was? #00:03:15-5#

B: Ja. 18 Monat. Fast 18, ja, 18 Monat hat gedauert; keine Antwort bekommen von der BAMF. Musste ich, ja, einfach warten, warte. Ich wusste nicht, bis wann? Oder (...) wie lange noch? Dann (...) es gibt schon Paragraph, von (...) hier in Deutschland. Wenn man in Deutschland vier Jahre ist, und (...) ja. Schule besucht. Das und, ja. Gute (...) Kenntnisse hat. Dann darf man schon hier bleiben, diese Paragraph beantragen. #00:03:51-8#

I: Das ist(...) #00:03:51-8#

B: //Dann habe ich schon diese Paragraph beantragt und den, und (...) und BAMF hab ich schon, ja. Dann habe ich (unv.) gesagt, das brauche ich nicht. Hab gesagt. (...) #00:04:05-1#

I: hm (bejahend) #00:04:05-1#

B: Ja. #00:04:07-4#

I: ähm das ist dieser Paragraph 25a) oder? #00:04:12-4#

B: Paragraph 25a), genau. #00:04:13-1#

I: hm (bejahend) #00:04:14-4#

B: Ja. #00:04:16-0#

I: ähm, was sin ddann jetzt deine nächsten Schritte so?  #00:04:24-0#

B: Meine nächsten Schritt (...) Meine nächsten Schritt (...) dass ich meinen Ausbildung schaffe. (lacht) Und (...) ja, danns chaun ma mal (lacht). #00:04:32-2#

I: ähm, wie lang machst'n die jetzad schon? #00:04:36-5#

B: Jetzt fast fünf Monat. Seit fünf Monat#00:04:37-7#

I: Seit fünf Monaten? #00:04:38-8#

B: hm (bejahend) #00:04:39-9#

I: Du machst Jackierer gell? #00:04:41-3#

B: Nein, Karosseriebau.  #00:04:44-3#

I: (unv.) war Lackierer! #00:04:44-2#

B: (unv.) macht Lackierer, genau.  #00:04:45-8#

I: Ah, ja. Okay. #00:04:48-3#

B: Genau. #00:04:48-3#

I: ähm (...) gefällt's dir? #00:04:55-4#

B: Ganz gut. #00:04:55-4#

I: Ok. #00:04:55-4#

B: Ja. #00:04:56-9#

I: ähm wie hast'n die Ausbildung bekommen wenn ich fragen darf? #00:05:01-8#

B: Mh, den Ausbildung (...) ich wollten eigentlich (...) Mh, nicht Karosseriebauer werden. Und, ich wollte was anderes machen weil, ich hab schon im Iran (...) vier bis (...) vier Jahre gearbeitet als Mechaniker. #00:05:17-2#

I: hm (bejahend) #00:05:18-3#

B: Und ich wollte was anderes (...) anders machen. #00:05:19-5#

I: hm (bejahend) #00:05:19-5#

B: Und (...) nicht Hände schmutzig machen. (lacht) #00:05:23-5#

I: (lacht) #00:05:23-5#

B: Ja, dann habe ich schon alles probiert, und nicht geklappt. #00:05:27-8#

I: Mh, was hast'n alles ausprobiert so? #00:05:28-9#

B: Ja, ja. Ich hab Zahntechniker. Das habe ich probiert. Das hat mir gut gefallen, aber (...) es gibt schon weniger (...) Labor in Regensburg, und die brauchen, die meißten brauchen keine Ausländer #00:05:45-6#

I: Hm? #00:05:46-8#

B: Ja.(lacht) Ja, ist so, ja. Die meißten brauchen keine Ausländer. Die akzeptieren keine Ausländer. Die einer (...) in Doktor Gessler Straße, ich weiß es nicht ob du kennst oder nicht. Da oben beim Königswiesen fast in der Nähe #00:06:01-0#

I: (...)ähm, ne ich glaub ned. #00:06:04-8#

B: Ja. Dort habe ich schon zwei Wochen Praktikum gemacht, und (...) ja, da habe ich schon eine Stelle gefunden, und (...) ja. Und (...) dann (...) ich habe so  zu spät meine Bewerbung abgegeben und die haben eine andere genommen, und (...) der Junge hat schon zwei Monate, drei Monat gearbeitet, und einfach weg(...) gehaut. #00:06:30-7#

I: Mh? #00:06:33-0#

B: Ja. #00:06:34-1#

I: Wie? #00:06:35-5#

B: Der hat nicht weiter gemacht. Der hat einfach (...) #00:06:38-7#

I: Der, der die Stelle anstatt dir bekommen hat? #00:06:39-1#

B: Genau, ja genau. Ja, der hat schon zwei, drei Monaten gearbeiten und hat einfach (...) weggehabt. Ja, und (...) dann, wir haben schon wieder versucht anzurufen, und die haben schon gesagt: Nein. Wir brauchen jetzt keine Aus(...) keine Ausländer, weil wir können nicht vertrauen. Die, die sind nicht so fleißig, und (...) #00:07:03-4#

I: Scheiße #00:07:03-4#

B: Ja, deswegen wir haben schon gesagt: 'Ja, es (...) es kommt (...) ein Mensch hald, wie du in der Arbeit bist undso.' #00:07:14-1#

I: hm (bejahend) #00:07:14-1#

B: Immer nicht gleich. Jeder ist nicht gleich undso. Aber trotdem die haben gesagt: 'Nein. Wir werden Ihn jetzt nicht.' Ja. Und dann habe ich schon im Internet gesehen, dass die Firma Reisinger suchen eine Lehrling, und ich hab mich (...) Telefonnummer rausgeholt und angerufen und Termin ausgemacht, Praktikum. #00:07:40-0#

I: hm (bejahend) #00:07:40-0#

B: Drei Tage habe ich schon Praktikum gemacht, und erster Tag habe ich schon (unv.) (...) Stelle bekommen. (lacht) #00:07:47-3#

I: Cool! #00:07:48-2#

B: Genau, ja. Weil ich konnte mich schon mit dem Auto undso klarkommen undso. Und deswegen die haben schon gesagt: 'Ja, wir nehmen dich.' Obwohl es schon so spät war und (...) vier Monaten zu spät war. Die haben gesagt: 'Gerne!' #00:08:04-1#

I: Gut, da hast du schon (...)ähm vier Jahre im, im Iran gearbeitet. Des kannst du dann. #00:08:10-4#

B: Genau. Ja. Ja. Ja, im Iran ist so, kann man schon arbeiten, aber (...) Schule und (...) lernt man nur einfach. Geht man nicht in die Schule. Lernt man nur  in der Werkstatt. #00:08:23-4#

I: hm (bejahend) #00:08:23-4#

B: Besucht man keine Schule. Es ist so. #00:08:26-7#

I: ähm(...), hast du dann im, ähm im Iran oder in Afghanistan grundsätzlich ne Schule besucht? #00:08:33-8#

B: Nein. #00:08:35-0#

I: Ok. Also, erst hier in Deutschland dann. #00:08:38-8#

B: Es (...) ja, habe ich schon angefangen, ja. Null. (lacht) #00:08:41-6#

I: Du hast komplett von Null angefangen? #00:08:41-6#

B: Komplett, ja. #00:08:45-4#

I: Das muss echt hart gewesen sein oder? #00:08:46-0#

B: War schon, ja, sehr hart. Und, ich war schon (...) im Deutschkurs, und (...) wenn, im Lernwerkstatt hab ich schon drei, drei? (...) Ja, drei Monate habe ich schon Deutschkurs besucht. Und, erster Tag bis (...) erster Monat war sehr, sehr, sehr, sehr schwierig für mich. #00:09:09-1#

I: hm (bejahend) #00:09:09-6#

B: Weil, ich konnte garnix. Ich konnte keine Buchstaben, einfach. Und, ich konnte auch nicht schreiben (...) und lesen. #00:09:17-1#

I: hm (bejahend) #00:09:17-1#

B: Das war sehr, sehr, sehr hart, und ich hab mal auch gewein, und (lacht), ja, die anderen hat schon geschrieben und ich hab, ich hab gesagt: 'Hä, wieso, wieso kann ich nicht?(...) Und, wieso kann ich nicht, wenn (...) wieso kann ich nicht schreiben?' Und dann auf einmal hab ich schon Gas gegeben. (...) Schreiben gelernt. Sprache gelernt. (...) Ja. #00:09:42-2#

I: Also jetz (...) also, du sprichst auf jeden Fall gut Deutsch. #00:09:45-8#

B: Ja, jetzt geht schon. #00:09:49-1#

I: ähm, hast du dann auch (...) die persische Schrift hier gelernt? #00:09:53-9#

B: Nein, auch nicht. Wenn man nicht in die Schule geht, dann, wie kann man schon in Persisch lesen? #00:09:59-0#

I: hm (bejahend) ähm, ich hab's nur mitbekommen von a paar andren aus'm Jugendcafé, die eben dann hier in Deutschland Persisch, also die Schrift gelernt haben dazu. #00:10:10-2#

B: Ahso? Ok. Ne, habe ich nicht. #00:10:16-5#

I: hm (bejahend) ähm, über welche Fluchtroute bist du denn nach Deutschland gekommen wenn ich fragen darf? #00:10:26-9#

B: Ja, war sehr schwierig. (lacht) (...) Iran habe ich , bin nach Türkei gekommen und (...) Türkei waren wir schon so (...) keine Ahnung was, sechs, sieben Leute. Ja, sowas (...) ähm, mit eine LKW. Ja, dann (...) Österreich, Österreich, Deutschland. #00:10:55-5#

I: Also du bist in der Türkei in nen LKW gestiegen #00:10:56-0#

B: hm (bejahend) #00:10:56-0#

I: und in Österreich wieder raus. #00:10:57-1#

B: Genau. (...)  #00:11:02-8#

I: Das hat länger gedauert oder? #00:11:04-0#

B: WAr schon länger, keine Ahnung. Für uns war alles Nacht. (lacht) #00:11:09-3#

I: Ohh. #00:11:12-4#

B: War auch alles Nacht, ja. (...) #00:11:15-3#

I: Ow. #00:11:17-7#

B: hm (bejahend) (...) Trinken haben wir gehabt, aber Essen (...)? #00:11:21-6#

I: Es gab nichts zu Essen? #00:11:23-1#

B: hm (verneinend) #00:11:23-1#

I: (...) ähm (...) und in Österreich seid Ihr dann wieder raus gekommen? #00:11:36-2#

B: Österreich, die haben uns raus geschmissen. #00:11:39-0#

I: WIe, einfahc die Türen raus, und (...) alle raus? #00:11:42-3#

B: Mh, genau. Ja, den Fahrer hat uns raus geschmissen. Tür raus, ausgemacht. Und hat gesagt: 'Geht (...) raus.' Und wir waren schon auf den Straße und ich hab schon einen Taxi geruft und bis Hauptbahnhof und den Taxifahrer war schon (...) gut. Und die haben (...) die hat gesagt: 'Wo, wo wollen Sie hin?' Natürlich mit Englisch, und (...) wir konnten schon ein bisschen Englisch #00:12:11-5#

I: Als Englisch konntest, ähm konntet Ihr? Okay. #00:12:11-2#

B: Ja, genau. Ja. Und (...) die haben gesagt: Ja, jetzt gehen wir nach Deutschland. Und dann (...) die anderen, jeder hat schon einfach und (...) weggegangen. UNd ich war schon mit meinem Kumpel und (...) wir warten zu zweit und die hat (...) den Fahrer hat schon uns, wir haben Geld gegeben und der hat schon Ticket gekauft und (...) Bar bezahlt und (...) ja. #00:12:38-6#

I: hm (bejahend) #00:12:38-6#

B: Dann (...) #00:12:41-7#

I: Also, der Taxifahrer? #00:12:41-1#

B: Der Taxifahrer. #00:12:44-4#

I: Des is nett! #00:12:44-4#

B: Ja, der war Moslem und der hat schon gesagt: 'Ja, seid Ihr Moslem?' Hab gesagt, wir haben gesagt: 'Ja, wir sind Moslem' und (...) der hat schon, einfach (...) menschlich, ja. Geholfen und (...) #00:13:01-7#

I: Cool. #00:13:01-7#

B: Ja. #00:13:01-7#

I: (...) ähm, wie hat's denn ausgschaut, sobald Ihr nach Deutschland gekommen seid? #00:13:13-7#

B: Mh, war schon ganz, ganz (...) gut. War, war echt ganz gut weil (...) bei uns ist (...) Polizei, die schlagen uns, ne? Wenn wir keine (...) Fehler machen oder (...) in Türkei oder im Iran undso. #00:13:28-6#

I: hm (bejahend) #00:13:29-1#

B: Aber, damals. Wir waren schon im Zug und die Polizei, die Kontrollen sind gekommen und Sie haben uns gefragt nach dem Pass oder Reisepass oder Ausweis und wir haben gesagt: 'Ja, wir haben garkeins. Nur Ticket.' Und (...) die haben ganz nett, ganz nett gefragt: 'Ja, Ihr müsst zu uns kommen und (...) IHr müsst garnicht Angst haben, und Ihr sind (...) sind in Sicherheit und (...) wir gehen zum Polizeistadtion und nur einfach fragen wir wo (...) woher kommen (...) Sie und (...) #00:14:10-2#

I: War des auf (...) Englisch oder war da nen Dolmetscher dabei? #00:14:13-8#

B: Nein. War ein Kumpel von mir, der konnte schon persische Schrift und der hat schon auf Deutsch, die Polizisten hat schon auf Deutsch geschrieben und auf Persisch übersetzt und, und uns gezeigt. #00:14:25-3#

I: Habt Ihr mim Google Übersetzer oder wie? #00:14:28-0#

B: Genau. (lacht) #00:14:28-0#

I: Nein! (lacht) #00:14:28-2#

B: Wirklich(lacht) #00:14:29-9#

I: Hah, cool! (lacht) #00:14:31-8#

B: War ganz netter Polizist. (...) Ja, der war so chillig. Ja. #00:14:37-6#

I: (...) Des hatte ich mal in ähm(...) im Flughafen #00:14:44-5#

B: hm (bejahend)? #00:14:46-0#

I: Da waren ähm(...) nen paar Leute aus Russland, die(...) des war in, ähm, Reykjavík. ähm, war ich mit nem Freund #00:14:54-2#

B: hm (bejahend) #00:14:54-2#

I: Und ähm (...) da waren nen paar Leute aus Russland, die umbedingt nach Barcelona mussten #00:15:00-1#

B: Okay? #00:14:59-1#

I: Aber absolut kein Englisch gesprochen haben und keine Ahnung hatten, wie Sie da hinkommen sollten. #00:15:05-7#

B: Oh. (lacht) #00:15:06-2#

I: Das heißt, da haben wir das aller selbe gemacht: Auch über den Google Übersetzer, ähm, und irgenwann haben die dann ne provisorische Kreditkarte bekommen #00:15:19-9#

B: hm (bejahend) #00:15:19-9#

I: ähm, haben's eben bei ner Bank ähm (...) beantragt und dann konnten Sie fliegen. Aber find ich irgenwie nett, dass der Google Übersetzer so (...) #00:15:27-0#

B: Genau, ja. Die Polizist hat schon dort geschrieben, und auf, auf Persisch dann übersetzt und (...) und sein Zeit und (...) dann die Kumpel von mir, die konnte schon Persisch lesen und schreiben, und die hat schon wieder die Antwort geschrieben und (...) hingegeben. (lacht) #00:15:46-8#

I: (lacht) #00:15:46-8#

B: Wir haben so gemacht, ja. Aber die waren ganz, ganz, ganz, ganz nett. Und wir haben schon (...) wir haben, ja. Wir waren schon ganz (...) hunger, und hungrig, und die haben uns Essen gegeben und Obst gegeben und (...) #00:16:03-6#

I: (...) Cool. #00:16:03-6#

B: Ja. #00:16:05-8#

I: War das dann direkt hier in Regensburg? #00:16:09-4#

B: WIr wollten durch, ähm(...), Österreich nach ähm(...) Frankfurt gehen #00:16:18-1#

I: hm (bejahend) #00:16:18-1#

B: Ja. Wir wussten da nicht. Bayern? Wo ist Bayern? Wo ist Regensburg? #00:16:23-5#

I: Ja. #00:16:23-5#

B: Ja, wir haben gesagt, ja gehen wir nach Frankfurt und (...) Kumpel von uns war dort. Und der hat gesagt, er muss nachfragen (...) und dann wir haben schon Österreich nach Frankfurt Ticket gekauft. Und dann, Kontrolle. Den Polizei war von Regensburg.  #00:16:46-9#

I: hm (bejahend) #00:16:47-5#

B: Und die haben schon uns kontrolliert, und (Handy des Interviewten leutet) #00:16:50-4#

I: Ups, t'schuldigung. #00:16:51-2#

B: Ja, dann wieder nach Regensburg gebracht. Eigentlich wir waren schon #00:16:50-7#

I: Ich pausier eben. #00:16:56-8#

B: Ja. #00:16:57-6#

I: ähm, Ihr seid auf der Strecke von Österreich nach Frankfurt  #00:17:03-1#

B: genau #00:17:04-3#

I: ähm, in Nürnberg dann kontrolliert worden? #00:17:06-1#

B: Nein, in Regensburg die Polizei hat schon (...) ähm, ja, eingestiegen. und dann, nach dem Zug Kontrolle einfach alle hat kontrolliert, und lange gedauert, und wir waren schon ganz vorne beim Zug #00:17:20-7#

I: hm (bejahend) #00:17:21-1#

B: Und dann (...) ja. Die haben schon gefragt und wir haben gesagt: 'Keine Pass und keine Beizeug.' #00:17:32-8#

I: hm (bejahend) #00:17:32-8#

B: Dann wir waren schon dann im (unv.). Die haben schon (...) gezeigt, wir müssen jetzt aussteigen und wieder nach Regensburg fahren. #00:17:41-1#

I: hm (bejahend) #00:17:41-8#

B: hm (bejahend) #00:17:43-3#

I: Okay. (...) ähm, aber (...) hattest du dann zum Beispiel im (...) du hattest im Iran vermutlich auch keinen Pass oder? #00:17:57-0#

B: Nein. Ich hab schon illegal im Iran gelebt. #00:18:02-9#

I: ähm, weil (...) ähm, die anderen, mit denen ich jetz geredet hab #00:18:06-8#

B: hm (bejahend) #00:18:06-8#

I: die meinten auch, also (...) Sie wären eben ursprünglich aus (...) Afghanistan und hätten im Iran dann gelebt #00:18:15-8#

B: hm (bejahend) #00:18:15-8#

I: ähm, aber (...) dass ma als Afghane im Iran quasi immer Illegaler is, außer, OK, man schließt sich dem Militär an oder so? #00:18:26-6#

B: hm (bejahend) #00:18:27-3#

I: Aber das des so ziemlich die einzige Möglichkeit is, sich (...) nen legalen Status da zu verschaffen #00:18:33-8#

B: Ja. #00:18:34-9#

I: Das is hart. #00:18:36-8#

B: Das is echt hart, wenn, wenn jemand (...) illegal lebt dort, dann mit Polizei. (...) Ja. In (...) kontrolliert, und (...) sag, dass du Afghaner bist oder, keine Ahnung, als Afghaner dort lebst, dann (...) legal dann kriegst du schwierig Probleme. Erstes Mal musst du Gefängnis, dann (...) keine Ahnung. Ein Monat (...) ein Monat, eineinhalb Monat, dann, die schicken, die schieben dich nach Afghanistan. #00:19:06-0#

I: hm (bejahend) #00:19:06-4#

B: Weil, damals. Ich konnte schon die Sprache, weil ich als Kind dort war. #00:19:11-6#

I: hm (bejahend) #00:19:12-3#

B: Ich konnte schon gut den Sprache, und die haben schon garnicht (...) gemerkt, dass ich Afghaner bin undso. Ja. Aber (...) paarmal hatte ich schon Schwierigkeiten, ganz ganz ganz Schwierigkeiten bekommen. Ich war schon im Gefängnis undso. #00:19:30-6#

I: Shiit. #00:19:30-6#

B: Ja, genau. Und dann, wir haben schon (...) ja, den Richter gefunden, und wir haben schon viel Geld gegeben, und dann die haben schon uns raus gelassen. Eigentlich (unv.) #00:19:42-5#

I: Also es lief (...) quasi nur über Bestechung, da wieder raus zu kommen. #00:19:46-0#

B: Ja. Nur dass du Afghaner bist und legal dort lebst und (...)ja. (...) #00:19:56-5#

I: Ich hoff mal hier läuft's besser? (lacht) #00:19:58-1#

B: (lacht) Jetzt läuft gut. (...) Ja. Bisschen Schwierigkeiten mit den (...) Behörden aber sonst passt alles. #00:20:12-6#

I: Mh, was denn? #00:20:11-9#

B: Mit den Behörden, ne? #00:20:14-1#

I: Was lief denn da? #00:20:17-3#

B: Mh, mit Ausweis und Aufenthaltserlaubnis und Asyl und sowas. Soweit. #00:20:21-6#

I: hm (bejahend) #00:20:23-4#

B: Dass man keine Aufenthaltserlaubnis hat. Darf man nicht irgendwohin Reisen undso, deswegen sage ich. Eigentlich, sonst passt.(...) #00:20:21-8#

I: Gut, des is echt ungut irgendwie. #00:20:24-4#

B: Ja (...) es ist schon ungut. Nunja, manchmal du willst (...) Disco gehen, und (...) gehst so. Dann die Security fragt: 'Ja, Ausweis?' Dann (...) gibst du ein Papier und der sagt: 'Ne was, was für ein Ausweis ist? Darfst du nicht mit solche Dokumenten rein!' (...) Dann hast du andere Gefühl, weißt du? Und (...) ja. Du hast keine richtige Ausweis, das (...)  #00:20:58-2#

I: Warte, die Aufenthaltserlaubnis wird dann nicht als gültiges, ähm(...) Ausweisdokument akzeptiert? #00:21:05-5#

B: Die akzeptieren manchmal nicht. Manchmal sagen: 'Ja (...) du darfst nicht. Also (...) Ausweiß nicht da rein. #00:21:17-1#

I: Das hört sich irgendwie nach Blödsinn an? #00:21:17-1#

B: Ja ich war schon paarmal, und die haben mich gefragt: 'Bitte Ausweis.' Ich hab schon meine Ausweis abgegeben! Geben mir die einfach und (...) alle gesehen und die hat gesagt: 'Ja mit solche Dokumente darfst du nicht rein. Zeig deine richtige Ausweis.' Ich hab gesagt: 'Das ist mein richtiger Ausweis!' Und, ja (...) hat gesagt: 'Nein! Musst du ein richtig Ausweis!' (...) Ja. Keine Ahnung. Es ist so. (lacht) #00:21:49-2#

I: Oh, Scheiße. #00:21:48-1#

B: Ja. #00:21:48-9#

I: ähm, aber wenn du jetzt die Ausbildung dann hast, dann hast du ja, ähm, ne Chance ne permanente Aufenthaltsgenehmigung zu kriegen oder? #00:22:01-1#

B: Zum Aufenthaltsgenehmigung die §25 brauchst du nicht umbedingt den Aubildung schaffen, oder (...) ob du Ausbildung hast oder nicht.  #00:22:14-0#

I: hm (bejahend) #00:22:14-0#

B: Du kriegst das, weil du (...) vier Jahre hier bist und (...) #00:22:18-6#

I: Die hast du jetzt dann. #00:22:18-6#

B: Ja, genau. Und Schule besucht hast und alles. Keine (...) Strafentäter bist oder keine Ahnung, keine Schlägerei und (...) keine Scheiße gebaut hast. Ja, dann kriegst du. Ja. Aber natürlich musst du für dich einen Job suchen. Und selbstständig sein; ob Ausbildung oder (...) Arbeiten einfach. #00:22:40-9#

I: hm (bejahend) #00:22:40-9#

B: Ja. Nicht, dass du daheim einfach sitzt und (...) nix machst und (...)  #00:22:46-0#

I: Ok. #00:22:46-8#

B: Das darfst du nicht. #00:22:47-5#

I: ähm, das willst du machen? #00:22:53-6#

B: Ja. Letztes mal, ja, ich bin schon in der Ausbildung, wir haben (...) es ist schon echt schwierig, und mit der Ausbildung. Und schaue ich einfach weiter und (...) ob ich nicht schaffe, das ist auch wurscht. Aber versuche ich! #00:23:06-5#

I: Ja. #00:23:06-5#

B: Ich gebe meine besten undso. #00:23:09-5#

I: ähm, du, und wirklich, wenn  (...) wenn ich dir dabei helfen kann, dann meld dich einfach. #00:23:15-5#

B: Ja. #00:23:16-3#

I: Also, meine Nummer hast du. #00:23:17-5#

B: Ich hab schon dein Nummer. Schreiben wir. #00:23:19-2#

I: Ja. #00:23:20-6#

B: Ja. #00:23:21-7#

I: Also, ähm, wenn ich irgenwie Zeit hab helf ich dir da echt gerne. #00:23:26-5#

B: Ok, Danke. #00:23:28-8#

I: Ja, ähm, gerne. (...)ähm(...) Eins würd mich interessieren, des hab ich in, ähm, meinem ersten Interview gehabt. #00:23:40-9#

B: hm (bejahend) #00:23:41-3#

I: ähm(...) der meinte, dass (...) okay, bei, bei Ihm war's in der Schule, ähm, dass der, ähm, ziemliche Probleme hatte mit (...) Also hier in Regensburg wird da relativ viel Bayrisch noch gesprochen. #00:23:57-5#

B: (lacht) #00:23:59-5#

I: Dass er damit Probleme hatte. #00:24:02-0#

B: Ja (...) Hochdeutsch und Bayrisch ist natürlich hier (...) wenn jemand nicht bayrisch verstehst, dann (...) es ist schwierig. Echt, es ist schwierig.  #00:24:13-7#

I: hm (bejahend) #00:24:15-0#

B: Weil den (...) die manche (...) reden ganz Dialekt auf Bayrisch und (...) du verstehst (...) garnicht was der (...) sagt, oder was der meint er? Und, ja, es ist schon bei mir jetzt dann auch schon (...) auf an(...) ähm Anfang habe ich schon Probleme gehabt in der Werkst att, und weil (...) die Leute sind alle (...) die reden (...) #00:24:44-4#

I: Die reden alle Bayrisch? #00:24:44-4#

B: Bayrisch, und, ja (...) am Anfang habe ich garnicht verstanden. #00:24:48-7#

I: Oh. (lacht) #00:24:48-7#

B: Ja. Und, danach (...) bin mitbekommen, und einfach (...) #00:24:55-0#

I: Mittlerweile funktioniert's? #00:24:56-3#

B: Ja, genau. Funktioniert. #00:24:57-7#

I: Und, und (...) die ham dir dann quasi auch geholfen? ähm #00:25:01-1#

B: Ja, die reden garnicht mit dir Hochdeutsch! #00:25:03-8#

I: Nur Bayrisch? #00:25:03-8#

B: Nur Bayrisch. #00:25:06-0#

I: (lacht) #00:25:06-0#

B: Nur Bayrisch, ja. Obwohl du sagst: 'ja bitte, ich verstehe nicht Bayrisch.' Und die (...) #00:25:14-7#

I: ähm #00:25:15-3#

B: Die reden überhaupt. #00:25:18-1#

I: ähm, können Sie nicht oder wollen Sie nicht? #00:25:17-8#

B: Die wollen nicht. #00:25:18-6#

I: Wollen nicht? #00:25:19-7#

B: Vielleicht, die können nicht. Bayr(...)ähm Hochdeutsch. #00:25:22-3#

I: hm (bejahend) #00:25:22-8#

B: Kann sein, die können nicht Hochdeutsch, oder die haben schon immer Bayrisch gesprochen und(...) #00:25:29-4#

I: Was ist dein Eindruck, was glaubst du? #00:25:32-0#

B: Ich glaube die können schon Bayrisch, ähm, Hochdeutsch, und (...) die wollen einfach, dass ich Bayrisch lerne. Und (...) #00:25:40-7#

I: Also, also eher so (...)ähm (...) Als Spielerei, Freundschaftlich? #00:25:44-8#

B: Ja, genau. Wirklich, Spielerei, Freundschaftlich. #00:25:47-3#

I: Also jetz nicht irgendwie so in die Ausgrenzungsrichtung. #00:25:52-4#

B: Ne, ne. #00:25:53-9#

I: Okay. #00:25:53-7#

B: Ja. (...) #00:25:56-6#

I: Ja, gut. #00:25:58-2#

B: Die wollen, dass ich einfach Bayrisch, weil ich einfach haben (...) einmal gesagt: 'Wieso reden (...) Ihr nicht mit mir Hochdeutsch?' Und die haben geagt: 'Ja, wir wollen, dass du auch Bayrisch #00:26:07-5#

I: hm (bejahend) #00:26:07-5#

B: redest.' Und, ja, Bayrisch mit uns redest. Und dann, ja, wenn du ein Wort öfter (...) mitbekommst, dann, natürlich, dann (...) verstehst du und (...) kennst du. Aber es ist schon auch schwierig, bayrisch zu reden. (lacht) #00:26:21-7#

I: (lacht) #00:26:24-4#

B: Aber ich kann es irgendwann(lacht) #00:26:23-4#

I: //Jetzt muss (...) jetzt musst du Deutsch und Bayrisch lernen. (lacht) #00:26:25-1#

B: Ja, trotzdem Bayrisch. Ja, wenn, natürlich, wenn du in Bayern bist, dann (...) musst du beides können, ne? (...) Ja. #00:26:36-9#

I: (lacht) Vermutlich kannst du jetzt schon besser Bayrisch als mancher Deutscher hier. #00:26:37-2#

B: ja, ne, ne, ne (lacht) #00:26:38-0#

I: ähm, (...) aber (...) das hat quasi so für dich jetz nich wirklich prob(...) nen Problem her(...) dargestellt oder? #00:26:44-7#

B: jetzt ist nicht mehr. Jetzt ist nicht mehr. Manchmal  #00:26:47-3#

I: //Wie war's am Anfang? #00:26:47-3#

B: Manchmal vielleicht (...) ähm, wie bitte? #00:26:50-6#

I: Wie war's am Anfang? #00:26:50-7#

B: An Anfang war ganz schlecht. Ich hab schon so alle, keine Ahnung, Schrauben verkehrt gebracht und die haben schon was anderes gemeint und ich hab schon (...) was anderes gebracht.(lacht) #00:27:02-8#

I: (lacht) #00:27:03-9#

B: Genau. #00:27:05-7#

I: Ja gut, aber (...) das passiert am Anfang. #00:27:07-5#

B: Ja. #00:27:10-4#

I: (...) ähm (...)Aber so im großen und ganzen ist es für dich ne Positive (...) ist es (...) schön hier oder? Also, ne positive Erfahrung. #00:27:26-4#

B: Ja. #00:27:30-4#

I: ähm (...) wie viele Wohnungen hast du denn bisher gehabt? Da war nämlich letztes mal, kann ich mich erinnern (...)ähm(...) dass, da, da gab's doch irgendwann (...) ähm, meintest du, Reibereien in ner Wohngemeinschaft oder? #00:27:52-4#

B: In Wohngemeinschaft, ja. Es ist immer so, weil (...) dort sind (...) verschiedene Nationalitäten, wohnen dort #00:28:07-1#

I: hm (bejahend) #00:28:07-1#

B: Und (...) ja, wenn (...) die eine nett ist, dann der andere ist bestimmt (...) nicht nett ist. Und es ist so, die manche kommen mit dir klar, aber die manche wollen nicht mit dir klar kommen einfach. (...) Ja, es ist so. Ich hab schon dort, ähm, (unv.) gewohnt. Nicht lange, ein Woche, als ich neu war. Hier, in Regensburg. #00:28:40-1#

I: hm (bejahend) #00:28:40-1#

B: Und (...) ja. Ich hab schon einfach Probleme gehabt. Ich war schon, ich war, ich war auf mein Bett und (...) im Schlaf, hat eine gekommen und mich geschlagen. #00:28:57-2#

I: Was?! #00:28:57-2#

B: Ja, wirklich. Auf dem bett weggeworfen. Ja. #00:29:02-0#

I: Hä?! #00:29:02-0#

B: Der hat schon mich auf dem Bett weggeworfen. Ja, wirklich. Der war (...) betrunken. Der war Russe und ähm(...) keine Ahnung, dann (...) ich wusste das nicht. Ich hab aufgestanden, und (...) ja. #00:29:15-8#

I: Der hat (...) der hat auch da gewohnt oder wie? #00:29:17-5#

B: Ja. (...) Auf eine Zimmer, keine Ahnung, 20, 30 Bett? Und waren viele verschiedene Leute, und (...) der hat mich geschlagen und (...) ja. Und dann (...) hab gesagt: 'Hey, was, was, was ist los?' und 'was willst du?' Und der hat schon einfach weiter gemacht. Und dann habe ich schon Hilfe gerufen. #00:29:41-3#

I: //Der is einfach weiter auf dich, hat geschlagen? #00:29:44-1#

B: //ja, genau. ja. Ja. #00:29:47-1#

I: Besoffener Wichser. #00:29:47-1#

B: Ja, dann, dann Polizei gekommen und dann ich hab gesagt: 'Es ist so' und 'Ich will nicht hier wohnen' #00:29:54-5#

I: hm (bejahend) #00:29:54-5#

B: Und dann die haben mich (...) ein Zimmer gegeben. #00:29:59-7#

I: Dann hast du nen Einzelzimmer bekommen. #00:30:01-4#

B: Dann habe ich schon eine Eilzelzimmer bekommen. #00:30:04-8#

I: Ja. War denn sowas(...) #00:30:05-9#

B: //zwei Tage.  #00:30:07-5#

I: Zwei Tage? #00:30:09-5#

B: Ja, zwei Tage. Dann, die haben mich (...) nach (...) ja, St. Vinzent gebracht.  #00:30:17-0#

I: hm (bejahend) #00:30:17-0#

B: Dann war ich schon dort auch, dreieinhalb Monate bis vier Monate, habe ich schon dort gewohnt. #00:30:24-7#

I: Wie war's da? #00:30:25-6#

B: War gut. Ein, eine Apartement (...) eine, ein große Apartement war dort #00:30:34-4#

I: hm (bejahend) #00:30:34-4#

B: und sechs, sieben (...) acht Leute. (...) #00:30:43-4#

I: Und die waren alle okay? #00:30:43-2#

B: der war schon, ja, die meißten war okay. Die meißten war nicht, ja. (lacht) Es ist so, passt, war gut, einfach. #00:30:52-9#

I: hm (bejahend) #00:30:53-3#

B: Auch nicht so schlecht, war gut. Und, ja, ich hab schon ein Zimmer gehabt, mit eine geteilt. Weil, wir waren schon zu zweit in einem Zimmer. #00:31:04-1#

I: hm (bejahend) #00:31:04-1#

B: Dann war okay. Dann (...) Nach dem drei Monat, drei, vier Monat, die haben gesagt: Ja, jetzt, jetzt musst du weiter. (...) Dann die haben mich nach St. Vinzent gebracht in Liebigstraße. Dort war schon auch eine Wohngruppe #00:31:25-6#

I:hm (bejahend)  #00:31:25-6#

B: mit sechs Leute. Dann (...) sechs, sieben Leute. Ja, dann habe ich schon auch ein Zimmer bekommen, mit mein Kumpel. #00:31:35-7#

I: Der, mit dem du (...)ähm #00:31:39-3#

B: nach Detschland. #00:31:40-3#

I: Cool! #00:31:41-4#

B: gekommen. Ja, wir waren schon zusammen dort, und, ja, dann wir haben gesagt: 'wir wollen schon auch zusammen, wenn irgendwo hingehen, dann (...) gehen zusammen.' Dann die haben uns ein Zimmer gegeben. Dann war ich schon eineinhalb Jahre dort #00:31:58-4#

I: hm (bejahend) #00:31:58-4#

B: Und (unv.). (lacht). Ja, dann habe ich gesagt, dass es mir reicht, und ich will nicht ähm(...) in Wohngruppe. Und die haben schon mir betreutes Wohnen gegeben, hier. In drei (unv.) #00:32:12-7#

I: Also nach der, ähm, nach der Wohnung bist du hierher gezogen. #00:32:16-5#

B: Genau. #00:32:18-8#

I: Was macht dein Kumpel jetz? Is der noch drin? #00:32:20-4#

B: Der war schon hier! #00:32:22-0#

I: Der war hier? #00:32:22-0#

B: Der war schon hier. Genau. #00:32:25-1#

I: Und, wo ist der jetz? #00:32:26-2#

B: Der hat schon Aufenthaltserlaubnis bekommen, und der, der durfte schon einfach (...) #00:32:32-4#

I: der konnte jetzt ne normale Wohnung (...) #00:32:35-0#

B: Normale Wohnung mieten, genau. Genau. ja. Der hat schon eine Wohnung gefunden, und (...) weggezogen. #00:32:43-2#

I: Machst du des dann (...)ähm, aus Regensburg raus, oder (...)? #00:32:47-8#

B: Ne. Der wohnt jetzt in Pentling, weil #00:32:48-0#

I: //Schon hier, ok #00:32:48-0#

B: der arbeitet dort. Der arbeitet auch ähm(...) als Karosseriebauer. Als Lackierer. #00:32:54-7#

I: hm (bejahend) #00:32:55-1#

B: Und (...) ja. (...) Dort hat schon, ja, hat schon ein Zimmer dann bekomm(...) gefunden und er hat geagt: 'Ja, es ist (...) gut für Ihn', weil (...) der hat, der arbeitet auch in Pentling. #00:33:13-2#

I: hm (bejahend) #00:33:13-2#

B: Der hat gesagt: 'Wenn ich dort gehe, es ist schon besser. Es ist kürzer.' Von der Arbeit her. #00:33:19-8#

I: Ja. #00:33:19-8#

B: Ja. #00:33:20-9#

I: ähm, wie lang dauerts denn so am Tag, wenn du hier raus kommst zund zur Arbeit? Wie lang bistn da unterwgs? #00:33:30-3#

B: (...) Unterwegs? Meinst du jeden Tag? #00:33:35-1#

I: hm (bejahend) #00:33:37-1#

B: Ja, acht (...) acht Stunde arbeite ich. Und der Rest, komme ich nach Hause und (...) manche Tag (...) zweimal in der Woche gehe ich zum Sport. #00:33:49-9#

I: Ah? #00:33:49-9#

B: Ja. #00:33:49-9#

I: Was machst du? #00:33:52-2#

B: Boxen! (lacht) #00:33:52-2#

I: Okay #00:33:54-4#

B: Ja, genau. Mit anderen, mit Humind und (...) im Jugendcafé. #00:33:57-9#

I: Des geht um (...)ähm 17 Uhr los meintest du oder? #00:34:00-8#

B: genau, ja. Dienstag und Donnerstag. #00:34:07-0#

I: Am Dienstag ist des auch? #00:34:07-0#

B: hm (bejahend) #00:34:07-0#

I: Auch um 17 Uhr? #00:33:59-2#

B: Um 17 Uhr. (...) Ja (lacht) #00:34:02-2#

I: Ach, des Boxen würd mich schon interessieren. #00:34:15-4#

B: Eh, Boxen ist cool, cooles Sport, aber ein bisschen anstrengend. #00:34:20-5#

I: Ja bitte, muss es sein! #00:34:22-7#

B: Ja (lacht) #00:34:24-5#

I: (lacht) #00:34:24-5#

B: ist schon viel anstrengend, aber, das ist gut. (...) #00:34:29-5#

I: ähm, wie bist'n du zum Boxen gekommen? Hast du des im Iran schon gemacht oder hier dann? #00:34:34-6#

B: nein. Im Iran (...) war ich schon im Fitnessstudio, paarmal, so (...) habe ich schon auch ein bisschen Sport gemacht, aber nicht mit Boxen undso. (...) Und, Tae Kwon Do habe ich schon gemacht, ein Jahr #00:34:51-9#

I: Tae Kwon Do? #00:34:51-9#

B: hm (bejahend) Im Iran. Und (...) Fitnessstudio. Ich war schon auch ganz sportlich da. (...) Und dann, hier (...) zwei Jahre. Ein Jahr war ich schon im Fitnessstudio, und dann (...) war langweilig. Echt, war langweilig. #00:35:10-6#

I: hm (bejahend) #00:35:12-7#

B: Und wenn du jeden Tag so ne Stunde zum Fitnessstudio gehst, zum pumpst deine Muskeln. War echt langweilig. #00:35:21-0#

I: War ich noch nie, will ich nie machen. #00:35:21-0#

B: (lacht) #00:35:21-9#

I: (lacht) #00:35:22-6#

B: Ja, du bekommst ganz Muskeln. Und wenn du nicht machst, dann geht alles weg! #00:35:28-7#

I: Vor allem, des sind total nutzlose Muskeln #00:35:31-1#

B: Genau. #00:35:31-1#

I: Also (...) Ehh. #00:35:34-3#

B: Ja (lacht). Und, ja, dann Kumpel von mir war Boxer, und der hat gesagt: ´'Ja, komm mit! Gehen wir (...) Boxen. Ich nehme mit. Und es ist besser als Fitnessstudio.' Und dann ich hab gesagt: 'Ja, ok.' Dann war ich schon bei einem Boxverein #00:35:53-8#

I: hm (bejahend) #00:35:53-8#

B: Und (...) wir waren schon sechs bis (...)acht Monaten dort. Und jedes Monat, wir haben 30€ bezahlt. #00:36:04-5#

I: hm (bejahend) Mh, des is relativ viel Geld. #00:36:07-1#

B: ähm waren echt viel, 30€. Fitnessstudio 30€. #00:36:14-4#

I: Sind 360€ im Jahr. #00:36:16-5#

B: Genau, (...) dann 30€ Boxen, war 60€, dann Internet undso weiter. 100€ gibst du alles weg! #00:36:28-7#

I: hm (bejahend) #00:36:28-7#

B: Dann wir haben schon überlegt, und wo, wo müssen wir einen Platz, einen (...) Platz suchen. Eine Halle zum Boxen. #00:36:42-3#

I: hm (bejahend) #00:36:42-3#

B: Dort wir haben schon beim Jugendcafé gefragt und die haben uns gegeben ja. Wegen (unv.) #00:36:51-2#

I: //Ach des habt quasi Ihr euch selber aufgebaut im Jugendcafé? #00:36:55-1#

B: Genau. Wir haben schon selber gegründet, also(...) #00:36:58-1#

I: Cool! Also, ähm, der Raum war quasi leer, also(...)? #00:37:06-1#

B: Der Raum war nicht ganz leer. Die anderen auch schon damals benutzt, die anderen Tage. Die, ähm, die eine Frau hat schon (...) den kleine Mädels. Die hat schon, ähm(...) Tanzunterricht gegeben. #00:37:23-5#

I: hm (bejahend) #00:37:23-5#

B: Ja, genau. Und (...) so und war schon leer, und wir haben schon gesagt: 'Ja, dann andere Tage. Wenn frei ist, dann nehmen wir.' #00:37:36-5#

I: Cool. #00:37:36-5#

B: Und zweimal in der Woche trainieren wir. Ja, ok. (...) ja. (...) #00:37:44-1#

I: ähm(...), Wie bist du denn eigentlich zum Jugendcafé gekommen? (...) #00:37:54-9#

B: (...)Jo, gute Frage. Ich war schon (...) bei der B - Klasse. (...) Ja, absch(...) normal, Schulabschluss. #00:38:06-3#

I: hm (bejahend) #00:38:06-8#

B: drei, zwei Jahre war schon bei Berufsschule, zwei, habe ich schon Schulabschluss gemacht, und dort eine Sozialpädagogen hat schon uns (...) gesagt: Es gibt schon. So eine Jugendcafé. Wenn Ihr wollt, ähm, dann (...) könnt Ihr (...) kommen. Dann war ich schon einen Tag. Dann viele Leute gesehen, dann habe ich schon gesagt: 'Es ist gut, wenn ich herkomme. Und es ist schon auch gut von der (...) deutsche Sprache und kann man schon auch was lernen und die helfen uns!' und dann (...) war schon dabei. Bin immer gekommen, und jeder frei, jeder Donnerstag habe ich schon gekocht. #00:38:54-4#

I: Ja. Jeden Donnerstag kochst? #00:38:56-7#

B: Jeden Donnerstag, ja. Früher habe ich schon jeden Donnerstag habe ich schon gekocht. #00:39:00-0#

I: Aber jetzt nichtmehr oder? #00:39:02-6#

B: Jetzt nichtmehr. Jetzt habe ich keine Zeit.(lacht) #00:39:03-9#

I: (lacht) #00:39:05-2#

B: Weil Donnerstag trainiere ich schon #00:39:06-8#

I: hm (bejahend) #00:39:06-8#

B: und (...) kochen Zeit ist nur am Donnerstag. #00:39:14-2#

I: hm (bejahend) (...) Aber des heißt, du bist dann jetz quasi erst (...) ähm, knapp zwei Jahre da drin oder? #00:39:21-4#

B: (...) drei (...) Jahre. #00:39:21-3#

I: Drei Jahre? #00:39:24-2#

B: hm (bejahend) #00:39:24-2#

I: Ah! #00:39:24-2#

B: Drei. (...) Vier. 2015. Es ist 2015. #00:39:33-4#

I: hm (bejahend) #00:39:34-8#

B: Vier Jahre. #00:39:40-9#

I: ähm, du gehst doch morgen sicher weider relativ früh ins Jugendcafé oder? #00:39:48-8#

B: Morgen, nein. #00:39:48-8#

I: Morgen nicht? #00:39:49-3#

B: Ne. Weil Ramadan Zeit ist so. Und dann (...) müssen wir selbst was daheim kochen, und (...) #00:40:01-0#

I: hm (bejahend) #00:40:01-0#

B: Ja. #00:40:01-0#

I: ähm(...) also, beim Ramadan isst man auch zu Hause dann? #00:40:09-8#

B: nein, es ist so, musst du nicht umbedigt. Zu Hause. Essen. Wenn ich im Jugendcafe geht, dann kriege ich nur (...) Abendessen, ne? #00:40:17-9#

I: hm (bejahend) #00:40:17-9#

B: Dann muss ich auch mit(...) in die Mitternacht, muss ich auch aufstehen und was Essen. #00:40:22-0#

I: hm (bejahend) #00:40:22-0#

B: Drei Uhr. Dann muss ich auch selbst was kochen. #00:40:26-6#

I: Ja. #00:40:27-6#

B: Dass die übrig bleibt. Und (...) ja. deswegen. #00:40:33-3#

I: Du kannst nicht einfach was mitnehmen von dort? ich mein (...) #00:40:34-4#

B: Geht schon, aber (...) #00:40:36-7#

I: Ja gut, dann musst du's rumschleppen. #00:40:36-4#

B: Ja (...) es ist nich so, schleppen. ist kein Problem. Aber vielleicht kommen viel Leute, und bleibt nicht übrig. Und dann, was kannst du machen? #00:40:47-0#

I: Es muss ja nicht übrig bleiben. Also ich mein (...)ähm, du kannst dir ja direkt am Anfang was wegnehmen dann. #00:40:57-0#

B: Ja es ist so, dann (...) du kannst machen. Aber wenn viele Leute kommen, und nicht zu Essen (...) bekommen, dann fühlst du dich nicht gut. Weil (...) #00:41:09-5#

I: Ja, versteh ich. #00:41:09-5#

B: Weil dann der andere kriegen nichts Essen, dann krieg zwei Portion. Dann nicht so gut, ne? Deswegen (...) #00:41:16-1#

I: hm (bejahend) #00:41:16-1#

B: koche ich. Ja. (...) #00:41:22-2#

I: (...)ähm, aber des heißt, du gehst morgen auch garned hin. #00:41:26-7#

B: Ne. #00:41:28-1#

I: Also, jetzt über die komplette Zeit des Ramadan. #00:41:30-3#

B: Komplette Zeit Ramadan. #00:41:31-4#

I: Okay. #00:41:31-4#

B: Danach (unv.) gehe ich schon. Zweimal in der Woche. Einmal zum Training und einmal zum Training und auch zum Essen. (lacht) Ja. #00:41:45-3#

I: ähm(...) Wie lang geht denn der Ramadan noch? #00:41:51-4#

B: (...) Außer dieser Woche noch zwei Wochen. (...) Bis fünfte (...) jetzt Mai zuende? (...) Bis fünfte nächsten Monat. #00:42:03-6#

I: ähm Juni. #00:42:07-1#

B: Juni, genau. Bis fünfte Juni. #00:42:07-5#

I: Okay. ähm(...) hat's eigentlich für dich ähm nach (...) irgendwelche anderen Auswirkungen gehabt, dass du dich quasi an den Ramadan hältst? Also irgendwie in der Arbeit oder so? #00:42:23-6#

B: Wie meinst du? #00:42:26-8#

I: ähm, naja du (...)ähm isst ja dann zum Beispiel auch nix. #00:42:29-6#

B: Ja. #00:42:30-4#

I: ähm(...) hat des dann irgendwie (...) irgendwelche Auswirkungen? Hat des (...) keine Ahnung, nen Mitarbeiter mal kommentiert oder was weiß ich? #00:42:41-7#

B: (...)ne. normal. #00:42:44-9#

I: Okay. #00:42:44-9#

B: Oder (...) einfach normal, ich gehe (...) der Arbeit. Ich mache meine Arbeit. #00:42:48-8#

I: hm (bejahend) #00:42:48-8#

B: Egal, was ich bekomme. #00:42:51-1#

I: hm (bejahend) #00:42:51-1#

B: Egal, wie (...) schwierig meine Arbeit sein. Mache meine Aufgabe, bekomme ich. Und ja.(...) #00:43:01-6#

I: Ja. #00:43:02-8#

B: Ja. #00:43:05-5#

I: Deine ganzen Mitarbeiter sind dann eigentlich auch echt nett oder? #00:43:08-4#

B: Meine (...) ja. ja, schon. Die sind alle nett. (...)  #00:43:14-7#

I: Glücksgriff? #00:43:17-2#

B: Ja. #00:43:18-5#

I: ähm(...) vielleicht ne Dumme Frage, aber (...) hast du, seit du hier angekommen bist, ähm, offenen Rassismus erlebt? #00:43:32-8#

B: (...) mhm. #00:43:32-8#

I: Schon? #00:43:32-8#

B: ja. (lacht) ja.(...) Parrmal habe ich schon erlebt. Nicht oft, aber paarmal. Zweimal mit eine Frau, die gleiche Frau. #00:43:48-3#

I: Hmh? #00:43:49-9#

B: Und einmal eine (...) ein Mann. Irgendein Mann einmal. #00:44:00-4#

I: //ähm, was (...) #00:44:00-4#

B: Eine alter Mann. #00:44:02-0#

I: Ein alter Mann? #00:44:04-0#

B: hm (bejahend). Ich war unterwegs, und (...) in Maximilianstraße war ich schon, (...) Hotel (...) Bahnhof gehen, und im Arcaden gehen, was kaufen. Und (...) vorne (...) auf Vorne Seite, kam da ein Frau und, zu mir, und einfach sie hat schon auf mich gespucken. #00:44:24-1#

I: Gespuckt?! #00:44:24-1#

B: hm (bejahend). Gespuckt. Und (...) die hat schon (...) keine Ahnung, geschimpft. #00:44:32-4#

I: Wie? Du gehst, ähm, die Maxstraße entlang, in Richtung Arcaden. #00:44:35-9#

B: Ja, genau. #00:44:35-7#

I: Und dann kommt dir ne Frau entgegen, und beschimpft dich und bespuckt dich? #00:44:41-1#

B: Ja danach. Genau, ja. Die hat schon einfach entgegen gekommen und die hat schon (...) gespuckt. Und (...) geschimpft und ich hab gesagt: 'Ja, passt schon.' Es war einmal, nicht einmal. (lacht) Die gleiche Frau, die gleiche war. #00:45:00-5#

I: Es ist, ähm, dann, in der, wo is denn des andere passiert? #00:45:03-4#

B: Die andere war auch schon dort #00:45:05-2#

I: hm (bejahend) #00:45:05-2#

B: Die andere war schon bei der Ampel. (...)(lacht) Genau. bei der Ampel.  #00:45:12-0#

I: Ja? #00:45:12-0#

B: In der Max. Maximilianstraße. Und ich wollte schon auch im andere Seite gehen. Und (...) ich wollte den zehner Bus, den (...) Bus zehner nehmen, und (...) keine Ahnung, war schon irgendwo (...) unterwegs. Und da, bei der Ampel, eine alter Man gekommen und er hat uns geschimpft und er hat gesagt: 'Was nacht Ihr? Müsst  wieder zurück!' Und (...) ja. Ja, er will Geld von uns und keine Ahnung, will Arbeit von uns und irgendwo. #00:45:46-3#

I: Also, ähm, er, er is an der Ampel gestanden und hat dich beschimpft, dass (...) warte: 'Er will Geld, und Ihr sollt was arbeiten!' oder wie? #00:45:58-2#

B: Der hat gesagt: 'Ja, die ähm, die Ausländer, die blöde Ausländer! Was macht Ihr?' und ja, 'wir müssen weggehen!' und ja, einfach weiter geschimpft und (...) #00:46:09-7#

I: Hmm, was macht Ihr? Auf jeden Fall keine leute an der Ampel beschimpfen. (lacht) #00:46:14-2#

B: Ja. (lacht) Seine Arbeiten, jammer, 'unsere Arbeitsplatz geht weg und unsere Geld, unsere Geld und alles!' Und er hat schon weitergeschimpft und ich habe gesagt: 'Ja, passt. (...) Wiedersehen' #00:46:32-2#

I: Uh.(...) #00:46:32-2#

B: Ja. #00:46:32-6#

I: Hat dich das danach noch länger beschäftigt, oder (...) passt? #00:46:38-1#

B: Ne, einfach schon (...) geschimpft und weiter gegangen. #00:46:39-4#

I: Was für ein Idiot! #00:46:42-0#

B: Okay. (lacht) Ja. #00:46:42-6#

I: Uh. #00:46:42-6#

B: Ja, wir hatten einmal, war ich schon auch (...) in Dachauplatz. (...) ähm, mit eine (...) eine Junge. Ein deutscher Junge. War mein Kumpel. (...) Wir haben schon da einfach gestanden, und einige, einfach einige zu uns gekommen und (...) nicht zu mir direkt gekommen. Zu meine Kumpel gekommen. Und der hat gesagt: 'Hey, du bist doch Deutscher! Was machst du mit dem Ausländer?'(...) #00:47:20-1#

I: Was? #00:47:20-1#

B: Ja. Der hat gesagt: 'Du bist doch Deutscher! Was machst du mit der Ausländer? Schämst du dich nicht? (...) Ja. Schämst du dich nicht?' Er hat gesagt: 'Wieso? Wiesoll ich mich schämen?' Und er hat geasgt: 'Ist doch Ausländer! Du musst Ihn schlagen! Ist Ausländer!' #00:47:38-9#

I: //(lacht) #00:47:41-9#

B: 'Ist schon gut! Der versteht überhaupt nicht! Ja, de rmuss zurück gehen!' Ja, dann der hat gesagt: 'Ja, passt schon. Geh (...) geh weg.' #00:47:51-8#

I: ähm, der der (...) #00:47:56-1#

B: Der war eigentlich ganz sauer, (...) aber sagt: 'Ja, was soll ich machen? Nix!' #00:48:02-3#

I: ähm, der, der des gesagt hat, war des auch nen alter Mann oder (...)? #00:48:06-4#

B: Ne, war ein Junge. 25? 30? (...) #00:48:15-7#

I: (...) Und der is zu dem Freund von dir #00:48:19-0#

B: //Zu mein Freund, ja. Zu mein Freund von von mir gekommen, und der war Deeutscher und der hat gesagt: 'Hey, was machst du da?!' #00:48:24-9#

I: (lacht) #00:48:24-9#

B: (lacht)(...) Ja. #00:48:32-8#

I: Was is denn des für seine komische Situation? (...) #00:48:34-4#

B: Keine Ahnung. (lacht) Keine Ahnung. #00:48:37-6#

I: Uh. #00:48:38-1#

B: Ich hab schon echt, im Iran, ähm(...) solche Situationen habe ich schon garnicht erlebt. Ja, aber (...) nun, hast du im Iran mit die Polizei Probleme. Aber mit andere Leute? Wenn du normale Leute hast und ein Problem? Garkein Problem!  #00:48:55-4#

I: hm (bejahend) #00:48:55-8#

B: Ja, wenn du nicht natürlich selbst Scheiße baust, aber sonst (...) alle sind (...) meißtens sehr freundlich. Es gibt schon Arschloch, aber (...) (unv.) aber, er schon auch natürlich hier. Gibt schon auch andere Seite, viele nette Leute. (...)  #00:49:18-5#

I: (...) Des irritiert mich grad voll. (lacht) #00:49:20-0#

B: (lacht) #00:49:20-0#

I: Ja, wenigstens, ähm(...) hier auf die Polizei verlassen. #00:49:28-4#

B: hm (bejahend) Ja. kann man nicht machen. Wenn du (...) (unv.) dann, bist du auch selbst (unv.). (...) Wie Ihr oder wie Sie(...) #00:49:41-0#

I: Nochmal? #00:49:41-0#

B: ich habe geasgt, wenn du gegen (...) gegenseitig was schimpfst, (...) den Typ, dann bist du auch, so eine (...) so eine Idiot. Dann (...) Wie unterscheidest du zwischen dir und Ihm? #00:49:58-0#

I: Ja. Am besten einfach garnicht anfangen. #00:50:00-5#

B: Ja. (...) #00:50:03-3#

I: (lacht) Ich muss gestehen, ich hätt zwar vermutlich selber angefangen zu schimpfen, aber (...) das ist vermutlich die klügere Entscheidung, das nicht zu machen. (...) #00:50:17-5#

B: Ja, manchmal hast du auch keine Bock. Du willst auch (...) den schimpfen zurück, aber manchmal sagst: 'Hey, lass doch! Das (...) der geht weg.' und, sagst du: 'Überhaupt keine Stress. Geh. Bitte.' #00:50:41-5#

I: (...)ähm(...) aber sind dir noch andere Sachen aufgefallen, die(...) ähm, in Deutschland jetz anders sind als im Iran? Also(...) so: Die Leute im Iran sind dann wohl ein bisschen freundlicher?(lacht)(...) #00:51:02-5#

B: (...) in jede Land gibt es schon gute Leute und schlechte Leute. #00:51:05-4#

I: hm (bejahend) #00:51:05-8#

B: Ja. #00:51:07-6#

I: Klar, sind alles Menschen. #00:51:07-7#

B: Alle, alle sind Menschen. Ja. genau. Aber (...) im Iran hast du schon (...) eigentlich keine Sicherheit. #00:51:18-6#

I: hm (bejahend) #00:51:19-1#

B: Und (...) mehr Probleme. Aber hier nicht. Dafür manche Rassisten, die schimpfen dich. #00:51:26-0#

I: hm (bejahend) #00:51:26-6#

B: Die schlagen dich natürlich nicht, aber die schimpfen dich. Und (...) zum Schimpfen (...) bekommst du garnicht was. (...) Musst du nur einfach (...) aus und raus (...). #00:51:48-7#

I: Ja. (...) wenn du's einfach ignorieren kannst, dann (...) dann isses ja gut. #00:51:54-4#

B: Ja. #00:51:54-4#

I: Sollte aber trotzdem ned passieren, also ganz ehrlich. Es ist 2019. #00:51:59-1#

B: Ich kann schon auch verstehen, weil (...) #00:52:01-2#

I: Echt? #00:52:01-2#

B: Ja, natürlich. Kann schon auch den Leute verstehen. Weil (...) die, die manche Ausländer, die sind auch nicht, keine gute Leute. Die manchen sind einfach Idioten. Die manchen sind einfach Arschloch. Und die bauen einfach Scheiße und (...) #00:52:16-1#

I: Ja. #00:52:16-1#

B: Weil, wenn die (...) weniger, die Rassist sind. Die seht im, im Fernseher. Oder, keine Ahnung. Im Radio. Und im Zeitung. Die lest was. #00:52:26-3#

I: hm (bejahend) #00:52:26-3#

B: 'Heute hat eine Ausländer Scheiße gebaut.' #00:52:31-1#

I: hm (bejahend) #00:52:31-6#

B: Dann der sagt: 'Hey (...) Ja. Alle sind schlecht und alle sind gleich.' #00:52:39-9#

I: Ja, ähm, tatsächlich wenn (...) wenn nen Ausländer irgendwie Scheiße baut, dann (...) ähm wird das(...) sehr viel schneller von den Nachrichten aufgegriffen. Das ist (...) auch unpraktisch dann. #00:52:53-9#

B: Das ist natürlich auch unpraktisch. Weißt du, wenn, wenn ein Ausländer (...) keine Ahnung, hald Scheiße bauen, dann die sagen: 'Ja (...) Ausländer. Islamischer, keine Ahnung, so so so.' Dann, wenn eine Deutscher macht das, dann die sagen nicht, dass (...) keine Ahnung, die war Schuld, sondern die sagen: 'Ja, die war krank!' Ja. Die war (...) keine Ahnung, psychisch krank oder keine Ahnung #00:53:26-3#

I: hm (bejahend) #00:53:26-3#

B: Einfach, ähm, (...)die sagen, dass die krank war, die hat schon so gemacht. Aber, ja. #00:53:33-4#

I: Das wird dann grundlegend anders behandelt. #00:53:35-0#

B: Ja, ja. Genau. Wird anders behandelt, genau. (...) Die Behandlung sind nicht gleich (...) ja. #00:53:40-4#

I: ähm (...) gestern, oder heute? Habe ich in der Zeitung gelesen, dass (...) ähm rechtsextreme Straftaten haben jetzt (...) deutlich zugenommen tatsächlich. Also (...) ähm, im, ähm, die Ausländer machen (...) also, da hat sich die Zahl der Verbrechen nicht wirklich verändert, aber die Zahl der, ähm, rechtsextrem motivierten Straftaten (...) die is tatsächlich deutlich nach oben gegangen. (...) Also, mal schaun in welche Richtung das jetz geht. #00:54:16-5#

B: hm (bejahend) #00:54:17-1#

I: Aber (...) Ja. #00:54:20-2#

B: Das ist alles so (...) wenig, wenig, wenig auf Person. Wenn ich Aufenthaltserlaubnis habe #00:54:27-7#

I: hm (bejahend) #00:54:28-1#

B: dann dürfe ich alles machen, ne? Dann habe ich keinen Stress. #00:54:33-4#

I: hm (bejahend) #00:54:33-4#

B: Dann gehe ich jeden Tag zum Arbeit, komme ich zurück, heim. #00:54:35-3#

I: Ja. #00:54:36-1#

B: Dann habe ich was zu tun. #00:54:36-5#

I: Ja. #00:54:38-4#

B: Wenn ich keine Aufenthaltserlaubnis hätte? (...) Und, garnicht in der Hand (...) #00:54:44-0#

I: hm (bejahend) #00:54:44-0#

B: Dann darf ich nicht (unv.). Kann ich was machen? Ich kann nicht in die Stadt gehen. Ich darf nicht in der Disco gehen. Ich darf nicht, ähm(...), nach Österreich gehen. Ich darf nicht (...) arbeiten. #00:54:57-5#

I: hm (bejahend) #00:54:58-5#

B: Ich darf nicht garnicht, mich ähm, beschäftigen. Dann (...) das ist alles Problem, und das ist alles (...) Stress und (...) und dann überlegst du, wenn du auch gute Mensch bist. #00:55:07-8#

I: hm (bejahend) #00:55:07-8#

B: Dann baust du irgendwann Scheiße. Dann du kannst auch nicht (unv.).  #00:55:11-8#

I: hm (bejahend) #00:55:12-3#

B: Du gehst in die Straße, und keiner (unv.) eine dich? Und du hast ganz viele Stress im Kopf? #00:55:21-0#

I: hm (bejahend) #00:55:21-0#

B: Und du machst gleich ein Fehler! Und schlagst Ihn. #00:55:21-5#

I: hm (bejahend) #00:55:22-1#

B: Oder keine Ahnung, du gibst eine Antwort. Dann? So ist das. #00:55:28-0#

I: Dann stheht's in den Nachrichten. #00:55:28-9#

B: Genau, dann steht in der Nachricht: 'Ja, Ausländer hat schon Deutschen (...) ja, geschlagen', oder keine Ahnung. So. #00:55:40-0#

I: Ja, natürlich. wenn du (...) Tatsächlich, wenn ma keine Aufenthaltsgenehmigung hat, ähm(...) dann is man quasi noch in Deutschland und wartet auf die Abschiebung. #00:55:51-9#

B: Genau. #00:55:53-9#

I: Ab(...) und, diese Deadline kommt immer näher und man kann nichts machen! #00:55:57-6#

B: hm (bejahend). Du kannst nicht dich konzentrieren, du sagst jeden Tag: 'Ja, ob ich bleiben darf? Ob Ihr, ob (...) keine Ahnung. Ich bin dabeim. Vielleicht, die kommen. Die Bullen kommen jetzt und (...) die nehmen mich fest und die schieben mich ab! Nach Afghanistan.' Du bist garnicht, du fühlst dann nicht (...) du fühlst dich garnicht gut. #00:56:16-5#

I: Ne. #00:56:16-5#

B: Ja. Ist schon passiert schon bei dem, bei anderen Leute. Und (...) #00:56:22-3#

I: // Kennst du (...) kennst du Leute, denen des passiert is? #00:56:24-4#

B: (...) Ich kenne schon eine, ja genau. Der war schon ähm (...) in Regensburg, und die hat schon nach Re(...) ähm, nach Nürnberg gegangen. #00:56:35-9#

I: hm (bejahend) #00:56:36-5#

B: Und (...) zwölf Uhr in der Nacht (...) drei Uhr in der Nacht! Polizei (...) sind gekommen und die haben schon (...) den festgenommen und (...) abgeschoben nach (...) Afghanistan. #00:56:50-6#

I: Scheiße. #00:56:52-1#

B: Dann hörst du solche Geschichten, und dann hast du viel Angst. (...) #00:56:57-0#

I: Ja. (...) Aber hey, Afghanistan ist sicher! (...) #00:56:59-3#

B: Ja, die sagen einfach sicher. (...) #00:57:04-1#

I: Fuck. #00:57:05-3#

B: Die drei (...) drei Jahren, und dann (...) echt viel Leute gestorben. Ganz, ganz, ganz viele Leute gestorben. #00:57:11-9#

I: Ja. (...) ich versteh's nicht. #00:57:19-1#

B: Ich verstehe auch nicht. Weil (...) keine Ahnung #00:57:23-0#

I: (...) #00:57:25-5#

B: Die anderen machen Geschäft und wir sind im Arsch.(...) #00:57:42-0#

I: (...)ähm(...) des wär dann auch schon direkt das Letzte: ähm (...) mit wem bist du denn momentan noch so direkt im Kontakt? Also natürlich die Leut aus'm Jugendcafé. Hast du auch noch Kontakt in den Iran zu deiner Familie oder so? #00:58:10-3#

B: In die Iran zu meine Familie? Ja, klar. Wir reden schon telefonisch. #00:58:13-4#

I: Ihr könnt auch ganz normal telefonnieren? #00:58:17-1#

B: Ganz normal telefonnieren. #00:58:19-0#

I: Okay, weil (...) ähm, aus nem andren hab ich's mitbekommen, dass (...) der sich Gedanken gemacht hat, dass zum Beispiel des Telefonnat dann verwanzt ist? Weil's im Iran wohl teilweise, ähm(...) auf der Suche nach den IS-Leuten #00:58:41-1#

B: hm (bejahend) #00:58:41-1#

I: ähm(...) Telefonnate abgehört werden, und wenn's ähm, jemanden der illegal dorten lebt, ähm #00:58:50-4#

B: Ja. Es ist auch so. Es ist auch so, weil (...) als (...) eine Afghaner. Das ist (...) als eine Afghaner, du lebst dort und (...) du lebst auch illegal #00:59:05-8#

I: hm (bejahend) #00:59:05-8#

B: Wenn die Polizei dich sehen und festnehmen, dann sagen: 'Ja, wir geben dich Geld. Schicken wir dich nach Syrien. Oder nach Irak.' #00:59:20-0#

I: //und dann #00:59:20-0#

B: 'Du kämpfst. Für uns!' #00:59:22-4#

I: ja. #00:59:22-4#

B: 'Gegen. Gegen Leute. Wo dort sind.' Gegen Amerikaner, ne? Das sind Amerikaner oder keine Ahnung. Oder gegen IS, ähm Taliban oder keine Ahnung. Die sagen irgendwas! #00:59:34-3#

I: Ja. #00:59:34-3#

B: Die erzählen (...) irgendwelche Geschichte und sagen: 'Ja, du musst dort gehen, und du musst kämpfen, und ja' Möchten natürlich nicht gehen! 'Oh? ähm, du gehst in Gefängnis! (...) vier bis fünf Monaten. Dann, schieben wir dich nach Afghanistan danach.' (...) #00:59:55-5#

I: Also, entweder du ziehst in den Krieg, an die Front. Oder du gehst in's Gefängnis. #01:00:01-2#

B: Genau. #01:00:02-7#

I: verdammte Scheiße. (...) #01:00:06-9#

B: Du kannst garnicht machen. Du hast keine Rechte. #01:00:10-6#

I: hm (bejahend) #01:00:10-6#

B: Du hast garkeine Rechte. (...)  #01:00:14-8#

I: Ich glaub ich wär' in der Situation (...) wär ich auch, wär ich auch abgehauen. (...) #01:00:20-4#

B: Ja, das kommt drauf an. (lacht) Wo du lebst. #01:00:25-5#

I: Ja. (...) Aber du kannst mit deinen Leuten Zuhause ganz normal telefonnieren (...) #01:00:37-1#

B: Ich kann schon ganz normal telefonnieren. #01:00:36-3#

I: Das ist gut (...) #01:00:40-1#

B: Ja. #01:00:41-0#

I: ähm, mit wem bist du denn dann hier in Deutschland sonst noch in Kontakt? Also, vermutlich Campus Asyl denk ich mal? #01:00:50-3#

B: Campus Asyl gehe ich nicht. #01:00:51-3#

I: Ne? #01:00:52-8#

B: Ne, war ich garnicht in Campus Asyl. Und (...) ja. Nur Jugendcafé. #01:01:02-1#

I: hm (bejahend) #01:01:02-1#

B: Manche Leute aus Jugendcafé. Und ich kenne schon auch (...) paar Leute. (...) Auch in der Stadt. In Regensburg. Ich kenne schon viele Leute in Regensburg. Deutsche, Ausländer, alle. Kenne ich. #01:01:19-9#

I: Ja. #01:01:20-8#

B: Die gute Leute habe ich schon immer Kontakt. (lacht) Die schlechte Leute nicht. #01:01:22-0#

I: (lacht) Genau so musst du's machen! #01:01:27-4#

B: Ja. (...) Ja. (...) #01:01:33-2#

I: Dann glaub ich, ähm, lass ma's jetzt hier direkt mal. Danke, dass du (...) ähm, dass du mitgemacht hast! #01:01:41-0#

B: Gerne, gerne. #01:01:49-9# \newpage



\end{document}
