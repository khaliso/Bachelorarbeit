\section{Interview 3}

I: Des is mein Problem dann später wenn ich's abtippe. #00:00:05-0#

B: Okay. #00:00:05-4#

I: Also, muss ich nämlich dann noch an  PC und des quasi alles aufschreiben. ähm (...) Ja, erstmal: Wie war dein Name nochma? #00:00:15-9#

B: Mohsen. #00:00:14-6#

I: Ahso. Hast ja geschrieben, seh ich dann. ähm (...) Wie alt bist du? #00:00:25-7#

B: Ich? Bin (...) Schätz mal. #00:00:26-4#

I: (...) 22. #00:00:30-5#

B: Ja.  #00:00:33-1#

I: Bist du? #00:00:36-6#

B: Ja, also ich werde 22. #00:00:37-7#

I: Du wirst 22, also bist noch 21? #00:00:40-4#

B: Ja. #00:00:40-4#

I: Okay. ähm (...) Wo bist du denn ursprünglich her? #00:00:48-5#

B: Afghanistan, Iran. #00:00:49-4#

I: hm (bejahend)? #00:00:52-4#

B: Afghanistan/Iran. Ich meine meine Eltern kommen aus Afghanistan #00:00:54-0#

I: hm (bejahend) #00:00:54-0#

B: Aber ich bin im Iran geboren. #00:00:56-1#

I: Okay #00:00:59-1#

B: Aber ich bin trotzdem Afghane. #00:01:01-1#

I: Also du bist quasi als Afghane groß geworden. #00:01:03-7#

B: Ja. #00:01:01-9#

I: Im Iran. ähm und seit wann bist du dann hier in Deutschland? #00:01:11-0#

B: Seit wann? Vier Jahre. Im August werden 4 Jahre. #00:01:17-3#

I: Im August werden's 4 Jahre. Also zwanzigfünfzehn dann. Also 2015. #00:01:25-2#

B: Genau. #00:01:23-3#

I:hm (bejahend) ähm (...) Warum hast du dich auf'n Weg gemacht? #00:01:37-3#

B: Hmm, gute Frage. ähm hm, was soll ich dir dann erzählen bitte? Also, im Iran, (...)so etwas jung war so schwierig, weil die Afghaner, egal ob man dort geboren ist oder nicht, Hauptsache wenn man Ausländer ist, ne? Von Eltern Ausländer. #00:01:54-0#

I: Das hab ich von Mahdi mitbekommen. #00:01:56-6#

B: Genau. Wenn Eltern Ausländer ist, dann ist auch Ausländer. Also, zum Beispiel ich bin im Iran geboren und ist trotzdem Ausländer. #00:02:05-2#

I: Ja. #00:02:06-0#

B: Ich werde dann keine Dokumente aus dem Iran bekommen, (unv.) da (...) ja, leben darf undso. Arbeiten darf, und (...) in die Schule gehen darf sogar. ähm ist schon (...) ähm Ja.#00:02:22-6#

I: ähm #00:02:23-8#

B: Also, man darf hald (...) als Afghane im Iran nix machen. Außer, dass man alles illegal macht. Haben wir bis (...) vorher gemacht. #00:02:30-0#

I: Scheiße. #00:02:31-3#

B: Ja. Und es bringt eh nichts. Du, du arbeitest viel mehr als normale Leuten #00:02:36-8#

I: hm (bejahend) #00:02:36-8#

B: Aber kriegst du viel weniger, als die (...) also ich mein hald, Gehalt undso. #00:02:43-3#

I: ähm, war's dann bei dir auch so wie bei Mahdi, dass du hier, also hier dann erst die Schule besucht hast? #00:02:52-4#

B: Ja. #00:02:53-8#

I: Also, du hast dann auch hier (...) ähm die afghanische Schrift, die arabische Schrift gelernt oder? #00:03:02-0#

B: Hier? Nein.  #00:03:02-5#

I: Nicht? #00:03:02-5#

B: Ich konnte schonmal im Iran, also da habe ich was gelernt. Da durfte ich nicht, eigentlich in normale Schule gehen #00:03:07-1#

I: hm (bejahend) #00:03:07-1#

B: Aber, ich habe so privat (...) ich konnte solche Leute, die mir beibringen. #00:03:17-5#

I: Also, du hast quasi Leute gekannt, die dir das dann beigebracht haben? #00:03:22-9#

B: Also, Lesen und schreiben. #00:03:27-5#

I: ähm (...) also, ich würd jetz so nen bisschen die Fragen durchgehen wenn du magst? ähm (...) Wie bist du nach Deutschland gekommen? #00:03:39-6#

B: Fuß. #00:03:39-6#

I: Zu Fuß? (lacht) #00:03:39-6#

B: Nah, ähm von Iran nach Türkei komplett zu Fuß, da war ich so (...) ähm (...) also bis zum Grenze #00:03:50-9#

I: hm (bejahend) #00:03:50-9#

B: Iran zu Türkei das ist so mim Auto irgendwie #00:03:53-6#

I: Ja. #00:03:53-6#

B: Dann (...) diese Grenze waren glaub ich sieben Tage unterwegs. #00:04:02-1#

I: Sieben Tage? #00:04:05-3#

B: Weil wir durften auch nich (...) illegal aus dem Iran raus. Also keiner darf illegal aus dem Iran raus. #00:04:11-3#

I: Ja? #00:04:11-3#

B: Und, wenn man hald aus dem Iran raus will, dann muss man so viel Geld bezahlen. Egal ob der Iraner ist, oder (...) Ausländer. #00:04:18-8#

I: Nur wenn man aus dem Land raus will? #00:04:23-8#

B: Genau. (...) Weil diese Situation im Iran so (...) schwierig, dass die Leute, wenn die irgendwo anders hingehen, ist viel besser als Iran. Auch die Iraner. #00:04:39-8#

I: Also, die wollen alle abhauen oder (...)? #00:04:40-1#

B: Ja.  #00:04:41-8#

I: Die wollen alle raus? #00:04:43-6#

B: Ja, wegen Politik undso. #00:04:46-9#

I: hm (bejahend) #00:04:46-9#

B: Weil die machen so viel Scheiß mit den Leuten, aber wir können nix machen. (...) #00:04:53-3#

I: Shit. #00:04:55-3#

B: Ja. (...) Und die meinen hald, dass das Problem wegen den ähm, USA ist. Also die Politiker, die meinen, dass das nur wegen den USA ist. Aber das ist nicht so. #00:05:08-2#

I: ähm #00:05:08-2#

B: Weil, die sind nicht so gut miteinander, also die verstehen nicht einander. USA und Iran. #00:05:13-0#

I: Das hab ich mitbekommen, ja. #00:05:15-3#

B: Das weiß jeder glaube ich. (lacht) #00:05:15-3#

I: (lacht) #00:05:17-9#

B: Ja. #00:05:18-9#

I: ähm was denkst du ist der Grund? #00:05:21-0#

B: Was? #00:05:21-9#

I: Warum's so schlecht läuft. #00:05:28-5#

B: Ich glaube (...) also, ist mein Meinung. Weiß nicht ob das so ist oder nicht. Weil Politik und Religion gemischtrennt. Im Iran. #00:05:39-4#

I: hm (bejahend) #00:05:40-1#

B: Weil die Politiker, die machen alles was Sie wollen, und sagen: 'Ja, ähm, unsere Religion sagt so.' Also am Meißten. #00:05:49-4#

I: hm (bejahend) #00:05:49-9#

B: Zum Beispiel, die junge Leute, die wollen auch Freiheit haben. Weil, am Meißten du wollen kein Kopftuch haben und solche Sachen. #00:05:56-5#

I: Ja. #00:05:59-1#

B: Aber, Relig- aber, Politik sagt: 'Nein, es muss so bleiben wie es ist.' (...) #00:06:07-6#

I: Denkst du, das ändert sich? #00:06:09-7#

B: Hmm, bis die, diese (...) Politik ist, dann (...) (Schüttelt Kopf) #00:06:15-2#

I: Ne? #00:06:15-2#

B: Leute versuchen heute immer (Verschiedene Leute aus dem Jugendcafé stellen sich vor) #00:06:21-9#

I: ähm (...) und (...) warte, jetzt seit du hier in Deutschland bist, ähm mit wem bist du so in Kontakt? Also hier natürlich die Leute ausm Jugendcafé #00:06:46-0#

B: Ja. Ich meine, normale Kumpel hald. Zum Beispiel, mein Cousin, wir sind zusammen hergekommen. #00:06:48-1#

I: hm (bejahend) #00:06:48-1#

B: Der macht auch Ausbildung als Maler, der ist auch bald fertig, im Sommer (unv.). Ja, die Leute hier, ähm Fußballmannschaft, also die Leute ausm (...) #00:07:02-1#

I: //Spielst du Fußball? #00:07:03-3#

B: Und (...) meine Freundin, seine Familie, also Ihre Familie #00:07:07-3#

I: hm (bejahend) (...) cool. ähm, wo wohnst du dann? #00:07:15-3#

B: In Regensburg, bei Steinregen (unv.) #00:07:18-1#

I: Okay. Wie viele Wohnungen hast du dir jetz aktuell (...) (unv.) #00:07:26-6#

B: ähm #00:07:28-1#

I: Und insgesamt ähm (...) wie hat denn des funktioniert, so (...) also, du bist in Deutschland quasi angekommen, und wo ging's für dich quasi dann los? #00:07:39-0#

B: In Passau. #00:07:41-2#

I: In Passau.  #00:07:41-6#

B: Ja ich komme in Passau rein #00:07:41-7#

I: hm (bejahend) #00:07:41-9#

B: Dann wurde ich von Polizei erwischt, dann wurde ich wo, bei, keine Ahnung, Polizei hald angemeldet, dass wir in Deutschland angekommen sind. #00:07:53-8#

I: Ja. #00:07:53-8#

B: Genau, dann (...) haben die uns weiter nach Regensburg geschickt (...) (Leute aus dem Jugendcafé stellen sich vor) (...) Passau sind wir angekommen #00:08:36-9#

I: hm (bejahend) #00:08:36-9#

B: Und der hat uns hald (...) waren (...) ein Uhr in der Nacht in Passau, dann morgen Mittag haben uns die in Regensburg geschickt, mit dem Bus nach Regensburg geschickt.  #00:08:49-1#

I: Ja. #00:08:51-4#

B: Da war dann (...) bei Landratsamt (...) ein, ein Gebäude. Da ist jetzt Parkplatz, bei (...) #00:08:56-2#

I: Okay? #00:08:56-2#

B: Hmm (...) Aldi? Oder Lidl? Lidl, genau! Irgendwas mit L. War ein Gebäude, da war so (...) so eine Wohngruppe. #00:09:12-5#

I: ähm, ja. Also (...) da war quasi ein Gebäude, und da seid ihr am Anfang untergebracht worden? #00:09:16-7#

B: Genau. Da waren wir dann (...) keine Ahnung, sechs Monaten vielleicht? Da sind wir dann in die Deutschkurs gegangen. Da war super. Weil (...) es war so (...) wir, wir konnten nix, keine Sprache, keine Regel undso #00:09:33-2#

I: hm (bejahend) #00:09:33-7#

B: Also garnix, null. Wir haben da angefangen. #00:09:36-3#

I: Ja. #00:09:37-0#

B: Dann, nach zwei, ja. Zwei Monaten, sind wir zu Deutschkurs gegangen. Davor hatten wir nix. Also zwei Monaten lang, hald nix zu tun. (lacht). Ja, da hatten wir Platz zum Schlafen, hald was zu Essen gekriegt, und dann wir sind da schon (unv.) undso. #00:09:56-3#

I: hm (bejahend) (...) Da hattet Ihr quasi (...) zwei Monate lang komplett frei? #00:10:02-7#

B: Genau. Dann haben wir mit durch (unv.) unso weiter haben wir immer ein bisschen mehr verstanden, Regeln und die Sprache undso. Das wird immer schwieriger dann,  #00:10:12-8#

I: hm (bejahend) #00:10:12-8#

B: wenn man nicht versteht und macht irgendwas, dann denkt man nicht ob das (...) schlimm ist undso ne? #00:10:16-6#

I: Ja #00:10:16-6#

B: Ob das peinlich ist? Weil, mann weiß ja nicht  #00:10:20-3#

I: Alles egal #00:10:24-3#

B: Alles egal, genau. Dann (unv.) immer, ähm irgendwo Kleidung und sowas besorgen, ne? Fußball. #00:10:29-9#

I: hm (bejahend) #00:10:30-7#

B: Wir sind immer, jeden Tag, zum Fußball gegangen und wieder zurück. Jeden. Das war unsere Ding. Und in die Stadt rumlaufen. Genau. (lacht) #00:10:41-9#

I: ähm und dann habt ihr quasi nach zwei Monaten den Deutschkurs da angenfangen #00:10:47-3#

B: Genau. Da war ich glaube ich vier Monate in Deutschkurs. Dann, (...) also ich persönlich so (unv.) war vier Monate in Deutschkurs, dann habe ich (...) #00:10:56-5#

I: // wie lief der? #00:10:56-5#

B: Hm? Wie meinst du? #00:11:00-9#

I: ähm, also hat des einfach so funktioniert? #00:11:05-7#

B: Schon, da haben wir schon bisschen was verstanden. Was gelernt undso.  #00:11:07-2#

I: hm (bejahend) #00:11:07-7#

B: Also die Lehrerin war irgendwie so gut, dass wir uns gut verstanden haben #00:11:12-5#

I: hm (bejahend) #00:11:12-5#

B: Also wir konnten nichts, also wir konnten kein Englisch, und die konnte auch natürlich unsere Sprache nicht. #00:11:17-4#

I: hm (bejahend) #00:11:17-4#

B: War eine Deutsche. Und, aber die konnte irgendwie uns Sachen beibringen undso. (Begrüßung).  #00:11:28-8#

I: hm (bejahend) #00:11:28-8#

B: Ja hald mit Zeigen (...) und Bilder zeigen und solche Sachen. #00:11:33-1#

I: WAr ne gute Lehrerin #00:11:33-1#

B: Genau, ja #00:11:33-1#

I: Cool #00:11:33-1#

B: (...) Da war ich vielleicht vier Monaten. #00:11:37-1#

I: hm (bejahend) #00:11:37-1#

B: Ungefähr hald. Da war ich dann (...) danach war ich in der Berufsschule (...) da habe ich dann neunte Klasse angefangen. #00:11:45-1#

I: Die neunte Klasse. #00:11:46-0#

B: Ja. #00:11:47-1#

I: Also direkt der Einstieg in der neunten Klasse dann? #00:11:49-6#

B: Genau. #00:11:50-7#

I: Okay #00:11:52-2#

B: Weil da sind dann, also die Lehrerin sieht, wie (...) wie gut sind wir?
 #00:11:56-0#

I: hm (bejahend) #00:11:56-0#

B: Und die sagen, so wir sind hald zum Gymnasium gehen und (unv.) oder hald ähm, ähm, wie heißt das? FOS gehen oder hald irgend was anderes. Oder weiter im Deutschkurs bleiben.  #00:12:10-8#

I: hm (bejahend) #00:12:11-4#

B: Dann bin ich hald zu Dings gangen, neunte Klasse. #00:12:14-5#

I: hm (bejahend) #00:12:15-0#

B: Habe ich neute Klasse fertig gemacht, habe ich, also ich hab mir gedacht, dass ich jetz am, ähm, ähm gut genug bin für eine Ausbildung #00:12:24-2#

I: Ja. #00:12:24-2#

B: Habe ich meine Ausbildung angefangen. #00:12:30-4#

I: Was hast du für eine Ausbildung gemacht? #00:12:30-4#

B: Fachkraft im Gastgewerbe. Bin bald dann fertig. #00:12:31-0#

I: Gefällt's dir? #00:12:34-8#

B: Schon.  #00:12:36-5#

I: Wo bist du denn? #00:12:39-4#

B: (unv.) #00:12:36-8#

I: Hm? #00:12:36-8#

B: (unv.) #00:12:39-1#

I: (...) #00:12:40-8#

B: Ist bei (unv.)  #00:12:46-0#

I: Ich glaub des kenn ich nicht. #00:12:46-6#

B: Hmh, egla. (lacht) #00:12:46-7#

I: Okay. ähm (...) wie, wie war's denn in der Schule? Wie viele, ähm (...) wie viele Deutsche waren da drin? #00:12:59-4#

B: Keine Deutsche. #00:13:00-1#

I: Keine? #00:12:59-2#

B: Ja. Wir waren alle die Ausländer, die (...) paar Jahren in Deutschland sind undso. #00:13:04-4#

I: hm (bejahend) Und das hat funktioniert? #00:13:08-0#

B: Ja. (Begrüßung) (...) Genau. #00:13:20-4#

I: ähm (...) gefällt's dir hier? (...) #00:13:22-5#

B: Also, auf jeden Fall besser als im Iran. #00:13:26-9#

I: (lacht) Okay. #00:13:28-8#

B: Ja. #00:13:30-6#

I: ähm eins interessiert mich ähm, des (...) hat mich bei Mahdi auch interessiert, des hab ich nur dann vergessen zu fragen: Ich mein, im Iran - du kannst der Polizei ned vertrauen. Es (...) geht quasi ned. Oder? #00:13:48-4#

B: Ne. Wenn man auch (...) als Afghane zur Polizei geht, dann (...) erwischen uns, ne? Dann die schicken gleich nach Afghanistan zurück. #00:13:58-2#

I: Ja. #00:13:58-2#

B: Ja. Weil wir haben nichts vom Iran, ne? Wir haben keine (...) Duldung oder irgendwas, dass man dableiben darf.  #00:14:09-4#

I: Ja. ähm(...) Wie, wie sieht's dann hier in Deutschland aus? Kannst, kannst du den Polizisten bei uns auf der Straße irgendwie vertrauen, oder (...) ? #00:14:19-0#

B: Im Iran? #00:14:23-7#

I: Ne, ich mein (...) des hab ich dämlich formuliert. Ich meine (...) ähm Hast du irgendwie (...) mit den (unv.) hier oder mit der Polizei, das läuft alles oder? #00:14:38-8#

B: Ja, schon. Ich mache meine Sachen und (unv.). #00:14:40-2#

I: Passt. #00:14:42-4#

B: Also, ich kann auf jeden Fall vertrauen, weil da läuft alles über Regeln, ne? Aber im Iran ist das nicht so. #00:14:48-8#

I: hm (bejahend) (...) Da läuft's nicht über Regeln? #00:14:52-6#

B: : Ne. #00:14:53-9#

I: Keine Gesetze? #00:14:54-2#

B: Schon, aber die macht nix. Und ist auch egal. Ja. Man kann auch nix dagegen, wenn wir Polizei sehen, dann (...) wir haben was zu machen ne? Wir können das machen. (...). Also im Iran, das war auch (...) so schlimm. Bevor ich nach Deutschland komme, da konnte Afghaner selber keine SIM-card selber kaufen.  #00:15:19-0#

I: Was? #00:15:20-3#

B: Also, hald anmelden. #00:15:22-6#

I: hm (bejahend) #00:15:22-6#

B: Das musste ein Iraner machen, also ich hatte auch so Irane Kumpel undso, Freunde. #00:15:28-4#

I: hm (bejahend) #00:15:28-4#

B: Die haben immer für mich sowas gemacht. Ich durfte auch keine Auto kaufen, kein (...) Haus, wenn ich mal Geld hab, hätte. Also man dort garnix machen. Es ist so schlimm. Wenn man hald nix hat, ne? #00:15:45-6#

I: hm (bejahend) #00:15:46-4#

B: Keine (...) Aufenthalt undso. #00:15:49-8#

I: Zum Glück bist du jetz hier? #00:15:53-2#

B: Ja, da kann man auf jeden Fall was machen, ne? #00:15:55-7#

I: hm (bejahend) (...) ähm also, ein Ding hätte ich, ähm (...) hattest du (...) des is so, meine Frage, mit der sich meine Arbeit beschäftigt is quasi, ähm, ob du, seit du hier in Deutschland angekommen bist, irgendwann mal (...) so nen Moment hattest: 'Oh! So hätte des funktionieren sollen! Des wusste ich nicht.' ähm (...) jetz könntge ich quasi ein Problem, des ich hab, besser lösen, weil ich irgend ne INformation jetz hab. #00:16:39-3#

B: Ja schon. Ist auch klar, wenn du irgenwo ankommst, verstehst du garnix, stell mal vor, du kannst nur Deutsch.  #00:16:48-1#

I: hm (bejahend) #00:16:48-1#

B: Dann gehst du irgendwo anders hin, verstehst du nix. #00:16:48-8#

I: hm (bejahend) #00:16:49-3#

B: Du baust normalerweise Scheiße, ne? #00:16:52-8#

I: Hmm #00:16:52-8#

B: Bist du hald, (unv.). #00:16:54-2#

I: hm (bejahend) #00:16:56-4#

B: So, bis du verstehst undso. #00:16:55-7#

I: ähm ist, ist irgendwann bei den Ämtern auch mal was schief gelaufen oder so? #00:17:03-1#

B: (...) Pff, weiß nicht. Glaube ich nicht. #00:17:07-8#

I: Also nicht, dass du irgendwas mitbekommen hättest? #00:17:09-5#

B: Joa. #00:17:08-8#

I: (...) ähm, und bei #00:17:15-5#

B: // Außer, warte noch was: Weiß nicht, war das (...) 2016 glaube ich. Das ist, das läuft immer noch. Ich hab dann wieder mal ein Termin beim Gericht.  #00:17:27-5#

I: Wa~s? #00:17:29-5#

B: Ja. Das war, so, bei ein Fußballspiel gegen Regenstauf #00:17:32-2#

I: hm (bejahend) #00:17:33-5#

B: Da wurde jemand geschlagen, also es war so irgendwie eine Schlägerei dabei.  #00:17:39-4#

I: Ja. #00:17:40-2#

B: Da war hald, ein Kamerad dabei. Also es, also die hald diese Ding passiert, gefilmt hat ne? #00:17:46-1#

I: hm (bejahend) #00:17:46-1#

B: Dann wurde beim Gericht, also ich war schonmal auch beim Gericht #00:17:49-5#

I: hm (bejahend) #00:17:49-9#

B: Wegen das da (...) Und, da muss ich jetzt wieder hin wegen Prüfen undso, weil ich hab gesagt: 'Es stimmt nicht, was die sagen' weil irgendjemand hat gesagt, dass ich Ihn geschlagen. #00:18:05-0#

I: hm (bejahend) #00:18:05-1#

B: Das stimmt nicht, weil in diese Videos sieht man auch ganz wenig. Also auf jeden Fall sieht man wo ich bin und (...) wie ist das passiert, ne? #00:18:11-9#

I: Ja. #00:18:12-9#

B: Der Typ hat gemeint, deswegen dass ich Ihn geschlagen. Das war auch schwere (...)ähm Körperverletzung. #00:18:19-2#

I: ähm (...) #00:18:23-5#

B: Und die, und der (...) hald, das Gericht wollen, dass nicht, hald, akzeptieren. #00:18:26-7#

I: hm (bejahend) #00:18:27-5#

B: Die meinen, ja, das sieht man garnicht in diese Video undso, aber das sieht man schon bisschen was. Auf jeden Fall, die können auf jeden Fall beweisen mit diese Videomaterial dass das nicht passiert ist, was der sagt, ne? Weil der hat mal gesagt, dass ich Ihm (...) keine Ahnung, dreimal oder so (...) mit ähm, Fußballschuhe in den Bauch geschlagen. ähm Das sieht man, dass das ein Person nur mit eim (unv.) und seim (unv.) hat. #00:18:56-5#

I: Wie is es denn dazu gekommen überhaupt? #00:18:58-3#

B: Keine Ahnung. Und irgendwann, der Typ hat gesagt, dass, wie schaut aus mit ähm Schmerzgeld, weißt du? #00:19:03-6#

I: hm (bejahend) #00:19:03-6#

B: Ich glaub, der braucht eigentlich nur diese Schmerzgeld #00:19:06-8#

I: Ja. #00:19:06-8#

B: (...) Und das Gericht wollten eigentlich das nicht akzeptiern. Dann habe ich mir gedacht: 'Das geht nicht! Ich bin, wir wollen zum Prüfung gehen.' #00:19:16-7#

I: hm (bejahend) #00:19:17-2#

B: Also es ist, das schon, aber sonst nix. Also ich hab nix zu Tun mit Gericht undso. (...) #00:19:24-7#

I: ähm und auch so in die Richtung, zum Beispiel bei (...) ähm Mahdi war's so: Zum Arzt ähm zum Beispiel wenn ähm wenn ma geht, dass irgendwie der Dolmetscher zum Beispiel nicht dabei war oder sowas in die Richtung? #00:19:40-7#

B: Ja früher schon, ja #00:19:40-7#

I: Ist früher schon passiert? #00:19:41-8#

B: Also als wir nichts konnten, also wir konnten ja nicht reden, und zurück #00:19:44-3#

I: und der (...) #00:19:44-3#

B: Aber müssen wir mit jemandem reden. #00:19:48-6#

I: Wie oft war der denn dabei, wenn Ihr was erledigen musstet? (...) #00:19:54-5#

B: ähm (...) (zuckt mit den Schultern) #00:20:00-3#

I: Okay. #00:20:03-1#

B: (...) Ich bin nach Deutschland gekommen, und wir mussten hald so, Bluttest abgeben und sowas wegen DNA und solche Sachen. #00:20:07-0#

I: Yup. Aber seitdem musstest du auch nimmer, nichtmehr zum Arzt oder so? #00:20:14-8#

B: Ich persönlich (...) nicht. Wenn dann konnte ich selber machen. #00:20:21-3#

I: ähm (...) Hast du (...) hier auch mal Rassismus erlebt zum Beispiel? #00:20:33-4#

B: (...) Schon, aber danach die meinen: 'Ja, das ist ja Spaß, ne?' Ist voll gemein erstmal, dann wenn die merken, dass es (...) Scheiße war, dann ja 'es war Spaß' undso. #00:20:47-5#

I: hm (bejahend) #00:20:49-6#

B: Ja. #00:20:50-7#

I: Was ist denn zum Beispiel passiert? #00:20:52-6#

B: (...) Beispiel am Anfang, als ich die Ausbildung angefangen, ne, in der Arbeit. #00:20:58-3#

I: hm (bejahend) #00:20:58-3#

B: Wann geht, wann (unv.) haben wir irgendwas gemacht oder irgendwas gesagt, also wirklich, das ist wirklich (...) traurig is ne? #00:21:05-0#

I: hm (bejahend) #00:21:05-5#

B: Danach war nur so: 'Ja, das war ein Spaß undso.' Oder (...) es gibt auch solche Sachen, sieht man ja auch wenn man auf die Straße läuft manchmal. Wenn die Leute so komisch schauen (...) so komisch anschauen und (...) vielleicht sagen die irgendwas, ne? #00:21:23-9#

I: hm (bejahend) #00:21:23-9#

B: Sowas gibts auch. Aber, aber in Regensburg nicht so viel. Aber (...) schon bisschen.  #00:21:31-0#

I: hm (bejahend). Ist dir das wo anders mehr passiert? #00:21:32-4#

B: (...) ähm (...) Ja. Beispiel, ich war mal in Dresden #00:21:41-6#

I: Du warst auch mal in Dresden? #00:21:42-0#

B: Ja. #00:21:43-5#

I: Warte, du bist in (...) #00:21:46-4#

B: Ne, ich bin hald zum, ähm, zu meinem Kumpel besuch gekommen. #00:21:48-2#

I: Ahso! Okay. (...) Ja gut (...) #00:21:52-2#

B: Dresden ist ja (...) ähm #00:21:52-4#

I: Dresden ist n bisschen berüchtigt. #00:21:53-6#

B: //Ja, ich weiß (lacht). #00:21:55-7#

I: Ahh. (...) #00:21:57-3#

B: Ja ich meine hald, Regensburg am besten. (...) Irgendsowas. #00:22:02-7#

I: hm (bejahend) #00:22:02-7#

B: Ja das kommt auch drauf an, weil (...) es ist, es ist auch so, dass die Leute denken, alle Ausländer sind gleich. Also die sehen ja ähm, am meißten diese (...) schlimme, also die, die Scheiß bauen, ne? #00:22:17-5#

I: hm (bejahend) #00:22:17-5#

B: (...) Und die denken hald, alle sind so. Die bauen immer Schieße. (...) #00:22:25-0#

I: Sind alles Menschen, und irgendwie (...) #00:22:29-8#

B: Ja, ja. #00:22:29-8#

I: Ganz egal, woher, es gibt Arschlöcher und es gibt gute Menschen #00:22:33-5#

B: Ja, klar, aber #00:22:34-9#

I: // Das ist überall (...) also (...). #00:22:36-3#

B: Das müssen alle Leute eigentlich wissen, ne? Nicht nur wir zwei! Das wissen wir, schon. (...) ALso, die Leute die da sind, ne? Sonnst würden nicht hier kommen. Glaube ich. #00:22:49-3#

I: hm (bejahend) (...) Ooh, for fuck's sake. (...) Okay, warte (...) ähm, wie informierst du dich denn normal? #00:23:07-7#

B: Was? #00:23:07-7#

I: Keine Ahnung, ähm (...) hast du noch Kontakt nach Hause zum Beispiel? #00:23:15-8#

B: Ja, per Internet. #00:23:17-2#

I: Über's Internet? Also, Facebook oder was bei dir? #00:23:18-8#

B: Ja, sowas. #00:23:17-5#

I: Ja. ähm (...) ähm, ist dein (...) du bist ein anerkannter Flüchtling oder? Also, läuft dein Asylverfahren noch? #00:23:39-5#

B: Ja. #00:23:40-4#

I: Das läuft noch. #00:23:41-8#

B: Ja. #00:23:42-6#

I: ähm (...) #00:23:45-4#

B: Also ich hab hald irgendwie so (...) eigene (...) eine Aufenthaltsgenehmigung, das heißt, ähm Para (...) Paragraph keine Ahnung, (unv.) dann Aus (...) Aus (...)  Ausschiebungsverbot? Weil dürfen mich nicht nach Afghanistan schicken, weil ich im Iran geboren #00:24:00-4#

I: hm (bejahend) #00:24:00-8#

B: Und die dürfen mich nicht nach Iran schicken, weil ich  (...) kein Iraner bin.  #00:24:04-8#

I: hm (bejahend) #00:24:06-0#

B: Also, hab ich hald Ausschiebungsverbot. #00:24:11-2#

I: Das ist auch ne eher seltene Situation oder? #00:24:14-8#

B: hm (bejahend) #00:24:14-8#

I: ähm, (...) aber (...) Hmm. Warte. Das, bei dem Verfahren gab's jetzt bisher (...) wenig Probleme, oder? #00:24:42-6#

B: Hm? #00:24:42-6#

I: Also irgendwas, dass (...) nach deiner Einschätzung nicht so gelaufen ist wie's laufen hätte sollen? #00:24:50-3#

B: ähm, hm. (...) Ah, als ich nach Deutschland, bevor dass ich nach Deutschland komme, habe mir gedacht: 'Ja, wenn ich in Deutschland bin, oder in Europa' #00:25:04-6#

I: hm (bejahend) #00:25:04-6#

B: ähm, 'kann ich dann normal leben vielleicht?' Also, ich hab mir so gedacht am Anfang. Dass ich dort normal Leben kann.  #00:25:12-1#

I: hm (bejahend) #00:25:12-5#

B: Undso. Also normal leben heißt, dass ich hald (...) irgendein (...) also dass ich hald sicher bleiben kann. #00:25:22-9#

I: Ja. #00:25:23-7#

B: Und dann normal hald arbeiten gehen undso #00:25:27-0#

I: hm (bejahend) #00:25:27-0#

B: Aber (...) Das war (...) ziemlich schwierig, mit solche Sachen.  #00:25:31-7#

I: Bisschen komplizierter. #00:25:33-1#

B: Ja. #00:25:31-3#

I: (...) Wie unterscheiden sich ähm, der Iran und Deutschland? Also für dich jetzt. Also, gab's da irgendwas, bei dem du dir dachtest: (...) 'Waaas? Also damit hast du zum Beispiel überhaupt ned gerechnet, dass das so läuft.' (...) #00:25:55-9#

B: Also ich bin jetzt ja nichtmehr Iraner, sondern Afghane. #00:25:59-1#

I: Ja. #00:25:59-3#

B: Und der Präsident von Afghanistan (...) (unv.) erstmal aufgehört.  #00:26:03-3#

I: hm (bejahend) #00:26:03-8#

B: Der, der ist oft nach Deutschland gekommen, und der hat so (...) mit deutsche Regierung und deutsche (...) #00:26:12-2#

I: hm (bejahend) #00:26:12-2#

B: Politik geredet, dass die hald Afghanistan bisschen Geld gibt. Also er hald unsere Präsident Geld gibt. #00:26:22-7#

I: Ja. #00:26:22-7#

B: Und er kann dann in Afghanistan, keine Ahnung, solche (...) Arbeit für die Leute, die nach Afghanistan wieder zurück kommen, aufbauen und sowas machen. Wohnungen aufbauen. #00:26:33-4#

I: hm (bejahend) #00:26:34-2#

B: Und der hat, der hat immer so viel Geld von Deutschland oder Europa gekriegt, und er hat nix gemacht. Ich habe auch mal mitgekriegt, dass die Leute, die, die aus (...) Europa raus, aus Deutschland nach Afghanistan zurück (...) abgeschoben wurden #00:26:53-0#

I: hm (bejahend) #00:26:53-0#

B: Die wurden (...) also ich hab auf jeden Fall ge (...) also mitgekriegt, dass zwei Leute da beim Kämpfen, oder (...) solche Sachen, gestorben wurden. #00:27:05-2#

I: hm (bejahend) #00:27:05-6#

B: Die hald hier waren, und zurück. Weil die dachten: 'Ja, wir können dort gleich das machen, weil (...) Präsident hat so viel Geld von hier mitgenommen. Aber (...) macht er nix.  #00:27:17-2#

I: Scheiße. #00:27:18-9#

B: Ja. Jeder denkt an sich selber dort.  #00:27:22-5#

I: hm (bejahend) (Begrüßung) (...) Ähm, lass ma's für jetz mal?  #00:27:34-6#

B: Bist schon fertig oder was? #00:27:39-0#

I: ähm (...) ich muss mich erstmal bisschen ordnen muss ich sagen. ähm (...) #00:27:44-3#

B: Ja du kannst mal überlegen, wenn du noch Frage hast, nächste Woche #00:27:45-9#

I: Ja, cool! Danke. #00:27:49-1#