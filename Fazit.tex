\section{Fazit und Ausblick}

Das Ziel dieser Arbeit war es, Informationslücken im Integrationsprozess von Geflüchteten und Asylsuchenden in Deutschland zu untersuchen. Im Rahmen der Arbeit wurden semi-strukturierte, retrospektive Tiefeninterviews mit sechs 19 bis 23 - jährigen, männlichen Geflüchteten aus dem nahen Osten durchgeführt.\newline
Die Interviews wurden nach der Transkription thematisch analysiert.\newline
Die festgestellten Informationslücken deckten sich überwiegend mit denen in der vorhandenen Literatur. Die itralinguistische Sprachbarriere, wie in diesem Fall durch den bayrischen Dialekt, wurde in vorausgehenden Arbeiten zu meinem Wissen allerdings noch nicht angesprochen.\newline
Auch wirkungsvolle Methoden, um mit erlebtem Rassismus umzugehen, wurden nur in Teilen ergründet.\newline
Es lässt sich abschließend sagen, dass den Teilnehmern dieser Studie eine reelle Chance zur Integration in die deutsche Gesellschaft gegeben wurde. Dennoch gibt es noch Bereiche, bei denen Nachjustirungen im System oder systematische Veränderungen notwendig sind.

\begin{quote}
    ``The comity of European peoples went to pieces when, and because, it allowed its weakest member to be excluded and persecuted.''
\end{quote}
\centerline{\textit{Hannah Arendt, \textit{We Refugees}. 1943}}