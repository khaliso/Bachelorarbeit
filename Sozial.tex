\subsection{Soziales Umfeld}

\subsubsection{Kulturell}
Informationslücken im sozialen Bereich erstreckten sich über verschiedene Gebiete. Der syrische Befragte erzählte, dass sich in seinem sozialen Umfeld durchaus auch deutsche befinden würden. Allerdings seien diese durchweg älter als der 23-jährige, und er hatte Probleme, sich mit Seinen ehemaligen Klassenkameraden in der Ausbildung zurecht zu finden. Dies begründe sich zum einen darin, dass Schweinefleisch und Bier in der bayrischen Esskultur ein hoher Stellenwert eingeräumt wird. IT1 ist in Syrien aufgewachsen, einem muslimisch geprägten Land. Dort werden weder der Konsum von Schweinefleisch noch von Alkohol gesellschaftlich akzeptiert.\newline
Er selbst gibt an, gelegentlich Alkohol zu trinken - bei sozialen Ereignissen in Syrien werde aber vor allem Kaffee und Tee getrunken. Als Er mit seinen damaligen Mitschülern nach der Schule etwas unternahm, stellte Er fest, dass deutsche junge Erwachsene vor allem an Bier trinken. Er gab an, dass Er den Eindruck hatte, von der Gruppe nicht akzeptiert worden zu sein, als Er nur Kaffee trank.
\begin{quote}
    ``Also warum(...)wir trinken schon, warum trinkst du nicht? Ich sag's ganz ehrlich du hast ja, also.. (unv.) Die Antwort kommt sofort: Du bist hier in Bayern. Du musst trinken![..]Du musst Schwein essen!''
\end{quote}
\caption{IT1.1, min 7}
\begin{quote}
    `` Wenn man nicht mittrinkt, dann (...) es gibt schon immer Abstand. (...)''
\end{quote}
\caption{IT1.1, min 9}
Dieses Verhalten schrieb er überwiegend deutschen jungen Erwachsenen zu; diese Situation habe Er mit älteren Deutschen nicht erlebt.\newline
Eine andere kulturell basierte soziale Informationslücke lag im Verhältnis zu den Mitmenschen. Die Herkunftsgesellschaft und die neue Gesellschaft wiesen merkliche Unterschiede auf, welche Anfangs für Verwirrung sorgten.\newline
IT1 erzählt, dass seine Herkunftsgesellschaft neuen Bekanntschaften gegenüber aufgeschlossener sei:
\begin{quote}
    ``Zum Beispiel wir unterscheiden nicht das (...) das ist ein Kollege, also Kollegen in der Arbeit oder das ist ein Freund von mir, das ist (...) ne. Die sind alle Freunde.''
\end{quote}
\caption{IT1.1 min14}
IT4 hat die selbe Erfahrung gemacht; er erzählt, dass Er im Deutschkurs bereits auf diese Problematik vorbereitet worden sei.
\begin{quote}
    ``Genau, davor haben wir in die Schule auch gelernt, dass wenn bei deutsche arbeitet oder irgendwas macht, dann die brauchen länger, dass die dich kennt [..] und dir vertraut undso.''
\end{quote}
\caption{IT4 min17}

    -irakischer dude

\subsubsection{Alltag}
Für viele der Interviewten nimmt Sport einen hohen Stellenwert in Ihrem Leben ein. Ein Interviewter war auf der Suche nach einer für Ihn passenden Sportart. Er war zwei jahre Mitglied eines örtlichen Fitnessstudio, die finanzielle Belastung und monotonie des Fitnessstudios bewegten Ihn dazu, etwas neues auszuprobieren. Ein Freund bewegte Ihn dazu, sich am Boxen zu versuchen. Das sagte Ihm zu, und auf der Suche nach einer passenden Lokalität wandten Sie sich an das EJSA - Jugendcafé, welches der Interviewte frequentiert. Einer der Nebenräume wurde zu diesem Zeitpunkt gelegentlich für Tanzunterricht verwendet. Die zuständigen Betreuer erklärten sich bereit, den Geflüchteten diese Räumlichkeit einmal wöchentlich zur Verfügung zu stellen. Seitdem wird diese Möglichkeit von einer wachsenden Zahl interessierter wahrgenommen.\newline
Dieser Boxclub ist nun fester Bestandteil des wöchentlichen Ablaufs vieler Geflüchteter. Der Interviewte gibt weiter an, hier auch Unterstützung in anderen Bereichen gefunden zu haben; Vor Seiner Aktivität im Boxclub hatte er für die anderen Teilnehmer des Jugendcafé immer Donnerstags gekocht.
\begin{quote}
    ``[..]'Es ist gut, wenn ich herkomme. Und es ist schon auch gut von der (...) deutsche Sprache und kann man schon auch was lernen und die helfen uns!' und dann (...) war schon dabei. Bin immer gekommen, und jeder frei, jeder Donnerstag habe ich schon gekocht.''
\end{quote}
Die fünf afghanisch - und iranischstämmigen Geflüchteten frequentieren das EJSA Jugendcafé in regelmäßigen Abständen; alle äußerten sich positiv über die Angebote und Belegschaft. Die EJSA bietet Ihnen die Möglichkeit, Menschen mit einer ähnlichen persönlichen Geschichte zu treffen. Es wird eine Schnittstelle zwischen der Kultur des Herkunftslandes und der des Aufnahmelandes geschaffen.

\subsubsection{Rassismus}

Einige Interviewte wurden im laufe der Zeit in Deutschland mit Rassismus konfrontiert. Ein Interviewter wurde etwa an seinem alten Ausbildungsplatz als Kassierer von einem älteren Herren angesprochen:
\begin{quote}
    `` Der hat einfach, angefangen mit dem Sprechen so laut, hat gesagt: 'Die (...) Flüchtlingen sind nach Deutschland gekommen, die (...) können nicht arbei(...) in der Arbeit. Die kriegen von der Stadt Geld, und Sie kriegen unsere Geld, wir geben gerade der Stadt Geld an die (...) die sind faul[..]'' 
\end{quote}
\caption{IT2.2, min58}
Daraufhin war IT2 mit der Frage konfrontiert, wie Er auf diese Situation reagieren sollte. Er entschloss sich, sich sachlich vor seinem Kunden zu rechtfertigen; dieser Versuch blieb jedoch fruchtlos. Die Diskussion wurde von außenstehenden Ausländern unterbrochen, welche Ihm rieten, nicht mit dem Mann zu reden.
\begin{quote}
    ``Dann ich bin einfach (...) von Geschäft rausgekommen, nach Hause, ich war (...) so ein Tag, zwei Tage, war ich so ein bisschen traurig. Danach ist, habe ich vergessen alles. Ja.''
\end{quote}
\caption{IT2.2. min59}
Die Konfrontation mit diesem Mann beschäftigt den Interviewten nach wie vor; Er gibt an, noch keine Antwort auf diese Frage gefunden zu haben.
\begin{quote}
    ``Ich finde, vielleicht der hat recht? Weiß ich nicht.''
\end{quote}
\caption{IT2.2 min 59}
Wie bereits kurz im Teilabschnitt Bildung angesprochen, hatte ein Proband mit Rassismus am Arbeitsplatz zu kämpfen. Seine Kollegen hatten Ihn beleidigt und die Aussage als Scherz abgetan, als Ihnen die Tragweite Ihrer Aktion bewusst wurde.\newline
Ein weiterer Interviewter hatte mehrere Fälle von Rassismus im Alltag auf der Straße erlebt. Diese Fälle rangierten von Beschimpfungen über Bespucken bis hin zu einem Fall, an dem ein Deutscher Freund des Probanden zur Rede gestellt wurde:
\begin{quote}
    ``'Du bist doch Deutscher! Was machst du mit der Ausländer? Schämst du dich nicht? (...) Ja. Schämst du dich nicht?'[..]'Ist doch Ausländer! Du musst Ihn schlagen! Ist Ausländer!'
\end{quote}
\caption{IT6, min48}
Der Interviewte erzählte, dass er in keinem der Fälle weiter auf die betreffende Person eingegangen sei und diese ignoriert habe.\newline
Es lässt sich also feststellen, dass die Frage, wie mit Rassismus umgegangen werden kann, für einen Großteil der Beteiligten noch nicht beantwortet wurde. Mögliche Lösungswege, welche angewendet wurden, waren entweder einen Dialog zu suchen oder die Situation zu ignorieren.

    
    %T1.1: Wollen die mich dabei haben? (IT1, min17:51), wollen keinen Kontakt (IT1, min19)
    %IT2: Wie verhalte ich mich, wenn Ich rassismus erlebe? (IT2, min57)( IT3, min21)
    %Wie gehe ich mit Rassismus am Arbeitsplatz um? (IT3, min21. vgl Bildung)
    % Basiert Rassismus teilweise auf MEdienberichten? (IT3, min22:17)(IT6 -> das mit dem unmotivierten Zahnarzhelfer)
    %Was mache ich, wenn mir der Zutritt zur Disco verwehrt wird, weil die Aufenthaltsgenehmigung nicht als gültiges Ausweisdokument anerkannt wurde? (IT6, min21)
    
    %IT1.1
    %B: Aber die reden nur bayrisch. Und wenn ich mit denen rede, dann.. Pff ja also wenn ich denen nicht verstehe und Sie mich nicht verstehen, da gibts keinen Kontakt dabei #00:16:10-7#
    %    Sozial:
    %Trinkkultur (IT1.1: min8:15, 11:14)
    %    Trinkkultur in Bayern / Alkoholkonsum (wurde nur in IT1 abgefragt.)
    %Anderes Verhältnis zu Mitmenschen zwischen D und Syrien (min13:12, alle Freunde)
    %Wie kommuniziere ich mit meiner Familie, ohne Sie in Gefahr zu    bringen? (IT2, min15)
    %WIe kontaktiere ich meine Familie (IT1.1, min10:16) -> nicht integration bezogen
    %Bayrisch (IT1.1 min15:43, vgl sprachlich)
    %Was gibt es in diesem Land für soziale Regeln? (IT3, min10:20 -> peinlich?)
    %Wie lerne ich neue Leute kennen? (IT4, min17)
    %    In Deutschland ist es sozial akzeptiert, nachzufragen, und zu hinterfragen? (IT5, min54/55)
        
        
    %IT1: Verbidung mit Bildung: Mitschüler haben Ihm Papiere nicht gegeben.  