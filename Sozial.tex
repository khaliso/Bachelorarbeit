	\item [soziales Umfeld]
		extra-curricular
		
\subsection{Soziales Umfeld}

    Sozial:
    Trinkkultur (IT1.1: min8:15, 11:14)
        Trinkkultur in Bayern / Alkoholkonsum (wurde nur in IT1 abgefragt.)
    Anderes Verhältnis zu Mitmenschen zwischen D und Syrien (min13:12, alle Freunde)
    Wie kommuniziere ich mit meiner Familie, ohne Sie in Gefahr zu    bringen? (IT2, min15)
    WIe kontaktiere ich meine Familie (IT1.1, min10:16)
    Bayrisch (IT1.1 min15:43, vgl sprachlich)
    Was gibt es in diesem Land für soziale Regeln? (IT3, min10:20 -> peinlich?)
    Wie lerne ich neue Leute kennen? (IT4, min17)
        In Deutschland ist es sozial akzeptiert, nachzufragen, und zu hinterfragen? (IT5, min54/55)
        
        
    IT1: Verbidung mit Bildung: Mitschüler haben Ihm papiere nicht gegeben.    

Rassismus:
    IT1.1: Wollen die mich dabei haben? (IT1, min17:51), wollen keinen Kontakt (IT1, min19)
    IT2: Wie verhalte ich mich, wenn Ich rassismus erlebe? (IT2, min57)( IT3, min21)
    Wie gehe ich mit Rassismus am Arbeitsplatz um? (IT3, min21. vgl Bildung)
    Basiert Rassismus teilweise auf MEdienberichten? (IT3, min22:17)(IT6 -> das mit dem unmotivierten Zahnarzhelfer)
    Was mache ich, wenn mir der Zutritt zur Disco verwehrt wird, weil die Aufenthaltsgenehmigung nicht als gültiges Ausweisdokument anerkannt wurde? (IT6, min21)
    
    IT1.1
    B: Aber die reden nur bayrisch. Und wenn ich mit denen rede, dann.. Pff ja also wenn ich denen nicht verstehe und Sie mich nicht verstehen, da gibts keinen Kontakt dabei #00:16:10-7#