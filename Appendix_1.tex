\title{Aufenthaltsgesetz}
\chapter{2 - Einreise und Aufenthalt im Bundesgebiet (§§ 3 - 42)}
\section{ 5 - Aufenthalt aus völkerrechtlichen, humanitären oder politischen Gründen (§§ 22 - 26)}

https://dejure.org/gesetze/AufenthG/25a.html

§ 25a
Aufenthaltsgewährung bei gut integrierten Jugendlichen und Heranwachsenden
(1) Einem jugendlichen oder heranwachsenden geduldeten Ausländer soll eine Aufenthaltserlaubnis erteilt werden, wenn

1.	er sich seit vier Jahren ununterbrochen erlaubt, geduldet oder mit einer Aufenthaltsgestattung im Bundesgebiet aufhält,
2.	er im Bundesgebiet in der Regel seit vier Jahren erfolgreich eine Schule besucht oder einen anerkannten Schul- oder Berufsabschluss erworben hat,
3.	der Antrag auf Erteilung der Aufenthaltserlaubnis vor Vollendung des 21. Lebensjahres gestellt wird,
4.	es gewährleistet erscheint, dass er sich auf Grund seiner bisherigen Ausbildung und Lebensverhältnisse in die Lebensverhältnisse der Bundesrepublik Deutschland einfügen kann und
5.	keine konkreten Anhaltspunkte dafür bestehen, dass der Ausländer sich nicht zur freiheitlichen demokratischen Grundordnung der Bundesrepublik Deutschland bekennt.
Solange sich der Jugendliche oder der Heranwachsende in einer schulischen oder beruflichen Ausbildung oder einem Hochschulstudium befindet, schließt die Inanspruchnahme öffentlicher Leistungen zur Sicherstellung des eigenen Lebensunterhalts die Erteilung der Aufenthaltserlaubnis nicht aus. Die Erteilung einer Aufenthaltserlaubnis ist zu versagen, wenn die Abschiebung aufgrund eigener falscher Angaben des Ausländers oder aufgrund seiner Täuschung über seine Identität oder Staatsangehörigkeit ausgesetzt ist.
(2) Den Eltern oder einem personensorgeberechtigten Elternteil eines minderjährigen Ausländers, der eine Aufenthaltserlaubnis nach Absatz 1 besitzt, kann eine Aufenthaltserlaubnis erteilt werden, wenn

1.	die Abschiebung nicht aufgrund falscher Angaben oder aufgrund von Täuschungen über die Identität oder Staatsangehörigkeit oder mangels Erfüllung zumutbarer Anforderungen an die Beseitigung von Ausreisehindernissen verhindert oder verzögert wird und
2.	der Lebensunterhalt eigenständig durch Erwerbstätigkeit gesichert ist.
2Minderjährigen Kindern eines Ausländers, der eine Aufenthaltserlaubnis nach Satz 1 besitzt, kann eine Aufenthaltserlaubnis erteilt werden, wenn sie mit ihm in familiärer Lebensgemeinschaft leben. 3Dem Ehegatten oder Lebenspartner, der mit einem Begünstigten nach Absatz 1 in familiärer Lebensgemeinschaft lebt, soll unter den Voraussetzungen nach Satz 1 eine Aufenthaltserlaubnis erteilt werden. 4§ 31 gilt entsprechend. 5Dem minderjährigen ledigen Kind, das mit einem Begünstigten nach Absatz 1 in familiärer Lebensgemeinschaft lebt, soll eine Aufenthaltserlaubnis erteilt werden.
(3) Die Erteilung einer Aufenthaltserlaubnis nach Absatz 2 ist ausgeschlossen, wenn der Ausländer wegen einer im Bundesgebiet begangenen vorsätzlichen Straftat verurteilt wurde, wobei Geldstrafen von insgesamt bis zu 50 Tagessätzen oder bis zu 90 Tagessätzen wegen Straftaten, die nach diesem Gesetz oder dem Asylgesetz nur von Ausländern begangen werden können, grundsätzlich außer Betracht bleiben.
(4) Die Aufenthaltserlaubnis kann abweichend von § 10 Absatz 3 Satz 2 erteilt werden und berechtigt zur Ausübung einer Erwerbstätigkeit.